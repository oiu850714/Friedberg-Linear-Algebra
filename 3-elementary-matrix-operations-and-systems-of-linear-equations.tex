\chapter{Elementary Matrix Operations and Systems
of Linear Equations} \label{ch 3}

This chapter is devoted to two related objectives:
\begin{enumerate}
\item[1.] the study of certain \emph{``rank-preserving'' operations} on matrices;
\item[2.] the application of these operations and the theory of \LTRAN{}s to the \emph{solution} of systems of linear equations.
\end{enumerate}

As a consequence of objective 1, we obtain a simple method for computing the rank of a linear transformation between finite-dimensional vector spaces
\emph{by applying these rank-preserving matrix operations} to a matrix that represents that transformation.

Solving a system of linear equations is probably the most important application of linear algebra.
The familiar method of \emph{elimination} for solving systems of linear equations, which was discussed in \SEC{1.4}, involves the elimination of variables so that a simpler system can be obtained.
The technique by which the variables are eliminated \emph{utilizes three types of operations}:
\begin{enumerate}
\item [1.] interchanging any two equations in the system;
\item [2.] multiplying any equation in the system by a \emph{nonzero} constant;
\item[3.] adding a multiple of one equation to another.
\end{enumerate}

In \SEC{3.3}, we express a system of linear equations as a single matrix equation (\(Ax = b\)).
In this representation of the system, the three operations above are the ``\textbf{elementary row operations}'' for matrices(See \DEF{3.1}).
These operations provide a convenient \emph{computational} method for determining all solutions to a system of linear equations.

\section{Elementary Matrix Operations and Elementary Matrices} \label{sec 3.1}

In this section, we define the elementary operations that are used throughout the chapter.
In subsequent sections, we use these operations to obtain simple computational methods for determining the rank of a linear transformation and the solution of a system of linear equations.
There are two types of elementary matrix operations -- row operations and column operations.
As we will see, the \emph{row} operations are more useful;
they \emph{arise from} the three operations that can be \emph{used to eliminate variables} in a system of linear equations.

\begin{definition} \label{def 3.1}
Let \(A\) be an \(m \X n\) matrix.
Any one of the following three operations on the rows [columns] of \(A\) is called an \textbf{elementary row [column] operation}:
\begin{enumerate}
\item[(1)] interchanging any two rows [columns] of \(A\);
\item[(2)] multiplying any row [column] of \(A\) by a \emph{nonzero} scalar;
\item[(3)] adding any scalar multiple of a row [column] of \(A\) to \emph{other} row [column].
\end{enumerate}

Any of these three operations is called an \textbf{elementary operation}.
Elementary operations are of \textbf{type 1}, \textbf{type 2} or \textbf{type 3} depending on whether they are obtained by (1), (2), or (3).
\end{definition}

\begin{note}
I use abbreviations e.r.o. and e.c.o. to represent elementary row operation and elementary column operation, respectively.
\end{note}

\begin{example} \label{example 3.1.1}
Let
\[
    A = \begin{pmatrix}
        1 & 2 & 3 & 4 \\
        2 & 1 & -1 & 3 \\
        4 & 0 & 1 & 2
    \end{pmatrix}.
\]
Interchanging the second row of \(A\) with the first row is an example of an e.r.o. of type 1. 
The resulting matrix is
\[
    B = \begin{pmatrix}
        2 & 1 & -1 & 3 \\
        1 & 2 & 3 & 4 \\
        4 & 0 & 1 & 2
    \end{pmatrix}.
\]
Multiplying the second column of \(A\) by \(3\) is an example of an e.c.o. of type 2.
The resulting matrix is
\[
    C = \begin{pmatrix}
        1 & 6 & 3 & 4 \\
        2 & 3 & -1 & 3 \\
        4 & 0 & 1 & 2
    \end{pmatrix}.
\]
Adding \(4\) times the third row of \(A\) to the first row is an example of an e.r.o. of type 3.
In this case, the resulting matrix is
\[
    M = \begin{pmatrix}
        17 & 2 & 7 & 12 \\
        2 & 1 & -1 & 3 \\
        4 & 0 & 1 & 2
    \end{pmatrix}.
\]
\end{example}

\begin{remark} \label{remark 3.1.1}
Notice that if a matrix \(Q\) can be obtained from a matrix \(P\) by means of an e.r.o., 
then \(P\) can be obtained from \(Q\) by an e.r.o. \textbf{of the same type}.
(See \EXEC{3.1.8}.)
So, in \EXAMPLE{3.1.1}, \(A\) can be obtained from \(M\) by adding \(-4\) times the third row of \(M\) to the first row of \(M\).
\end{remark}

\begin{definition} \label{def 3.2}
An \(n \X n\) \textbf{elementary matrix} is a matrix obtained by performing an(one) elementary operation on \(I_n\).
The elementary matrix is said to be of \textbf{type 1}, \textbf{2}, or \textbf{3} according to whether the elementary operation
performed on \(I_n\) is a type 1, 2, or 3 operation, respectively.
\end{definition}

For example, interchanging the first two \emph{rows} of \(I_3\) produces the elementary matrix
\[
    E = \begin{pmatrix}
        0 & 1 & 0 \\
        1 & 0 & 0 \\
        0 & 0 & 1
    \end{pmatrix}
\]

\begin{remark} \label{remark 3.1.2}
Note that \(E\) can also be obtained by interchanging the first two \emph{columns} of \(I_3\).
In fact, any elementary matrix can be obtained in \emph{at least two ways} -- either by performing an elementary \emph{row} operation on \(I_n\) or by performing an elementary \emph{column} operation on \(I_n\).
(See \EXEC{3.1.4}.)
Similarly,
\[
    \begin{pmatrix}
        1 & 0 & -2 \\
        0 & 1 & 0 \\
        0 & 0 & 1
    \end{pmatrix}
\]
is an elementary matrix since it can be obtained from \(I_3\) by an e.c.o. of type 3
(adding \(-2\) times the first \emph{column} of \(I_3\) to the third \emph{column})
or by an e.r.o. of type 3 (adding \(-2\) times the third \emph{row} to the first \emph{row}).
\end{remark}

Our first theorem shows that \emph{performing an e.r.o. on a matrix is equivalent to multiplying the matrix by an elementary matrix}.

\begin{theorem} \label{thm 3.1}
Let \(A \in M_{m \X n}(F)\), and suppose that \(B\) is obtained from \(A\) by performing an elementary row [column] operation.
Then there exists an \(m \X m\) [\(n \X n\)] elementary matrix \(E\) such that \(B = EA\) [\(B = AE\)].
In fact, \(E\) is obtained from \(I_m\) [\(I_n\)] by \textbf{performing the same} elementary row [column] operation \textbf{as} that which was performed on \(A\) to obtain \(B\).

Conversely, if \(E\) is an elementary \(m \X m\) [\(n \X n\)] matrix, then \(EA\) [\(AE\)] is the matrix obtained from \(A\) by \textbf{performing the same} elementary row [column] operation \textbf{as} that which produces \(E\) from \(I_m\) [\(I_n\)].
\end{theorem}

\begin{proof}
See \EXEC{3.1.7}.
\end{proof}

\begin{example} \label{example 3.1.2}
Consider the matrices \(A\) and \(B\) in \EXAMPLE{3.1.1}.
In this case, \(B\) is obtained from \(A\) by interchanging the first two rows of \(A\).
Performing this \emph{same} operation on \(I_{3}\), we obtain the elementary matrix
\[
    E = \begin{pmatrix}
        0 & 1 & 0 \\
        1 & 0 & 0 \\
        0 & 0 & 1
    \end{pmatrix}
\]
Note that \(EA = B\).

In the second part of \EXAMPLE{3.1.1}, \(C\) is obtained from \(A\) by multiplying the second column of \(A\) by \(3\).
Performing this same operation on \(I_{4}\), we obtain the elementary matrix

\[
    E = \begin{pmatrix}
        1 & 0 & 0 & 0 \\
        0 & 3 & 0 & 0 \\
        0 & 0 & 1 & 0 \\
        0 & 0 & 0 & 1
    \end{pmatrix}
\]
Observe that \(AE = C\).
\end{example}

It is a useful fact that the \emph{inverse} of an elementary matrix is \emph{also} an elementary matrix.

\begin{theorem} \label{theorem} \label{thm 3.2}
Elementary matrices are \emph{invertible}, and the inverse of an elementary matrix is an elementary matrix \emph{of the same type}.
\end{theorem}

\begin{proof}
Let \(E\) be an elementary \(n \X n\) matrix.
Then (by \DEF{3.1}) \(E\) can be obtained by an e.r.o. on \(I_n\).
By \emph{reversing the steps} used to transform \(I_n\) into \(E\), we can transform \(E\) back into \(I_n\).
The result is that \(I_n\) can be obtained from \(E\) by an e.r.o. of the same type.
By \THM{3.1}, there is an elementary matrix \(\overline{E}\) such that \(\overline{E}E = I_n\).
Therefore, by \ATHM{2.38}(``one-sided'' inverse is a ``two-sided'' inverse), \(E\) is invertible and \(E^{-1} = \overline{E}\).
\end{proof}

\exercisesection

\begin{exercise} \label{exercise 3.1.1}
Label the following statements as true or false.
\begin{enumerate}
\item An elementary matrix is always square.
\item The only entries of an elementary matrix are zeros and ones. \item The \(n \X n\) identity matrix is an elementary matrix.
\item The product of two \(n \X n\) elementary matrices is an elementary matrix.
\item The inverse of an elementary matrix is an elementary matrix.
\item The sum of two \(n \X n\) elementary matrices is an elementary matrix.
\item The transpose of an elementary matrix is an elementary matrix.
\item If \(B\) is a matrix that can be obtained by performing an e.r.o. on a matrix \(A\), then \(B\) can also be obtained by performing an e.\textbf{\RED{c}}.o. on \(A\).
\item If \(B\) is a matrix that can be obtained by performing an e.r.o. on a matrix \(A\), then \(A\) can be obtained by performing an e.r.o. on \(B\).
\end{enumerate}
\end{exercise}

\begin{proof} \ 

\begin{enumerate}
\item True. Since every elementary matrix is obtained from \(I_n\) by an elementary operation that does not change the size of the matrix, and \(I_n\) is square.
\item False. \(\begin{pmatrix} 2 & 0 & 0 \\ 0 & 1 & 0 \\ 0 & 0 & 1\end{pmatrix}\) is elementary of type 2.
\item True. \(I_n\) can be considered as a type 2 elementary matrix with the first row multiplied by \(1\).
\item False. \(\begin{pmatrix} 2 & 0 \\ 0 & 1 \end{pmatrix} \begin{pmatrix} 1 & 0 \\ 0 & 2 \end{pmatrix} = \begin{pmatrix} 2 & 0 \\ 0 & 2 \end{pmatrix}\), which needs ``two'' elementary operations to obtain from \(I_2\).
\item True by \THM{3.2}.
\item False. \(\begin{pmatrix} 1 & 0 \\ 0 & 1 \end{pmatrix} + \begin{pmatrix} 1 & 0 \\ 0 & 1 \end{pmatrix} = \begin{pmatrix} 2 & 0 \\ 0 & 2 \end{pmatrix}\), which is not elementary by part(d).
\item True by \EXEC{3.1.5}.
\item False. The matrix \(\begin{pmatrix} 5 & 5 \\ 1 & 1 \end{pmatrix}\) can be obtained from \(\begin{pmatrix} 1 & 1 \\ 1 & 1 \end{pmatrix}\) by a single e.r.o. of type 3, but not by any number of \emph{column} operations, since any e.c.o. on \(\begin{pmatrix} 1 & 1 \\ 1 & 1 \end{pmatrix}\) will still leave columns that are multiple of \(\begin{pmatrix} 1 \\ 1 \end{pmatrix}\).
\item True by \EXEC{3.1.8}.
\end{enumerate}
\end{proof}

\begin{exercise} \label{exercise 3.1.2}
Let
\[
    A = \left(\begin{array}{rrr}
        1 & 2 & 3 \\
        1 & 0 & 1 \\
        1 & -1 & 1
    \end{array}\right),
    B = \left(\begin{array}{rrr}
        1 & 0 & 3 \\
        1 & -2 & 1 \\
        1 & -3 & 1
    \end{array}\right), \text { and }
    C = \left(\begin{array}{rrr}
        1 & 0 & 3 \\
        0 & -2 & -2 \\
        1 & -3 & 1
    \end{array}\right)
\]
Find an elementary operation that transforms \(A\) into \(B\) and an elementary operation that transforms \(B\) into \(C\).
\emph{By means of several additional operations}, transform \(C\) into \(I_3\).
\end{exercise}

\begin{proof}
Using type 3 e.c.o., we add column \(1\) of \(A\) times \(-2\) to column \(2\) of \(A\) to obtain \(B\);
and using type 3. e.r.o., we add row \(1\) of \(B\) times \(-1\) to row \(2\) of \(B\) to obtain \(C\);

Finally, the process of obtaining \(I_3\) from \(C\) is:
\[
    \begin{aligned}
    C & =\left(\begin{array}{ccc}
        1 & 0 & 3 \\
        0 & -2 & -2 \\
        1 & -3 & 1
    \end{array}\right)
    \rightsquigarrow
    \left(\begin{array}{ccc}
        1 & 0 & 3 \\
        0 & \RED{1} & \RED{1} \\
        1 & -3 & 1
    \end{array}\right) \\
      & \rightsquigarrow
    \left(\begin{array}{ccc}
        1 & 0 & 3 \\
        0 & 1 & 1 \\
        \RED{0} & -3 & \RED{-2}
    \end{array}\right)
    \rightsquigarrow
    \left(\begin{array}{ccc}
        1 & 0 & 3 \\
        0 & 1 & 1 \\
        0 & \RED{0} & \RED{1}
    \end{array}\right) \\
    & \rightsquigarrow
    \left(\begin{array}{ccc}
        1 & 0 & \RED{0} \\
        0 & 1 & 1 \\
        0 & 0 & 1
    \end{array}\right)
    \rightsquigarrow\left(\begin{array}{ccc}
        1 & 0 & 0 \\
        0 & 1 & \RED{0} \\
        0 & 0 & 1
    \end{array}\right)
\end{aligned}
\]
\end{proof}

\begin{exercise} \label{exercise 3.1.3}
Use the proof of \THM{3.2} to obtain the \emph{inverse} of each of the following elementary matrices.
\[
    \text{(a)} \left(\begin{array}{lll}0 & 0 & 1 \\ 0 & 1 & 0 \\ 1 & 0 & 0\end{array}\right)
    \text{ (b) } \left(\begin{array}{lll}1 & 0 & 0 \\ 0 & 3 & 0 \\ 0 & 0 & 1\end{array}\right)
    \text{ (c) } \left(\begin{array}{rrr}1 & 0 & 0 \\ 0 & 1 & 0 \\ -2 & 0 & 1\end{array}\right)
\]
\end{exercise}

\begin{proof} \ 

\begin{enumerate}
\item This is type 1 e.r.o. by changing row \(1, 3\).
So the inverse is also type 1 e.r.o. by changing row \(1, 3\) back;
that is, the inverse is \(\begin{pmatrix} 0 & 0 & 1 \\ 0 & 1 & 0 \\ 1 & 0 & 0 \end{pmatrix}\) itself.

\item This is type 2 e.r.o. by multiplying row \(2\) by \(3\).
So the inverse is type 2 e.r.o. by multiplying row \(2\) by \(\frac1{3}\);
that is, \(\begin{pmatrix} 1 & 0 & 0 \\ 0 & \frac1{3} & 0 \\ 0 & 0 & 1 \end{pmatrix}\).

\item This is type 3 e.r.o. by adding row \(1\) times \(-2\) to row \(3\).
So the inverse is type 3 e.r.o. by add row \(1\) time \(2\) to row \(3\).
that is,\(\begin{pmatrix} 1 & 0 & 0 \\ 0 & 1 & 0 \\ 2 & 0 & 1 \end{pmatrix}\).
\end{enumerate}
\end{proof}

\begin{exercise} \label{exercise 3.1.4}
Prove the assertion made on \RMK{3.1.2}:
Any elementary \(n \X n\) matrix can be obtained in at least two ways -- either by performing an e.r.o. on \(I_n\) or by performing an e.c.o. on \(I_n\).
\end{exercise}

\begin{proof}
Let \(E\) be an \(n \X n\) elementary matrix.
There are three types of elementary operations:
\begin{enumerate}
\item[1.] Interchange two columns or two rows.
\item[2.] Multiply any row or any column of \(E\) by a nonzero scalar.
\item[3.] Add any scalar multiple of a row of \(E\) to another row, or add any scalar multiple of a column of \(E\) to another column.
\end{enumerate}

We say that an elementary matrix is type (1), (2) or (3) if it is obtained performing an elementary operation type 1, 2 or 3, respectively, on \(I_n\).

If we suppose that \(E\) is type (1) and suppose that an elementary operation is performed on rows \(i\) and \(j\) of  \(I_n\),
then by interchanging \emph{columns} \(i\) and \(j\) of \(I_n\) we get the same matrix.

If we suppose that \(E\) is type (2) by multiplying row \(i\) of \(I_n\) using a nonzero scalar \(a\), than \(E\) is a matrix where all entries are same as in \(I_n\), except an item on a position \((i, i)\) in \(E\), which is equal \(a\).
If we multiply a \emph{column} \(i\) by a nonzero scalar \(a\) on \(I_n\), we can get the same matrix \(E\).

Finally, suppose that \(E\) is type (3) by multiplying row \(i\) of \(I_n\) by a nonzero scalar \(a\) and adding to row \(j\) of \(I_n\),
then \(E\) is a matrix where all entries are same as in \(I_n\), except an item on a position \((j, i)\), which is equal to \(a\).
If we multiply the \emph{column} \(j\) of \(I_n\) by the scalar \(a\) and add to the column \(i\) of \(I_n\) we can get the same matrix \(E\).

In all cases, we can find the second way to create matrix \(E\).
\end{proof}

\begin{exercise} \label{exercise 3.1.5}
Prove that \(E\) is an elementary matrix if and only if \(E^\top\) is.
\end{exercise}

\begin{proof}
We can easily check that type 1 and type 2 elementary matrices are symmetric, hence the transpose of them are also elementary matrices.

For type 3, suppose \(E\) is obtained from adding row \(i\) of \(I_n\) times scalar \(c\) to row \(j\) of \(I_n\).
Then \(E\) can be considered as \(I_n\) except that the position \((j, i)\) is equal to \(c\).
In particular, \(E^\top\) can be considered as \(I_n\) except that the position \((i, j)\) is equal to \(c\).
Then it's obvious that \(E^\top\) is obtained from adding row \(j\) of \(I_n\) times scalar \(c\) to row \(i\) of \(I_n\), so \(E^\top\) is also an elementary matrix.
\end{proof}

\begin{exercise} \label{exercise 3.1.6}
Let \(A\) be an \(m \X n\) matrix.
Prove that if \(B\) can be obtained from \(A\) by an elementary row [column] operation, then \(B^\top\) can be obtained from
\(A^\top\) by the corresponding elementary \emph{column} [row] operation.
\end{exercise}

\begin{proof}
Let
\[
    A = \begin{pmatrix} a_1 \\ a_2 \\ \vdots \\ a_m \end{pmatrix}
\]
where \(a_i\) are rows of \(A\).
Then in particular
\[
    A^\top = \begin{pmatrix} a_1^\top a_2^\top ... a_m^\top \end{pmatrix}
\]

Suppose \(B\) is obtain from \(A\) with a type 1 e.r.o., that is, changing row \(i\) and row \(j\) of \(A\) (and WLOG where \(i \le j\)).
Then \(B\) can be considered as
\[
    B = \begin{pmatrix} a_1 \\ a_2 \\ \vdots \\ \RED{a_j} \\ \vdots \\ \RED{a_i} \\ \vdots \\ a_m \end{pmatrix}
\]
Then in particular
\[
    B^\top = \begin{pmatrix} a_1^\top a_2^\top ... \RED{a_j^\top} ... \RED{a_i^\top} ... a_m^\top \end{pmatrix}
\]
which can be obtain from \(A^\top\) with a type 1 e.\RED{c}.o., that is, changing column \(i\) and column \(i\) of \(A^\top\).

Suppose \(B\) is obtain from \(A\) with a type 2 e.r.o., that is, multiplying row \(i\) of \(A\) by a nonzero scalar \(c\).
Then \(B\) can be considered as
\[
    B = \begin{pmatrix} a_1 \\ a_2 \\ \vdots \\ \RED{ca_i} \\ \vdots \\ a_m \end{pmatrix}
\]
Then in particular
\[
    B^\top = \begin{pmatrix} a_1^\top a_2^\top ... \RED{ca_i^\top} ... a_m^\top \end{pmatrix}
\]
which can be obtain from \(A^\top\) with a type 2 e.\RED{c}.o., that is, multiplying column \(i\) of \(A^\top\) by the nonzero scalar \(c\).

Finally, suppose \(B\) is obtain from \(A\) with a type 3 e.r.o., that is, adding row \(i\) of \(A\) times a scalar \(c\) to row \(j\) of \(A\).
Then \(B\) can be considered as
\[
    B = \begin{pmatrix} a_1 \\ a_2 \\ \vdots \\ \RED{ca_i + a_j} \\ \vdots \\ a_m \end{pmatrix}
\]
Then in particular
\[
    B^\top = \begin{pmatrix} a_1^\top a_2^\top ... \RED{(ca_i + a_j)^\top} ... a_m^\top \end{pmatrix} = \begin{pmatrix} a_1^\top a_2^\top ... \RED{(ca_i^\top + a_j^\top)} ... a_m^\top \end{pmatrix}
\]
(where the last equal sign is from \ATHM{1.2}(1)) which can be obtain from \(A^\top\) with a type 3 e.\RED{c}.o., that is, adding column \(i\) of \(A^\top\) times a scalar \(c\) to column \(j\) of \(A^\top\).

The case of e.c.o.s is similar to prove.
\end{proof}

\begin{exercise} \label{exercise 3.1.7}
Prove \THM{3.1}.
\end{exercise}

\begin{proof}
Let
\[
    A=\left[\begin{array}{cccc}
        a_{11} & a_{12} & \cdots & a_{1 n} \\
        a_{21} & a_{22} & \cdots & a_{2 n} \\
        \vdots & \vdots & \ddots & \vdots \\
        a_{m 1} & a_{m 2} & \cdots & a_{m n}
    \end{array}\right] \in M_{m \X n}(F)
\]

\begin{enumerate}
\item 
Suppose \(B\) is obtained from \(A\) with a type 1 e.r.o., that is, changing row \(i\) and row \(j\) of \(A\).
And if we let \(E\) be the same type of elementary matrix that changes row \(i\) and row \(j\) of \(I_n\);
that is,
\[
    E=\left[\begin{array}{cccccccc}
        1 & 0 & \cdots & 0 & \cdots & 0 & \cdots & 0 \\
        0 & 1 & \cdots & 0 & \cdots & 0 & \cdots & 0 \\
        \vdots & \vdots & \ddots & \vdots & & \vdots & & \vdots \\
        0 & 0 & \cdots & 0 & \cdots & \RED{1} & \cdots & 0 \\
        \vdots & \vdots & & \vdots & & \vdots & & \vdots \\
        0 & 0 & \cdots & \RED{1} & \cdots & 0 & \cdots & 0 \\
        \vdots & \vdots & & \vdots & & \vdots & & \vdots \\
        0 & 0 & \cdots & 0 & \cdots & 0 & \cdots & 1
    \end{array}\right] \in M_{m \X m}(F)
\]
(where the top-right \RED{\(1\)} appears in position \((i, j)\), the bottom-left \RED{\(1\)} appears in position \((j, i)\).)
Then
\[
    E A=\left[\begin{array}{cccccccc}
        a_{11} & a_{12} & \cdots & a_{1 i} & \cdots & a_{1 j} & \cdots & a_{1 n} \\
        a_{21} & a_{22} & \cdots & a_{2 i} & \cdots & a_{2 j} & \cdots & a_{2 n} \\
        \vdots & \vdots & & \vdots & & \vdots & & \vdots \\
        \RED{a_{j 1}} & \RED{a_{j 2}} & \RED{\cdots} & \RED{a_{j i}} & \RED{\cdots} & \RED{a_{j j}} & \RED{\cdots} & \RED{a_{j n}} \\
        \vdots & \vdots & & \vdots & & \vdots & & \vdots \\
        \RED{a_{i 1}} & \RED{a_{i 2}} & \RED{\cdots} & \RED{a_{i i}} & \RED{\cdots} & \RED{a_{i j}} & \RED{\cdots} & \RED{a_{i n}} \\
        \vdots & \vdots & & \vdots & & \vdots & & \vdots \\
        a_{m 1} & a_{m 2} & \cdots & a_{m i} & \cdots & a_{m j} & \cdots & a_{m n}
    \end{array}\right]
\]
is exactly the matrix obtained from changing row \(i\) and row \(j\) of \(A\); that is, \(B\).

\item
Suppose \(B\) is obtained from \(A\) with a type 2 e.r.o., that is, multiplying row \(i\) by scalar \(c\).
And if we let \(E\) be the same type of elementary matrix that multiplies row \(i\) by scalar \(c\),
that is,
\[
    E = \left[\begin{array}{ccccccc}
        1 & 0 & \cdots & 0 & 0 & \cdots & 0 \\
        0 & 1 & \cdots & 0 & 0 & \cdots & 0 \\
        \vdots & \vdots & \ddots & \vdots & \vdots & & \vdots \\
        0 & 0 & \cdots & \RED{c} & 0 & \cdots & 0 \\
        0 & 0 & \cdots & 0 & 1 & \cdots & 0 \\
        \vdots & \vdots & & \vdots & \vdots & \ddots & \vdots \\
        0 & 0 & \cdots & 0 & 0 & \cdots & 1
    \end{array}\right] \in M_{m \X m}(F)
\]
(where that red \(\RED{c}\) appears on position \((i, i)\).)
Then
\[
    E \cdot A=\left[\begin{array}{cccccc}
        a_{11} & a_{12} & \cdots & a_{1 i} & \cdots & a_{1 n} \\
        a_{21} & a_{22} & \cdots & a_{2 i} & \cdots & a_{2 n} \\
        \vdots & \vdots & & \vdots & & \vdots \\
        \RED{c a_{i 1}} &  \RED{c a_{i 2}} & \RED{\cdots} & \RED{c a_{i i}} & \RED{\cdots} & \RED{c a_{i n}} \\
        \vdots & \vdots & & \vdots & & \vdots \\
        a_{m 1} & a_{m 2} & \cdots & a_{m i} & \cdots & a_{m n}
    \end{array}\right]
\]
is exactly the matrix obtained from multiplying row \(i\) of \(A\) by scalar \(c\), that is, \(B\).

\item 
Suppose \(B\) is obtained from \(A\) with a type 3 e.r.o., that is, add row \(i\) of \(A\) times scalar \(c\) to row \(j\) of \(A\).
And if we let \(E\) be the same type of elementary matrix that adds row \(i\) of \(I_n\) times \(c\) to row \(j\) of \(I_n\);
that is,
\[
    E=\left[\begin{array}{cccccccc}
        1 & 0 & \cdots & 0 & \cdots & 0 & \cdots & 0 \\
        0 & 1 & \cdots & 0 & \cdots & 0 & \cdots & 0 \\
        \vdots & \vdots & \ddots & \vdots & & \vdots & & \vdots \\
        0 & 0 & \cdots & 1 & \cdots & 0 & \cdots & 0 \\
        \vdots & \vdots & & \vdots & & \vdots & & \vdots \\
        0 & 0 & \cdots & \RED{c} & \cdots & 1 & \cdots & 0 \\
        \vdots & \vdots & & \vdots & & \vdots & & \vdots \\
        0 & 0 & \cdots & 0 & \cdots & 0 & \cdots & 1
    \end{array}\right] \in M_{m \X m}(F)
\]
(where \RED{\(c\)} appears in position \((j, i)\).)
Then
\[
    E A= \left[\begin{array}{cccccccc}
        a_{11} & a_{12} & \cdots & a_{1 i} & \cdots & a_{1 j} & \cdots & a_{1 n} \\
        a_{21} & a_{22} & \cdots & a_{2 i} & \cdots & a_{2 j} & \cdots & a_{2 n} \\
        \vdots & \vdots & & \vdots & & \vdots & & \vdots \\
        a_{i 1} & a_{i 2} & \cdots & a_{i i} & \cdots & a_{i j} & \cdots & a_{i n} \\
        \vdots & \vdots & & \vdots & & \vdots & & \vdots \\
        \RED{a_{j 1} + c a_{i 1}} & \RED{a_{j 2} + c a_{i 2}} & \RED{\cdots} & \RED{a_{j i}+ c a_{i i}} & \RED{\cdots} & \RED{a_{j j} + c a_{i j}} & \RED{\cdots} & \RED{a_{j n}+ c a_{i n}} \\
        \vdots & \vdots & & \vdots & & \vdots & & \vdots \\
        a_{m 1} & a_{m 2} & \cdots & a_{m i} & \cdots & a_{m j} & \cdots & a_{m n}
\end{array}\right]
\]
is exactly the matrix obtained from add row \(i\) of \(A\) times scalar \(c\) to row \(j\) of \(A\), that is, \(B\).
\end{enumerate}

For elementary \emph{column} operations, if \(B \in M_{m \X n}\) is obtained from \(A \in M_{m \X n}\) by performing any type of e.c.o.,
then from \EXEC{3.1.6}, \(B^\top \in M_{n \X m}\) is obtained from \(A^\top \in M_{n \X m}\) by performing the corresponding type of e.\RED{r}.o..
Then by the previous case, we have \(B^\top = EA^\top\) \MAROON{(1)} for some elementary matrix \(E\), and
\begin{align*}
    B & = (B^\top)^\top & \text{by \ATHM{1.2}(2)} \\
      & = (EA^\top)^\top & \text{by \MAROON{(1)}} \\
      & = (A^\top)^\top E^\top & \text{by \ATHM{2.24}} \\
      & = A E^\top & \text{by \ATHM{1.2}(2)}
\end{align*}
And by \EXEC{3.1.5}, \(E^\top\) is also elementary, so we have found a elementary matrix \(E^\top\) s.t. \(B = A E^\top\), as desired.
\end{proof}

\begin{exercise} \label{exercise 3.1.8}
Prove that if a matrix \(Q\) can be obtained from a matrix \(P\) by an e.r.o., then \(P\) can be obtained from \(Q\) by an e.r.o. of the same type.
Hint: Treat each type of e.r.o. separately.
\end{exercise}

\begin{proof}
If we interchange the \(i\)th and \(j\)th rows of \(P\) to get \(Q\), we can change the \(i\)th and \(j\)th rows \(Q\) again to get \(P\), and the operation is type 1 e.r.o..

If we multiply the \(i\)th row of \(P\) by nonzero scalar \(c\) to get \(Q\), we can multiply the \(i\)th row of \(Q\) by nonzero scalar \(\frac1{c}\) to get \(P\), and the operation is type 2 e.r.o..

If we add the \(i\)th row of \(P\) times scalar \(c\), to the \(j\)th row of \(P\), to get \(Q\), then we can add the \(i\)th row of \(Q\) times scalar \(-c\), to the \(j\)th row of \(Q\), to get \(P\), and the operation is type 3 e.r.o..
\end{proof}

\begin{proof} [Another proof for \EXEC{3.1.8}]
By \THM{3.1}, we can write \(EP = Q\) where \(E\) corresponds to an e.r.o..
And by \THM{3.2}, \(E^{-1}\) is also elementary of the same type.
And
\[
    E^{-1} Q = E^{-1} E P = P.
\]
So by \THM{3.1}, \(P\) can be obtained from \(Q\) by an e.r.o. of the same type as \(E^{-1}\), which is the same type as \(E\).
\end{proof}

\begin{exercise} \label{exercise 3.1.9}
Prove that any elementary row [column] operation of type 1 can be obtained by a succession of three elementary row [column] operations of type 3 followed by one elementary row [column] operation of type 2.
\end{exercise}

\begin{proof}
Let \[A = \begin{pmatrix} a_1 \\ a_2 \\ \vdots \\ a_m \end{pmatrix}\].
Then if \(B\) can be obtain from \(A\) by performing a type 1 e.r.o., that is,
\[
    B = \begin{pmatrix} a_1 \\ a_2 \\ \vdots \\ \RED{a_j} \\ \vdots \\ \RED{a_i} \\ \vdots \\ a_m \end{pmatrix}
\]
Then
\[
    A = \begin{pmatrix} a_1 \\ a_2 \\ \vdots \\ \RED{a_i} \\ \vdots \\ \RED{a_j} \\ \vdots \\ a_m \end{pmatrix}
    \text{\RED{(*1)}} \to \begin{pmatrix} a_1 \\ a_2 \\ \vdots \\ \RED{a_i + a_j} \\ \vdots \\ \RED{a_j} \\ \vdots \\ a_m \end{pmatrix}
    \text{\RED{(*2)}} \to \begin{pmatrix} a_1 \\ a_2 \\ \vdots \\ \RED{a_i + a_j} \\ \vdots \\ \RED{-a_i} \\ \vdots \\ a_m \end{pmatrix}
    \text{\RED{(*3)}} \to \begin{pmatrix} a_1 \\ a_2 \\ \vdots \\ \RED{a_j} \\ \vdots \\ \RED{-a_i} \\ \vdots \\ a_m \end{pmatrix}
    \text{\RED{(*4)}} \to \begin{pmatrix} a_1 \\ a_2 \\ \vdots \\ \RED{a_j} \\ \vdots \\ \RED{a_i} \\ \vdots \\ a_m \end{pmatrix}
    = B
\]
where \RED{(*1)} is e.r.o. type 3 by adding row \(j\) to row \(i\);
\RED{(*2)} is e.r.o. type 3 by adding \(-1 \X\) row \(i\) to row \(j\);
\RED{(*3)} is e.r.o. type 3 by adding row \(j\) to row \(i\);
\RED{(*4)} is e.r.o. type 2 by multiplying row \(j\) by \(-1\).

The case of e.r.c. is very similar to prove.
\end{proof}

\begin{exercise} \label{exercise 3.1.10}
Prove that any elementary row [column] operation of type 2 can be obtained by dividing some row [column] by a nonzero scalar.
\end{exercise}

\begin{proof}
Well, multiplying nonzero scalar \(c\) is equal to dividing scalar \(\frac1{c}\).
\end{proof}

\begin{exercise} \label{exercise 3.1.11}
Prove that any elementary row [column] operation of type 3 can be obtained by subtracting a multiple of some row [column] from another row [column].
\end{exercise}

\begin{proof}
Well, adding a row times scalar \(c\) is equal to subtracting a row times \(c\).
\end{proof}

\begin{exercise} \label{exercise 3.1.12}
Let \(A\) be an \(m \X n\) matrix.
Prove that there exists a sequence of e.r.o.s of types 1 and 3 that transforms \(A\) into an upper triangular matrix.
\end{exercise}

\begin{proof}
The \emph{forward pass} of Gaussian elimination(see \THM{3.14}) give a sequence of e.r.o.s of type 1, 2, and 3 that transform any matrix into upper triangular.
In particular, the forward pass only uses type 2 e.r.o.s to transforms the leading term of each row into \(1\).
So if we remove type 2 e.r.o.s in the sequence, then we have found a sequence of e.r.o.s of type 1 and 3 that transforms \(A\) into an upper triangular matrix.
\end{proof}

\begin{additional theorem} \label{athm 3.1}
This is the placeholder theorem for \EXEC{3.1.5}: \(E\) is an elementary matrix if and only if \(E^\top\) is.
\end{additional theorem}

\begin{additional theorem} \label{athm 3.2}
This is the placeholder theorem for \EXEC{3.1.6}: Let \(A\) be an \(m \X n\) matrix.
If \(B\) can be obtained from \(A\) by an e.r.o. [e.c.o.], then \(B^\top\) can be obtained from \(A^\top\) by the corresponding e.c.o. [e.r.o.].
\end{additional theorem}

\begin{additional theorem} \label{athm 3.3}
This is the placeholder theorem for \EXEC{3.1.8}: If a matrix \(Q\) can be obtained from a matrix \(P\) by an e.r.o., then \(P\) can be obtained from \(Q\) by an e.r.o. of the same type.
\end{additional theorem}

\section{The Rank of a Matrix and Matrix Inverses} \label{sec 3.2}

In this section, we define the \emph{rank} of a matrix.
We then use elementary operations to compute the rank of a matrix and a \LTRAN{}.
(Recall that the definition of the rank of a \LTRAN{} is in \DEF{2.3}.)
The section concludes with \emph{a procedure for computing the inverse} of an \emph{invertible} matrix.

\begin{definition} \label{def 3.3}
If \(A \in M_{m \X n}(F)\), we define the \textbf{rank} of \(A\), denoted \(\rank(A)\), to be the rank of the \LTRAN{} \(\LMTRAN_A: F^n \to F^m\).
\end{definition}

\begin{remark} \label{remark 3.2.1}
Many results about the rank of a matrix follow immediately from the corresponding facts about a \LTRAN{}.
An important result of this type, which follows from \ATHM{2.35}(3) and \CORO{2.18.2}, (p. 103), is that \textbf{an \(n \X n\) matrix is invertible if and only if its rank is \(n\)}.
(That is, \(A\) is invertible, iff (by \CORO{2.18.2}) \(\LMTRAN_A\) is invertible, iff (by \ATHM{2.35}(3)) \(\rank(\LMTRAN_A) = \dim(F^n) = n\).) 
\end{remark}

\begin{remark} \label{remark 3.2.2}
Every matrix \(A\) is the matrix representation of the \LTRAN{} \(\LMTRAN_A\) with respect to the appropriate \emph{standard} ordered bases.
(See \THM{2.15}(a).)
And by \DEF{3.3} we have \(\rank(A) = \rank(\LMTRAN_A)\).
Thus the rank of the linear transformation \(\LMTRAN_A\) is the same as the rank of \RED{\textbf{one of}} its matrix representations, namely, \(A\).
The next theorem extends this fact to \RED{\textbf{any}} matrix representation of any linear transformation defined on finite-dimensional vector spaces.
\end{remark}

\begin{theorem} \label{thm 3.3}
Let \(\T : \V \to \W\) be a \LTRAN{} between finite-dimensional vector spaces, and let \(\beta\) and \(\gamma\) be \emph{arbitrary} ordered bases for \(\V\) and \(\W\), respectively.
Then \(\rankT = \rank([\T]_{\beta}^{\gamma})\).
\end{theorem}

\begin{proof}
From \ATHM{2.42}, we have \(\rankT = \rank(\LMTRAN_A)\) where \(A = [\T]_{\beta}^{\gamma}\).
But by \DEF{3.3}, \(\rank(A) = \rank(\LMTRAN_A)\), so we have \(\rankT = \rank(A)\);
that is, \(\rankT = \rank([\T]_{\beta}^{\gamma})\).
\end{proof}

\begin{proof}[Another proof for \THM{3.3}]
This proof uses \ATHM{2.41}(2), which is the hint of \ATHM{2.42}.
(I think it's more intuitive to use \ATHM{2.41}(2) in this context.)

Again let \(A = [\T]_{\beta}^{\gamma}\).
So, from \RMK{2.4.6}, we have \(\phi_{\gamma} \T = \LMTRAN_A \phi_{\beta}\), which (of course) implies \(\T = \phi_{\gamma}^{-1} \LMTRAN_A \phi_{\beta}\) \MAROON{(1)}.
Then
\begin{align*}
    \rankT & = \dim(\RANGET) = \dim(\T(\V)) & \text{just by definition} \\
           & = \dim(\phi_{\gamma}^{-1} \LMTRAN_A \phi_{\beta}(\V)) & \text{by \MAROON{(1)}} \\
           & = \dim(\phi_{\gamma}^{-1} \LMTRAN_A (\phi_{\beta}(\V))) & \text{by def of composition} \\
           & = \dim(\phi_{\gamma}^{-1} \LMTRAN_A (F^n)) & \text{since in particular \(\phi_{\beta}\) is onto} \\
           & = \dim(\phi_{\gamma}^{-1} (\LMTRAN_A (F^n))) & \text{by def of composition} \\
           & = \dim(\LMTRAN_A (F^n)) & \text{by \ATHM{2.41}(2),} \\
           & & \text{and since \(\phi_{\gamma}^{-1}\) is also an isomorphism} \\
           & = \dim(\RANGE(\LMTRAN_A)) = \rank(\LMTRAN_A) & \text{just by definition} \\
           & = \rank(A) & \text{by \DEF{3.3}} \\
           & = \rank([\T]_{\beta}^{\gamma})
\end{align*}
\end{proof}

\begin{remark} \label{remark 3.2.3}
Now that from \THM{3.3}, the problem of finding the rank of a \LTRAN{} has been \emph{reduced to} the problem of finding the rank of a matrix.
But given a matrix representation \(A\) of a \LTRAN{}, it could be difficult to compute its rank;
(that is, it could also be difficult to compute \(\rank(\LMTRAN_A)\))
but we can perform some \emph{``rank-preserving''} operations on the matrix to get a matrix of which it is easy to compute the rank.
The next theorem and its corollary tell us how to do this.
\end{remark}

\begin{theorem} \label{thm 3.4}
Let \(A\) be an \(m \X n\) matrix.
If \(P\) and \(Q\) are \textbf{invertible} \(m \X m\) and \(n \X n\) matrices, respectively,
then
\begin{enumerate}
\item \(\rank(AQ) = \rank(A)\),
\item \(\rank(PA) = \rank(A)\), and \emph{therefore},
\item \(\rank(PAQ) = \rank(A)\).
\end{enumerate}
\end{theorem}

\begin{proof}
First observe that
\begin{align*}
    \rank(AQ) & = \rank(\LMTRAN_{AQ}) & \text{by \DEF{3.3}} \\
              & = \dim(\RANGE(\LMTRAN_{AQ})) & \text{by def of rank of \LTRAN{}} \\
              & = \dim(\RANGE(\LMTRAN_A \LMTRAN_Q)) & \text{by \THM{2.15}(e)} \\
              & = \dim(\LMTRAN_A \LMTRAN_Q(F^n)) & \text{by def of range} \\
              & = \dim(\LMTRAN_A(\LMTRAN_Q(F^n))) & \text{by def of composition} \\
              & = \dim(\LMTRAN_A(F^n)) & \text{\(Q\) invertible, so \(\LMTRAN_Q\) invertible, and in particular onto} \\
              & = \dim(\RANGE(\LMTRAN_A)) & \text{by def of range} \\
              & = \rank(\LMTRAN_A) & \text{by def of rank of \LTRAN{}} \\
              & = \rank(A) & \text{by \DEF{3.3}}
\end{align*}
This establishes part(a).

To establish part(b),
\begin{align*}
    \rank(PA) & = \rank(\LMTRAN_{PA}) & \text{by \DEF{3.3}} \\
              & = \rank(\LMTRAN_P \LMTRAN_A) & \text{by \THM{2.15}(e)} \\
              & = \dim(\RANGE(\LMTRAN_P \LMTRAN_A)) & \text{by def of rank of \LTRAN{}} \\
              & = \dim(\LMTRAN_P \LMTRAN_A(F^n)) & \text{by def of range} \\
              & = \dim(\LMTRAN_P(\LMTRAN_A(F^n)) & \text{by def of composition} \\
              & = \dim(\LMTRAN_A(F^n)) & \text{by \ATHM{2.41}(2),} \\
              & & \text{since (\(P\) is invertible then) \(\LMTRAN_P\) is invertible} \\
              & = \dim(\RANGE(\LMTRAN_A)) & \text{by def of range} \\
              & = \rank(\LMTRAN_A) & \text{by def of rank of \LTRAN{}} \\
              & = \rank(A) & \text{by \DEF{3.3}}
\end{align*}

Finally, applying (a) and (b), we have
\begin{align*}
    \rank(PAQ) & = \rank((PA)Q) & \text{of course} \\
               & = \rank(PA) & \text{by part(a)} \\
               & = \rank(A). & \text{by part(b)}
\end{align*}
\end{proof}

\begin{corollary} \label{corollary 3.4.1}
Elementary row and column operations on a matrix are \emph{rank-preserving}.
\end{corollary}

\begin{proof}
If \(B\) is obtained from a matrix \(A\) by an elementary row operation, then by \THM{3.1} there exists an elementary matrix \(E\) such that \(B = EA\).
By \THM{3.2}, \(E\) is invertible, and hence \(\rank(B) = \rank(EA) = \rank(A)\) by \THM{3.4}(b).

If \(B\) is obtained from a matrix \(A\) by an elementary column operation, then again by \THM{3.1} there exists an elementary matrix \(E\) such that \(B = AE\).
By \THM{3.2}, \(E\) is invertible, and hence \(\rank(B) = \rank(AE) = \rank(A)\) by \THM{3.4}(a).
\end{proof}

Now that we have a class of matrix operations(i.e., e.r.o. and e.c.o.) that preserve rank, we need a way of \emph{examining a transformed matrix to ascertain its rank}.
The next theorem is the first of several in this direction.

\begin{note}
這邊的脈絡是,直接把一個\ \LTRAN{} 寫成矩陣代表,我們常常還是不容易看出該矩陣的\ rank。
但我們現在要給定一些方法使得我們可以很輕易的判斷某些類型的矩陣的\ rank。
然後再用\ e.r.o. (跟\ e.c.o.) 這些\ rank-preserving 操作把那個矩陣代表轉成這類型的矩陣,判斷這類矩陣的\ rank,就能得到原本的矩陣代表的\ rank 了。
\end{note}

\begin{theorem} \label{thm 3.5}
The rank of any matrix equals the \emph{maximum} number of its \emph{\LID{}} columns;
that is, the rank of a matrix is the \emph{dimension of the subspace generated by its columns}.
\end{theorem}

\begin{note}
In the theorem above, the second statement is equivalent to the first statement, but it needs proof;
however, it's trivial(proof by contradiction).
And the proof below just proves the second statement.
\end{note}

\begin{note}
有了這個定理,我們就不用去看\ \(\LMTRAN_A\) 的\ rank 來得出\ \(A\) 的\ rank,而是可直接從「\(A\) 的長相」或者\ \(A\) 的\ columns span 出來的\ subspace(的\ dimension) 來知道\ \(A\) 的\ rank。
\end{note}

\begin{proof}
Let arbitrary \(A \in M_{m \X n}(F)\),
\(a_1, a_2, ..., a_n\) be the columns of \(A\),
and \(\beta = \{ e_1, e_2, ..., e_n \}\) be the \emph{standard} ordered basis for \(F^n\).
Then
\begin{align*}
    \rank(A) & = \rank(\LMTRAN_A) & \text{by \DEF{3.3}} \\
             & = \dim(\RANGE(\LMTRAN_A)) & \text{by def of rank of \LTRAN{}} \\
             & = \dim(\spann(\LMTRAN_A(\beta))) & \text{by \THM{2.2}} \\
             & = \dim(\spann(\{ \LMTRAN_A(e_1), \LMTRAN_A(e_2), ..., \LMTRAN_A(e_n) \})) \\
             & = \dim(\spann(\{ Ae_1, Ae_2, ..., Ae_n \})) & \text{just by def of left multiplication} \\
             & = \dim(\spann(\{ a_1, a_2, ..., a_n \})) & \text{by \THM{2.13}(b)}
\end{align*}
\end{proof}

\begin{example} \label{example 3.2.1}
Let
\[
    A=\left(\begin{array}{lll}
        1 & 0 & 1 \\
        0 & 1 & 1 \\
        1 & 0 & 1
    \end{array}\right)
\]
Observe that the first and second columns of \(A\) are \LID{} and that the third column is a linear combination of the first two.

Thus
\[
    \rank(A) =
    \dim
        \left(
        \spann\left(\left\{
            \left(\begin{array}{l}
                1 \\
                0 \\
                1
            \end{array}\right),
            \left(\begin{array}{l}
                0 \\
                1 \\
                0
            \end{array}\right),
            \left(\begin{array}{l}
                1 \\
                1 \\
                1
            \end{array}
        \right)\right\}\right)
        \right)
    = 2
\]
\end{example}

\begin{remark} \label{remark 3.2.4}
To compute the rank of a matrix \(A\), it is frequently useful to \emph{postpone the use of \THM{3.5}} until \(A\) has been suitably \emph{modified by} means of appropriate \emph{elementary row and column operations} so that the number of \LID{} columns is obvious.
\CORO{3.4.1} guarantees that the rank of the modified matrix is the same as the rank of \(A\).

One such modification of \(A\) can be obtained by using elementary row and column operations to \emph{introduce zero entries}.
The next example illustrates this procedure.
\end{remark}

\begin{example} \label{example 3.2.2}
Let
\[
    A = \begin{pmatrix} 1 & 2 & 1 \\ 1 & 0 & 3 \\ 1 & 1 & 2 \end{pmatrix}.
\]
If we subtract the first row of \(A\) from rows 2 and 3 (type 3 elementary row operations), the result is
\[
    \begin{pmatrix} 1 & 2 & 1 \\ \RED{0} & -2 & 2 \\ \RED{0} & -1 & 1 \end{pmatrix}.
\]
If we now subtract twice the first \emph{column} from the second and subtract the first \emph{column} from the third (type 3 elementary \emph{column} operations), we obtain
\[
    \begin{pmatrix} 1 & \RED{0} & \RED{0} \\ 0 & -2 & 2 \\ 0 & -1 & 1 \end{pmatrix}.
\]
It is now obvious that the maximum number of \LID{} columns of this matrix is \(2\).
Hence the rank of \(A\) is \(2\).
\end{example}

The next theorem uses this process to transform a matrix into a particularly simple form.
The power of this theorem can be seen in its corollaries.

\begin{theorem} \label{thm 3.6}
Let \(A\) be an \(m \X n\) matrix and its rank is \(r\).

\BLUE{(1)} Then by means of a \emph{finite} number of elementary row and column operations, \(A\) can be transformed into the matrix
\[
    D = \begin{pmatrix} I_r & O_1 \\ O_2 & O_3 \end{pmatrix}
\]
where \(O_1, O_2\), and \(O_3\) are zero matrices.
Furthermore, \(r \le m\) \textbf{and} \(r \le n\), which is a direct consequence of \BLUE{(1)} since trivially by \THM{3.5}, \(\rank(D) = r\), and by \CORO{3.4.1}, \(\rank(A) = \rank(D)\), and from the structure of \(D\) we have \(r \le m\) and \(r \le n\).
Thus \(D_{ii} = 1\) for \(i \le r\) and \(D_{ij} = 0\) otherwise.
\end{theorem}

\THM{3.6} and its corollaries are quite important.
Its proof, though easy to understand, is tedious to read.
As an aid in following the proof, we first consider an example.

\begin{example} \label{example 3.2.3}
Consider the matrix
\[
    A = \begin{pmatrix} 0 & 2 & 4 & 2 & 2 \\ 4 & 4 & 4 & 8 & 0 \\ 8 & 2 & 0 & 10 & 2 \\ 6 & 3 & 2 & 9 & 1 \end{pmatrix}.
\]
By means of a succession of elementary row and column operations, we can transform \(A\) into a matrix \(D\) as in \THM{3.6}.
We list many of the intermediate matrices, but on several occasions a matrix is transformed from the preceding one by means of \emph{several} elementary operations.
The number above each arrow indicates how many elementary operations are involved. Try to identify the nature of each elementary operation (row or column and type) in the following matrix transformations
\begin{align*}
    \left(\begin{array}{rrrrr}
        0 & 2 & 4 & 2 & 2 \\
        4 & 4 & 4 & 8 & 0 \\
        8 & 2 & 0 & 10 & 2 \\
        6 & 3 & 2 & 9 & 1
    \end{array}\right) \stackrel{1\MAROON{(1)}}{\longrightarrow}
    \left(\begin{array}{rrrrr}
        4 & 4 & 4 & 8 & 0 \\
        0 & 2 & 4 & 2 & 2 \\
        8 & 2 & 0 & 10 & 2 \\
        6 & 3 & 2 & 9 & 1
    \end{array}\right) \stackrel{1\MAROON{(2)}}{\longrightarrow}
    \left(\begin{array}{rrrrr}
        1 & 1 & 1 & 2 & 0 \\
        0 & 2 & 4 & 2 & 2 \\
        8 & 2 & 0 & 10 & 2 \\
        6 & 3 & 2 & 9 & 1
    \end{array}\right) \stackrel{2\MAROON{(3)}}{\longrightarrow} \\
    \left(\begin{array}{rrrrrr}
        1 & 1 & 1 & 2 & 0 \\
        0 & 2 & 4 & 2 & 2 \\
        0 & -6 & -8 & -6 & 2 \\
        0 & -3 & -4 & -3 & 1
    \end{array}\right) \stackrel{3\MAROON{(4)}}{\longrightarrow}
    \left(\begin{array}{rrrrrr}
        1 & 0 & 0 & 0 & 0 \\
        0 & 2 & 4 & 2 & 2 \\
        0 & -6 & -8 & -6 & 2 \\
        0 & -3 & -4 & -3 & 1
    \end{array}\right) \stackrel{1\MAROON{(5)}}{\longrightarrow} \\
    \left(\begin{array}{rrrrrr}
        1 & 0 & 0 & 0 & 0 \\
        0 & 1 & 2 & 1 & 1 \\
        0 & -6 & -8 & -6 & 2 \\
        0 & -3 & -4 & -3 & 1
    \end{array}\right) \stackrel{2\MAROON{(6)}}{\longrightarrow}
    \left(\begin{array}{rrrrr}
        1 & 0 & 0 & 0 & 0 \\
        0 & 1 & 2 & 1 & 1 \\
        0 & 0 & 4 & 0 & 8 \\
        0 & 0 & 2 & 0 & 4
    \end{array}\right) \stackrel{3\MAROON{(7)}}{\longrightarrow}
    \left(\begin{array}{rrrrrr}
        1 & 0 & 0 & 0 & 0 \\
        0 & 1 & 0 & 0 & 0 \\
        0 & 0 & 4 & 0 & 8 \\
        0 & 0 & 2 & 0 & 4
    \end{array}\right) \stackrel{1\MAROON{(8)}}{\longrightarrow} \\
    \left(\begin{array}{rrrrr}
        1 & 0 & 0 & 0 & 0 \\
        0 & 1 & 0 & 0 & 0 \\
        0 & 0 & 1 & 0 & 2 \\
        0 & 0 & 2 & 0 & 4
    \end{array}\right) \stackrel{1\MAROON{(9)}}{\longrightarrow}
    \left(\begin{array}{rrrrr}
        1 & 0 & 0 & 0 & 0 \\
        0 & 1 & 0 & 0 & 0 \\
        0 & 0 & 1 & 0 & 2 \\
        0 & 0 & 0 & 0 & 0
    \end{array}\right) \stackrel{1\MAROON{(10)}}{\longrightarrow}
    \left(\begin{array}{rrrrr}
        1 & 0 & 0 & 0 & 0 \\
        0 & 1 & 0 & 0 & 0 \\
        0 & 0 & 1 & 0 & 0 \\
        0 & 0 & 0 & 0 & 0
    \end{array}\right) = D
\end{align*}
\begin{enumerate}
\item[\MAROON{(1)}] one e.r.o.(1), changing row 1 and row 2.
\item[\MAROON{(2)}] one e.r.o.(2), multiplying \(1/4\) to row 1
\item[\MAROON{(3)}] two e.r.o.(3), subtracting row 3, 4 from row 1
\item[\MAROON{(4)}] three e.c.o.(3), subtracting col 1 from col 2, 3, 4
\item[\MAROON{(5)}] one e.r.o.(2), multiplying \(1/2\) to row 2
\item[\MAROON{(6)}] two e.r.o.(3), subtracting row 2 from row 3, 4
\item[\MAROON{(7)}] three e.c.o.(3), subtracting row 2 from col 3, 4, 5
\item[\MAROON{(8)}] one e.r.o.(2), multiplying \(1/4\) to row 3
\item[\MAROON{(9)}] one e.r.o.(3), subtracting row 3 from row 4
\item[\MAROON{(10)}] one e.c.o.(3), subtracting col 3 from col 5
\end{enumerate}
By the \CORO{3.4.1}, \(\rank(A) = \rank(D)\).
Clearly, however, \(\rank(D) = 3\); so \(\rank(A) = 3\).
\end{example}

\begin{note}
Note that the first two elementary operations in \EXAMPLE{3.2.3} result in a \(1\) in the \(1,1\) position,
and the next several operations (type 3) result in \(O\)'s everywhere in the first row and first column except for the \(1,1\) position.
\emph{And the remaining elementary operations do not change the first row and first column}.
With this example in mind, we proceed with the proof of \THM{3.6}.
\end{note}

\begin{proof}[Proof for \THM{3.6}]
If \(A\) is the zero matrix, \(r = 0\) by \EXEC{3.2.3} and of course \(r \le m\) and \(r \le n\).
In this case, the conclusion follows with \(D = A\).

Now suppose that \(A \ne O_{m \X n}\) and \(r = \rank(A)\);
then \(r > 0\).
The proof is by mathematical induction on \(m\), the number of rows of \(A\).

Suppose that \(m = 1\).
We construct a sequence of finite operations to transform \(A\) into a matrix of the desired form.
By means of at most one type 1 column operation and at most one type 2 column operation, \(A\) can be transformed into a matrix with a \(1\) in the 1,1 position.
By means of at most \(n - 1\) type 3 column operations, this matrix can in turn be transformed into the matrix
\[
    \begin{pmatrix} 1 & 0 & ... & 0 \end{pmatrix},
\]
which has the desired form.
And note that there is one \LID{} \emph{column} in \(D\).
So
\begin{align*}
    \rank(A) & = \rank(D) & \text{by the \CORO{3.4.1}} \\
             & = 1. & \text{by \THM{3.5}}
\end{align*}
Hence \(\rank(A) \le 1\), the number of rows of \(A\), and \(\rank(A) \le n\), the number of columns of \(A\).
Thus the theorem is established for \(m = 1\).

Next assume that the theorem holds for any matrix with \(m - 1\) rows (for some \(m > 1\)).
We must prove that the theorem holds for any matrix with \(m\) rows.

Suppose that \(A\) is any \(m \X n\) matrix.
If \(n = 1\), \THM{3.6} can be established in a manner analogous to that for \(m = 1\) (see \EXEC{3.2.10}).

We now suppose that \(n > 1\).
Since \(A \ne O_{m \X n}\), \(A_{ij} \ne 0\) for \emph{some} \(i, j\).
By means of at most one e.r.o. and at most one e.c.o. (each of type 1), we can move the nonzero entry to the \(1,1\) position (just as was done in \EXAMPLE{3.2.3}).
By means of at most one additional type 2 operation, we can assure a \(1\) in the \(1,1\) position.
(Look at the second operation in \EXAMPLE{3.2.3}.)
By means of at most \(m - 1\) type 3 row operations and at most \(n - 1\) type 3 column operations,
we can eliminate all nonzero entries in the first row and the first column with the exception of the \(1\) in the \(1,1\) position.
(In \EXAMPLE{3.2.3}, we used two row and three column operations to do this.)
Thus, with a finite number of elementary operations, \(A\) can be transformed into a matrix
\[
    B= \left(\begin{array}{c|ccc}
        1 & 0 & \cdots & 0 \\
        \hline 0 & & & \\
        \vdots & & B^{\prime} & \\
        0 & & &
    \end{array}\right)
\]
where \(B'\) is an \(\RED{(m - 1)} \X (n - 1)\) matrix.
In \EXAMPLE{3.2.3}, for instance,
\[
    B' = \begin{pmatrix}
        2 & 4 & 2 & 2 \\
        -6 & -8 & -6 & 2 \\
        -3 & -4 & -3 & 1
    \end{pmatrix}.
\]
By \EXEC{3.2.11}, \(B'\) must have rank \emph{one less than} \(B\).
Since \(\rank(A) = \rank(B) = r\), \(\rank(B') = r - 1\).
Also by the \emph{induction hypothesis}, \(B'\) can be transformed by a \emph{finite} number of elementary row and column operations into the \((m - 1) \X (n - 1)\) matrix \(D'\) such that
\[
    D' = \begin{pmatrix}
        I_{r - 1} & O_4 \\
        O_5 & O_6
    \end{pmatrix}
\]
where \(O_4, O_5,\) and \(O_6\) are zero matrices, and \(r - 1 \le m - 1\) and \(r - 1 \le n - 1\).
And the latter simply implies \(r \le m\) and \(r \le n\).
That is, \(D'\) consists of all zeros except for its first \(r - 1\) diagonal entries, which are ones.
Let
\[
    D = \left(\begin{array}{c|ccc}
        1 & 0 & \cdots & 0 \\
        \hline 0 & & & \\
        \vdots & & D^{\prime} & \\
        0 & & &
    \end{array}\right)
\]
We see that the theorem now follows once we show that \(D\) can be obtained from \(B\) by means of a finite number of elementary row and column operations.
However this follows by repeated applications of \EXEC{3.2.12}.

Thus, since \(A\) can be transformed into \(B\) and \(B\) can be transformed into \(D\), each by a \emph{finite} number of elementary operations,
\(A\) can be transformed into \(D\) by a finite number of elementary operations.
Finally, since \(D'\) contains ones as its first \(r - 1\) diagonal entries, \(D\) contains ones as its first \(r\) diagonal entries and zeros elsewhere.
This establishes the theorem.
\end{proof}

\begin{corollary} \label{corollary 3.6.1}
Let \(A\) be an \(m \X n\) matrix of rank \(r\).
Then there exist \emph{invertible} matrices \(B\) and \(C\) of sizes \(m \X m\) and \(n \X n\), respectively, such that \(D = BAC\), where
\[
    D = \begin{pmatrix}
        I_r & O_1 \\
        O_2 & O_3
    \end{pmatrix}
\]
is the \(m \X n\) matrix in which \(O_1, O_2\), and \(O_3\) are zero matrices.
\end{corollary}

\begin{proof}
By \THM{3.6}, \(A\) can be transformed by means of a finite number of elementary row and column operations into the matrix \(D\).
We can appeal to \THM{3.1} each time we perform an elementary operation.
Thus there exist elementary \(m \X m\) matrices \(E_1, E_2, ..., E_p\) and elementary \(n \X n\) matrices \(G_1, G_2, ..., G_q\) such that
\[
    D = E_p E_{p - 1} ... E_2 E_1 A G_1 G_2 ... G_q.
\]
(Notice the \emph{reverse} order of \(E_i\) in the expression.)

By \THM{3.2}, each \(E_i\) and \(G_j\) is invertible.
Let \(B = E_p E_{p - 1} ... E_1\) and \(C = G_1 G_2 ... G_q\).
Then \(B\) and \(C\) are invertible by (induction and) \ATHM{2.36}(1), and \(D = BAC\).
\end{proof}

\begin{corollary} \label{corollary 3.6.2}
Let \(A\) be an \(m \X n\) matrix.
Then
\begin{enumerate}
\item \(\rank(A^\top) = \rank(A)\).
\item The rank of any matrix equals the maximum number of its \LID{} \textbf{rows} (this is a different statement from \THM{3.5});
    that is, the rank of a matrix is the \emph{dimension of the subspace generated by its rows}.
\item The rows and columns of any matrix generate subspaces of the \textbf{same dimension}, numerically equal to the rank of the matrix.
\end{enumerate}
\end{corollary}

\begin{note}
The corollary just says the dimensions of the ``column space'' and ``row space'' of the given matrix are the same,
but the generated vector space can be different(although isomorphic).
\end{note}

\begin{proof} \ 

\begin{enumerate}
\item By \CORO{3.6.1}, there exist invertible matrices \(B\) and \(C\) such that \(D = BAC\), where \(D\) satisfies the stated conditions of that corollary.
Taking transposes, (by \ATHM{2.24},) we have
\[
    D^\top = (BAC)^\top = C^\top A^\top B^\top.
\]
Since \(B\) and \(C\) are invertible, so are \(B^\top\) and \(C^\top\) by \ATHM{2.36}(2).
Hence
\begin{align*}
    \rank(A^\top) & = \rank(C^\top A^\top B^\top) & \text{by \THM{3.4}(c)} \\
                  & = \rank(D^\top) \MAROON{(1)}.
\end{align*}
Suppose that \(r = \rank(A)\) \MAROON{(2)}.
Then \(D^\top\) is an \(n \X m\) matrix with the form of the matrix \(D\) in \CORO{3.6.1}, and hence \(\rank(D^\top) = r\) \MAROON{(3)} by \THM{3.5}.
(That is, the column space of \(D^\top\) trivially has dimension \(r\))
Thus
\begin{align*}
    \rank(A^\top) & = \rank(D^\top) & \text{by \MAROON{(1)}} \\
                  & = r & \text{by \MAROON{(3)}} \\
                  & = \rank(A) & \text{by \MAROON{(1)}}.
\end{align*}

\item From part(a), given any matrix \(A\).
Let \(a_1, a_2, ..., a_m\) be \emph{rows} of \(A\), therefore \(a_1^\top, a_2^\top, ..., a_m^\top\) are the \emph{columns} of \(A^\top\).
Then
\begin{align*}
    \rank(A) & = \rank(A^\top) & \text{by part(a)} \\
             & = \dim(\spann(\{a_1^\top, a_2^\top, ... a_m^\top\}), & \text{by \THM{3.5}}
\end{align*}
which of course has the same dimension (or isomorphic) as the subspace generated by \(\{ a_1, a_2, ..., a_m \}\),
that is, the subspace generated by the rows of \(A\).

\item This is a direct consequence of \THM{3.5} and part(b).
\end{enumerate}
\end{proof}

\begin{corollary} \label{corollary 3.6.3}
Every \emph{invertible} (which implies square) matrix is a product of elementary matrices.
\end{corollary}

\begin{proof} \ 

If \(A\) is an invertible \(n \X n\) matrix, then (by \RMK{3.2.1}) \(\rank(A) = n\).
Hence the matrix \(D\) in \CORO{3.6.1} equals \(I_n\), and there exist invertible matrices \(B\) and
\(C\) such that
\[
    BAC = D = \I_n \MAROON{(1)}.
\]
As in the proof of \CORO{3.6.1}, note that \(B = E_p E_{p - 1} ... E_1\) and \(C = G_1 G_2 ... G_q\), where the \(E_i\)'s and \(G_i\)'s are elementary matrices.
Thus from \MAROON{(1)} we have \(A = B^{-1} I_n C^{-1} = B^{-1} C^{-1}\),
so that
\[
    A = E_1^{-1} E_2^{-1} ... E_p^{-1} G_q^{-1} G_{q - 1}^{-1} ... G_1^{-1}.
\]
The inverses of elementary matrices are elementary matrices, however, and hence \(A\) is the product of elementary matrices.
\end{proof}

We now use \CORO{3.6.2} to relate the rank of a matrix product to the rank of each factor.
Notice how the proof exploits the \emph{relationship between the rank of a matrix and the rank of a \LTRAN{}}.

\begin{theorem} \label{thm 3.7}
Let \(\T : \V \to \W\) and \(\U : \W \to Z\) be \LTRAN{}s on finite-dimensional vector spaces \(\V, \W\), and \(Z\), and let \(A\) and \(B\) be matrices such that the product \(AB\) is defined.
Then
\begin{enumerate}
\item \(\rank(\U\T) \le \rank(\U)\).
\item \(\rank(\U\T) \le \rank(\T)\).
\item \(\rank(AB) \le \rank(A)\).
\item \(\rank(AB) \le \rank(B)\).
\end{enumerate}
That is, the rank of the product [composition] of matrices [\LTRAN{}s] is less than or equal to \emph{both} of each factor.
\end{theorem}

\begin{proof}
We prove these items in the order: (a), (c), (d), and (b).
\begin{enumerate}
\item[(a)] Clearly, \(\RANGET \subseteq \W\) \MAROON{(a.1)}.
Hence
\begin{align*}
    \RANGE(\U\T) & = \U\T(\V) & \text{by def of range} \\
                 & = \U(\T(\V)) & \text{by def of composition} \\
                 & = \U(\RANGET) & \text{by def of range} \\
                 & \subseteq \U(\W) & \text{by \MAROON{(a.1)}} \\
                 & = \RANGE(\U) & \text{by def of range}
\end{align*}
So we have \(\RANGE(\U\T) \subseteq \RANGE(\U)\) \MAROON{(a.2)}.
Thus
\begin{align*}
    \rank(\U\T) & = \dim(\RANGE(\U\T)) & \text{by def of rank of \LTRAN{}} \\
                & \le \dim(\RANGE(\U)) & \text{by \MAROON{(a.2)}} \\
                & = \rank(\U). & \text{by def of rank of \LTRAN{}}
\end{align*}

\item[(c)] We have
\begin{align*}
    \rank(AB) & = \rank(\LMTRAN_{AB}) & \text{by \DEF{3.3}} \\
              & = \rank(\LMTRAN_A \LMTRAN_B) & \text{by \THM{2.15}(e)} \\
              & \le \rank(\LMTRAN_A) & \text{by part(a)} \\
              & = \rank(A). & \text{by \DEF{3.3}}
\end{align*}

\item[(d)] We have
\begin{align*}
    \rank(AB) & = \rank((AB)^\top) & \text{by \CORO{3.6.2}} \\
              & = \rank(B^\top A^\top) & \text{by \ATHM{2.36}(2)} \\
              & \le \rank(B^\top) & \text{by part(c)} \\
              & = \rank(B). & \text{by \CORO{3.6.2}}
\end{align*}

\item[(b)] Let \(\alpha, \beta\), and \(\gamma\) be ordered bases for \(\V\), \(\W\), and \(Z\), respectively,
and let \(A' = [\U]_{\beta}^{\gamma}\) and \(B' = [\T]_{\alpha}^{\beta}\).
Then \(A'B' = [\U\T]_{\alpha}^{\gamma}\) \MAROON{(b.1)} by \THM{2.11}.
And 
\begin{align*}
    \rank(\U\T) & = \rank([\U\T]_{\alpha}^{\gamma}) & \text{by \THM{3.3}} \\
                & = \rank(A' B') & \text{by \MAROON{(b.1)}} \\
                & \le \rank(B') & \text{by part(d)} \\
                & = \rank(\T) & \text{by \THM{3.3}}.
\end{align*}
\end{enumerate}
\end{proof}

\begin{remark} \label{remark 3.2.5}
It is important to be able to compute the rank of any matrix.
We can use the \CORO{3.4.1}, \THM{3.5} and \THM{3.6}, and \CORO{3.6.2} to accomplish this goal.

The object is to perform elementary row and column operations on a matrix to ``simplify'' it
(so that the transformed matrix has many zero entries)
to the point where a simple observation enables us \emph{to determine how many \LID{} rows or columns the matrix has}, and thus to determine its rank.
\end{remark}

\begin{example} \label{example 3.2.4} \ 

\begin{enumerate}
\item Let
\[
    A = \begin{pmatrix} 1 & 2 & 1 & 1 \\ 1 & 1 & -1 & 1 \end{pmatrix}.
\]
Note that the first and second rows of \(A\) are \LID{} since one is not a multiple of the other.
Thus \(\rank(A) = 2\).

\item Let
\[
    A = \begin{pmatrix} 1 & 3 & 1 & 1 \\ 1 & 0 & 1 & 1 \\ 0 & 3 & 0 & 0 \end{pmatrix}.
\]
In this case, there are several ways to proceed.
Suppose that we begin with an e.r.o. to obtain a zero in the \(2,1\) position.
Subtracting the first row from the second row, we obtain
\[
    B = \begin{pmatrix} 1 & 3 & 1 & 1 \\ \RED{0} & -3 & 0 & 0 \\ 0 & 3 & 0 & 0 \end{pmatrix}.
\]
Now note that the third row is a multiple of the second row, and the first and second rows are \LID{}. Thus \(\rank(A) = 2\).
\item Let
\[
    A = \begin{pmatrix} 1 & 2 & 3 & 1 \\ 2 & 1 & 1 & 1 \\ 1 & -1 & 1 & 0 \end{pmatrix}.
\]
Using elementary row operations, we can transform \(A\) as follows:
\[
    A \longrightarrow
    \left(\begin{array}{rrrr}
        1 & 2 & 3 & 1 \\
        \RED{0} & -3 & -5 & -1 \\
        \RED{0} & -3 & -2 & -1
    \end{array}\right)
    \longrightarrow
    \left(\begin{array}{rrrr}
        1 & 2 & 3 & 1 \\
        0 & -3 & -5 & -1 \\
        0 & \RED{0} & 3 & 0
    \end{array}\right).
\]
It is clear that the last matrix has three \LID{} \emph{rows} and hence has rank \(3\). 
\end{enumerate}
\end{example}

\subsection{The Inverse of a Matrix}
We have remarked that an \(n \X n\) matrix is invertible if and only if its rank is \(n\). (\RMK{3.2.1}.)
Since we know how to compute the rank of any matrix, we can always test a matrix to determine whether it is invertible.
We now provide a \emph{simple technique for computing the inverse} of a matrix that utilizes elementary row
operations.

\begin{definition} \label{def 3.4}
Let \(A\) and \(B\) be \(m \X n\) and \(m \X p\) matrices, respectively.
By the \textbf{augmented matrix} \((A|B)\), we mean the \(m \X (n + p)\) matrix \((A B)\),
that is, the matrix whose first \(n\) columns are the columns of \(A\), and whose last \(p\) columns are the columns of \(B\).
\end{definition}

\begin{remark} \label{remark 3.2.6}
Let \(A\) be an invertible \(n \X n\) matrix, and consider the \(n \X 2n\) augmented matrix \(C = (A|I_n)\).
By \EXEC{3.2.15}, we have
\begin{align*}
    A^{-1}C & = A^{-1} (A | I_n) \\
            & = (A^{-1}A | A^{-1} I_n) & \text{by \EXEC{3.2.15}} \\
            & = (I_n|\RED{A^{-1}}).
\end{align*}
By \CORO{3.6.3}, \(A^{-1}\) is the product of elementary matrices, say \(A^{-1} = E_p E_{p - 1} ... E_1\).
Thus equation above becomes
\[
    E_p E_{p - 1} ... E_1 C = (A^{-1}A | A^{-1} I_n) = (I_n|\RED{A^{-1}}).
\]
Because multiplying a matrix \emph{on the left} by an elementary matrix transforms the matrix by an elementary \emph{row} operation (\THM{3.1}), we have the following result:

\BLUE{(3.2.6.1)} If \(A\) is an invertible \(n \X n\) matrix, then it is possible to transform the matrix \((A | I_n)\) into the matrix \((I_n|A^{-1})\) by means of a finite number of elementary \emph{row} operations.

Conversely, suppose that \(A\) is invertible and that, for some \(n \X n\) matrix \(B\), the matrix \((A|I_n)\) can be transformed into the matrix \((I_n|B)\) by a finite number of elementary \emph{row} operations.
Let \(E_1, E_2, ..., E_p\) be the elementary matrices associated with these elementary \emph{row} operations as in \THM{3.1};
then
\[
    E_p E_{p -1} ... E_1 (A|I_n) = (I_n | B).
\]
Letting \(M = E_p E_{p - 1} ... E_1\), then from the equation above and \EXEC{3.2.15} again,
\[
    (MA | M) = M(A | I_n) = (I_n | B).
\]
Hence \(MA = I_n\) and \(M = B\).
It follows that \(M = A^{-1}\).
So \(B = A^{-1}\).
Thus we have the following result:

\BLUE{(3.2.6.2)} If \(A\) is an invertible \(n \X n\) matrix, and the matrix \((A | I_n)\) is transformed into a matrix of the form \((I_n |B)\) by means of a finite number of elementary \emph{row} operations, then \(B = A^{-1}\).

If, on the other hand, \(A\) is an \(n \X n\) matrix that is \textbf{not} invertible, then \(\rank(A) < n\) (by \RMK{3.2.1}).
Hence any attempt to transform \((A|I_n)\) into a matrix of the form \((I_n|B)\) by means of elementary row operations must fail because otherwise \(A\) can be transformed into \(I_n\) using the same row operations.
This is impossible, however, because elementary row operations preserve rank.
In fact, \(A\) can be transformed \textbf{into a matrix with a row containing only zero entries},
yielding the following result:

\BLUE{(3.2.6.3)} If \(A\) is an \(n \X n\) matrix that is \emph{not} invertible, then any attempt to transform \((A | I_n)\) into a matrix of the form \((I_n | B)\) by elementary row operations produces a row whose \emph{first \(n\) entries} are zeros.

The next two examples demonstrate these comments.
\end{remark}

\begin{example} \label{example 3.2.5}
We determine whether the matrix
\[
    A = \begin{pmatrix} 0 & 2 & 4 \\ 2 & 4 & 2 \\ 3 & 3 & 1 \end{pmatrix}
\]
is invertible, and if it is, we compute its inverse.
We attempt to use elementary row operations to transform
\[
    (A \mid I)
    = \left(\begin{array}{lll|lll}
        0 & 2 & 4 & 1 & 0 & 0 \\
        2 & 4 & 2 & 0 & 1 & 0 \\
        3 & 3 & 1 & 0 & 0 & 1
    \end{array}\right).
\]
into a matrix of the form \((I|B)\).
One method for accomplishing this transformation is to change each column of \(A\) successively, beginning with the first column, into the corresponding column of \(I\).
Since we need a nonzero entry in the \(1,1\) position, we begin by interchanging rows \(1\) and \(2\).
The result is
\[
    \left(\begin{array}{lll|lll}
        2 & 4 & 2 & 0 & 1 & 0 \\
        0 & 2 & 4 & 1 & 0 & 0 \\
        3 & 3 & 1 & 0 & 0 & 1
    \end{array}\right).
\]
In order to place a \(1\) in the \(1, 1\) position, we must multiply the first row by \(\frac1{2}\);
this operation yields
\[
    \left(\begin{array}{lll|lll}
        1 & 2 & 1 & 0 & \frac1{2} & 0 \\
        0 & 2 & 4 & 1 & 0 & 0 \\
        3 & 3 & 1 & 0 & 0 & 1
    \end{array}\right).
\]
We now complete work in the first column by adding \(-3\) times row \(1\) to row \(3\) to obtain
\[
    \left(\begin{array}{lll|lll}
        1 & 2 & 1 & 0 & \frac1{2} & 0 \\
        0 & 2 & 4 & 1 & 0 & 0 \\
        0 & -3 & -2 & 0 & -\frac{3}{2} & 1 
    \end{array}\right).
\]
In order to change the second column of the preceding matrix into the second column of \(I\), we multiply row \(2\) by \(\frac1{2}\) to obtain a \(1\) in the \(2, 2\) position.
This operation produces
\[
    \left(\begin{array}{lll|lll}
        1 & 2 & 1 & 0 & \frac1{2} & 0 \\
        0 & 1 & 2 & \frac1{2} & 0 & 0 \\
        0 & -3 & -2 & 0 & -\frac{3}{2} & 1 
    \end{array}\right).
\]
We now complete our work on the second column by adding \(-2\) times row \(2\) to row \(1\) and \(3\) times row \(2\) to row \(3\).
The result is
\[
    \left(\begin{array}{rrr|rrr}
        1 & 0 & -3 & -1 & \frac{1}{2} & 0 \\
        0 & 1 & 2 & \frac{1}{2} & 0 & 0 \\
        0 & 0 & 4 & \frac{3}{2} & -\frac{3}{2} & 1
    \end{array}\right).
\]
Only the third column remains to be changed.
In order to place a \(1\) in the \(3, 3\) position, we multiply row \(3\) by \(\frac1{4}\);
this operation yields
\[
    \left(\begin{array}{rrr|rrr}
        1 & 0 & -3 & -1 & \frac{1}{2} & 0 \\
        0 & 1 & 2 & \frac{1}{2} & 0 & 0 \\
        0 & 0 & 1 & \frac{3}{8} & -\frac{3}{8} & \frac1{4}
    \end{array}\right).
\]
Adding appropriate multiples of row \(3\) to rows \(1\) and \(2\) completes the process and gives
\[
    \left(\begin{array}{lll|rrr}
        1 & 0 & 0 & \frac{1}{8} & -\frac{5}{8} & \frac{3}{4} \\
        0 & 1 & 0 & -\frac{1}{4} & \frac{3}{4} & -\frac{1}{2} \\
        0 & 0 & 1 & \frac{3}{8} & -\frac{3}{8} & \frac{1}{4}
    \end{array}\right).
\]
Thus \(A\) is invertible, and
\[
    A^{-1} = \left(\begin{array}{rrr}
        \frac{1}{8} & -\frac{5}{8} & \frac{3}{4} \\
        -\frac{1}{4} & \frac{3}{4} & -\frac{1}{2} \\
        \frac{3}{8} & -\frac{3}{8} & \frac{1}{4}
    \end{array}\right).
\]
\end{example}

\begin{example} \label{example 3.2.6}
We determine whether the matrix
\[
    A=\left(\begin{array}{rrr}
        1 & 2 & 1 \\
        2 & 1 & -1 \\
        1 & 5 & 4
    \end{array}\right)
\]
is invertible, and if it is, we compute its inverse.
Using a strategy similar to the one used in \EXAMPLE{3.2.5}, we attempt to use elementary row operations to transform
\[
    (A \mid I) =\left(\begin{array}{rrr|rrr}
        1 & 2 & 1 & 1 & 0 & 0 \\
        2 & 1 & -1 & 0 & 1 & 0 \\
        1 & 5 & 4 & 0 & 0 & 1
    \end{array}\right)
\]
into a matrix of the form \((I \mid B)\).
We first add \(-2\) times row \(1\) to row \(2\) and \(-1\) times row \(1\) to row \(3\).
We then add row \(2\) to row \(3\).
The result,
\[
\begin{aligned}
    \left(\begin{array}{rrr|rrr}
        1 & 2 & 1 & 1 & 0 & 0 \\
        2 & 1 & -1 & 0 & 1 & 0 \\
        1 & 5 & 4 & 0 & 0 & 1
    \end{array}\right)
    \longrightarrow
    \left(\begin{array}{rrr|rrr}
        1 & 2 & 1 & 1 & 0 & 0 \\
        0 & -3 & -3 & -2 & 1 & 0 \\
        0 & 3 & 3 & -1 & 0 & 1
    \end{array}\right) \\
    \longrightarrow
    \left(\begin{array}{rrr|rrr}
        1 & 2 & 1 & 1 & 0 & 0 \\
        0 & -3 & -3 & -2 & 1 & 0 \\
        0 & 0 & 0 & -3 & 1 & 1
    \end{array}\right)
\end{aligned}
\]
is a matrix with a row whose first \(3\) entries are zeros.
Therefore \(A\) is not invertible.
\end{example}

Being able to test for invertibility and compute the inverse of a matrix allows us, with the help of \THM{2.18} and \CORO{2.18.1}, \CORO{2.18.2},
to test for invertibility and \textbf{compute the inverse of a linear transformation}.
The next example demonstrates this technique.

\begin{example} \label{example 3.2.7}
Let \(\T: \POLYRR \to \POLYRR\) be defined by \(\T(f(x)) = f(x) + f'(x) + f''(x)\),
where \(f(x)\) and \(f''(x)\) denote the first and second derivatives of \(f(x)\).
We use \CORO{2.18.1} to test \(\T\) for invertibility and compute the inverse if \(\T\) is invertible.
Taking \(\beta\) to be the \emph{standard} ordered basis of \(\POLYRR\), we have
\[
    [\T]_{\beta} = \begin{pmatrix}
        1 & 1 & 2 \\ 0 & 1 & 2 \\ 0 & 0 & 1
    \end{pmatrix}.
\]
Using the method of \EXAMPLE{3.2.5} and \EXAMPLE{3.2.6}, we can show that \([\T]_{\beta}\) is invertible with inverse
\[
    ([\T]_{\beta})^{-1} = \begin{pmatrix}
        1 & -1 & 0 \\ 0 & 1 & -2 \\ 0 & 0 & 1
    \end{pmatrix}.
\]
Thus (by \CORO{2.18.1}) \(\T\) is invertible, and \(([\T]_{\beta})^{-1} = [\T^{-1}]_{\beta}\).
Hence by \THM{2.14}, we have
\begin{align*}
    [\T^{-1}(a_0 + a_1 x + a_2 x^2)]_{\beta}
        & = [\T^{-1}]_{\beta} [(a_0 + a_1 x + a_2 x^2)]_{\beta} & \text{by \THM{2.14}} \\
        & = \left(\begin{array}{rrr}
                1 & -1 & 0 \\
                0 & 1 & -2 \\
                0 & 0 & 1
            \end{array}\right)
            \left(\begin{array}{l}
                a_0 \\ a_1 \\ a_2
            \end{array}\right) \\
        & = \left(\begin{array}{c}
                \BLUE{a_0 - a_1} \\ \RED{a_1 - 2 a_2} \\ \GREEN{a_2}
            \end{array}\right)
\end{align*}
Therefore
\[
    \T^{-1}(a_0 + + a_1 x + a_2 x^2) = \BLUE{(a_0 - a_1)} + \RED{(a_1 - 2 a_2)} x + \GREEN{a_2} x^2.
\]
\end{example}
\exercisesection

\begin{exercise} \label{exercise 3.2.1}
Label the following statements as true or false.
\begin{enumerate}
\item The rank of a matrix is equal to the number of its nonzero columns.
\item The product of two matrices always has rank equal to the lesser of the ranks of the two matrices.
\item The \(m \X n\) zero matrix is the only \(m \X n\) matrix having rank \(0\).
\item e.r.o.s preserve rank.
\item e.c.o.s do not necessarily preserve rank.
\item The rank of a matrix is equal to the maximum number of linearly independent rows in the matrix.
\item The inverse of a matrix can be computed exclusively by means of elementary row operations.
\item The rank of an \(n \X n\) matrix is at most \(n\).
\item An \(n \X n\) matrix having rank \(n\) is invertible.
\end{enumerate}
\end{exercise}

\begin{proof} \ 

\begin{enumerate}
\item False. Counterexample: \(\begin{pmatrix} 1 & 1 \\ 1 & 1 \end{pmatrix}\) has rank \(1\) but has two nonzero columns.

\item False. Counterexample: \(\begin{pmatrix} 1 & 0 \\ 0 & 0 \end{pmatrix} \begin{pmatrix} 0 & 0 \\ 1 & 0 \end{pmatrix} = O_{2 \X 2}\), the product has rank \(0\) but both of the factors have rank \(1\).

\item True. Any nonzero matrix has rank \(\ge 1\).
\item True by \CORO{3.4.1}.
\item False by \CORO{3.4.1}.
\item True by \CORO{3.6.2}(b).
\item True by the whole description in \RMK{3.2.6}.
\item True by \THM{3.6}.
\item True by \RMK{3.2.1}.
\end{enumerate}
\end{proof}

\begin{exercise} \label{exercise 3.2.2}
Calculation problem, skip.
\end{exercise}


\begin{exercise} \label{exercise 3.2.3}
Prove that for any \(m \X n\) matrix \(A\), \(\rank(A) = 0\) if and only if \(A\) is the zero matrix.
\end{exercise}

\begin{proof}
If \(A = O\) then by \THM{3.5} \(\rank(A) = 0\).
Now if \(\rank(A) = 0\), again by \THM{3.5}, \(\dim(\spann(\{a_1, ..., a_n\})) = 0\) where \(a_i\) is the \(i\)th column of \(A\).
But that implies \(\spann(\{a_1, ..., a_n\}) = \{ \OV \}\), which must implies \(a_1 = a_2 = ... = a_n = \OV\).
Hence \(A = O_{m \X n}\).
\end{proof}

\begin{exercise} \label{exercise 3.2.4}
Calculation problem, skip.
\end{exercise}

\begin{exercise} \label{exercise 3.2.5}
Calculation problem, skip.
\end{exercise}

\begin{exercise} \label{exercise 3.2.6}
For each of the following \LTRAN{}s \(\T\), determine whether \(\T\) is invertible, and compute \(\T^{-1}\) if it exists.
\begin{enumerate}
\item \(\T: \mathcal{P}_{2}(\SET{R}) \to \mathcal{P}_{2}(\SET{R})\) defined by \(\T(f(x)) = f''(x) + 2f'(x) - f(x)\).

\item \(\T:\mathcal{P}_{2}(\SET{R}) \to \mathcal{P}_{2}(\SET{R})\) defined by \(\T(f(x)) = (x + 1) f'(x)\).

\item \(\T: \SET{R}^{3} \to \SET{R}^{3}\) defined by
\[
    \T(a_1, a_2, a_3) = (a_1 + 2 a_2 + a_3, - a_1 + a_2 + 2 a_3, a_1 + a_3)
\]

\item \(\T: \SET{R}^{3} \to \mathcal{P}_{2}(\SET{R})\) defined by
\[
    \T(a_1, a_2, a_3) = (a_1 + a_2 + a_3) + (a_1 - a_2 + a_3)x + a_1 x^2.
\]

\item \(\T: \mathcal{P}_{2}(\SET{R}) \to \SET{R}^{3}\) defined by  \(\T(f(x)) = (f(-1), f(0), f(1))\).

\item \(\T: M_{2 \X 2}(\SET{R}) \to \SET{R}^4\) defined by
\[
    \T(A)= (\TRACE(A), \TRACE(A^\top), \TRACE(EA), \TRACE(AE)),
\]
where
\[
    E = \begin{pmatrix} 0 & 1 \\ 1 & 0 \end{pmatrix}.
\]
\end{enumerate}
\end{exercise}

\begin{proof}
We use \CORO{2.18.1} to test \(\T\) for invertibility and compute the inverse if \(\T\) is invertible.

In each item below, we let \(\alpha\) and \(\beta\) be the standard ordered basis of the domain and codomain of \(\T\), respectively.
(\(\alpha\) may be equal to \(\beta\).)

\begin{enumerate}
\item
We have
\begin{align*}
    \T(1) & = -1 = -1 \cdot 1 + 0 \cdot x + 0 \cdot x^2 \\
    \T(x) & = -2 - x = -2 \cdot 1 + (-1) \cdot x + 0 \cdot x^2 \\
    \T(x^2) & = -2 + 4x - x^2 = -2 \cdot 1 + 4 \cdot x + (-1) \cdot x^2
\end{align*}
Thus
\[
    [\T]_{\alpha}^{\beta} = \begin{pmatrix}
        -1 & -2 & -2 \\ 0 & -1 & 4 \\ 0 & 0 & -1
    \end{pmatrix}.
\]
Using the method of \EXAMPLE{3.2.5} and \EXAMPLE{3.2.6}, we can show that \([\T]_{\alpha}^{\beta}\) is invertible with inverse
\[
    ([\T]_{\alpha}^{\beta})^{-1} = \begin{pmatrix}
        -1 & -2 & -10 \\ 0 & -1 & -4 \\ 0 & 0 & -1
    \end{pmatrix}.
\]
Thus (by \CORO{2.18.1}) \(\T\) is invertible, and \(([\T]_{\alpha}^{\beta})^{-1} = [\T^{-1}]_{\beta}^{\alpha}\).
Hence by \THM{2.14}, we have
\begin{align*}
    [\T^{-1}(a_0 + a_1 x + a_2 x^2)]_{\alpha}
        & = [\T^{-1}]_{\beta}^{\alpha} [(a_0 + a_1 x + a_2 x^2)]_{\beta} & \text{by \THM{2.14}} \\
        & = \left(\begin{array}{rrr}
                -1 & -2 & -10 \\
                0 & -1 & -4 \\
                0 & 0 & -1
            \end{array}\right)
            \left(\begin{array}{l}
                a_0 \\ a_1 \\ a_2
            \end{array}\right) \\
        & = \left(\begin{array}{c}
                \BLUE{-a_0 - 2a_1 - 10a_2} \\ \RED{-a_1 - 4a_2} \\ \GREEN{-a_2}
            \end{array}\right)
\end{align*}
Therefore
\[
    \T^{-1}(a_0 + + a_1 x + a_2 x^2) = \BLUE{(-a_0 - 2a_1 - 10a_2)} + \RED{(-a_1 - 4a_2)} x + \GREEN{-a_2} x^2.
\]

\item We have
\begin{align*}
    \T(1) & = (x + 1) \cdot 0 = 0 \cdot 1 + 0 \cdot x + 0 \cdot x^2 \\
    \T(x) & = (x + 1) \cdot 1 = 1 \cdot 1 + 1 \cdot x + 0 \cdot x^2 \\
    \T(x^2) & = (x + 1) \cdot 2x = 0 \cdot 1 + 2 \cdot x + 2 \cdot x^2
\end{align*}
Thus
\[
    [\T]_{\alpha}^{\beta} = \begin{pmatrix} 0 & 1 & 0 \\ 0 & 1 & 2 \\ 0 & 0 & 2 \end{pmatrix}
\]
Which is not invertible.

\item Skip.

\item We have
\begin{align*}
    \T(1,0,0) & = 1 + x + x^{2} \\
    \T(0,1,0) & = 1 - x \\
    \T(0,0,1) & = 1 + x \\
\end{align*}
So
\[
    [\T]_{\alpha}^{\beta} = \begin{pmatrix}
        1 & 1 & 1 \\ 1 & -1 & 1 \\ 1 & 0 & 0
    \end{pmatrix}.
\]
Using the method of \EXAMPLE{3.2.5} and \EXAMPLE{3.2.6}, we can show that \([\T]_{\alpha}^{\beta}\) is invertible with inverse
\[
    ([\T]_{\alpha}^{\beta})^{-1} = \begin{pmatrix}
        0 & 0 & 1 \\
        \frac1{2} & -\frac1{2} & 0 \\
        \frac1{2} & \frac1{2} & -1
    \end{pmatrix}.
\]
Thus (by \CORO{2.18.1}) \(\T\) is invertible, and \(([\T]_{\alpha}^{\beta})^{-1} = [\T^{-1}]_{\beta}^{\alpha}\).
Hence by \THM{2.14}, we have
\begin{align*}
    [\T^{-1}(a_0 + a_1 x + a_2 x^2)]_{\alpha}
        & = [\T^{-1}]_{\beta}^{\alpha} [(a_0 + a_1 x + a_2 x^2)]_{\beta} & \text{by \THM{2.14}} \\
        & = \left(\begin{array}{rrr}
                0 & 0 & 1 \\
                \frac1{2} & -\frac1{2} & 0 \\
                \frac1{2} & \frac1{2} & -1
            \end{array}\right)
            \left(\begin{array}{l}
                a_0 \\ a_1 \\ a_2
            \end{array}\right) \\
        & = \left(\begin{array}{c}
                \BLUE{a_2} \\ \RED{\frac1{2}a_0 - \frac1{2}a_1} \\ \GREEN{\frac1{2}a_0 + \frac1{2}a_1 - a_2}
            \end{array}\right)
\end{align*}
Therefore
\[
    \T^{-1}(a_0 + + a_1 x + a_2 x^2) = (\BLUE{a_2}, \RED{\frac1{2}a_0 - \frac1{2}a_1}, \GREEN{\frac1{2}a_0 + \frac1{2}a_1 - a_2}).
\]

\item Skip.
\item Skip.
\end{enumerate}
\end{proof}

\begin{exercise} \label{exercise 3.2.7}
Express the invertible matrix
\[
    \begin{pmatrix} 1 & 2 & 1 \\ 1 & 0 & 1 \\ 1 & 1 & 2 \end{pmatrix}
\]
as a product of elementary matrices.
\end{exercise}

\begin{proof}
We can do the Gaussian elimination and record what operation we've done.
\begin{align*}
    \left(\begin{array}{lll}
        1 & 2 & 1 \\
        1 & 0 & 1 \\
        1 & 1 & 2
    \end{array}\right)
    & \leadsto
    \left(\begin{array}{ccc}
        1 & 2 & 1 \\
        0 & -2 & 0 \\
        1 & 1 & 2
    \end{array}\right)
    \text{, with } & E_1 = \left(\begin{array}{ccc}
        1 & 0 & 0 \\
        -1 & 1 & 0 \\
        0 & 0 & 1
    \end{array}\right) \\
    & \leadsto
        \left(\begin{array}{ccc}
        1 & 2 & 1 \\
        0 & -2 & 0 \\
        0 & -1 & 1
    \end{array}\right)
    \text{, with } & E_2 = \left(\begin{array}{ccc}
        1 & 0 & 0 \\
        0 & 1 & 0 \\
        -1 & 0 & 1
    \end{array}\right) \\
    & \leadsto
        \left(\begin{array}{ccc}
        1 & 2 & 1 \\
        0 & 1 & 0 \\
        0 & -1 & 1
    \end{array}\right)
    \text{, with } & E_3 = \left(\begin{array}{ccc}
        1 & 0 & 0 \\
        0 & -\frac1{2} & 0 \\
        0 & 0 & 1
    \end{array}\right) \\
    & \leadsto
        \left(\begin{array}{ccc}
        1 & 0 & 1 \\
        0 & 1 & 0 \\
        0 & -1 & 1
    \end{array}\right)
    \text{, with } & E_4 = \left(\begin{array}{ccc}
        1 & -2 & 0 \\
        0 & 1 & 0 \\
        0 & 0 & 1
    \end{array}\right) \\
    & \leadsto
    \left(\begin{array}{ccc}
        1 & 0 & 1 \\
        0 & 1 & 0 \\
        0 & 0 & 1
    \end{array}\right)
    \text{, with } & E_5 = \left(\begin{array}{ccc}
        1 & 0 & 0 \\
        0 & 1 & 0 \\
        0 & 1 & 1
    \end{array}\right) \\
        & \leadsto
    \left(\begin{array}{ccc}
        1 & 0 & 0 \\
        0 & 1 & 0 \\
        0 & 0 & 1
    \end{array}\right)
    \text{, with } & E_6 = \left(\begin{array}{ccc}
        1 & 0 & -1 \\
        0 & 1 & 0 \\
        0 & 1 & 1
    \end{array}\right)
\end{align*}
So we have \(\I_3 = E_6 E_5 E_4 E_3 E_2 E_1 A\), hence
\(A = E_{1}^{-1} E_{2}^{-1} E_{3}^{-1} E_{4}^{-1} E_{5}^{-1} E_{6}^{-1}\).
\end{proof}

\begin{exercise} \label{exercise 3.2.8}
Let \(A\) be an \(m \X n\) matrix.
Prove that if \(c\) is any nonzero scalar, then \(\rank(cA) = \rank(A)\).
\end{exercise}

\begin{proof}
Clearly,
\[
    cA = \begin{pmatrix} ca_1 \\ ca_2 \\ \vdots \\ ca_m
    \end{pmatrix}
\]
where \(a_i\) is the \(i\)th row of \(A\).
If we perform \(m\) type 2 e.r.o.s on each row of \(cA\) with scalar \(\frac1{c}\) then we get \(A\).
And since e.r.o.s are rank-preserving, we have \(\rank(cA) = \rank(A)\).
\end{proof}

\begin{exercise} \label{exercise 3.2.9}
Complete the proof of the \CORO{3.4.1} by showing that e.\RED{c}.o.s preserve rank.
\end{exercise}

\begin{proof}
See the second part of the proof of \CORO{3.4.1}.
\end{proof}

\begin{exercise} \label{exercise 3.2.10}
Prove \THM{3.6} for the case that \(A\) is an \(m \X 1\) matrix.
\end{exercise}

\begin{proof}
Given \(m \X 1\) nonzero matrix \(A\), there exists a position \((i, 1)\) s.t. \(A_{i1} \ne 0\).
By means of at most one type 1 row operation and at most one type 2 row operation, \(A\) can be transformed into a matrix with a \(1\) in the \((1, 1)\) position.
By means of at most \(m - 1\) type 3 row operations, this matrix can in turn be transformed into the matrix
\[
\begin{pmatrix} 1 \\ 0 \\ \vdots \\ 0 \end{pmatrix},
\]
which has the desired form, and from \THM{3.5}, it has rank \(1\).
Since e.r.o.s are rank-preserving, the original \(A\) has rank \(1\). which is both \(\le 1\), the number of columns of \(A\), and \(\le m\), the number of rows of \(A\).
\end{proof}

\begin{exercise} \label{exercise 3.2.11}
Let
\[
    B = \left(\begin{array}{c|ccc}
        1 & 0 & \cdots & 0 \\
        \hline 0 & & & \\
        \vdots & & B^{\prime} & \\
        0 & & &
    \end{array}\right),
\]
where \(B'\) is an \(m \X n\) submatrix of \(B\).
Prove that if \(\rank(B) = r\), then \(\rank(B') = r - 1\).
\end{exercise}

\begin{proof}
By \THM{3.5}, the maximum number of \(B\)'s \LID{} rows is equal to \(r\).
Furthermore, by the structure of the form of \(B\), the first column
\[
    \begin{pmatrix} 1 \\ 0 \\ \vdots \\ 0 \end{pmatrix}
\]
\emph{is \LID{} to the remaining columns of \(B\)} since these columns have \(0\) in the first entry.
And that means \(r = \rank(B) = \dim(\spann(\{ \RED{b_1}, b_2, ..., b_n \})) \RED{=} \dim(\spann(\{ b_2, b_3, ..., b_n \})) \RED{+ 1}\), or \(\rank(\spann(\{ b_2, b_3, ..., b_n \})) = r - 1\).

Now it suffices to show \(\rank(B') = \dim(\spann(\{ b_2, b_3, ..., b_n \}))\).
(This is in fact very intuitive since the first \(0\) of each \(b_i\) ``does not contribute any linear independency'', and by removing the it the columns become the corresponding columns of \(B'\), hence \(B'\) and \(\spann(\{ b_2, b_3, ..., b_n \})\) have same dimensions.)

To do this, we define \(\T : F^{m - 1} \to W\), where \(W\) is a subspace of \(F^m\) with elements having \(0\) as the first entry,
and
\[
    \T(a_1, a_2, ..., a_m) = (0, a_1, a_2, ..., a_m).
\]
It's clear that \(\T\) is an isomorphism.
Now let \(v_1, v_2, ..., v_{r - 1}\) be the selected \LID{} vectors of \(\{ b_2, b_3, ..., b_n \}\).
And let \(v_1', v_2', ..., v_{r - 1}'\) be the columns of \(B'\) s.t. \(v_i'\) lies in the same column position as \(v_i\).
Then it's clear that \(\T(v_i') = v_i\) for \(i = 1, ..., r - 1\).
And by \ATHM{2.2}(2.b), \(\{ v_1', v_2', ..., v_{r - 1}'\}\) is also \LID{}, which implies the dimension of subspace generated by the columns of \(B'\) is at least \(r - 1\).
But since \(\rank(\spann(\{ b_2, b_3, ..., b_n \})) = r - 1\) any \(r\) vectors  of \(b_2, ..., b_n\) must be \LDP{}.
And by \ATHM{2.2}(2.b), the corresponding \(r\) vectors mapped by \(\T\) in columns of \(B'\) must also be \LDP{}, and that implies \(\rank(B') < r\).
Hence \(\rank(B') = r - 1\), as desired.
\end{proof}

\begin{exercise} \label{exercise 3.2.12}
Let \(B'\) and \(D'\) be \(m \X n\) matrices, and let \(B\) and \(D\) be \((m + 1) \X (n + 1)\) matrices respectively defined by
\[
    B = \left(\begin{array}{c|ccc}
        1 & 0 & \cdots & 0 \\
        \hline 0 & & & \\
        \vdots & & B^{\prime} & \\
        0 & & &
    \end{array}\right)
    \text { and }
    D = \left(\begin{array}{c|ccc}
        1 & 0 & \cdots & 0 \\
        \hline 0 & & \\
        \vdots & & D^{\prime} & \\
        0 & &
    \end{array}\right).
\]
Prove that if \(B'\) can be transformed into \(D'\) by an elementary row [column] operation, then \(B\) can be transformed into \(D\) by an elementary row [column] operation.
\end{exercise}

\begin{proof}
Suppose \(B'\) can be transformed into \(D'\) by an elementary row operation.
Then by \THM{3.1} we can write \(D' = E'B'\) for some \(m \X m\) elementary matrix \(E'\).
Now let
\[
    E = \left(\begin{array}{c|ccc}
        1 & 0 & \ldots & 0 \\
        \hline 0 & & & \\
        \vdots & & E' & \\
        0 & & &
    \end{array}\right)
\]
Then it's clear that \(E\) is also an elementary matrix with dimension \((m + 1) \X (m + 1)\).
And
\[
    EB = \left(\begin{array}{c|ccc}
        1 & 0 & \ldots & 0 \\
        \hline 0 & & & \\
        \vdots & & E' & \\
        0 & & &
    \end{array}\right) \left(\begin{array}{c|ccc}
        1 & 0 & \cdots & 0 \\
        \hline 0 & & & \\
        \vdots & & B' & \\
        0 & & &
    \end{array}\right) \\
    \RED{=} \left(\begin{array}{c|ccc}
        1 & 0 & \ldots & 0 \\
        \hline 0 & & & \\
        \vdots & & E' B' & \\
        0 & & &
    \end{array}\right) =
    \left(\begin{array}{c|ccc}
        1 & 0 & \ldots & 0 \\
        \hline 0 & & & \\
        \vdots & & D' & \\
        0 & & &
    \end{array}\right)
    = D.
\]
(Note that the red \RED{\(=\)} needs proof, but it's similar to the proof of \EXEC{3.2.15}.)
Hence by \THM{3.1} again \(B\) can be transformed into \(D\) by an elementary row operation.

The case of column operation is similar to prove.
\end{proof}

\begin{exercise} \label{exercise 3.2.13}
Prove (b) and (c) of \CORO{3.6.2}.
\end{exercise}

\begin{proof}
See \CORO{3.6.2}.
\end{proof}

\begin{exercise} \label{exercise 3.2.14}
Let \(\T, \U: V \to W\) be linear transformations.
\begin{enumerate}
\item Prove that \(\RANGE(\T + \U) \subseteq \RANGET + \RANGE(\U)\).
(See the definition \ADEF{1.8} of the sum of subsets of a vector space.)
\item Prove that if \(W\) is finite-dimensional, then \(\rank(\T + \U) \le \rankT + \rank(\U)\).
\item Deduce from (b) that \(\rank(A + B) \le \rank(A) + \rank(B)\) for any \(m \X n\) matrices \(A\) and \(B\).
\end{enumerate}
\end{exercise}

\begin{proof} \ 

\begin{enumerate}
\item Suppose arbitrary \(w \in \RANGE(\T + \U)\).
Then \(\exists v \in V s.t. (\T + \U)(v) = w\), or \(\T(v) + \U(v) = w\).
Then by definition of sum, \(w \in \RANGET + \RANGE(\U)\) since \(w = \T(v) + \U(v)\) where \(\T(v) \in \RANGET\) and \(\U(v) \in \RANGE(\U)\).
Hence \(\RANGE(\T + \U) \subseteq \RANGET + \RANGE(\U)\).

\sloppy
\item From \ATHM{1.27}(1.1), \(\dim(\RANGET + \RANGE(\U)) = \dim(\RANGET) + \dim(\RANGE(\U)) - \dim(\RANGET \cap \RANGE(\U))\).
In particular, \(\dim(\RANGET + \RANGE(\U)) \le \dim(\RANGET) + \dim(\RANGE(\U))\).
And from part(a) we have \(\dim(\RANGE(\T + \U)) \le \dim(\RANGET + \RANGE(\U))\).
Put them together, we have \(\dim(\RANGE(\T + \U)) \le \dim(\RANGET) + \dim(\RANGE(\U))\).
That is, by definition of rank, \(\rank(\T + \U)) \le \rankT + \rank(\U)\).

\item
\begin{align*}
    \rank(A + B) & = \rank(\LMTRAN_{A + B}) & \text{by \DEF{3.3}} \\
                 & = \rank(\LMTRAN_A + \LMTRAN_B) & \text{by \THM{2.15}(c)} \\
                 & \le \rank(\LMTRAN_A) + \rank(\LMTRAN_B) & \text{by part(b)} \\
                 & = \rank(A) + \rank(B) & \text{by \DEF{3.3}}
\end{align*}
\end{enumerate}
\end{proof}

\begin{exercise} \label{exercise 3.2.15}
Suppose that \(A\) and \(B\) are matrices having \(n \X p\) and \(n \X q\) rows, respectively.
Prove that \(M(A|B) = (MA|MB)\) for any \(m \X n\) matrix \(M\).
\end{exercise}

\begin{proof}
We have
\begin{equation*}
    [M(A | B)]_{ij} =
    \begin{cases}
        \sum_{k = 1}^n M_{ik} A_{kj} = (MA)_{ij} & \text{ if } 1 \le j \le p \\
        \sum_{k = 1}^n M_{ik} B_{k(\RED{j - p})} = (MB)_{i(j - p)} & \text{ if } p < j \le p + q \\
    \end{cases}
\end{equation*}
So
\begin{align*}
    M(A | B) & = \left(\begin{array}{lll|lll}
        (MA)_{11} & ... & (MA)_{1p} & (MB)_{1(p+1-p)} & ... & (MB)_{1(p+q-p)} \\
        \vdots    &     & \vdots    & \vdots          &     & \vdots \\
        (MA)_{n1} & ... & (MA)_{np} & (MB)_{n(p+1-p)} & ... & (MB)_{n(p+q-p)} \\
    \end{array}\right) \\
             & = \left(\begin{array}{lll|lll}
        (MA)_{11} & ... & (MA)_{1p} & (MB)_{11} & ... & (MB)_{1q} \\
        \vdots    &     & \vdots    & \vdots          &     & \vdots \\
        (MA)_{n1} & ... & (MA)_{np} & (MB)_{n1} & ... & (MB)_{nq} \\
    \end{array}\right),
\end{align*}
which is equal to \((MA|MB)\).
\end{proof}

\begin{exercise} \label{exercise 3.2.16}
Supply the details to the proof of (b) of \THM{3.4}.
\end{exercise}

\begin{proof}
See the proof of \THM{3.4}.
\end{proof}

\begin{exercise} \label{exercise 3.2.17}
Prove that if \(B\) is a \(3 \X 1\) matrix and \(C\) is a \(1 \X 3\) matrix, then the \(3 \X 3\) matrix \(BC\) \textbf{has rank at most \(1\)}.
Conversely, show that if \(A\) is any \(3 \X 3\) matrix having rank \(1\), then there exist a \(3 \X 1\) matrix \(B\) and a \(1 \X 3\) matrix \(C\) such that \(A = BC\).
\end{exercise}

\begin{proof} \ 

\(\Longrightarrow\): By \THM{3.7}(c)(d), \(\rank(BC) \le \min(\rank(B), \rank(C)) = 1\).

\(\Longleftarrow\): Suppose \(A\) is any \(3 \X 3\) matrix having rank \(1\).
Then \(A\) must have the form
\[
    A = \begin{pmatrix} a & c_1 a & c_2 a \end{pmatrix} or \begin{pmatrix} c_1 a' & a' & c_2 a' \end{pmatrix} or \begin{pmatrix} c_1 a'' & c_2 a'' & a'' \end{pmatrix}
\]
where \(c_1, c_2 \in F\) and \(a, a', a''\) is the first/second/third column of \(A\).
(If \(A\) is not in one of these forms then it will trivially to get contradiction that \(\rank(A) \ne 1\).)
We just prove the case that \(A\) has the first form, the other cases are similar to prove.
Now let \(a = \begin{pmatrix} a_1 \\ a_2 \\ a_3 \end{pmatrix}\).
Then \(A\) can be represented as the product of \(3 \X 1\) ``matrix'' \(a\) and \(1 \X 3\) matrix \(c = \begin{pmatrix} 1 & c_1 & c_2 \end{pmatrix}\) since
\[
    a c = \begin{pmatrix} a_1 \\ a_2 \\ a_3 \end{pmatrix} \begin{pmatrix} 1 & c_1 & c_2 \end{pmatrix}
    = \begin{pmatrix}
        a_1 & c_1 a_1 & c_2 a_1 \\
        a_2 & c_1 a_2 & c_2 a_2 \\
        a_3 & c_1 a_3 & c_2 a_3 \\
    \end{pmatrix}
    = \begin{pmatrix} a & c_1 a & c_2 a \end{pmatrix} = A.
\]
\end{proof}

\begin{exercise} \label{exercise 3.2.18}
Let \(A\) be an \(m \X n\) matrix and \(B\) be an \(n \X p\) matrix.
Prove that \(AB\) can be written as a sum of \(n\) matrices of rank at most one.
\end{exercise}

\begin{proof}
In fact, we can split \(AB\) as:
\begin{align*}
    AB & = \begin{pmatrix}
                (AB)_{11} & ... & (AB)_{1p} \\
                \vdots    &     & \vdots \\
                (AB)_{m1} & ... & (AB)_{mp} \\
            \end{pmatrix} & \\
       & = \begin{pmatrix}
                \sum_{k = 1}^n A_{1k} B_{k1} & ... & \sum_{k = 1}^n A_{1k} B_{kp} \\
                \vdots    &     & \vdots \\
                \sum_{k = 1}^n A_{mk} B_{k1} & ... & \sum_{k = 1}^n A_{mk} B_{kp} \\
            \end{pmatrix} & \\
        & \RED{=} \begin{pmatrix}
                A_{1\RED{1}} B_{\RED{1}1} & ... & A_{1\RED{1}} B_{1p} \\
                \vdots    &     & \vdots \\
                A_{m\RED{1}} B_{\RED{1}1} & ... & A_{m\RED{1}} B_{\RED{1}p} \\
            \end{pmatrix}
            + \begin{pmatrix}
                A_{1\RED{2}} B_{\RED{2}1} & ... & A_{1\RED{2}} B_{\RED{2}p} \\
                \vdots    &     & \vdots \\
                A_{m\RED{2}} B_{\RED{2}1} & ... & A_{m\RED{2}} B_{\RED{2}p} \\
            \end{pmatrix}
            + ... + \begin{pmatrix}
                A_{1\RED{n}} B_{\RED{n}1} & ... & A_{1\RED{n}} B_{\RED{n}p} \\
                \vdots    &     & \vdots \\
                A_{m\RED{n}} B_{\RED{n}1} & ... & A_{m\RED{n}} B_{\RED{n}p} \\
            \end{pmatrix}, \\
            & \text{\ \ \ \ \ \ \ (by splitting into \(n\) matrices)}
\end{align*}
and each splitted matrix can be considered as the product of \(m \X 1\) and \(1 \X p\) matrices;
that is,
\[
    \begin{pmatrix} A_{11} \\ A_{21} \\ \vdots \\ A_{m1} \end{pmatrix} \begin{pmatrix} B_{11} & B_{12} & ... & B_{1p} \end{pmatrix}
    + \begin{pmatrix} A_{12} \\ A_{22} \\ \vdots \\ A_{m2} \end{pmatrix} \begin{pmatrix} B_{21} & B_{22} & ... & B_{2p} \end{pmatrix}
    + ... + \begin{pmatrix} A_{1n} \\ A_{2n} \\ \vdots \\ A_{mn} \end{pmatrix} \begin{pmatrix} B_{n1} & B_{n2} & ... & B_{np} \end{pmatrix}.
\]
By (the generalization of) \EXEC{3.2.17}, each product has at most rank one.
Hence we have written \(AB\) as the sum of \(n\) matrices of rank at most one, as desired.
\end{proof}

\begin{exercise} \label{exercise 3.2.19}
Let \(A\) be an \(m \X n\) matrix with rank \(m\) and \(B\) be an \(n \X p\) matrix with rank \(n\).
Determine the rank of \(AB\).
Justify your answer.
\end{exercise}

\begin{proof}
We have
\begin{align*}
             & \rank(A) = m \\
    \implies & \rank(\LMTRAN_A) = m & \text{by \DEF{3.3}} \\
    \implies & \LMTRAN_A \text{ is onto } & \text{since the codomain of \(\LMTRAN_A : F^n \to F^m\) has \(\dim = m\)}
\end{align*}
Similarly, \(\LMTRAN_B: F^p \to F^n\) is also onto.

Then
\begin{align*}
    \rank(AB) & = \rank(\LMTRAN_{AB}) & \text{by \DEF{3.3}} \\
              & = \rank(\LMTRAN_A \LMTRAN_B) & \text{by \THM{2.15}(e)} \\
              & = \dim(\LMTRAN_A(\LMTRAN_B(F^p))) & \text{just by def of range and rank and composition} \\
              & = \dim(\LMTRAN_A(F^n)) & \text{since \(\LMTRAN_B\) is onto} \\
              & = \dim(F^m) & \text{since \(\LMTRAN_A\) is onto} \\
              & = m.
\end{align*}
\end{proof}

\begin{exercise} \label{exercise 3.2.20}
Let
\[
    A = \begin{pmatrix}
        1 & 0 & -1 & 2 & 1 \\
        -1 & 1 & 3 & -1 & 0 \\
        -2 & 1 & 4 & -1 & 3 \\
        3 & -1 & -5 & 1 & -6
    \end{pmatrix}.
\]
\begin{enumerate}
\item Find a \(5 \X 5\) matrix \(M\) with rank \(\RED{2}\) such that \(AM = O\), where \(O\) is the \(4 \X 5\) zero matrix.

\item Suppose that \(B\) is a \(5 \X 5\) matrix such that \(AB = O\).
Prove that \(\rank(B) \le \RED{2}\).
\end{enumerate}
\end{exercise}

\begin{note}
I feel that the exercise is really strange, because the criteria of rank \(\RED{2}\) is just like a magic number.
It seems more reasonable to given some other fact, e.g. \(\rank(A) = 3\).
\end{note}

\begin{proof}
(Note that I use \THM{3.8}, which is in \SEC{3.3}, so this proof is a forward reference.)
\begin{enumerate}
\item
First, if \(AM = O\), that means \textbf{each} column of \(M\) is in the null space of \(\LMTRAN_A\), because \(A m_i = 0 \in F^m\) for each column \(m_i\) of \(M\).
Hence given any that kind of \(M\) the columns space of \(M\) is a subspace of \(\NULL(\LMTRAN_A)\).
Hence the \(\rank(M) \le \dim(\NULL(\LMTRAN_A))\).
Then why not we just first find the null space of \(\LMTRAN_A\)?
By the technique in \SEC{3.3}, this is equivalent to find the solution set of \(Ax = 0\),
which is
\[
    \left\{
        t_1 \begin{pmatrix} 1 \\ -2 \\ 1 \\ 0 \\ 0 \end{pmatrix}
        + t_2 \begin{pmatrix} 3 \\ 1 \\ 0 \\ -2 \\ 1 \end{pmatrix}
        : t_1, t_2 \in \SET{R}
    \right\}.
\]
Hence the dimension of \(\NULL(\LMTRAN_A)\) is \(2\), and the basis is
\[
    \left\{
        \begin{pmatrix} 1 \\ -2 \\ 1 \\ 0 \\ 0 \end{pmatrix}
        \begin{pmatrix} 3 \\ 1 \\ 0 \\ -2 \\ 1 \end{pmatrix}
    \right\}.
\]
So we just let
\[
    M = \left[\begin{array}{ccccc}
        1 & 3 & 0 & 0 & 0 \\
        -2 & 1 & 0 & 0 & 0 \\
        1 & 0 & 0 & 0 & 0 \\
        0 & -2 & 0 & 0 & 0 \\
        0 & 1 & 0 & 0 & 0
    \end{array}\right]
\]
then \(\rank(M) = 2\) and \(AM = O\).

\item We have shown in the previous case that any kind of \(B\) s.t. \(AB = O\) has the column space as a subspace of \(\NULL(\LMTRAN_A)\), and we have shown that \(\dim(\NULL(\LMTRAN_A)) = 2\), hence \(\rank(B) \le 2\).
\end{enumerate}
\end{proof}

\begin{exercise} \label{exercise 3.2.21}
Let \(A\) be an \(m \X n\) matrix with rank \(m\).
Prove that there exists an \(n \X m\) matrix \(B\) such that \(AB = I_m\).
\end{exercise}

\begin{proof}
With same argument in \EXEC{3.2.19}, \(\LMTRAN_A : F^n \to F^m\) is onto.
So for each vector \(e_1, e_2, ..., e_m\) in the standard basis of \(F^m\), we can find \(v_1, v_2, ..., v_m \in F^n\) s.t. \(\LMTRAN_A(v_i) = e_i\);
that is, \(A v_i = e_i\).
Now let \(B = \begin{pmatrix} v_1 & v_2 & ... & v_m \end{pmatrix}\).
Then
\begin{align*}
    AB & = \begin{pmatrix} A v_1 & A v_2 & ... & A v_m \end{pmatrix} & \text{by \THM{2.13}(a)} \\
       & = \begin{pmatrix} e_1 & e_2 & ... & e_m \end{pmatrix} = I_m,
\end{align*}
as desired.
\end{proof}

\begin{exercise} \label{exercise 3.2.22}
Let \(B\) be an \(n \X m\) matrix with rank \(m\).
Prove that there exists an \(m \X n\) matrix \(A\) such that \(AB = I_m\).
\end{exercise}

\begin{proof}
By \CORO{3.6.2}(a), \(B^\top\) is an \(m \X n\) matrix that also has rank \(m\).
And by \EXEC{3.2.21}, we can find an \(n \X m\) matrix \(A^\top\) s.t. \(B^\top A^\top = I_m\).
(I intentionally denote that matrix with transpose.)
Then we have \((B^\top A^\top)^\top = I_m^\top = I_m\);
that is \(A B = I_m\).
So we have found a \(m \X n\) matrix \(A\) s.t. \(A B = I_m\).
\end{proof}

\begin{additional theorem} \label{athm 3.5}
This is a placeholder theorem for \EXEC{3.2.8}, given any nonzero scalar \(c\), \(\rank(cA) = \rank(A)\) any any \(m \X n\) matrix \(A\).
\end{additional theorem}

\begin{additional theorem} \label{athm 3.6}
This is a placeholder theorem for \EXEC{3.2.11} and \EXEC{3.2.12}, which are intermediate steps of \THM{3.6}.
\end{additional theorem}

\begin{additional theorem} \label{athm 3.7}
This is a placeholder theorem for \EXEC{3.2.14}:
Let \(\T, \U: V \to W\) be linear transformations.
\begin{enumerate}
\item \(\RANGE(\T + \U) \subseteq \RANGET + \RANGE(\U)\).
\item If \(W\) is finite-dimensional, then \(\rank(\T + \U) \le \rankT + \rank(\U)\).
\item \(\rank(A + B) \le \rank(A) + \rank(B)\) for any \(m \X n\) matrices \(A\) and \(B\).
\end{enumerate}
\end{additional theorem}

\begin{additional theorem} \label{athm 3.8}
\sloppy This is a placeholder theorem for \EXEC{3.2.15}:
\(M(A|B) = (MA|MB)\) (where these matrices have compatible dimensions).
\end{additional theorem}

\begin{additional theorem} \label{athm 3.9}
This is a placeholder theorem for \EXEC{3.2.17}:
If \(B\) is a \(3 \X 1\) matrix and \(C\) is a \(1 \X 3\) matrix, then the \(3 \X 3\) matrix \(BC\) \textbf{has rank at most \(1\)}.
Conversely, if \(A\) is any \(3 \X 3\) matrix having rank \(1\), then there exist a \(3 \X 1\) matrix \(B\) and a \(1 \X 3\) matrix \(C\) such that \(A = BC\).

This theorem can be generalized with \(n \X 1\) and \(1 \X n\) matrices using induction.
\end{additional theorem}

\begin{additional theorem} \label{athm 3.10}
This is a placeholder theorem for \EXEC{3.2.18}:
Let \(A\) be an \(m \X n\) matrix and \(B\) be an \(n \X p\) matrix.
Then \(AB\) can be written as a sum of \(n\) matrices of rank at most one.
\end{additional theorem}

\begin{additional theorem} \label{athm 3.11}
This is a placeholder theorem for \EXEC{3.2.19}, which has concepts related to ``onto''.
\end{additional theorem}

\begin{additional theorem} \label{athm 3.12}
This is a placeholder theorem for \EXEC{3.2.21} and \EXEC{3.2.22}, which let us find ``some kind of inverse'' for \emph{non-square} matrices.
\end{additional theorem}
\section{Systems of Linear Equations -- Theoretical Aspects} \label{sec 3.3}

This section and the next are devoted to the study of systems of linear equations, which arise naturally in both the physical and social sciences.
In this section, we apply results from \CH{2} to describe the \textbf{solution sets} of systems of linear equations as \emph{subsets}(not necessarily subspaces) of a vector space.
In \SEC{3.4}, we will use e.r.o.s to provide a \emph{computational method} for finding all solutions to such systems.

\begin{additional definition} \label{adef 3.1}
The system of equations

\[
(S)
\begin{array}{c}
a_{11} x_{1}+a_{12} x_{2}+\cdots+a_{1 n} x_{n}=b_{1} \\
a_{21} x_{1}+a_{22} x_{2}+\cdots+a_{2 n} x_{n}=b_{2} \\
\vdots \\
a_{m 1} x_{1}+a_{m 2} x_{2}+\cdots+a_{m n} x_{n}=b_{m}
\end{array}
\]

where \(a_{ij}\) and  \(b_{i}\) (\(1 \le i \le m\) and \(1 \le j \le n\)) are scalars in a field \(F\) and \(x_1, x_2, ..., x_n\) are \(n\) variables taking values in \(F\), is called \textbf{a system of \(m\) linear equations in \(n\) unknowns over the field \(F\)}.

The \(m \X n\) matrix
\[
    A = \left(\begin{array}{cccc}
        a_{11} & a_{12} & ... & a_{1 n} \\
        a_{21} & a_{22} & ... & a_{2 n} \\
        \vdots & \vdots & & \vdots \\
        a_{m 1} & a_{m 2} & ... & a_{m n}
    \end{array}\right)
\]
is called the \textbf{coefficient matrix} of the system \((S)\).

If we let
\[
    x = \left(\begin{array}{c} x_{1} \\ x_{2} \\ \vdots \\ x_{n} \end{array}\right)
    \text { and }
    b = \left(\begin{array}{c} b_{1} \\ b_{2} \\ \vdots \\ b_{m} \end{array}\right),
\]
then the system \((S)\) \emph{may be rewritten as a single matrix equation}
\[
    A x = b.
\]
To exploit the results that we have developed, \emph{we often consider a system of linear equations as a single matrix equation}.

A \textbf{solution} to the system \((S)\) is an \(n\)-tuple
\[
    s = \left(\begin{array}{c} s_{1} \\ s_{2} \\ \vdots \\ s_{n} \end{array}\right) \in \SET{R}^{n}
\]
such that \(As = b\).
The \emph{set of all solutions} to the system \((S)\) is called the \textbf{solution set} of the system.
System \((S)\) is called \textbf{consistent} if its solution set is nonempty;
otherwise it is called \textbf{inconsistent}.
\end{additional definition}

\begin{example} \label{example 3.3.1} \ 

\begin{enumerate}
\item Consider the system
\[
    \begin{array}{l}
        x_1 + x_2 = 3 \\
        x_1 - x_2 = 1
    \end{array}.
\]
By use of familiar techniques, we can solve the preceding system and conclude that there is only one solution: \(x_1 = 2, x_2 =1\);
that is,
\[
    s = \left(\begin{array}{l} 2 \\ 1 \end{array}\right).
\]

In matrix form, the system can be written
\[
    \left(\begin{array}{rr} 1 & 1 \\ 1 & -1 \end{array}\right)
    \left(\begin{array}{l} x_1 \\ x_2 \end{array}\right)
    = \left(\begin{array}{l} 3 \\ 1 \end{array}\right).
\]

So
\[
    A = \left(\begin{array}{rr} 1 & 1 \\ 1 & -1 \end{array}\right) \text { and }
    b = \left(\begin{array}{l} 3 \\ 1 \end{array}\right)
\]

\item Consider
\[
    \sysdelim..\systeme{
        2 x_1 + 3 x_2 + x_3 = 1,
        x_1 - x_2 + 2 x_3 = 6
    };
\]
that is,
\[
    \left(\begin{array}{rrr} 2 & 3 & 1 \\ 1 & -1 & 2 \end{array}\right)
    \left(\begin{array}{l} x_1 \\ x_2 \\ x_3 \end{array}\right)
    = \left(\begin{array}{l} 1 \\ 6 \end{array}\right).
\]
This system has many solutions, such as
\[
    s = \left(\begin{array}{r} -6 \\ 2 \\ 7 \end{array}\right)
    \text { and }
    s = \left(\begin{array}{r} 8 \\ -4 \\ -3 \end{array}\right).
\]

\item Consider
\[
    \begin{array}{l} x_{1}+x_{2}=0 \\ x_{1}+x_{2}=1 \end{array}
\]
that is,
\[
    \left(\begin{array}{ll} 1 & 1 \\ 1 & 1 \end{array}\right)
    \left(\begin{array}{l} x_{1} \\ x_{2} \end{array}\right)
    = \left(\begin{array}{l} 0 \\ 1 \end{array}\right)
\]
It is evident that this system has no solutions.
Thus we see that a system of linear equations can have one, many, or no solutions.
\end{enumerate}
\end{example}

We must be able to \emph{recognize when} a system has a solution and then be able to describe all its solutions.
This section and the next are devoted to this end.

We begin our study of systems of linear equations by examining the class of \emph{homogeneous systems} of linear equations.
Our first result (\THM{3.8}) shows that the set of solutions to a homogeneous system of \(m\) linear equations in \(n\) unknowns forms a \textbf{subspace} of \(F^n\).
Hence we can then apply the theory of vector spaces to this set of solutions.
For example, a basis for the solution space can be found, and any solution can be expressed as a linear combination of the vectors in the basis.

\begin{definition} \label{def 3.5}
A system \(Ax = b\) of \(m\) linear equations in \(n\) unknowns is said to be \textbf{homogeneous} if \(b = 0 \in F^m\).
Otherwise the system is said to be \textbf{nonhomogeneous}.
\end{definition}

\begin{theorem} \label{thm 3.8}
Let \(Ax = 0\) be a homogeneous system of \(m\) linear equations in \(n\) unknowns over a field \(F\).
Let \(K\) denote the set of all solutions to \(Ax = 0\).
Then \(K = \NULL(\LMTRAN_A)\);
hence \(K\) is a subspace of \(F^n\) of dimension (by \THM{2.3}) \(\dim(F^n) - \dim(\RANGE(\LMTRAN_A)) = n - \rank(A)\).
\end{theorem}

\begin{proof}
Clearly, \(K = \{s \in F^n : As = 0 \}\), which by definition of \(\LMTRAN_A\) is equal to \(\{s \in F^n : \LMTRAN(s) = 0 \}\), which by definition of null space is equal to \(\NULL(\LMTRAN_A)\).
\end{proof}

\begin{corollary} \label{corollary 3.8.1}
If \(m < n\), the system \(Ax = 0\) has a \emph{nonzero} solution.
\end{corollary}

\begin{proof}
Suppose that \(m < n\).
Then
\begin{align*}
    \rank(A) & = \rank(\LMTRAN_A) & \text{by \DEF{3.3}} \\
             & \le \min(m, n) & \text{by \THM{3.6}} \\
             & = m. \MAROON{(1)}
\end{align*}
And let \(K = \NULL(\LMTRAN_A)\), then
\begin{align*}
    \dim(K) & = n - \rank(A) & \text{by \THM{3.8}} \\
            & \ge n - m & \text{by \MAROON{(1)}} \\
            & > 0 & \text{since \(m < n\)}
\end{align*}
Since \(\dim(K) > 0\), \(K \ne \{ 0 \}\).
Thus there exists a \emph{nonzero} vector \(s \in K\);
so \(s\) is a nonzero solution to \(Ax = 0\).
\end{proof}

\begin{example} \label{example 3.3.2} \ 

\begin{enumerate}
\item Consider the system
\[
    \sysdelim..\systeme{
        x_1 + 2 x_2 + x_3 = 0,
        x_1 - x_2 - x_3 = 0
    }
\]
Let
\[
    A = \left(\begin{array}{rrr}
        1 & 2 & 1 \\
        1 & -1 & -1
    \end{array}\right)
\]
be the coefficient matrix of this system.
It is clear that \(\rank(A) = 2\).
If \(K\) is the solution set of this system, then (by \THM{3.8}) \(\dim(K) = 3 - \rank(A) = 1\).
Thus any nonzero solution constitutes a basis for \(K\).
For example, since
\[
    \begin{pmatrix} 1 \\ -2 \\ 3 \end{pmatrix}
\]
is a solution to the given system,
\[
    \left\{ \begin{pmatrix} 1 \\ -2 \\ 3 \end{pmatrix} \right\}
\]
is a basis for \(K\).
Thus any vector in \(K\) is of the form
\[
    t \begin{pmatrix} 1 \\ -2 \\ 3 \end{pmatrix} = \begin{pmatrix} 1t \\ -2t \\ 3t \end{pmatrix},
\]
where \(t \in \SET{R}\).

\item Consider the system \(x_1 - 2x_2 + x_3 = 0\) of \emph{one} equation in \emph{three} unknowns.
If \(A = \begin{pmatrix} 1 & -2 & 1 \end{pmatrix}\) is the coefficient matrix, then \(\rank(A) = 1\).
Hence if \(K\) is the solution set, then (by \THM{3.8}) \(\dim(K) = 3 - \rank(K) = 2\).
Note that
\[
    \begin{pmatrix} 2 \\ 1 \\ 0 \end{pmatrix}
    \text{ and }
    \begin{pmatrix} -1 \\ 0 \\ 1 \end{pmatrix}
\]
are \LID{} vectors in \(K\).
Thus they constitute a basis for \(K\), so that
\[
    K = \left\{
        t_1 \begin{pmatrix} 2 \\ 1 \\ 0 \end{pmatrix}
        + t_2 \begin{pmatrix} -1 \\ 0 \\ 1 \end{pmatrix}
        : t_1, t_2 \in \SET{R} \right\}.
\]
\end{enumerate}
\end{example}

\begin{note}
The example above just mysteriously gave some solution(s) of a system of linear equations.
In \SEC{3.4}, \emph{explicit computational methods} for finding a basis for the solution set of a homogeneous system are discussed.
\end{note}

We now turn to the study of \textbf{non}\emph{homogeneous} systems.
Our next result shows that the solution set of a nonhomogeneous system \(Ax = b\) can be \emph{described in terms of} the solution set of the homogeneous system \(Ax = 0\).

\begin{additional definition} \label{adef 3.2}
We refer to the equation \(Ax = 0\) as the \textbf{homogeneous system corresponding to} \(Ax = b\).
\end{additional definition}

\begin{theorem} \label{thm 3.9}
Let \(K\) be the solution set of a \emph{consistent} system of linear equations \(Ax = b\), (hence \(K\) is non-empty;)
and let \(\mathrm{K_H}\) be the solution set of the \emph{corresponding homogeneous system} \(Ax = 0\).
Then for any solution \(s\) to \(Ax = b\),
\[
    K = \{ s \} + \mathrm{K_H} = \{ s + k : k \in K_H \}.
\]
\end{theorem}

\begin{note}
Related exercise: \EXEC{2.1.24}. (Or \ATHM{2.6}.)
\end{note}

\begin{proof}
Let \(s\) be \emph{any} solution to \(Ax = b\).
We must show that \(K = \{ s \} + \mathrm{K_H}\).
Suppose arbitrary \(w \in K\), then \(Aw = b\).
Hence
\begin{align*}
    A(w - s) & = Aw - As & \text{by \THM{2.12}(a)} \\
             & = b - b \\
             & = 0,
\end{align*}
so \(w - s \in \mathrm{K_H}\).
In particular, \(w = s + (w - s)\), where \(s \in \{ s \}\) and \(w - s \in \mathrm{K_H}\), so by definition of \( \{ s \} + \mathrm{K_H}\), \(w \in \{ s \} + \mathrm{K_H}\).
Since \(w\) is arbitrary, we have \(K \subseteq \{ s \} + \mathrm{K_H}\).

Conversely, suppose arbitrary \(w \in \{ s \} + \mathrm{K_H}\);
then \(w = s + k\) for some \(k \in \mathrm{K_H}\).
But then \(Aw = A(s + k) = As + Ak = b + 0 = b\);
so \(w \in K\).
Therefore \(\{ s \} + \mathrm{K_H} \subseteq K\).

Thus \(K = \{ s \} + \mathrm{K_H}\).

Note that \(K\) is \emph{not} a subspace of \(F^n\) when \(b \ne 0 \in F^m\).
\end{proof}

\begin{example} \label{example 3.3.3} \ 

\begin{enumerate}
\item Consider the system
\[
    \sysdelim..\systeme{
        x_1 + 2 x_2 + x_3 = 7,
        x_1 - x_2 - x_3 = -4
    }
\]

The corresponding homogeneous system is the system in \EXAMPLE{3.3.2}(a).
It is easily verified that
\[
    s = \begin{pmatrix} 1 \\ 1 \\ 4 \end{pmatrix}
\]
is a solution to the preceding nonhomogeneous system.
So the solution set of the system is (by \THM{3.9})
\[
    K = \left\{ s + k : k \in \mathrm{K_H} \right\}
      = \left\{ \begin{pmatrix} 1 \\ 1 \\ 4 \end{pmatrix}
           + t \begin{pmatrix} 1 \\ -2 \\ 3 \end{pmatrix}
           : t \in \SET{R}
        \right\}
\]

\item
Consider the system \(x_1 - 2x_2 + x_3 = 4\).
The corresponding homogeneous system is the system in \EXAMPLE{3.2.2}(b).
Since
\[
    s = \begin{pmatrix} 4 \\ 0 \\ 0 \end{pmatrix}
\]
is a solution to the given system, the solution set \(K\) can be written as
\[
    K = s + \mathrm{K_H}
      = \bigg\{
        \begin{pmatrix} 4 \\ 0 \\ 0 \end{pmatrix}
        + t_1 \begin{pmatrix} 2 \\ 1 \\ 0 \end{pmatrix}
        + t_2 \begin{pmatrix} -1 \\ 0 \\ 1 \end{pmatrix}
        : t_1, t_2 \in \SET{R} \bigg\}.
\]
\end{enumerate}
\end{example}

The following theorem provides us with a means of computing solutions to \emph{certain}(\(n\) equations, \(n\) unknowns) systems of linear equations.

\begin{theorem} \label{thm 3.10}
Let \(Ax = b\) be a system of \(n\) linear equations in \(n\) unknowns.
If \(A\) is invertible, then the system \textbf{has exactly one solution}, namely, \(A^{-1} b\).
Conversely, if the system has exactly one solution, then \(A\) is invertible.
\end{theorem}

\begin{note}
如果是\ \(n\) 條\ equations,\(n\) 個未知數,則\ \THM{3.10} 可用參數矩陣是否有反矩陣來判斷解是否唯一。
如果不是\ \(n\) 條\ equations,\(n\) 個未知數,則就用\ \SEC{3.4} 的解法,也就是高斯消去法。
\end{note}

\begin{proof} \

\(\Longrightarrow\):
Suppose that \(A\) is invertible.
Substituting \(A^{-1}b\) (as \(x\)) into the system, we have \(A(A^{-1}b) = (AA^{-1})b = I_n b = b\).
Thus \(A^{-1}b\) is a solution.
If \(s\) is an arbitrary solution, then \(As = b\).
Multiplying both sides by \(A^{-1}\) gives \(s = A^{-1}b\).
Thus the system has one and only one solution, namely, \(A^{-1}b\).

\(\Longleftarrow\):
Conversely, suppose that the system has exactly one solution \(s\).
That is, if we let \(K\) be the solution set of the system, then \(K = \{ s \}\).
Now let \(\mathrm{K_H}\) denote the solution set for the corresponding homogeneous system \(Ax = 0\).
By \THM{3.9}, \(K = \{ s \} + \mathrm{K_H}\), that is, \(\{ s \} = \{ s \} + \mathrm{K_H}\).
But this is so \emph{only if} \(\mathrm{K_H} = \{ 0 \}\).
Thus (by \THM{3.8}) \(\NULL(\LMTRAN_A) = \{ 0 \}\), which implies \(\LMTRAN_A\) is invertible (since both domain and codomain are \(F^n\));
that is, \(A\) is invertible.
\end{proof}

\begin{example} \label{example 3.3.4}
Consider the following system of three linear equations in three unknowns:
\[
    \sysdelim..\systeme{
        2 x_2 + 4 x_3 = 2,
        2 x_1 + 4 x_2 + 2 x_3 = 3,
        3 x_1 + 3 x_2 + 1 x_3 = 1}
\]
In \EXAMPLE{3.2.5}, we computed the inverse of the coefficient matrix \(A\) of this system.
Thus (by \THM{3.10}) the system has exactly one solution, namely,
\[
    \begin{pmatrix} x_1 \\ x_2 \\ x_3 \end{pmatrix} = A^{-1}b 
    = \left(\begin{array}{rrr}
        \frac{1}{8} & -\frac{5}{8} & \frac{3}{4} \\
        -\frac{1}{4} & \frac{3}{4} & -\frac{1}{2} \\
        \frac{3}{8} & -\frac{3}{8} & \frac{1}{4}
    \end{array}\right)
    \begin{pmatrix}
        2 \\ 3 \\ 1
    \end{pmatrix}
    = \begin{pmatrix}
        -\frac{7}{8} \\ \frac{5}{4} \\ -\frac1{8}
    \end{pmatrix}.
\]
\end{example}

We use this technique for solving systems of linear equations having invertible coefficient matrices in the application that concludes this section.

In \EXAMPLE{3.3.1}(c), we saw a system of linear equations that has \emph{no} solutions.
We now establish a criterion for determining when a system has solutions.
This criterion involves the rank of the coefficient matrix of the system \(Ax = b\) and the rank of the matrix \((A|b)\).
The matrix \((A|b)\) is called the \textbf{augmented matrix of the system} \(Ax = b\).

\begin{theorem} \label{thm 3.11}
Let \(Ax = b\) be a system of linear equations.
Then the system is consistent if and only if \(\rank(A) = \rank(A|b)\).
\end{theorem}

\begin{note}
直觀的說,就是\ \(Ax = b\) 有解,若且唯若\ \(A\) 的\ columns 能組出\ \(b\)。
\end{note}

\begin{proof}
To say that \(Ax = b\) has a solution is equivalent to saying that \(b \in \RANGE(\LMTRAN_A)\).
(See \EXEC{3.3.8}.)
In the proof of \THM{3.5}, we saw that
\[
    \RANGE(\LMTRAN_A) = \spann(\{ a_1, a_2, ..., a_n \}),
\]
the span of the \emph{columns of} \(A\).
Thus \(Ax = b\) has a solution if and only if \(b \in \spann(\{ a_1, a_2, ..., a_n \})\).
But \(b \in \spann(\{ a_1, a_2, ... , a_n \})\) if and only if \(\spann(\{ a_1, a_2, ..., a_n \}) = \spann(\{ a_1, a_2, ..., a_n, \RED{b} \})\). (See \CH{1}.)
This last statement is equivalent to
\[
    \dim(\spann(\{ a_1, a_2, ..., a_n \})) = \dim(\spann(\{ a_1, a_2, ..., a_n, b \})).
\]
So by \THM{3.5}, the preceding equation reduces to \(\rank(A) = \rank(A | b)\).
\end{proof}

\begin{example} \label{example 3.3.5}
Recall the system of equations
\[
    \sysdelim..\systeme{
        x_1 + x_2 = 0,
        x_1 + x_2 = 1
    }
\]
in \EXAMPLE{3.3.1}(c).
Since
\[
    A = \begin{pmatrix} 1 & 1 \\ 1 & 1 \end{pmatrix}
    \text{ and }
    (A | b) = \begin{pmatrix} 1 & 1 & 0 \\ 1 & 1 & 1 \end{pmatrix},
\]
\(\rank(A) = 1\) and \(\rank(A|b) = 2\).
Because the two ranks are unequal, the system has no solutions.
\end{example}

\begin{example} \label{example 3.3.6}
We can use \THM{3.11} to determine whether \((3, 3, 2)\) is in the range of the \LTRAN{} \(\T : \SET{R}^3 \to \SET{R}^3\) defined by
\[
    \T(a_1, a_2, a_3) = (a_1 + a_2 + a_3, a_1 - a_2 + a_3, a_1 + a_3).
\]
Now \((3, 3, 2) \in \RANGET\) if and only if there exists a vector \(s = (x_1, x_2, x_3)\) in \(\SET{R}^3\) such that \(\T(s) = (3, 3, 2)\).
Such a vectors must be a solution to the system
\[
    \sysdelim..\systeme{
        x_1 + x_2 + x_3 = 3,
        x_1 - x_2 + x_3 = 3,
        x_1       + x_3 = 2
    }.
\]
Since the ranks of the coefficient matrix and the augmented matrix of this system are \(2\) and \(3\), respectively, it follows by \THM{3.11} that this system has no solutions.
Hence \((3, 3, 2) \notin \RANGET\).
\end{example}

\subsection{An Application}
\begin{note}
This is also related to \CH{5}.
\end{note}

In 1973, Wassily-Leontief won the Nobel prize in economics for his work in developing a mathematical model that can be used to \emph{describe various economic phenomena}.
We close this section by applying some of the ideas we have studied to illustrate two special cases of his work.

We begin by considering a simple society composed of three people
(industries)
\begin{itemize}
\item a farmer who grows all the food
\item a tailor who makes all the clothing,
\item and a carpenter who builds all the housing. 
\end{itemize}
We assume that each person \emph{sells to and buys from a central pool} and that \textbf{everything produced is consumed}.
Since no commodities either enter or leave the system, this case is referred to as the \textbf{closed model}.

Each of these three individuals \emph{consumes all} three of the commodities \emph{produced} in the society.
Suppose that the \emph{proportion} of each of the commodities consumed by each person is given in the following table.
\[
    \begin{array}{lccc}
        \hline & \text { Food } & \text { Clothing } & \text { Housing } \\
        \hline \text { Farmer } & 0.40 & 0.20 & 0.20 \\
        \text { Tailor } & 0.10 & 0.70 & 0.20 \\
        \text { Carpenter } & 0.50 & 0.10 & 0.60 \\
        \hline
    \end{array}
\]
Notice that each of the columns of the table must sum to \(1\).
Let \(p_1, p_2\), and \(p_3\) denote the \textbf{incomes} of the farmer, tailor. and carpenter, respectively.
To ensure that this society survives, \emph{we require that the \textbf{consumption} of each individual equals his or her income}.
Note that the farmer consumes \(20\)\% of the clothing.
Because the total cost of all clothing is \(p_2\), the tailor's income, the amount spent by the farmer on clothing is \(0.20 p_2\).
Moreover, \emph{the amount spent} by the farmer on food, clothing, and housing \emph{must equal} the \emph{farmer's income}, and so we obtain the equation
\[
    0.40 p_1 + 0.20 p_2 + 0.20 p_3 = p_1.
\]
Similar equations describing the expenditures of the tailor and carpenter produce the following system of linear equations:
\[
    \sysdelim..\systeme{
        0.40 p_1 + 0.20 p_2 + 0.20 p_3 = p_1,
        0.10 p_1 + 0.70 p_2 + 0.20 p_3 = p_2,
        0.50 p_1 + 0.10 p_2 + 0.60 p_3 = p_3.
    }
\]
This system can be written as \(Ap = p\), where
\[
    p = \begin{pmatrix} p_1 \\ p_2 \\ p_3 \end{pmatrix}
\]
and \(A\) is the coefficient matrix of the system.
(You can see the more general form like \(Ax = x\), which is related the concept called Diagonalization, see \CH{5}.)
In this context, \(A\) is called the \textbf{input-output (or consumption) matrix}, and \(Ap = p\) is called the \textbf{equilibrium condition}.
For vectors \(b = (b_1, b_2, ..., b_n)\) and \(c = (c_1, c_2, ..., c_n)\) in \(\SET{R}^n\), we use the notation \(b \ge c\) [\(b > c\)] to mean \(b_i \le c_i\) [\(b_i > c_i\)] for all \(i\).
The vector \(b\) is called \textbf{nonnegative} [\textbf{positive}) if \(b \ge 0 \in \SET{R}^n\) [\(b > 0 \in \SET{R}^n\)].

At first, it may seem reasonable to replace the equilibrium condition by the inequality \(Ap \le p\), that is, the requirement that consumption \emph{not exceed} production.
(Or more concretely,
\begin{align*}
    0.40 p_1 + 0.20 p_2 + 0.20 p_3 & \RED{\le} p_1,
    0.10 p_1 + 0.70 p_2 + 0.20 p_3 & \RED{\le} p_2,
    0.50 p_1 + 0.10 p_2 + 0.60 p_3 & \RED{\le} p_3.
\end{align*}
)
But, in fact, \(Ap \le p\) \textbf{implies that} \(Ap = p\) \textbf{in the closed model}.
(That is, \(Ap < p\) will never be true.)
For otherwise, there exists a \(k\) s.t. the \(k\)'s people's consumption is \emph{less than} its income;
that is,
\[
    A_{k1} p_{1} + A_{k2} p_{2} + ... + A_{kn} p_n < p_k.
\]
But, since the \emph{columns} of \(A\) sum to \(1\),
\begin{align*}
    \sum_{i = 1}^n p_i & > \sum_{i = 1}^n (\sum_{j = 1}^n A_{ij} p_j) & \text{since \(p_k > \sum_{j = 1}^n A_{kj} p_j\)}\\
                       & = \sum_{j = 1}^n (\sum_{i = 1}^n A_{ij} p_j) & \text{change order of finite summations} \\
                       & = \sum_{j = 1}^n (\sum_{i = 1}^n A_{ij}) p_j & \text{of course} \\
                       & = \sum_{j = 1}^n \RED{1} p_j & \text{since the \(i\)th column of \(A\) sums to \(1\)} \\
                       & = \sum_{j = 1}^n p_j, & \text{of course}
\end{align*}
which is a contradiction.

One solution to the \emph{homogeneous system} \((I - A)x = 0\), which is equivalent\RED{*} to the equilibrium condition, is
\[
    p = \begin{pmatrix} 0.25 \\ 0.35 \\ 0.40 \end{pmatrix}.
\]
We may interpret this to mean that the society \emph{survives} if the farmer, tailor, and carpenter have incomes in the \emph{proportions} \(25 : 35 : 40\) (or \(5 : 7 : 8\)).
\begin{note}
\RED{*}, the meaning of the ``equivalent'' is really the \DEF{3.6}, which is in the next section.
And we have \(s\) is a solution of \((I - A)x = 0\), iff \((I - A)s = 0 \iff Is - As = 0 \iff s - As = 0 \iff As = s\), iff \(s\) is a solution of \(Ap = p\).
\end{note}
Notice that we are not simply interested in any nonzero solution to the system, but in one that is \emph{nonnegative}.
(The meaning of ``negative'' solution in this model makes no sense.)
Thus we must consider the question of whether the system \((I - A)x = 0\) has a nonnegative solution, where \(A\) is a matrix with nonnegative entries whose columns sum to \(1\).
A useful theorem(whose proof may be found in ``Applications of Matrices to Economic Models and Social Science Relationships,'' by Ben Noble, Proceedings of the Summer Conference for College Teachers on Applied Mathematics, 1971, CUPM, Berkeley, California) in this direction is stated below.

\begin{theorem} \label{thm 3.12}
Let \(A\) be an \(n \X n\) input output matrix(So the columns of \(A\) sum to \(1\)) having the form
\[
    A = \begin{pmatrix} B & C \\ D & E \end{pmatrix}
\]
where \(D\) is a \(1 \X (n - 1)\) positive vector and \(C\) is an \((n - 1) \X 1\) positive vector.
(So \(B\) is \((n - 1) \X (n - 1)\) square and \(E\) is a single entry.)
Then \((I - A)x = 0\) has a \emph{one-dimensional} solution set that is \emph{generated by a nonnegative vector}.
\end{theorem}

Observe that any input-output matrix with all positive entries satisfies the hypothesis of this theorem. The following matrix does also:
\[
    \begin{pmatrix}
        0.75 & 0.50 & 0.65 \\
        0    & 0.25 & 0.35 \\
        0.25 & 0.25 & 0
    \end{pmatrix}.
\]

\TODOREF{}
\begin{note}
Currently the open model below does not make any sense to me.
I will go back when finishing \CH{5}.
\end{note}

In the \textbf{open model}, we assume that there is an \emph{outside demand} for each of the commodities produced.
Returning to our simple society, let \(x_1, x_2\), and \(x_3\) be the monetary values of food, clothing, and housing \emph{produced with respective outside demands} \(d_1, d_2\), and \(d_3\).
Let \(A\) be the \(3 \X 3\) matrix such that \(A_{ij}\) represents the amount (in a fixed monetary unit such as the dollar) of commodity \(i\) required to produce one monetary unit of commodity \(j\).
Then the value of the surplus of \emph{food} in the society is
\[
    x_1 - (A_{11} x_1 + A_{12} x_2 + A_{13} x_3),
\]
that is, the value of food produced minus the value of food consumed while producing the three commodities.

\begin{note}
這邊\ \(A_{ij}\) 的意思是,生產價值一元的\ \(j\) 商品要耗掉多少元的\ \(i\) 商品。
然後\ open model 的結果是,外部對每種商品的需求等於該商品的剩餘價值。
\end{note}

The assumption that everything produced is consumed gives us a similar equilibrium condition for the open model, namely,
that \emph{the surplus of each of the three commodities must equal the corresponding outside demands}.
Hence
\[
    x_i - \sum_{j = 1}^3 A_{ij} x_j = d_i \text{ for } i = 1, 2, 3.
\]
In general, we must find a \emph{nonnegative} solution to \((I - A)x = d\), where \(A\) is a matrix \emph{with nonnegative entries} such that the sum of the entries of each column of \(A\) \emph{does not exceed one}, and \(d \ge 0\).
It is easy to see that if \((I - A)^{-1}\) exists and is nonnegative, then the desired solution is \((I - A)^{-1}d\).

Recall that for a real number \(a\), (by Calculus,) the series \(1 + a + a^2 + ...\) \emph{converges} to \((1 - a)^{-1}\) if \(\abs{a} < 1\).
Similarly, it can be shown (using the concept of convergence of matrices developed in \SEC{5.3}) that the series \(I + A + A^2 + ...\) converges to \((I - A)^{-1}\) if \(A^n\) converges to the zero matrix.
In this case, \((I - A)^{-1}\) is nonnegative since the matrices \(I, A, A^2, ...\) are nonnegative.

To illustrate the open model, suppose that \(30\) cents worth of food, \(10\) cents worth of clothing, and \(30\) cents worth of housing are required for the production of \$\(1\) worth of food.
Similarly, suppose that \(20\) cents worth of food, \(40\) cents worth of clothing, and \(20\) cents worth of housing are required for the production of \$\(1\) of clothing.
Finally, suppose that \(30\) cents worth of food, \(10\) cents worth of clothing, and \(30\) cents worth of housing are required
for the production of \$\(1\) worth of housing. 
Then the input-output matrix is
\[
    A = \begin{pmatrix}
        0.30 & 0.20 & 0.30 \\
        0.10 & 0.40 & 0.10 \\
        0.30 & 0.20 & 0.30
    \end{pmatrix}.
\]
so
\[
    I - A = \left(\begin{array}{rrr}
        0.70 & -0.20 & -0.30 \\
        -0.10 & 0.60 & -0.10 \\
        -0.30 & -0.20 & 0.70
    \end{array}\right)
    \text { and }
    (I - A)^{-1} = \left(\begin{array}{lll}
        2.0 & 1.0 & 1.0 \\
        0.5 & 2.0 & 0.5 \\
        1.0 & 1.0 & 2.0
    \end{array}\right).
\]

Since \((I - A)^{-1}\) is nonnegative, we can find a (unique) nonnegative solution to \((I - A)x = d\) for any demand \(d\).
For example, suppose that there are outside demands for \$\(30\) billion in food, \$\(20\) billion in clothing, and \$\(10\) billion in housing.
If we set
\[
    d = \begin{pmatrix} 30 \\ 20 \\ 10 \end{pmatrix}.
\]
then
\[
    x = (I - A)^{-1}d = \begin{pmatrix} 90 \\ 60 \\ 70 \end{pmatrix}.
\]
So a gross production of \$\(90\) billion of food, \$\(60\) billion of clothing, and \$\(70\) billion of housing is necessary to meet the required demands.

\exercisesection

\begin{exercise} \label{exercise 3.3.1}
Label the following statements as true or false.
\begin{enumerate}
\item Any system of linear equations has at least one solution.
\item Any system of linear equations has at most one solution.
\item Any \emph{homogeneous} system of linear equations has at least one solution.
\item Any system of \(n\) linear equations in \(n\) unknowns has at most one solution.
\item Any system of \(n\) linear equations in \(n\) unknowns has at least one solution.
\item If the homogeneous system corresponding to a given system of linear equations has a solution, then the given system has a solution.
\item If the coefficient matrix of a homogeneous system of \(n\) linear equations in \(n\) unknowns is invertible, then the system has no nonzero solutions.
\item The solution set of any system of \(m\) linear equations in \(n\) unknowns is a subspace of \(F^n\).
\end{enumerate}
\end{exercise}

\begin{proof} \ 
(Counterexamples for (a), (b) can be found in this section.)
\begin{enumerate}
\item False.
\item False.
\item True, the zero vector is the trivial solution.
\item False. Counterexample: \(\begin{pmatrix} 1 & 1 \\ 2 & 2 \end{pmatrix}\begin{pmatrix} x_1 \\ x_2 \end{pmatrix} = \begin{pmatrix} 0 \\ 0 \end{pmatrix}\) has the solution of the form \(\begin{pmatrix} -t \\ t \end{pmatrix}\) for any \(t \in \SET{R}\).
    More generally, when \(\NULL(\LMTRAN_A) > 0\) where \(A\) is the coefficient matrix of the system, the system has multiple solutions.
\item False. Counterexample: \(\begin{pmatrix} 1 & 1 \\ 1 & 1 \end{pmatrix}\begin{pmatrix} x_1 \\ x_2 \end{pmatrix} = \begin{pmatrix} 1 \\ 2 \end{pmatrix}\) has no solutions.
\item False. The system \(0x = 1\) has no solution but the corresponding homogeneous system \(0x = 0\) has a lot of solutions.
\item True. By \THM{3.10} the system has exactly one solution, but we know zero vector is the trivial solution, hence zero vector is that unique solution, hence the system has no nonzero solutions.
\item False. Counterexample is easy to find.
    In particular, by \THM{3.8} if the system is \emph{homogeneous}, the solution set of any system is a subspace of \(F^n\).
\end{enumerate}
\end{proof}

\begin{exercise} \label{exercise 3.3.2}
For each of the following homogeneous systems of linear equations, find the dimension of and a basis for the solution set.

\[
    (a) \sysdelim..\systeme{x_1 + 3 x_2 = 0, 2 x_1 + 6 x_2 = 0},
    (b) \sysdelim..\systeme{x_1 + x_2 - x_3 = 0, 4 x_1 + x_2 - 2 x_3 = 0},
    (c) \sysdelim..\systeme{x_1 + 2 x_2 - x_3 = 0, 2 x_1 + x_2 + x_3 = 0}
\]

\[
    (d) \sysdelim..\systeme{2 x_1 + x_2 - x_3 = 0, x_1 - x_2 + x_3 = 0, x_1 + 2 x_2 - 2 x_3 = 0},
    (e) \sysdelim..\systeme{x_1 + 2 x_2 - 3 x_3 + x_4 = 0},
    (f) \sysdelim..\systeme{x_1 + 2 x_2 = 0, x_1 - x_2 = 0}
\]
\[
    (g) \sysdelim..\systeme{x_1 + 2 x_2 + x_3 + x_4 = 0, x_2 - x_3 + x_4 = 0}
\]
\end{exercise}

\begin{proof} Calculation problem, skip. \end{proof}

\begin{exercise} \label{exercise 3.3.3}
Using the results of previous exercise, find all solutions to the following systems.
\[
    (a) \sysdelim..\systeme{x_1 + 3 x_2 = 5, 2 x_1 + 6 x_2 = 10},
    (b) \sysdelim..\systeme{x_1 + x_2 - x_3 = 1, 4 x_1 + x_2 - 2 x_3 = 3},
    (c) \sysdelim..\systeme{x_1 + 2 x_2 - x_3 = 3, 2 x_1 + x_2 + x_3 = 6}
\]

\[
    (d) \sysdelim..\systeme{2 x_1 + x_2 - x_3 = 5, x_1 - x_2 + x_3 = 1, x_1 + 2 x_2 - 2 x_3 = 4},
    (e) \sysdelim..\systeme{x_1 + 2 x_2 - 3 x_3 + x_4 = 1},
    (f) \sysdelim..\systeme{x_1 + 2 x_2 = 5, x_1 - x_2 = -1}
\]
\[
    (g) \sysdelim..\systeme{x_1 + 2 x_2 + x_3 + x_4 = 1, x_2 - x_3 + x_4 = 1}
\]
\end{exercise}

\begin{proof} Calculation problem, skip. \end{proof}

\begin{exercise} \label{exercise 3.3.4}
For each system of linear equations with the \emph{invertible} coefficient matrix \(A\),
\begin{enumerate}
\item[(1)] Compute \(A^{-1}\).
\item[(2)] Use \(A^{-1}\) to solve the system.

\[
    (a) \sysdelim..\systeme{x_1 + 3 x_2 = 4, 2 x_1 + 5 x_2 = 3},
    (b) \sysdelim..\systeme{x_1 + x_2 - x_3 = 5, x_1 + x_2 + x_3 = 1, 2 x_1 - 2 x_2 + x_3 = 4}
\]
\end{enumerate}
\end{exercise}

\begin{proof} Calculation problem, skip. \end{proof}

\begin{exercise} \label{exercise 3.3.5}
Give an example of a system of \(n\) linear equations in \(n\) unknowns with \emph{infinitely many} solutions.
\end{exercise}

\begin{proof}
See \EXEC{3.3.1}(d).
\end{proof}

\begin{exercise} \label{exercise 3.3.6}
Let \(\T: \SET{R}^3 \to \SET{R}^2\) be defined by \(\T(a, b, c) = (a + b, 2a - c)\).
Determine \(\T^{-1}(1, 11)\).
\end{exercise}

\begin{note}
Precisely, \(\T^{-1}(1, 11)\) is really \(\T^{-1}(\{ (1, 11) \})\), the set \(\{x \in \SET{R}^3: \T(x) \in \{ (1, 11) \}\).
In this context, the symbol \(\T^{-1}\) \emph{alone} dose not have any meaning since \(\T\) has no inverse.
\end{note}

\begin{proof}
The problem is equivalent to solve the system
\[
    \sysdelim..\systeme{
        a + b = 1,
        2a - c = 11
    }.
\]
But if we let \(a\) be a free variable, then we have \(b = 1 - a\) and \(c = 2a - 11\).
So the solution set is
\[
    \left\{ (a, 1 - a, 2a - 11) : a \in \SET{R} \right\}.
\]
\end{proof}

\begin{exercise} \label{exercise 3.3.7}
Calculation problem, skip.
But we can use \THM{3.11} to determine whether each system in this problem has a solution.
\end{exercise}

\begin{exercise} \label{exercise 3.3.8}
Let \(\T: \SET{R}^3 \to \SET{R}^3\) be defined by \(\T(a, b, c) = (a + b, b - 2c, a + 2c)\).
For each vector \(v \in \SET{R}^3\), determine whether \(v \in \RANGET\).

(a) \(v = (1, 3, -2)\) (b) \(v = (2, 1, 1)\).
\end{exercise}

\begin{proof}
Calculation problem, skip.
\end{proof}

\begin{exercise} \label{exercise 3.3.9}
Prove that the system of linear equations \(Ax = b\) has a solution if and only if \(b \in \RANGE(\LMTRAN_A)\).
\end{exercise}

\begin{proof}
\(Ax = b\) has a solution \(s\), if and only if (by def of left multiplication) we can find \(s\) in the domain of \(\LMTRAN_A\) s.t. \(\LMTRAN_A(s) = b\), if and only if (by def of range) \(b \in \RANGE(\LMTRAN_A)\).
\end{proof}

\begin{exercise} \label{exercise 3.3.10}
Prove or give a counterexample to the following statement:
If the coefficient matrix of a system of \(m\) linear equations in \(n\) unknowns has rank \(m\), then the system has a solution.
\end{exercise}

\begin{proof}
By the similar argument in \ATHM{3.11}(or \EXEC{3.2.19}), given any system of \(m\) equations and \(n\) unknowns s.t. the coefficient matrix \(A\) has rank \(m\), we have \(\LMTRAN_A : F^n \to F^m\) is onto.
And since \(\LMTRAN_A\) is onto, given \(b \in F^m\) we can find \(x \in F^n\) s.t. \(\LMTRAN_A(x) = b\).
That is, \(Ax = b\), hence the system has a solution.
\end{proof}

\TODOREF{} Skip exercise 11 to exercise 14.
Complete these exercises when finishing \CH{5}.

\begin{additional theorem} \label{athm 3.13}
This is the placeholder theorem for \EXEC{3.3.9}: The system of linear equations \(Ax = b\) has a solution if and only if \(b \in \RANGE(\LMTRAN_A)\).
\end{additional theorem}

\begin{additional theorem} \label{athm 3.14}
This is the placeholder theorem for \EXEC{3.3.10}:
If the coefficient matrix of a system of \(m\) linear equations in \(n\) unknowns has rank \(m\), then the system has a solution.
\end{additional theorem}

\begin{additional theorem} \label{athm 3.15}
This is the placeholder theorem for exercise 11 to exercise 14, which are not done yet.
\end{additional theorem}

\section{Systems of Linear Equations -- Computational Aspects} \label{sec 3.4}

In \SEC{3.3}. we obtained a necessary and sufficient condition for a system of linear equations to have solutions (\THM{3.11}) and learned how to express the solutions to a \textbf{non}homogeneous system in terms of solutions to the corresponding homogeneous system (\THM{3.9}).
The latter result enables us to determine \textbf{all} the solutions to a given system if we can \textbf{find one solution} to the given system and \textbf{a basis} for the solution set of the corresponding homogeneous system.

In this section, we use e.\RED{r}.o.s to \emph{accomplish these two objectives simultaneously}.
(That is, checking consistency and expressing solutions.)
The essence of this technique is to \textbf{transform} a given system of linear equations \textbf{into a system having the same solutions}, but which is \emph{easier to solve} (as in \SEC{1.4}).

\begin{definition} \label{def 3.6}
Two systems of linear equations are called \textbf{equivalent} if they have the \emph{same solution set}.
\end{definition}

The following theorem and corollary give a useful method for \emph{obtaining equivalent systems}.

\begin{theorem} \label{thm 3.13}
Let \(Ax = b\) be a system of \(m\) linear equations in \(n\) unknowns, and let \(C\) be an \textbf{invertible} \(m \X m\) matrix.
Then the system \((CA)x = Cb\) is equivalent to \(Ax = b\).
\end{theorem}

\begin{proof}
Let \(K\) be the solution set for \(Ax = b\) and \(K'\) the solution set for \((CA)x = Cb\).
If \(w \in K\), then
\begin{align*}
             & Aw = b \\
    \implies & C(Aw) = Cb \\
    \implies & (CA)w = Cb, & \text{by \THM{2.16}}
\end{align*}
and hence \(w \in K'\).
Thus \(K \subseteq K'\).

Conversely, if \(w \in K'\), then \((CA)w = Cb\).
Hence \(Aw = C^{-1}(CAw) = C^{-1}(Cb) = b\);
so \(w \in K\).
Thus \(K' \subseteq K\), and therefore, \(K = K'\).
\end{proof}

\begin{corollary} \label{corollary 3.13.1}
Let \(Ax = b\) be a system of m linear equations in \(n\) unknowns.
IE \((A'|b')\) is obtained from \((A | b)\) by a finite number of \emph{elementary \textbf{row} operations}, then the system \(A'x = b'\) is equivalent to the original system.
\end{corollary}

\begin{note}
Note that elementary \textbf{column} operations may change the solution set.
Counterexample is easy to find.
\end{note}

\begin{proof}
Suppose that \((A'|b')\) is obtained from \((A|b)\) by elementary row
operations.
(By \THM{3.3}) These may be executed by multiplying \((A|b)\) by elementary \(m \X m\) matrices \(E_1, E_2, ..., E_p\).
Let \(C = E_p ... E_2 E_1\);
then
\begin{align*}
    (A' | b') & = C(A | b) \\
              & = (CA | Cb). & \text{by \ATHM{3.7}}
\end{align*}
Since each \(E_i\) is invertible, so is \(C\). Now \(A' = CA\) and \(b' = Cb\).
Thus by \THM{3.13}, the system \(A'x = b'\) is equivalent to the system \(Ax = b\).
\end{proof}

\begin{remark} \label{remark 3.4.1}
We now describe \emph{a method for solving any} system of linear equations.
Consider, for example, the system of linear equations
\[
    \sysdelim..\systeme{
        3 x_1 + 2 x_2 + 3 x_3 - 2 x_4 = 1,
          x_1 +   x_2 +   x_3         = 3,
          x_1 + 2 x_2 +   x_3 -   x_4 = 2
    }
\]
First, we form the augmented matrix
\[
    \left(\begin{array}{rrrr|r}
        3 & 2 & 3 & -2 & 1 \\
        1 & 1 & 1 & 0 & 3 \\
        1 & 2 & 1 & -1 & 2
    \end{array}\right).
\]
By using elementary row operations, we transform the augmented matrix into an \textbf{upper triangular} matrix in which \textbf{the first nonzero entry of each row is \(1\)},
and it occurs in a column \emph{to the right of} the first nonzero entry of each \emph{preceding} row.
(Recall that matrix \(A\) is upper triangular if \(A_{ij} = 0\) whenever \(i > j\).)

\begin{enumerate}
\item[1.] \emph{In the leftmost nonzero column, create a \(1\) in the first row}.
    In our example, we can accomplish this step by interchanging the first and third rows.
    The resulting matrix is
    \[
        \left(\begin{array}{rrrr|r}
            1 & 2 & 1 & -1 & 2 \\
            1 & 1 & 1 & 0 & 3 \\
            3 & 2 & 3 & -2 & 1
        \end{array}\right).
    \]
\item[2.] \emph{By means of type \(3\) row operations, use the first row to obtain zeros in the remaining positions of the leftmost nonzero column}.
    In our example, we must add \(-1\) times the first row to the second row and then add \(-3\) times the first row to the third row to obtain
    \[
        \left(\begin{array}{rrrr|r}
            1 & 2 & 1 & -1 & 2 \\
            \RED{0} & -1 & 0 & 1 & 1 \\
            \RED{0} & -4 & 0 & 1 & -5
        \end{array}\right).
    \]
\item[3.] \emph{Create a \(1\) in the next row in the \textbf{leftmost possible} column, without using previous row(s)}.
    In our example, the second column is the leftmost possible column, and we can obtain a \(1\) in the second row, second column by multiplying the second row by \(-1\).
    This operation produces
    \[
        \left(\begin{array}{rrrr|r}
            1 & 2 & 1 & -1 & 2 \\
            0 & \RED{1} & 0 & -1 & -1 \\
            0 & -4 & 0 & 1 & -5
        \end{array}\right).
    \]
\item[4.] \emph{Now use type 3 elementary row operations to obtain zeros below the \(1\) created in the preceding step}.
    In our example, we must add four times the second row to the third row.
    The resulting matrix is
    \[
        \left(\begin{array}{rrrr|r}
            1 & 2 & 1 & -1 & 2 \\
            0 & 1 & 0 & -1 & -1 \\
            0 & \RED{0} & 0 & -3 & -9
        \end{array}\right).
    \]
\item[5.] \emph{Repeat steps 3 and 4 on each succeeding row until no nonzero rows remain}.
    (This creates zeros below the first nonzero entry in each row.)
    In our example, this can be accomplished by multiplying the third row by \(-\frac1{3}\).
    This operation produces
    \[
        \left(\begin{array}{rrrr|r}
            1 & 2 & 1 & -1 & 2 \\
            0 & 1 & 0 & -1 & -1 \\
            0 & 0 & 0 & \RED{1} & 3
        \end{array}\right).
    \]
We have now obtained the desired (upper-triangular) matrix.
To complete the simplification of the augmented matrix, \emph{we must make the first nonzero entry in each row the only nonzero entry in its column}.
(This corresponds to eliminating certain unknowns from all but one of the equations.)
\item[6.] \emph{\textbf{Work upward}, beginning with the last nonzero row, and add multiples of each row to the rows above}.
    (This creates zeros \textbf{above} the first nonzero entry in each row.)
    In our example, the third row is the last nonzero row, and the first nonzero entry of this row lies in column \(4\).
    Hence we add the third row to the first and second rows to obtain zeros in row \(1\), column \(4\) and row \(2\), column \(4\).
    The resulting matrix is
    \[
        \left(\begin{array}{rrrr|r}
            1 & 2 & 1 & \RED{0} & 5 \\
            0 & 1 & 0 & \RED{0} & 2 \\
            0 & 0 & 0 & 1 & 3
        \end{array}\right).
    \]
\item[7.] \emph{Repeat the process described in step 6 for each preceding row until it is performed with the second row, at which time the reduction process is complete}.
    In our example, we must add \(-2\) times the second row to the first row in order to make the first row, second column entry become zero. This operation produces
    \[
        \left(\begin{array}{rrrr|r}
            1 & \RED{0} & 1 & 0 & 1 \\
            0 & 1 & 0 & 0 & 2 \\
            0 & 0 & 0 & 1 & 3
        \end{array}\right).
    \]
\end{enumerate}
We have now obtained the desired reduction of the augmented matrix.
This matrix corresponds to the system of linear equations
\[
    \sysdelim..\systeme{
        x_1 + x_3 = 1,
        x_2 = 2,
        x_4 = 3
    }.
\]
Recall that, by the \CORO{3.13.1}, this system is \emph{equivalent to the original} system.
But this system is easily solved.
Obviously \(x_2 = 2\) and \(x_4 = 3\).
Moreover, \(x_1\) and \(x_3\) can have any values provided their sum is \(1\).
Letting \(x_3 = t\), we then have \(x_1 = 1 - t\).
Thus an arbitrary solution to the original system has the form
\[
    \begin{pmatrix} 1 - t \\ 2 \\ t \\ 3 \end{pmatrix}
    = \begin{pmatrix} 1 \\ 2 \\ 0 \\ 3 \end{pmatrix}
    + t \begin{pmatrix} -1 \\ 0 \\ 1 \\ 0 \end{pmatrix}.
\]
Observe that
\[
    \left\{ \begin{pmatrix} -1 \\ 0 \\ 1 \\ 0 \end{pmatrix} \right\}
\]
is a basis for the solutions of \textbf{homogeneous} system of equations corresponding to the given system.
\end{remark}

In the preceding example we performed elementary \textbf{row} operations on the augmented matrix of the system until we obtained the augmented matrix of a system having properties 1, 2, and 3 in \RMK{1.4.1}.
Such a matrix has a special name.

\begin{definition} \label{def 3.7}
A matrix is said to be in \textbf{reduced row echelon form} if the following three conditions are satisfied.
\begin{enumerate}
\item Any row containing a nonzero entry precedes any row in which all the entries are zero (if any).
\item The first nonzero entry in each row is the only nonzero entry in its column.
\item The first nonzero entry in each row is \(1\) and it occurs in a column to the right of the first nonzero entry in the preceding row.
    (Or, its column position is greater than the column position of the preceding row's first nonzero entry.)
\end{enumerate}
\end{definition}

\begin{remark} \label{remark 3.4.2}
By the structure of the reduced row echelon form, it's of course that the rank of any matrix in reduced row echelon form is equal to \emph{the number of nonzero rows} of the matrix.
\end{remark}

\begin{note}
英文苦手(的話),可去看\href{https://www.wikiwand.com/zh-tw/\%E9\%98\%B6\%E6\%A2\%AF\%E5\%BD\%A2\%E7\%9F\%A9\%E9\%98\%B5}{中文的定義},雖然也是非常的拗口。

或者根據莊重:
\begin{enumerate}
\item[(1)] 所有全\ \(0\) 的列一定要在所有不為全\ \(0\) 的列下方。
\item[(2)] 每一列的第一個非\ \(0\) 元素,其對應的那一行的其它元素都為\ \(0\)。
\item[(3)] 任一列\ \(r\) 的第一個非\ \(0\) 元素必須是\ \(1\),而且它的行數會比前一列的第一個非\ \(0\) 元素的行數還要大。
\end{enumerate}
\end{note}

\begin{example} \label{example 3.4.1} \ 

\begin{enumerate}
\item 
The (last) matrix in step 7 of \RMK{3.4.1} is in reduced row echelon form.
Note that the first nonzero entry of each row is \(1\) and that the \emph{column} containing each such entry has all zeros otherwise.
Also note that each time we move downward to a new row, \emph{we must move to the right one or more columns} to find the first nonzero entry of the new row.

\item
The matrix
\[
    \begin{pmatrix} 1 & 1 & 0 \\ 0 & 1 & 0 \\ \RED{1} & 0 & 1 \end{pmatrix}
\]
is not in reduced row echelon form, because the first column, which contains the first nonzero entry in row \(1\), contains another nonzero entry.
Similarly, the matrix
\[
    \begin{pmatrix} 1 & 1 & 0 & 2 \\ \RED{1} & 0 & 0 & 1 \\ 0 & 0 & 1 & 1 \end{pmatrix}
\]
is not in reduced row echelon form, because the first nonzero entry of the second row is not to the right of the first nonzero entry of the first row.

Finally, the matrix
\[
    \begin{pmatrix} \RED{2} & 0 & 0 \\ 0 & 1 & 0 \end{pmatrix}
\]
is not in reduced row echelon form, because the first nonzero entry of the first row is not \(1\).
\end{enumerate}
\end{example}

\begin{remark} \label{remark 3.4.3}
It can be shown (see the \CORO{3.16.1}) that \textbf{the reduced row echelon form of a matrix is unique};
that is, if different sequences of elementary row operations are used to transform a matrix into matrices \(Q\) and \(Q'\) in reduced row echelon form, then \(Q = Q'\).
Thus, although there are many different sequences of elementary row operations that can be used to transform a given matrix into reduced row echelon form, they all produce the same result.

The procedure described in \RMK{3.4.1} for reducing an augmented matrix to reduced row echelon form is called \textbf{Gaussian elimination}.
It consists of two separate parts: \emph{forward pass} and \emph{backward pass}.

\begin{enumerate}
\item[1.]
In the \emph{forward pass} (steps 1 - 5), the augmented matrix is transformed into an \emph{upper triangular} matrix in which the first nonzero entry of each row is \(1\), and it occurs in a column to the right of the first nonzero entry of each preceding row.

\item[2.]
In the \emph{backward pass} or \emph{back-substitution} (steps 6 - 7), the upper triangular matrix is transformed into reduced row echelon form by making the first nonzero entry of each row \emph{the only} nonzero entry of its column.
\end{enumerate}
\end{remark}

\begin{remark} \label{remark 3.4.4}
Of all the methods for transforming a matrix into its reduced row echelon form, Gaussian elimination requires the fewest arithmetic operations.
(For large matrices, it requires approximately 50\% fewer operations than the Gauss-Jordan method, in which the matrix is transformed into reduced row echelon form by using the first nonzero entry in each row to make zero all other entries in its column.)
Because of this efficiency, Gaussian elimination is the preferred method when solving systems of linear equations \emph{on a computer}.

In this context, the Gaussian elimination procedure is usually \emph{modified in order to minimize \textbf{roundoff errors}}.
Since discussion of these techniques is inappropriate here, readers who are interested in such matters are referred to books on \textbf{numerical analysis}.
\end{remark}

\begin{remark} \label{remark 3.4.5}
When a matrix is in reduced row echelon form, the corresponding system of linear equations is easy to solve.
We present below a procedure for solving any system of linear equations for which the augmented matrix is in reduced row echelon form.

First, however, we note that \emph{every matrix} can be transformed into reduced row echelon form by Gaussian elimination.
In the forward pass, we satisfy conditions (a) and (c) in the definition of reduced row echelon form and thereby make zero all entries below the first nonzero entry in each row.
Then in the backward pass, we make zero all entries above the first nonzero entry in each row, thereby satisfying condition (b) in the
definition of reduced row echelon form.
\end{remark}

\begin{theorem} \label{thm 3.14}
Gaussian elimination transforms \emph{any matrix} into its reduced row echelon form.
\end{theorem}

\begin{proof}
See \RMK{3.4.5}.
\end{proof}

We now describe a method for solving a system in which the augmented matrix is in reduced row echelon form.
To illustrate this procedure, we consider the system
\[
    \sysdelim..\systeme{
        2 x_1 + 3 x_2 +   x_3 + 4 x_4 - 9 x_5 = 17,
          x_1 +   x_2 +   x_3 +   x_4 - 3 x_5 = 6,
          x_1 +   x_2 +   x_3 + 2 x_4 - 5 x_5 = 8,
        2 x_1 + 2 x_2 + 2 x_3 + 3 x_4 - 8 x_5 = 14
    }
\]
for which the augmented matrix is
\[
    \left(\begin{array}{rrrrr|r}
        2 & 3 & 1 & 4 & -9 & 17 \\
        1 & 1 & 1 & 1 & -3 & 6 \\
        1 & 1 & 1 & 2 & -5 & 8 \\
        2 & 2 & 2 & 3 & -8 & 14
    \end{array}\right).
\]
Applying Gaussian elimination to the augmented matrix of the system produces the following sequence of matrices.
\[
\begin{array}{l}
    \left(\begin{array}{rrrrr|r}
        2 & 3 & 1 & 4 & -9 & 17 \\
        1 & 1 & 1 & 1 & -3 & 6 \\
        1 & 1 & 1 & 2 & -5 & 8 \\
        2 & 2 & 2 & 3 & -8 & 14
    \end{array}\right) 
    \longrightarrow
    \left(\begin{array}{rrrrr|r}
        1 & 1 & 1 & 1 & -3 & 6 \\
        2 & 3 & 1 & 4 & -9 & 17 \\
        1 & 1 & 1 & 2 & -5 & 8 \\
        2 & 2 & 2 & 3 & -8 & 14
    \end{array}\right)
    \longrightarrow \\
    \left(\begin{array}{rrrrr|r}
        1 & 1 & 1 & 1 & -3 & 6 \\
        \RED{0} & 1 & -1 & 2 & -3 & 5 \\
        \RED{0} & 0 & 0 & 1 & -2 & 2 \\
        \RED{0} & 0 & 0 & 1 & -2 & 2
    \end{array}\right) 
    \longrightarrow
    \left(\begin{array}{rrrrr|r}
        1 & 1 & 1 & 1 & -3 & 6 \\
        0 & 1 & -1 & 2 & -3 & 5 \\
        0 & 0 & 0 & 1 & -2 & 2 \\
        0 & 0 & 0 & \RED{0} & 0 & 0
    \end{array}\right)
    \longrightarrow \\
    \left(\begin{array}{llllr|r}
        1 & 1 & 1 & \BLUE{0} & -1 & 4 \\
        0 & 1 & -1 & \BLUE{0} & 1 & 1 \\
        0 & 0 & 0 & 1 & -2 & 2 \\
        0 & 0 & 0 & 0 & 0 & 0
    \end{array}\right)
    \longrightarrow
    \left(\begin{array}{rrrrr|r}
        1 & \BLUE{0} & 2 & 0 & -2 & 3 \\
        0 & 1 & -1 & 0 & 1 & 1 \\
        0 & 0 & 0 & 1 & -2 & 2 \\
        0 & 0 & 0 & 0 & 0 & 0
    \end{array}\right)
\end{array}
\]
The system of linear equations corresponding to this last matrix is
\[
    \sysdelim..\systeme{
        x_1 + 2 x_3 - 2 x_5 = 3,
        x_2 - x_3 + x_5 = 1,
        x_4 - 2 x_5 = 2.
    }
\]
Notice that \emph{we have ignored the last row since it consists entirely of zeros}.

To solve a system for which the augmented matrix is in reduced row echelon form, \textbf{divide the variables into two sets}.
The first set consists of those variables that appear as \emph{leftmost variables} in one of the equations of the system (in this case the set is \(\{ x_1, x_2, x_4 \}\)).
The second set consists of all the remaining variables (in this case, \(\{ x_3, x_5 \}\)).
To each variable in the second set, assign a \emph{parametric} value \(t_1, t_2, ...\) (\(x_3 = t_1, x_5 = t_2\)),
and then solve for the variables of the first set in terms of those in the second set:
\begin{align*}
    x_{1} & = -2 x_{3}+2 x_{5}+3 = -2 t_{1}+2 t_{2}+3 \\
    x_{2} & = x_{3}-x_{5}+1= t_{1}-t_{2}+1 \\
    x_{4} & = 2 x_{5}+2= 2 t_{2}+2 .
\end{align*}

Thus an arbitrary solution is of the form
\[
    \left(\begin{array}{l} x_{1} \\ x_{2} \\ x_{3} \\ x_{4} \\ x_{5} \end{array}\right)
    = \left(\begin{array}{c}
        -2 t_{1}+2 t_{2}+3 \\
        t_{1}-t_{2}+1 \\
        t_{1} \\
        2 t_{2}+2 \\
        t_{2}
    \end{array}\right)
    = \left(\begin{array}{l} 3 \\ 1 \\ 0 \\ 2 \\ 0 \end{array}\right)
    + t_{1} \left(\begin{array}{r} -2 \\ 1 \\ 1 \\ 0 \\ 0 \end{array}\right)
    + t_{2}\left(\begin{array}{r} 2 \\ -1 \\ 0 \\ 2 \\ 1 \end{array}\right),
\]
where \(t_1, t_2 \in \SET{R}\).
Notice that
\[
    \left\{
    \left(\begin{array}{r} -2 \\ 1 \\ 1 \\ 0 \\ 0 \end{array}\right),
    \left(\begin{array}{r} 2 \\ -1 \\ 0 \\ 2 \\ 1 \end{array}\right)
    \right\}
\]
is a basis for the solution set of the \emph{corresponding homogeneous system} of equations and
\[
    \left(\begin{array}{l} 3 \\ 1 \\ 0 \\ 2 \\ 0 \end{array}\right)
\]
is a particular solution to the original system.

Therefore, in simplifying the augmented matrix of the system to reduced row echelon form, \textbf{we are in effect simultaneously finding} a particular solution to the original system \textbf{and} a basis for the solution set of the associated homogeneous system.
Moreover, this procedure \textbf{detects when} a system \textbf{is inconsistent}, for by \EXEC{3.4.3},
solutions exist if and only if, in the reduction of the augmented matrix to reduced row echelon form, we \emph{do not obtain a row in which the only nonzero entry lies in the last column}.

Thus to use this procedure for solving a system \(Ax = b\) of \(m\) linear equations in \(n\) unknowns, we need only begin to transform the augmented matrix \((A|b)\) into its reduced row echelon form \((A'|b')\) by means of Gaussian elimination.
If a row is obtained in which the only nonzero entry lies in the last column, then the original system is inconsistent.
Otherwise, discard any zero rows from \((A'|b')\), and write the corresponding system of equations.
Solve this system as described above to obtain an arbitrary solution of the form
\[
    s = s_0 + t_1 u_1 + t_2 u_2 + ... + t_{n - r} u_{n - r},
\]
where \(r\) is the \emph{number of nonzero rows} in \(A'\) (\(r \le m\)).
The preceding equation is called a \textbf{general solution} of the system \(Ax = b\).
It \emph{expresses an arbitrary solution \(s\) of \(Ax = b\) in terms of \(n - r\) parameters}.
(\(t_1, t_2, ..., t_{n - r}\).)

The following theorem states that \(s\) \emph{cannot} be expressed in fewer than \(n - r\) parameters.

\begin{theorem} \label{thm 3.15}
Let \(Ax = b\) be a system of \(r\) nonzero equations in \(n\)
unknowns.
Suppose that \(\rank(A) = \rank(A|b)\) and that \((A|b)\) is in reduced row echelon form.
Then
\begin{enumerate}
\item \(\rank(A) = r\).
\item If the general solution obtained by the procedure above is of the form
\[
    s = s_0 + t_1 u_1 + t_2 u_2 + ... + t_{n - r} u_{n - r},
\]
then \(\{ u_1, u_2, ..., u_{n-r} \}\) \emph{is a basis} for the solution set of the \emph{corresponding homogeneous} system,
and \(s_0\) is a solution to the original system.
(Hence the general solution must be expressed in at least \(n - r\) parameters.)
\end{enumerate}
\end{theorem}

\begin{proof}
Since \(Ax = b\) has \(r\) nonzero equations, \(A\) has \(r\) nonzero rows, hence of course \((A|b)\) also has \(r\) nonzero rows.
And since \((A|b)\) is in reduced row echelon form, clearly these rows are \emph{\LID{}} by the definition of reduced row echelon form, so \(\rank(A|b) = r\).
And by supposition, \(\rank(A|b) = \rank(A)\), so \(\rank(A) = r\).

Let \(K\) be the solution set for \(Ax = b\), and let \(\mathrm{K_H}\) be the solution set for \(Ax = 0\).
We see from the form of the general solution that \(s = s_0 \in K\) if we set \(t_1 = t_2 = ... = t_{n - r} = 0\).
But by \THM{3.9}, \(K = \{ s_0 \} + \mathrm{K_H}\).
Hence
\[
    \mathrm{K_H} = \{ -s_0 \} + K = \spann(\{ u_1, u_2, ..., u_{n-r}\}).
\]
Why? Because by definition
\begin{align*}
    \{ -s_0 \} + K & = \{ -s_0 + k : k \in K \}) \\
                   & = \{ -s_0 + k : k \in \{ s_0 \} + \mathrm{K_H} \}) \\
                   & = \{ -s_0 + k : k \in \{ s_0 + k' : k' \in \mathrm{K_H} \} \}) \\
                   & = \{ -s_0 + (s_0 + k') : k' \in \mathrm{K_H} \}) \\
                   & = \{ k' : k' \in \mathrm{K_H} \}) \\
                   & = \spann(\{ u_1, u_2, ..., u_{n-r}\}) & \text{of course by the form of general solution}
\end{align*}

And we have
\begin{align*}
    \dim(\mathrm{K_H}) & = \dim(\NULL(\LMTRAN_A)) & \text{by \THM{3.8}} \\
                       & = n - \rank(\LMTRAN_A) & \text{by \THM{2.3}} \\
                       & = n - \rank(A) & \text{by \DEF{3.3}} \\
                       & = n - r & \text{by part(a)}
\end{align*}
So since \(\dim(\mathrm{K_H}) = n - r\) and \(\mathrm{K_H}\) is generated by a set \(\{ u_1, u_2, ..., u_{n - r} \}\) containing \emph{at most} \(n - r\) vectors, (by \CORO{1.10.3}(a)) we conclude that this set is a basis for \(\mathrm{K_H}\).
\end{proof}

\subsection{An Interpretation of the Reduced Row Echelon Form}
Let \(A\) be an \(m \X n\) matrix with columns \(a_1, a_2, ..., a_n\), and let \(B\) be the reduced row echelon form of \(A\).
Denote the columns of \(B\) by \(b_1, b_2, ..., b_n\).
If the rank of \(A\) is \(r\), then the rank of \(B\) is also \(r\) by the \CORO{3.4.1}.
Because \(B\) is in reduced row echelon form, (from the structure of the form,) no nonzero row of \(B\) can be a linear combination of the other rows of \(B\).
Hence \(B\) must have exactly \(r\) nonzero rows.
And if \(r \ge 1\), (then by the structure of r.r.e.f. again,) the vectors \(e_1, e_2, ..., e_r\), the first \(r\) vectors of the standard ordered basis of \(F^n\), must occur among the \textbf{columns} of \(B\).

Now for \(i = 1, 2, ..., r\), let \(j_i\) denote a \emph{column number} of \(B\) such that \(b_{j_i} = e_i\).
We \textbf{claim that} \(a_{j_1}, a_{j_2}, ..., a_{j_r}\), the columns of \(A\) corresponding to these columns of \(B\), are also \LID{}.

For suppose that there are scalars \(c_1, c_2, ..., c_r\) such that
\[
    c_1 a_{j_1} + c_2 a_{j_2} + ... + c_r a_{j_r} = 0,
\]
Because \(B\) can be obtained from \(A\) by a sequence of elementary \emph{row} operations, there exists an invertible \(m \X m\) matrix \(M = E_p E_{p - 1} ... E_2 E_1\) where \(E_i\) represents elementary row operations, such that \(MA = B\).
Multiplying the preceding equation by \(M\) yields
\[
    c_1 M a_{j_1} + c_2 M a_{j_2} + ... + c_r M a_{j_r} = 0.
\]
By \THM{2.13}(a), The \(j_i\)th column of \(MA\) is equal to \(M a_{j_i}\), but since \(MA = B\), the \(j_i\)th column of \(B\) is equal to \(M a_{j_i}\).
That is, \(b_{j_i} = M a_{j_i}\) \MAROON{(1)}.
But \(b_{j_i} = e_i\), we have \(M a_{j_i} = e_i\).
It follows that
\[
    c_1 e_1 + c_2 e_2 + ... + c_r e_r = 0.
\]
Hence \(c_1 = c_2 = ... = c_r = 0\), proving that the vectors \(a_{j_1}, a_{j_2}, ..., a_{j_r}\) are \LID{}.

Because \(B\) has only \(r\) nonzero rows, every column of \(B\) has the form
\[
    \begin{pmatrix} d_1 \\ d_2 \\ \vdots \\ d_r \\ \RED{0} \\ \RED{\vdots} \\ \RED{0} \end{pmatrix}
\]
for scalars \(d_1, d_2, ..., d_r\).
And from the equation \MAROON{(1)}, we have \(b_i = M a_i\), hence \(a_i = M^{-1} b_i\) \MAROON{(2)}.
But from the form of column of \(B\) above, \(a_i = M^{-1} b_i = M^{-1} (d_1 e_1 + d_2 e_2 + ... + d_r e_r)\).
Hence \textbf{the corresponding column of \(A\) must be}
\begin{align*}
    M^{-1} (d_1 e_1 + d_2 e_2 + ... + d_r e_r)
    & = d_1 M^{-1} e_1 + d_2 M^{-1} e_2 + ... + d_r M^{-1} e_r \\
    & = d_1 M^{-1} b_{j_1} + d_2 M^{-1} b_{j_2} + ... + d_r M^{-1} b_{j_r} \\
    & = d_1 a_{j_1} + d_2 a_{j_2} + ... + d_r a_{j_r} & \text{by \MAROON{(2)}}
\end{align*}

The next theorem summarizes these results.

\begin{theorem} \label{thm 3.16}
Let \(A\) be an \(m \X n\) matrix of rank \(r\), where \(r > 0\), and let \(B\) be the reduced row echelon form of \(A\).
Then
\begin{enumerate}
\item The number of nonzero rows in \(B\) is \(r\).
\item For each \(i = 1, 2, ... , r\), there is a column \(b_{j_i}\), of \(B\) such that \(b_{j_i} = e_i\)
\item The columns of \(A\) numbered \(j_1, j_r, ..., j_r\) are \LID{}.
\item For each \(k = 1, 2, ... , n\), if column \(k\) of \(B\) is \(d_1 e_1 + d_2 e_2 + ... + d_r e_r\), then column \(k\) of \(A\) is \(d_1 a_{j_1} + d_2 a_{j_2} + ... + d_r a_{j_r}\).
\end{enumerate}

The proof is described in the (long) previous discussion.
BTW, for part(d), there is a related exercise: \EXEC{2.3.15}.
\end{theorem}

\begin{corollary} \label{corollary 3.16.1}
The reduced row echelon form of a matrix is \textbf{unique}.
\end{corollary}

\begin{proof}
See \EXEC{3.4.15}.
\end{proof}

\begin{example} \label{example 3.4.2}
Let
\[
    A = \left(\begin{array}{ccccc}
        2 & 4 & 6 & 2 & 4 \\
        1 & 2 & 3 & 1 & 1 \\
        2 & 4 & 8 & 0 & 0 \\
        3 & 6 & 7 & 5 & 9
    \end{array}\right)
\]

The reduced row echelon form of \(A\) is
\[
    B=\left(\begin{array}{rrrrr}
        \RED{1} & 2 & 0 & 4 & 0 \\
        0 & 0 & \RED{1} & -1 & 0 \\
        0 & 0 & 0 & 0 & \RED{1} \\
        0 & 0 & 0 & 0 & 0
    \end{array}\right)
\]
Since \(B\) has three nonzero rows, (by \THM{3.16}(a)) the rank of \(A\) is \(3\).
The first, third, and fifth columns of \(B\) are \(e_1, e_2\), and \(e_3\);
so \THM{3.16}(c) \emph{asserts that the first, third, and fifth columns} of \(A\) are \LID{}.

Let the columns of \(A\) be denoted \(a_1, a_2, a_3, a_4\), and \(a_5\).
Because the \RED{second} column of \(B\) is \(2 e_{1}\), wehere \(e_{1} = b_{\BLUE{1}}\), it follows from \THM{3.16}(d) that \(a_{\RED{2}} = 2 a_{\BLUE{1}}\), as is easily checked.
Moreover, since the fourth column of \(B\) is \(4 e_1 + (-1) e_2\), where \(e_1 = b_{\GREEN{1}}\) and \(e_2 = b_{\BLUE{3}}\), the same result shows that
\[
    a_4 = 4 a_{\GREEN{1}} + (-1) a_{\BLUE{3}}.
\]
\end{example}

In \EXAMPLE{1.6.6}. we extracted a basis for \(\SET{R}^3\) from the \emph{generating} set
\[
    S = \{ (2, -3, 5), (8, -12, 20), (1, 0, -2), (0, 2, -1), (7, 2, 0) \}.
\]
The procedure described there (precisely, the procedure described in \THM{1.9}) can be \emph{streamlined} by using \THM{3.16}.
We begin by noting that if \(S\) were \LID{}, then S would be a basis for \(\SET{R}^3\).
In this case, it is clear that \(S\) is \LDP{} because
\(S\) contains more than \(\dim(\SET{R}^3) = 3\) vectors.
Nevertheless, it is \emph{instructive to consider the \textbf{calculation}} that is needed to determine whether \(S\) is \LDP{} or \LID{}.
Recall that \(S\) is linearly dependent if there are scalars \(c_1, c_2, c_3, c_4\), and \(c_5\), \emph{not all zero}, such that
\[
    c_1(2, -3, 5) + c_2(8, -12, 20) + c_3(1, 0, -2) + c_4(0, 2, -1) + c_5(7, 2, 0) = (0, 0, 0).
\]
Thus \(S\) is \LDP{} if and only if the system of linear equations
\[
    \sysdelim..\systeme{
         2 c_1 +  8 c_2 +   c_3         + 7 c_5 = 0,
        -3 c_1 - 12 c_2         + 2 c_4 + 2 c_5 = 0,
         5 c_1 + 20 c_2 - 2 c_3 -   c_4         = 0
    }
\]
has a \textbf{nonzero} solution.
The augmented matrix of this system of equations is
\[
    A=\left(\begin{array}{rrrrrr}
        2 & 8 & 1 & 0 & 7 & 0 \\
        -3 & -12 & 0 & 2 & 2 & 0 \\
        5 & 20 & -2 & -1 & 0 & 0
    \end{array}\right),
\]
and its reduced row echelon form is
\[
    B=\left(\begin{array}{llllll}
        1 & 4 & 0 & 0 & 2 & 0 \\
        0 & 0 & 1 & 0 & 3 & 0 \\
        0 & 0 & 0 & 1 & 4 & 0
    \end{array}\right).
\]
Using the technique described earlier in this section, that is, find the form of the general solution, we can find nonzero solutions of the preceding system, confirming that \(S\) is \LDP{}.
However, \THM{3.16}(c) gives us additional information.
Since the first, third, and fourth columns of \(B\) are \(e_1, e_2\), and \(e_3\), we conclude that the first, third, and fourth columns of \(A\) are \LID{}.
But the columns of \(A\) other than the last column (which is the zero vector) are vectors in \(S\).
Hence
\[
    \beta = \{ (2, -3, 5), (1, 0, -2), (0, 2, -1) \}
\]
is a \LID{} subset of \(S\).
If follows from \CORO{1.10.2}(b) that \(\beta\) is a basis for \(\SET{R}^3\).

\begin{note}
這邊這坨敘述的意思是,給定一個\ vector space 的\ generating set,我們可以做一個轉換把它變成一個\ system of linear equations,然後用高斯消去法加上\ \THM{3.16},就可以得到這個\ generating set 的一個\ (subset) basis。
\end{note}

Because every finite-dimensional vector space over \(F\) is isomorphic to \(F^n\) for some \(n\), a similar approach can be used to reduce any \emph{finite} generating set to a basis.
This technique is illustrated in the next example.

\begin{example} \label{example 3.4.3}
The set
\[
    S = \{
        2 + x + 2x^2 + 3x^3,
        4 + 2x + 4x^2 + 6x^3,
        6 + 3x + 8x^2 + 7x^3,
        2 + x + 5x^3,
        4 + x + 9x^3
    \}
\]
generates a subspace \(V\) of \(\mathcal{P}_{3}(\SET{R})\).
To find a subset of \(S\) that is a basis for \(V\), we consider the subset
\[
    S' = \{(2,1,2,3),(4,2,4,6),(6,3,8,7),(2,1,0,5),(4,1,0,9)\}
\]
consisting of the images of the polynomials in \(S\) \emph{under the standard representation} of \(\mathcal{P}_{3}(\SET{R})\) with respect to the standard ordered basis.
Note that the \(4 \X 5\) matrix in which the \emph{columns} are the vectors in \(S'\) is the matrix \(A\) in \EXAMPLE{3.4.2}.
From the reduced row echelon form of \(A\), which is the matrix \(B\) in \EXAMPLE{3.4.2},
we see that the first, third, and fifth columns of \(A\) are \LID{} and the second and fourth columns of  \(A\) are linear combinations of the first, third, and fifth columns.
Hence
\[
    \{(2,1,2,3),(6,3,8,7),(4,1,0,9)\}
\]
is a basis for the subspace of \(\SET{R}^4\) that is generated by \(S'\).
It follows that
\[
    \{
        2 + x + 2 x^2 + 3x^3,
        6 + 3x + 8x^2 + 7x^3,
        4 + x + 9x^3
    \}
\]
is a basis for the subspace \(V\) of \(\mathcal{P}_{3}(\SET{R})\).
\end{example}

We conclude this section by describing a method for \textbf{extending} a \LID{} subset \(S\) of a finite-dimensional vector space \(V\) to a basis for \(V\).
Recall that this is always possible by \CORO{1.10.2}(c).
Our approach is based on the replacement theorem and assumes that we can find an explicit basis \(\beta\) for \(V\).

Let \(S'\) be the ordered set consisting of the vectors in \(S\) \textbf{followed by those in} \(\beta\).
Since \(\beta \subseteq S'\), the set \(S'\) generates \(V\).
We can then apply the technique described above to reduce
this generating set to a basis for \(V\) containing \(S\).

\begin{remark} \label{remark 3.4.6}
Note that the order for putting \(S\) and \(\beta\) into \(S'\) is important,
because we prefer picking the vectors in \(S\) instead of those in \(\beta\).
If \(S'\) consists vectors in \(\beta\) followed by those in \(\S\), \textbf{then the reduced basis does not necessarily contain the whole \(S\)!}
\end{remark}

\begin{note}
這邊說要把一個線性獨立\ subset \(S\) extend 成一個\ basis 的目的是,我「就是要」一個有包含\ \(S\) 的\ basis。
(\TODOREF{} 這在實務上很常見,補一下例子。)
我們可以先造出一個有包含\ \(S\) 跟任一個\ basis \(\beta\) 的\ generating set,然後再用前面的方法把它\ reduce 成另一個(有包含 \(S\) 的) basis。
\end{note}

\begin{example} \label{example 3.4.4}
Let
\[
    V = \{
        (x_1, x_2, x_3, x_4, x_5) \in \SET{R}^5
        : x_1 + 7x_2 + 5x_3 - 4x_4 + 2x_5 = 0
    \}.
\]

It is easily verified that \(V\) is a subspace of \(\SET{R}^{5}\) with dimension \(4\) and that
\[
    S = \{(-2,0,0,-1,-1),(1,1,-2,-1,-1),(-5,1,0,1,1)\}
\]
is a \LID{} subset of \(V\).

To extend \(S\) to a basis for \(V\), we first obtain a basis  \(\beta\) for \(V\).
To do so, we solve the system of linear equations that defines \(V\).
Since in this case \(V\) is defined by a single equation, we need only write the equation as
\[
    x_1 = -7 x_2 - 5 x_3 + 4 x_4 - 2 x_5
\]
and assign parametric values to \(x_2, x_3, x_4\), and \(x_5\).
If \(x_2 = t_1, x_3 = t_2, x_4 = t_3\), and \(x_5 = t_4\), then the vectors in \(V\) have the form
\[
\begin{array}{l}
    (x_1, x_2, x_3, x_4, x_5) = (-7 t_1 - 5 t_2 + 4 t_3 - 2 t_4, t_1, t_2, t_3, t_4) \\
    = t_1(-7,1,0,0,0) + t_2(-5,0,1,0,0) + t_3(4,0,0,1,0) + t_4(-2,0,0,0,1)
\end{array}
\]
Hence
\[
    \beta = \{(-7,1,0,0,0),(-5,0,1,0,0),(4,0,0,1,0),(-2,0,0,0,1)\}
\]

is a basis for \(V\) by \THM{3.15}(b).

The matrix whose columns consist of the vectors in \(S\) \emph{followed by those in} \(\beta\) is
\[
    \left(\begin{array}{rrrrrrr}
        \RED{-2} & \RED{1}  & \RED{-5} & -7 & -5 & \BLUE{4} & -2 \\
        \RED{0}  & \RED{1}  & \RED{1}  &  1 &  0 & \BLUE{0} & 0 \\
        \RED{0}  & \RED{-2} & \RED{0}  &  0 &  1 & \BLUE{0} & 0 \\
        \RED{-1} & \RED{-1} & \RED{1}  &  0 &  0 & \BLUE{1} & 0 \\
        \RED{-1} & \RED{-1} & \RED{1}  &  0 &  0 & \BLUE{0} & 1
    \end{array}\right)
\]
and its reduced row echelon form is
\[
    \left(\begin{array}{llllrlr}
        \RED{1} & \RED{0} & \RED{0} & 1 &    1 & \BLUE{0} & -1 \\
        \RED{0} & \RED{1} & \RED{0} & 0 & -0.5 & \BLUE{0} & 0 \\
        \RED{0} & \RED{0} & \RED{1} & 1 &  0.5 & \BLUE{0} & 0 \\
        \RED{0} & \RED{0} & \RED{0} & 0 &    0 & \BLUE{1} & -1 \\
        \RED{0} & \RED{0} & \RED{0} & 0 &    0 & \BLUE{0} & 0
\end{array}\right).
\]
Thus (by \THM{3.16}(c))
\[
    \{(-2,0,0,-1,-1),(1,1,-2,-1,-1),(-5,1,0,1,1),(4,0,0,1,0)\}
\]
(is \LID{}, hence) is a basis for \(V\) containing \(S\).
\end{example}

\exercisesection

\begin{exercise} \label{exercise 3.4.1}
Label the following statements as true or false.
\begin{enumerate}
\item If \((A'|b')\) is obtained from \((A|b)\) by a finite sequence of elementary \emph{column} operations, then the systems \(Ax = b\) and \(A'x = b'\) are equivalent.
\item If \((A'|b')\) is obtained from \((A|b)\) by a finite sequence of e.r.o.s, then the systems \(Ax = b\) and \(A' x\) are equivalent.
\item If \(A\) is an \(n \X n\) matrix with rank \(n\), then the reduced row echelon form of \(A\) is \(I_n\).
\item Any matrix can be put in reduced row echelon form by means of a finite sequence of elementary row operations.
\item If \((A|b)\) is in reduced row echelon form, then the system \(Ax = b\) is consistent.
\item Let \(Ax = b\) be a system of \(m\) linear equations in \(n\) unknowns for which the augmented matrix is in reduced row echelon form.
If this system is consistent, then the dimension of the solution set of \(Ax = 0\) is \(n - r\), where \(r\) equals the number of nonzero rows in \(A\).
\item If a matrix \(A\) is transformed by elementary row operations into a matrix \(A'\) in reduced row echelon form, then the number of nonzero rows in \(A'\) equals the rank of \(A\).
\end{enumerate}
\end{exercise}

\begin{proof} \ 

\begin{enumerate}
\item False. Counterexample is easy to find.
\item True by \CORO{3.13.1}.
\item True. Let \(B\) be the r.r.e.f. of \(A\), by \THM{3.16}(b), we have that the \(i\)th column of is \(e_i\) for \(1 \le i \le \RED{n}\), which implies \(B = I_n\).
\item True by \THM{3.14}.
\item False by \EXEC{3.4.3}.
\item True.
    \(A\) has \(r\) nonzero rows, so \(Ax = b\) is a system of \(r\) nonzero equations of \(n\) unknowns.
    And \((A|b)\) is in r.r.e.f..
    Also \(Ax = b\) is consistent.
    So the requirements of \THM{3.15} are satisfied.
    So by \THM{3.15}(b), the dimension of the corresponding homogeneous system \(Ax = 0\) is \(n - r\), as desired.
\item True by \RMK{3.4.2} and \CORO{3.4.1}.
\end{enumerate}
\end{proof}

\begin{exercise} \label{exercise 3.4.2}
Calculation problem, skip.
\end{exercise}

\begin{exercise} \label{exercise 3.4.3}
Suppose that the augmented matrix of a system \(Ax = b\) is transformed into a matrix \((A'|b')\) in reduced row echelon form by a finite sequence of elementary row operations.
\begin{enumerate}
\item Prove that \(\rank(A') \ne \rank(A'|b')\) if and only if \((A'|b')\) contains a row in which the only nonzero entry lies in the last column.

\item Deduce that \(Ax = b\) is consistent if and only if \((A'|b')\) contains no row in which the only nonzero entry lies in the last column.
\end{enumerate}
\end{exercise}

\begin{proof}
WLOG, suppose \(A\) is of size \(m \X n\).
\begin{enumerate}
\item
\(\Longrightarrow\): For the sake of contradiction, suppose \(r = \rank(A') \ne \rank(A'|b')\), \RED{but} \((A'|b')\) does not contain a row in which the only nonzero entry lies in the last column.
Then (by \RMK{3.4.2}) since \(A'\) only contains \(r\) nonzero rows, \((A|b')\) must be in the form
\[
    (A|b') =
    \left(\begin{array}{rrr|r}
             A'_{11} &       ... &      A'_{1n} & b'_1 \\
              \vdots &           &       \vdots & \vdots \\
             A'_{r1} &       ... &      A'_{rn} & b'_r \\
             \RED{0} & \RED{...} &      \RED{0} & \RED{0} \\
        \RED{\vdots} &           & \RED{\vdots} & \RED{\vdots} \\
             \RED{0} & \RED{...} &      \RED{0} & \RED{0}
    \end{array}\right),
\]
that is, the \(r + 1, ..., m\) components of \(b'\) must be zero. (Otherwise \((A'|b')\) \emph{does} contain a row in which the only nonzero entry lies in the last column.)
But then since by \THM{3.16}(b) we know \(e_1, e_2, ..., e_r\) are columns of \(A'\), we have \(b' = b_1 e_1 + b_2 e_2 + ... + b_r e_r\), so \(b'\) is in the column space of \(A'\).
But that implies \(\rank(A') = \rank(A'|b')\), which is also a contradiction!
So \((A'|b')\) \textbf{must contain} a row in which the only nonzero entry lies in the last column.

\(\Longleftarrow\):
If \((A'|b')\) contains a row in which the only nonzero entry lies in the last column, then by the structure of r.r.e.f., the last column is of course \LID{} to the remaining columns of \((A'|b')\), that is, the last column is \LID{} to the columns of \(A'\).
Hence \(\rank(A') \ne \rank(A'|b')\).

\item \(Ax = b\) is consistent, if and only if (by \THM{3.13}) \(A'x = b'\) is consistent, if and only if (by part(a)) \((A'|b')\) does \emph{not} contain a row in which the only nonzero entry lies in the last column.
\end{enumerate}
\end{proof}

\begin{exercise} \label{exercise 3.4.4}
Calculation problem, skip.
\end{exercise}

\begin{exercise} \label{exercise 3.4.5}
Let the reduced row echelon form of \(A\) be
\[
    B = \left(\begin{array}{rrrrr}
        \RED{1} & \RED{0} & 2 & \RED{0} & -2 \\
        \RED{0} & \RED{1} & -5 & \RED{0} & -3 \\
        \RED{0} & \RED{0} & 0 & \RED{1} & 6
    \end{array}\right)
\]
Determine \(A\) if the first, second, and fourth columns of \(A\) are
\[
    \begin{pmatrix} 1 \\ -1 \\ 3 \end{pmatrix},
    \begin{pmatrix} 0 \\ -1 \\ 1 \end{pmatrix},
    \text { and }
    \begin{pmatrix} 1 \\ -2 \\ 0 \end{pmatrix}
\]
respectively.
\end{exercise}

\begin{proof}
Let \(a_i\) be the \(i\)th column of \(A\), and \(b_i\) be the \(i\)th column of \(B\).
Since \(b_3 = 2 e_1 + (-5) e_2 = 2 b_1 + (-5) b_2\), by \THM{3.16}(d), \(a_3 = 2 a_1 + (-5) a_2\).
So
\[
    a_3 = 2 \begin{pmatrix} 1 \\ -1 \\ 3 \end{pmatrix}
        + (-5) \begin{pmatrix} 0 \\ -1 \\ 1 \end{pmatrix}
        = \begin{pmatrix} 2 \\ 3 \\ 1 \end{pmatrix}.
\]
Similarly, since \(b_5 = (-2) e_1 + (-3) e_2 + 6 e_3 = (-2) b_1 + (-3) b_2 + 6 b_4\), by \THM{3.16}(d), \(a_5 = (-2) a_1 + (-3) a_2 + 6 a_4\).
So
\[
    a_5 = (-2) \begin{pmatrix} 1 \\ -1 \\ 3 \end{pmatrix}
        + (-3) \begin{pmatrix} 0 \\ -1 \\ 1 \end{pmatrix}
        + (6) \begin{pmatrix} 1 \\ -2 \\ 0 \end{pmatrix}
        = \begin{pmatrix} 4 \\ -7 \\ -9 \end{pmatrix}.
\]
Hence
\[
    A = \left(\begin{array}{ccccc}
        1 & 0 & 2 & 1 & 4 \\
        -1 & -1 & 3 & -2 & -7 \\
        3 & 1 & 1 & 0 & -9
    \end{array}\right)
\]
\end{proof}

\begin{exercise} \label{exercise 3.4.6}
Let the reduced row echelon form of \(A\) be
\[
    B = \left(\begin{array}{rrrrrr}
        \RED{1} & -3 & \RED{0} & 4 & \RED{0} & \GREEN{5} \\
        \RED{0} & 0 & \RED{1} & 3  & \RED{0} & \GREEN{2} \\
        \RED{0} & 0 & \RED{0} & 0  & \RED{1} & \GREEN{-1} \\
        \RED{0} & 0 & \RED{0} & 0  & \RED{0} & \GREEN{0}
    \end{array}\right)
\]
Determine \(A\) if the \RED{first}, \RED{third}, and \GREEN{six} columns of \(A\) are
\[
    \begin{pmatrix} 1 \\ -2 \\ -1 \\ 3 \end{pmatrix},
    \begin{pmatrix} -1 \\ 1 \\ 2 \\ -4 \end{pmatrix},
    \text { and }
    \begin{pmatrix} 3 \\ -9 \\ 2 \\ 5 \end{pmatrix},
\]
respectively.
\end{exercise}

\begin{proof}
This problem is a variation of \EXEC{3.4.5}.

Let \(a_i\) be the \(i\)th column of \(A\), and \(b_i\) be the \(i\)th column of \(B\).
We should solve \(\RED{a_5}\) first, since (by \THM{3.16}(c)) it is the corresponding \LID{} vector of \(b_5\).

Since \(b_6 = 5 e_1 + 2 e_2 + (-1) e_3 = 5 b_1 + 2 b_3 + (-1) b_5\), by \THM{3.16}(d), \(a_6 = 5 a_1 + 2 a_3 + (-1) a_5\).
That is, \(a_5 = 5 a_1 + 2 a_3 + (-1) a_6\).
So
\[
    a_5 = 5 \begin{pmatrix} 1 \\ -2 \\ -1 \\ 3 \end{pmatrix}
        + 2 \begin{pmatrix} -1 \\ 1 \\ 2 \\ -4 \end{pmatrix}
        + (-1) \begin{pmatrix} 3 \\ -9 \\ 2 \\ 5 \end{pmatrix}
        = \begin{pmatrix} 0 \\ 1 \\ -3 \\ 2 \end{pmatrix}.
\]
And Since \(b_2 = (-3) e_1 = (-3) b_1\), by \THM{3.16}(d), \(a_2 = (-3) a_1\).
So
\[
    a_2 = (-3) \begin{pmatrix} 1 \\ -2 \\ -1 \\ 3 \end{pmatrix}
        = \begin{pmatrix} -3 \\ 6 \\ 3 \\ -9 \end{pmatrix}.
\]
Finally, since \(b_4 = 4 e_1 + 3 e_2 = 4 b_1 + 3 b_3\), by \THM{3.16}(d), \(a_4 = 4 a_1 + 3 a_3\).
So
\[
    a_4 = 4 \begin{pmatrix} 1 \\ -2 \\ -1 \\ 3 \end{pmatrix}
        + 3 \begin{pmatrix} -1 \\ 1 \\ 2 \\ -4 \end{pmatrix}
        = \begin{pmatrix} 1 \\ -5 \\ 2 \\ 0 \end{pmatrix}.
\]
Hence
\[
    A = \left(\begin{array}{cccccc}
        1 & -3 & -1 & 1 & 0 & 3 \\
        -2 & 6 & 1 & -5 & 1 & -9 \\
        -1 & 3 & 2 & 2 & -3 & 2 \\
        3 & -9 & -4 & 0 & 2 & 5
    \end{array}\right).
\]
\end{proof}

\begin{exercise} \label{exercise 3.4.7}
\sloppy It can be shown that the vectors \(u_1 = (2, -3, 1), u_2 = (1, 4, -2), u_3 = (-8, 12, -4), u_4 = (1, 37, -17)\), and \(u_5 = (-3, -5, -8)\) generate \(\SET{R}^3\).
Find a subset of \(\{ u_1, u_2, u_3, u_4, u_5 \}\) that is a basis for \(\SET{R}^3\).
\end{exercise}

\begin{proof}
We mimic the procedure described after \EXAMPLE{3.4.2}.
We first find the nonzero solution \((c_1, c_2, c_3, c_4, c_5)\) s.t. \(c_1 u_1 + ... + c_5 u_5 = 0\).
That is,
\[
    \sysdelim..\systeme{
         2 c_1 + 1 c_2 -  8 c_3 +  1 c_4 - 3 c_5 = 0,
        -3 c_1 + 4 c_2 + 12 c_3 + 37 c_4 - 5 c_5 = 0,
         1 c_1 - 2 c_2 -  4 c_3 - 17 c_4 - 8 c_5 = 0
    }
\]
Then the corresponding r.r.e.f. system (by calculation) is,
\[
    \sysdelim..\systeme{
         1 c_1 + 0 c_2 - 4 c_3 - 3 c_4 -           11 c_5 = 0,
         0 c_1 + 1 c_2 + 0 c_3 + 7 c_4 - \frac{2}{19} c_5 = 0,
         0 c_1 + 0 c_2 + 0 c_3 + 0 c_4 +            1 c_5 = 0
    }
\]
\sloppy Then by \THM{3.16}(c), \(\{ u_1, u_2, u_5 \}\) is \LID{}.
And since \(\dim(\SET{R}^3 = 3\), \(\{ u_1, u_2, u_5 \}\) is a basis for \(\SET{R}^3\).
\end{proof}

\begin{exercise} \label{exercise 3.4.8}
Let \(W\) denote the subspace of \(\SET{R}^5\) consisting of all vectors having coordinates that \emph{sum to zero}.
The vectors
\begin{align*}
    u_1 = (2, -3, 4, -5, 2), &\ u_2 = (-6, 9, -12, 15, -6), \\
    u_3 = (3, -2, 7, -9, 1), &\ u_4 = (2, -8, 2, -2, 6), \\
    u_5 = (-1, 1, 2, 1, -3), &\ u_6 = (0. -3, -18, 9, 12), \\
    u_7 = (1, 0, -2, 3, -2), &\ u_8 = (2, -1, 1, -9, 7),
\end{align*}
generate \(W\).
Find a subset of \(\{ u_1, u_2, ..., u_8 \}\) that is a basis for \(W\).
\end{exercise}

\begin{proof}
(BTW, it's trivial that \(W\) has dimension \(4\).)
(And BTW, the vectors \(u_1\) to \(u_8\) are the same as \EXEC{1.6.8}.)
Similar to previous exercise, we reduce the problem to the system
\[
    A = [u_1\ u_2\ ... u_8] =
    \left[\begin{array}{cccccccc}
        2 & -6 & 3 & 2 & -1 & 0 & 1 & 2 \\
        -3 & 9 & -2 & -8 & 1 & -3 & 0 & -1 \\
        4 & -12 & 7 & 2 & 2 & -18 & -2 & 1 \\
        -5 & 15 & -9 & -2 & 1 & 9 & 3 & -9 \\
        2 & -6 & 1 & 6 & -3 & 12 & 2 & 7
    \end{array}\right]
\]
And a equivalent system (by calculation) is
\[
    B = \left(\begin{array}{cccccccc}
        1 & -3 & \frac{3}{2} & 1 & -\frac{1}{2} & 0 & \frac{1}{2} & 1 \\
        0 & 0 & 1 & -2 & -\frac{1}{5} & -\frac{6}{5} & \frac{3}{5} & \frac{4}{5} \\
        0 & 0 & 0 & 0 & 1 & -4 & -\frac{23}{21} & -\frac{19}{21} \\
        0 & 0 & 0 & 0 & 0 & 0 & 1 & -1 \\
        0 & 0 & 0 & 0 & 0 & 0 & 0 & 0
    \end{array}\right)
\]
If we continue to transform \(B'\) into r.r.e.f., then it's trivial that the \(1, 3, 5, 7\)th column of the r.r.e.f. is \LID{}.
By \THM{3.16}(c), \(\{ a_1, a_3, a_5, a_7 \} =\{ u_1, u_3, u_5, u_7 \}\) is \LID{}.
And since \(\dim(W) = 4\), \(\{ u_1, u_3, u_5, u_7 \}\) is a basis for \(W\).
\end{proof}

\begin{exercise} \label{exercise 3.4.9}
Let \(W\) be the subspace of \(M_{2 \X 2}(\SET{R})\) consisting of the \emph{symmetric} \(2 \X 2\) matrices.
The set
\[
    S= \left\{
        \left(\begin{array}{rr}
            0 & -1 \\
            -1 & 1
        \end{array}\right),
        \left(\begin{array}{ll}
            1 & 2 \\
            2 & 3
        \end{array}\right),
        \left(\begin{array}{ll}
            2 & 1 \\
            1 & 9
        \end{array}\right),
        \left(\begin{array}{rr}
            1 & -2 \\
            -2 & 4
        \end{array}\right),
        \left(\begin{array}{rr}
            -1 & 2 \\
            2 & -1
        \end{array}\right)
    \right\}
\]
generates \(W\).
Find a subset of \(S\) that is a basis for \(W\).
\end{exercise}

\begin{proof}
We use the \emph{standard representation} of each vector of \(S\) to make the system
\[
    A = \left(\begin{array}{ccccc}
        0 & 1 & 2 & 1 & -1 \\
        -1 & 2 & 1 & -2 & 2 \\
        -1 & 2 & 1 & -2 & 2 \\
        1 & 3 & 9 & 4 & -1
    \end{array}\right)
\]
And the r.r.e.f. (by calculation) is
\[
    B = \left(\begin{array}{ccccc}
        1 & -2 & -1 & 2 & -2 \\
        0 & 1 & 2 & 1 & -1 \\
        0 & 0 & 0 & 1 & -2 \\
        0 & 0 & 0 & 0 & 0
    \end{array}\right)
\]
So by \THM{3.16}(c), \(a_1, a_2, a_4\) are \emph{\LID{}} vectors in \(F^4\).
That is, the corresponding matrices
\[
    S' = \left\{
        \left(\begin{array}{rr}
            0 & -1 \\
            -1 & 1
        \end{array}\right),
        \left(\begin{array}{ll}
            1 & 2 \\
            2 & 3
        \end{array}\right),
        \left(\begin{array}{rr}
            1 & -2 \\
            -2 & 4
        \end{array}\right),
    \right\}
\]
are \LID{} in \(M_{2 \X 2}(\SET{R})\).
And since \(\dim(M_{2 \X 2}(\SET{R})) = 4\), \(S'\) is a basis of the set consisting of the symmetric \(2 \X 2\) matrices.
\end{proof}

\begin{exercise} \label{exercise 3.4.10}
Let
\[
    V = \{(x_1, x_2, x_3, x_4, x_5) \in \SET{R}^5 : x_1 - 2x_2 + 3x_3 - x_4 + 2x_5 = 0
 \}.
\]
\begin{enumerate}
\item Show that \(S = \{(0, 1, 1, 1, 0)\}\) is a \LID{} subset of \(V\).
\item Extend \(S\) to a basis for \(V\).
\end{enumerate}
\end{exercise}

\begin{proof}
part(a) is a really stupid question; singleton nonzero set is \LID{}, and \(x_1 - 2x_2 + 3x_3 - x_4 + 2x_5 = 0 - 2 + 3 - 1 + 0 = 0\), hence \(S\) is a \LID{} subset of \(V\).

For part(b), the procedure is exactly the same as \EXAMPLE{3.4.4}. Skip.
\end{proof}

\begin{exercise} \label{exercise 3.4.11}
Let \(V\) be as in \EXEC{3.4.10}.
\begin{enumerate}
\item Show that \(S = \{ (1 , 2, 1, 0, 0) \}\) is a \LID{} subset of \(V\).
\item Extend \(S\) to a basis for \(V\).
\end{enumerate}
\end{exercise}

\begin{proof}
Calculation problem which is similar to \EXEC{3.4.10}. Skip.
\end{proof}

\begin{exercise} \label{exercise 3.4.12}
Let \(V\) denote the set of all solutions to the system of linear equations
\[
    \sysdelim..\systeme{
        x_1 - x_2 + 2x_4 - 3x_5 + x_6 = 0,
        2x_1 - x_2 - x_3 + 3x_4 - 4x_5 + 4x_6 = 0
    }
\]
\begin{enumerate}
\item Show that \(S = \{ (0,-1,0,1,1,0), (1,0,1,1,1,0) \}\) is a \LID{} subset of \(V\).
\item Extend \(S\) to a basis for \(V\).
\end{enumerate}
\end{exercise}

\begin{proof}
Calculation problem which is similar to \EXEC{3.4.10}. Skip.
\end{proof}

\begin{exercise} \label{exercise 3.4.13}
Let \(V\) be as in \EXEC{3.4.12}.
\begin{enumerate}
\item Show that \(S = \{ (1,0,1,1,1,0), (0,2,1,1,0,0) \}\) is a \LID{} subset of \(V\).
\item Extend \(S\) to a basis for \(V\).
\end{enumerate}
\end{exercise}

\begin{proof}
Calculation problem which is similar to \EXEC{3.4.10}. Skip.
\end{proof}

\begin{exercise} \label{exercise 3.4.14}
If \((A|b)\) is in reduced row echelon form, prove that \(A\) is also in reduced row echelon form.
\end{exercise}

\begin{proof}
Suppose \((A|b)\) is in r.r.e.f..
It's enough to check that \(A\) still satisfies the definition of reduced row echelon form(\DEF{3.7}).
Given any row number \(i\) s.t. row \(i\) of \(A\) is zero row, there are two cases:
\begin{enumerate}
\item Zero row \(i\) of \(A\) is also zero row of \((A|b)\).
    Then given any nonzero row \(j\) of \(A\), of course row \(j\) of \((A|b)\) is nonzero, hence by condition \(1\) of \((A|b)\),
    row \(j\) of \((A|b)\) is above row \(i\) of \((A|b)\), and that implies row \(j\) of \(A\) is above row \(i\) of \(A\).
    Hence condition 1 of \(A\) is satisfied.
\item Zero row \(i\) of \(A\) is a nonzero row of \((A|b)\).
    Then given any nonzero row \(j\) of \(A\), of course row \(j\) of \((A|b)\) is nonzero.
    But by the \textbf{condition 3 for \((A|b)\)}, row \(j\) of \((A|b)\) must be \emph{above} row \(i\) of \((A|b)\), and that implies row \(j\) of \(A\) is above row \(i\) of \(A\).
    Hence condition 1 of \(A\) is again satisfied.
\end{enumerate}
So in all cases, the condition 1 of \(A\) is satisfied.

The second condition for \(A\) is really, automatically, implied by the second condition for \((A|b)\).
The third condition for \(A\) is also automatically satisfied.
\end{proof}

\begin{exercise} \label{exercise 3.4.15}
Prove the \CORO{3.16.1}:
The reduced row echelon form of a matrix \textbf{is unique}.
\end{exercise}

\begin{proof}
Note that if \(A\) is a zero matrix, then every elementary row operation performed on \(A\) leaves \(A\) \emph{unchanged}.
Thus the r.r.e.f. of a zero matrix is the matrix itself, and hence is unique.

Suppose then that \(A\) is an \(m \X n\) matrix of rank \(r > 0\) whose r.r.e.f. is \(B\).
Then there is an invertible \(m \X m\) matrix \(M\) representing the sequence of e.r.o.s, such that \(MA = B\).
For \(j = 1, 2, ..., n\), let \(a_j\) denote the \(j\)th column of \(A\) and \(b_j\) denote the \(j\)th column of \(B\).
By \THM{3.16}(b), \(e_i\) is ``a'' column of \(B\) for \(i = 1, 2, ..., r\).
Now let \(j_i\) denote the column number of the \emph{leftmost column} of \(B\) s.t. \(b_{j_i} = e_i\) \MAROON{(1)}.
Then by \THM{3.16}(c), \(\{ a_{j_1}, a_{j_2}, ..., a_{j_r} \}\) is \LID{}, and (since the dimension of column space of \(A\) is \(r\)) therefore is a basis for the column space of \(A\).
So, for \(k = 1, 2, ..., n\), \(a_k = c_1 a_{j_1} + c_2 a_{j_2} + ... + c_r a_{j_r}\) for \textbf{unique} scalars \(c_1, c_2, ..., c_r\).

\RED{
But then, by \ATHM{2.29}(or \EXEC{2.3.15}), we have \(b_k = M a_k = c_1 b_{j_1} + c_2 b_{j_2} + ... + c_r b_{j_r}\).
}
(Nasty, nasty, nasty.)
That is, by \MAROON{(1)}, we have \(b_k = c_1 e_1 + c_2 e_2 + ... + c_r e_r\).
Since \(c_1, c_2, ..., c_r\) are \emph{unique} and completed determined by \(A\), \(b_k\) is completely determined by \(A\).
Hence each column of \(B\) is completely determined by \(A\), hence \(B\) is unique.
\end{proof}

\begin{note}
There is \href{https://www.youtube.com/watch?v=EcgaeUUYV1U&ab_channel=DrPeyam}{Another proof}, which is more humble.
\end{note}

\begin{note}
只用噁心的\ \ATHM{2.29} 還不夠,必須要靠\ \(c_1, ..., c_r\) 是\ unique 才能唯一決定\ \(B\)。
\end{note}

\begin{additional theorem} \label{athm 3.15}
This is the placeholder theorem for \EXEC{3.4.3}:
Suppose that the augmented matrix of a system \(Ax = b\) is transformed into a matrix \((A'|b')\) in reduced row echelon form by a finite sequence of elementary row operations.
\begin{enumerate}
\item \(\rank(A') \ne \rank(A'|b')\) if and only if \((A'|b')\) contains a row in which the only nonzero entry lies in the last column.
\item \(Ax = b\) is consistent if and only if \((A'|b')\) contains no row in which the only nonzero entry lies in the last column.
\end{enumerate}
\end{additional theorem}

\begin{additional theorem} \label{athm 3.16}
This is the placeholder theorem for \EXEC{3.4.14}:
If \((A|b)\) is in reduced row echelon form, then \(A\) is also in reduced row echelon form.
\end{additional theorem}

\begin{additional theorem} \label{athm 3.17}
This is the placeholder theorem for \EXEC{3.4.15}:
r.r.e.f. is \textbf{unique}.
\end{additional theorem}
