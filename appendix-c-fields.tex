\section{Fields} \label{sec 8.c}

The set of real numbers is an example of an \href{https://www.wikiwand.com/en/Algebraic_structure}{\emph{algebraic structure}} called a \textbf{field}.
More precisely, a field is defined as follows.

\begin{appendix definition} \label{def c.1}
A Field \(F\) is a set on which two operations \(+\) and \(\cdot\) (called \textbf{addition} and \textbf{multiplication}. respectively) are defined so that, for each pair of elements \(x, y\) in \(F\), there \textbf{exist unique} elements in \(F\), denoted \(x + y\) and \(x \cdot y\).
and such that the following conditions hold for all elements \(a, b, c \in F\).
\begin{enumerate}
\item [(F 1)] \(a + b = b + a\) and \(a \cdot b = b \cdot a\) \\
    (\emph{commutativity} of addition and multiplication)
\item [(F 2)] \((a + b) + c = a + (b + c)\) and \((a \cdot b) \cdot c = a \cdot (b \cdot c)\) \\
    (\emph{associativity} of addition and multiplication)
\item [(F 3)] There \emph{exist} \textbf{distinct} elements \(0\) and \(1\) in \(F\) such that \(0 + a = a\) and \(1 \cdot a = a\) \\
    (existence of \emph{identity elements} for addition and multiplication)
\item [(F 4)] For each element \(a\) in \(F\) and each \emph{nonzero} element \(b\) in \(F\), there exist elements \(c\) and \(d\) in \(F\) such that
\[
    a + c = 0 \quad \text{ and } \quad b \cdot d = 1
\]
    (existence of \emph{inverses} for addition and multiplication)
\item [(F 5)] \(a \cdot (b + c) = a \cdot b + a \cdot c\) \\
    (\emph{distributivity} of multiplication over addition)
\end{enumerate}
The elements \(x + y\) and \(x \cdot y\) are called the \textbf{sum} and \textbf{product}, respectively, of \(x\) and \(y\).
The elements \(0\) (read ``\textbf{zero}'') and \(1\) (read ``\textbf{one}'') mentioned in (F 3) are called \textbf{identity elements} for addition and multiplication, respectively,
and the elements \(c\) and \(d\) referred to in (F 4) are called an \textbf{additive inverse} for \(a\) and a \textbf{multiplicative inverse} for \(b\), respectively.

Also note that the statements for \(+\) and \(\cdot\) implies these operations are required to be \emph{well-defined}.
So given \(a, b\) in a field, if \(a = b\), then we have \(a + c = b + c\) for any element \(c\) in the field.
Multiplication is similar.
In some literature, (e.g. \href{https://terrytao.wordpress.com/books/analysis-i/}{Analysis by Terence Tao}), this property is called \textbf{Axiom of Substitution}.
\end{appendix definition}

\begin{example} \label{example c.1}
(Precisely, from Calculus or Real Analysis, we can construct the set of real numbers, such that) The set of real numbers \(\SET{R}\) with the \emph{usual} definitions of addition and multiplication is a field.
\end{example}

\begin{example} \label{example c.2}
(Again, precisely, from Calculus or Real Analysis, we can construct the set of \emph{rational} numbers, such that) The set of \emph{rational} numbers \(\SET{Q}\) with the \emph{usual} definitions of addition and multiplication is a field.
\end{example}

\begin{example} \label{example c.3}
(It can be shown that, by using the definition and properties of \(\SET{R}\) and \(\SET{Q}\),)
The set of all real numbers \textbf{of the form} \(a + b \sqrt{2}\), where \(a\) and \(b\) are \emph{rational} numbers, with addition and multiplication as in \(\SET{R}\) is a field.
\end{example}

\begin{example} \label{example c.4} It can be shown that the set \(Z_2\) consists of two elements \(0\) and \(1\) with the operations of addition
and multiplication defined by the equations
\[
    \begin{array}{r}
        0+0=0 \quad 1+1=0 \quad 0+1=1+0=1 \\
        0 \cdot 0=0 \quad 1 \cdot 1=1 \quad 0 \cdot 1=1 \cdot 0=0
    \end{array}
\]
satisfies \DEF{c.1} and hence is a field.
\end{example}

\begin{example} \label{example c.5}
Neither the set of positive integers nor the set of integers with the usual definitions of addition and multiplication is a field, for in either case (F 4) does not hold.
\end{example}

The identity and inverse elements guaranteed by (F 3) and (F 4) are \textbf{unique}; this is a consequence of the following theorem.

\begin{appendix theorem} [Cancellation Laws] \label{thm c.1}
For arbitrary elements \(a, b\), and \(c\) in a field, the following statements are true.
\begin{enumerate}
\item If \(a + b = c + b\), then \(a = c\).
\item If \(a \cdot b = c \cdot b\) \emph{and} \(b \ne 0\), then \(a = c\).
\end{enumerate}
\end{appendix theorem}

\begin{proof} \ 

\begin{enumerate}
\item Suppose \(a + b = c + b\).
First, by (F 4), there exists an element \(d\) in the given field such that \(b + d = 0\). \MAROON{(a.1)}
And we have
\begin{align*}
             & a + b = c + b \\
    \implies & (a + b) + d = (c + b) + d & \text{since \(+\) is well-defined} \\
    \implies & a + (b + d) = c + (b + d) & \text{by (F 2), associativity of \(+\)} \\
    \implies & a + 0 = b + 0 & \text{by \MAROON{(a.1)}} \\
    \implies & 0 + a = a + 0 = b + 0 = 0 + b & \text{by (F 1)} \\
    \implies & a = 0 + a = a + 0 = b + 0 = 0 + b = b & \text{by (F 3)} \\
    \implies & a = b & \text{of course}
\end{align*}

\item If \(b \ne 0\), then (F 4) guarantees the existence of an element \(d\) in the field such that \(b \cdot d = 1\) \MAROON{(b.1)}.
Multiply both sides of the equality \(a \cdot b = c \cdot b\) by \(d\) to obtain \((a \cdot b) \cdot d = (c \cdot b) \cdot d\).
Consider the left side of this equality: By (F 2) and (F 3), we have
\begin{align*}
    (a \cdot b) \cdot d & = a \cdot (b \cdot d) & \text{by (F 2)} \\
        & = a \cdot 1 & \text{by \MAROON{(b.1)}} \\
        & = a. & \text{by (F 3)}
\end{align*}
Similarly, the right side of the equality reduces to \(c\).
Thus \(a = c\).
\end{enumerate}
\end{proof}

\begin{appendix corollary} \label{corollary c.1.1}
The elements \(0\) and \(1\) mentioned in (F 3), are unique.
And given any element \(a\) in \(F\) and nonzero element \(b\) in \(F\), the corresponding elements \(c\) and \(d\) mentioned in (F 4), are unique.
\end{appendix corollary}

\begin{proof}
Suppose that \(0' \in F\) satisfies (F 3); that is, \(0' + a = a\) for each \(a \in F\).
Since \(0 + a = a\) for each \(a \in F\), we have \(0' + a = 0 + a\) for each \(a \in F\).
Thus \(0' = 0\) by \THM{c.1}(a).

Suppose that \(1' \in F\) satisfies (F 3); that is, \(1' \cdot a = a\) for each \(a \in F\).
First, we have \(1'\) is \emph{nonzero}, since by \DEF{c.1}, if \(1'\) satisfies (F 3), then it must be distinct to \(0\).
And since \(1 \cdot a = a\) for each \(a \in F\), we have \(1' \cdot a = 1 \cdot a\) for each \(a \in F\).
Thus the two conditions for \THM{c.1}(b) are satisfied, hence \(1' = 1\).

Suppose that \(a + c = 0\).
If we also have \(a + c' = 0\), then \(a + c = a + c'\), so by \THM{c.1}(a), \(c = c'\).
So the additive inverse is unique.

Suppose that \(b\) is nonzero, and \(b \cdot d = 1\).
If we also have \(b \cdot d' = 1\), then \(b \cdot d = b \cdot d'\), so by \THM{c.1}(b), \(d = d'\).
So the multiplicative inverse is unique.
\end{proof}

\begin{remark} \label{remark c.1}
Thus each element \(b\) in a field has a \emph{unique} additive inverse and, if \(b \ne 0\), a \emph{unique} multiplicative inverse.
(It is shown in the \CORO{c.2.1} that \(0\) has \emph{no} multiplicative inverse.)
Hence we can \emph{denote} the inverse.
The additive inverse and the multiplicative inverse of \(b\) are denoted by \(-b\) and \(b^{-1}\), respectively.
Note that \(-(-b)= b\) and \((b^{-1})^{-1} = b\);
We give the proof of \(-(-b) = b\) here, and left \((b^{-1})^{-1} = b\) \emph{after \THM{c.2}}:

\(-(-b)\), by definition, is the inverse of \(-b\), so we have \(-b + (-(-b)) = 0\).
But \(-b\) by definition is the inverse of \(b\), so we have \(b + (-b) = 0\).
So we have
\begin{align*}
             & b + (-b) = 0 \land (-b) + (-(-b)) = 0 \\
    \implies & b + (-b) = 0 \land (-(-b)) + (-b) = 0 & \text{by (F 1)} \\
    \implies & b + (-b) = (-(-b)) + (-b) \\
    \implies & b = (-(-b)) & \text{by \THM{c.1}(a)}
\end{align*}
\end{remark}

\begin{remark} \label{remark c.2}
\textbf{Subtraction} and \textbf{division} can be defined \emph{in terms of addition and multiplication} by using the additive and multiplicative inverses.
Specifically, subtraction of \(b\) is defined to be addition of \(-b\) and division by \(b \ne 0\) is defined to be multiplication by \(b^{-1}\);
that is,
\[
    a - b = a + (-b) \quad \text{ and } \quad \frac{a}{b} = a \cdot b^{-1}.
\]
In particular, the symbol \(\frac{1}{b}\) can just be denoted as \(b^{-1}\), since by definition \(\frac{1}{b} = 1 \cdot b^{-1}\), which by (F 3) is equal to \(b^{-1}\).

Division by zero is (left) undefined, but, with this exception, the sum, product, difference, and quotient of any two elements of a field are defined.
Many of the familiar properties of multiplication of real numbers are true in any field, as the next theorem shows.
\end{remark}

\begin{appendix theorem} \label{thm c.2}
Let \(a\) and \(b\) be arbitrary elements of a field.
Then the following statements are true.
\begin{enumerate}
\item \(a \cdot 0 = 0\).
\item \((-a) \cdot b = a \cdot (-b) = -(a \cdot b)\).
\item \((-a) \cdot (-b) = a \cdot b\).
\end{enumerate}
\end{appendix theorem}

\begin{proof}
Since by (F 3), \(\BLUE{0} + \RED{0} = \RED{0}\) \MAROON{(a.1)}, we have
\begin{align*}
    0 + a \cdot 0 & = a \cdot 0 & \text{by (F 3)} \\
        & = a \cdot (0 + 0) & \text{by \MAROON{(a.1)}} \\
        & = a \cdot 0 + a \cdot 0 & \text{by (F 5)}
\end{align*}
Thus \(0 = a \cdot 0\) by \THM{c.1}(a).

\item By definition, \(-(a \cdot b)\) is the \emph{unique} element of \(F\) with the property \(a \cdot b + [-(a \cdot b)] = 0\).
So in order to prove that \(\RED{(-a) \cdot b} = -(a \cdot b)\), it suffices to show that \(a \cdot b + \RED{(-a) \cdot b} = 0\).
But \(-a\) is the element of \(F\) such that \(a + (-a) = 0\);
so
\begin{align*}
    a \cdot b + (-a) \cdot b & = [a + (-a)] \cdot b & \text{by (F 5)} \\
        & = 0 \cdot b & \text{by (F 4)} \\
        & = b \cdot 0 & \text{by (F 1)} \\
        & = 0 & \text{by part(a)}
\end{align*}
Thus \((-a) \cdot b = -(a \cdot b)\).
Similarly, we have to show that \(a \cdot b + \RED{a \cdot (-b)} = 0\);
so
\begin{align*}
    a \cdot b + a \cdot (-b) & = a \cdot [b + (-b)] & \text{by (F 5)} \\
        & = a \cdot 0 & \text{by (F 4)} \\
        & = 0 & \text{by part(a)}
\end{align*}
Thus \(a \cdot (-b) = -(a \cdot b)\).
Hence \((-a) \cdot b = a \cdot (-b) = -(a \cdot b)\).

\item Using part(b), we have
\begin{align*}
    & (-a) \cdot (-b) \\
    & = -[a \cdot (-b)] & \text{by (b), ``exp 1'' of (b) equals ``exp 3'' of (b)} \\
    & = -[-(a \cdot b)] & \text{by (b), ``exp 2'' of (b) equals ``exp 3'' of (b)} \\
    & = a \cdot b & \text{by \RMK{c.1}}
\end{align*}
\end{proof}

\begin{appendix corollary} \label{corollary c.2.1}
By \THM{c.2}, the \emph{additive identity} of a field has no multiplicative inverse, since for any \(a\) in the field, \(a \cdot 0 = 0\), and \(0 \ne 1\) by (F 3).
\end{appendix corollary}

\begin{proof}[Proof of the second statement in \RMK{c.1}]
We have to show that for any \(b \in 0\) in a field, \(b = (b^{-1})^{-1}\).

So suppose \(b \ne 0\) so that by (F 4) we have \(b^{-1}\) such that \(b \cdot b^{-1} = 1\). \MAROON{(1)}.
Note that \(b^{-1}\) must be \emph{nonzero}, otherwise \(1 = b \cdot b^{-1} = b \cdot 0 = 0\) by \THM{c.2} and we get \(1 = 0\), which contradicts with \DEF{c.1}(F 3) that \(1\) and \(0\) are distinct.
Hence again by (F 4), we have \((b^{-1})^{-1}\), the multiplicative inverse of \(b^{-1}\), such that \(b^{-1} \cdot (b^{-1})^{-1} = 1\) \MAROON{(2)}.
Combining \MAROON{(1)(2)}, we have \(b \cdot b^{-1} = b^{-1} \cdot (b^{-1})^{-1}\), and by (F 1) we have \(b \cdot b^{-1} = (b^{-1})^{-1} \cdot b^{-1}\), and by \THM{c.1}, \(b = (b^{-1})^{-1}\), as desired.
\end{proof}

\begin{remark} \label{remark c.3}
In an arbitrary field \(F\), it \textbf{may happen} that a sum \(1 + 1 + ... + 1\) (\(p\) summands) equals \(0\) for some \emph{positive integer} \(p\).
For example, in the field \(Z_2\) (defined in \EXAMPLE{c.4}), \(1 + 1 = 0\).
In this case, the \emph{smallest positive integer} \(p\) for which a sum of \(p\) \(1'\)s equals \(0\) is called the \textbf{characteristic} of \(F\);
if \emph{no such positive integer exists}, then \(F\) is said to have \textbf{characteristic zero}.
Thus \(Z_2\) has characteristic two, and \(\SET{R}\) has characteristic zero.
Observe that if \(F\) is a field of characteristic \(p \ne 0\), then \(x + x + ... + x\) (\(p\) summands) equals \(0\) for all \(x \in F\).
In a field having nonzero characteristic (especially characteristic two), \textbf{many unexpected problems arise}.
For this reason, some of the results about vector spaces stated in this book \textbf{require that the field over which the vector space is defined} be of characteristic zero (or, at least, of some characteristic \emph{other than two}).
Finally, note that in other sections of this book, the product of two elements \(a\) and \(b\) in a field is usually denoted \(ab\) rather than \(a \cdot b\).

Refer to abstract algebra course for these kind of fields.
\end{remark}
