\section{Normal and Self-Adjoint Operators} \label{sec 6.4}

We have seen the importance of diagonalizable operators in \CH{5}.
For an operator on a vector space \(\V\) to be diagonalizable, (by \THM{5.1}) it is necessary and sufficient for \(\V\) to contain a basis of eigenvectors for this operator.
As \(\V\) is an inner product space in this chapter, \emph{it is reasonable to seek conditions that guarantee that \(\V\) has an \textbf{orthonormal} basis of eigenvectors}.
A very important result that helps achieve our goal is Schur's theorem (\THM{6.14}).
The formulation that follows is in terms of linear operators.
The next section contains the more familiar matrix form.
We begin with a lemma.

\begin{note}
小總結,這一節探討的是一個線性算子可用\textbf{標準正交基底}來對角化的充分必要條件,並且把複數內積空間跟實數內積空間分開討論。
但這個對角化實際上比第五章來地嚴格。意思是說,就算有一個線性算子不符合這些充分必要條件,也不代表他不能被對角化,很多時候它可以被對角化,只是對應的基底不是正交的。
\end{note}

\begin{lemma} \label{lem 6.4}
Let \(\T\) be a linear operator on a \emph{finite}-dimensional inner product
space \(\V\).
If \(\T\) has an eigenvector, then so does \(\T^*\).
\end{lemma}

\begin{proof}
Suppose that \(v\) is an eigenvector \textbf{of \(\T\)} with corresponding eigenvalue \(\lambda\);
that is, \(\T(v) = \lambda v\) and hence \((\T - \lambda\ITRAN{})(v) = \OV\). \MAROON{(1)}
Then for any \(x \in \V\),
\begin{align*}
    0 & = \LG \OV, x \RG & \text{by \THM{6.1}(c)} \\
      & = \LG (\T - \lambda\ITRAN{})(v), x \RG & \text{by \MAROON{(1)}} \\
      & = \LG v, (\T - \lambda\ITRAN{})^*(x) \RG & \text{by \THM{6.9}} \\
      & = \LG v, (\T^* - \conjugatet{\lambda} \ITRAN{})(x) \RG, & \text{by \THM{6.11}(a)(b)(e)}
\end{align*}
hence \(v\) is orthogonal to the \emph{range} of \(\T^* - \conjugatet{\lambda} \ITRAN{}\), so by \DEF{6.7}, \(v \in \RANGE(\T^* - \conjugatet{\lambda} \ITRAN{})^{\perp}\).
But since \(v \ne \OV\), so (since \(\RANGE(\T^* - \conjugatet{\lambda} \ITRAN{}) \cap \RANGE(\T^* - \conjugatet{\lambda} \ITRAN{})^{\perp} = \{ \OV \}\),) \(v \notin \RANGE(\T^* - \conjugatet{\lambda} \ITRAN{})\).
Then this implies \(\T^* - \conjugatet{\lambda} \ITRAN{}\) is not onto, (since there exists a vector that is not in its range) hence (by \THM{2.5}) is not one-to-one.
So there exists a nonzero vector \(u\) such that \(\T^* - \conjugatet{\lambda} \ITRAN{} (u) = \OV\), which implies \(\T^*(u) = \conjugatet{\lambda} u\), so \(u\) is an eigenvector of \(\T^*\) with corresponding eigenvalue \(\conjugatet{\lambda}\).
\end{proof}

Recall \DEF{5.14} that a subspace \(\W\) of \(\V\) is said to be \(\T\)-invariant if \(\T(\W)\) is contained in \(\W\).
If \(\W\) is \(\T\)-invariant, we may define the restriction \(\T_\W : \W \to \W\) by \(\T_\W(x) = \T(x)\) for all \(x \in \W\).
It is clear that \(\T_\W\) is a linear operator on \(\W\).
Recall from \DEF{5.5} that a polynomial is said to \textbf{split} if it factors into \emph{linear} polynomials.

\begin{theorem} [Schur] \label{thm 6.14}
Let \(\T\) be a linear operator on a \emph{finite}-dimensional inner product space \(\V\).
Suppose that the \CPOLY{} of \(\T\) splits.
Then there exists an \textbf{orthonormal} basis \(\gamma\) for \(\V\) such that the matrix \([\T]_{\gamma}\) is \textbf{upper triangular}.
\end{theorem}

\begin{proof}
Since the \CPOLY{} of \(\T\) splits, by \EXEC{5.2.12}(a), there exists an ordered basis \(\beta = \{ w_1, w_2, ..., w_n \}\) for \(\V\) such that \([\T]_{\beta}\) is upper triangular.
Now apply the Gram-Schmidt process (\THM{6.4}) to \(\beta\) to obtain an ortho\emph{gonal} basis \(\beta' = \{ v_1, v_2, ..., v_n \}\) for \(\V\).
For each \(k\), \(1 \le k \le n\), let
\[
    S_k = \{ w_1, w_2, ..., w_k \} \quad \text{ and } \quad S'_k = \{ v_1, v_2, ..., v_k \}.
\]
As in the proof of \THM{6.4}, \(\spann(S_k) = \spann(S'_k)\) for all \(k\). \MAROON{(1)}
By \EXEC{2.2.12}, (since \([\T]_{\beta}\) is upper triangular,) \(\T(w_k) \in \spann(S_k)\) for all \(k\). \MAROON{(2)}
So by \MAROON{(1)(2)}, \(\T(v_k) \in \spann(S'_k)\) for all \(k\), and by \EXEC{2.2.12} again, \([\T]_{\beta'}\) is also upper triangular.
Finally, let \(z_i = \frac{1}{\norm{z_i}} v_i\) for all \(1 \le i \le n\) and \(\gamma = \{ z_1, z_2, ..., z_n \}\).
Then \(\gamma\) is an ortho\textbf{normal} basis for \(\V\), and \([\T]_{\gamma}\) is (of course still) upper triangular.
\end{proof}

\begin{proof} [\MAROON{Proof in the fourth edition}]
The proof is by mathematical induction \textbf{on the dimension} \(n\) of \(\V\).
The result is immediate if n = 1, since by definition an one-by-one matrix is upper triangular.

So suppose that the result is true for linear operators on \((n - 1)\)-dimensional inner product spaces whose \CPOLY{} split, for some \(n > 1\).
Now let \(\T\) be a linear operator on \(n\)-dimensional inner product space \(\V\) whose \CPOLY{} splits (hence \(\T\) has eigenvector(s)).
Then by \LEM{6.4}, we can assume that \(\T^*\) has a \emph{unit} eigenvector \(z\).
Suppose that \(\T^*(z) = \lambda z\) and that \(\W = \spann(\{ z \})\).

We first show that \(\W^{\perp}\) is \(\T\)-invariant.
So suppose \(y \in \W^{\perp}\), and suppose arbitrary \(x \in \W\), then
\begin{align*}
    \LG \T(y), x \RG & = \LG \T(y), cz \RG \text{\ for come \(c\)} & \text{since \(x \in \W = \spann(\{z\})\)} \\
        & = \LG y, \T^*(cz) \RG & \text{by \THM{6.9}} \\
        & = \LG y, c\T^*(z) \RG & \text{since \(\T^*\) is linear} \\
        & = \LG y, c \lambda z \RG \\
        & = \conjugatet{c \lambda} \LG y, z \RG & \text{by \THM{6.1}(b)} \\
        & = \conjugatet{c \lambda} \cdot 0 = 0. & \text{since \(y \in \W^{\perp}\) and \(z \in \W\)}
\end{align*}
So by \DEF{6.7}, \(\T(y) \in \W^{\perp}\), hence \(\W^{\perp}\) is \(\T\)-invariant.
By \THM{5.20} the \CPOLY{} of \(\T_{\W^{\perp}}\)
\emph{divides} the \CPOLY{} of \(\T\) and hence splits.
By \THM{6.7}(c), \(\dim(\W^{\perp}) = \dim(\V) - \dim(\W) = n - 1\), \emph{so we may apply the induction hypothesis} to \(\T_{\W^{\perp}}\) and obtain an orthonormal basis \(\gamma\) of \(\W^{\perp}\) such that \([\T_{\W^{\perp}}]_{\gamma}\) is upper triangular.
Clearly, \(\beta = \gamma \cup \{ z \}\) is an orthonormal basis for \(\V\) such that \([\T]_{\beta}\) is upper triangular.
\end{proof}

We now return to our original goal of finding an orthonormal basis of eigenvectors of a linear operator \(\T\) on a finite-dimensional inner product space \(\V\).

\begin{remark} \label{remark 6.4.1}
Note that if such an orthonormal basis \(\beta\) exists, then (by \THM{6.10}) \([\T]_{\beta}\) is a diagonal matrix, and hence \([\T^*]_{\beta} = [\T]^*_{\beta}\) is \emph{also a diagonal} matrix.
Because diagonal matrices (of course, trivially, obvious, by definition of matrix product) \emph{commute}, we conclude that \(\T\) and \(\T^*\) commute, since
\begin{align*}
    [\T\T^*]_{\beta} & = [\T]_{\beta}[\T^*]_{\beta} & \text{by \THM{2.11}} \\
        & = [\T^*]_{\beta}[\T]_{\beta} & \text{since diagonal matrices commute} \\
        & = [\T^*\T]_{\beta} & \text{by \THM{2.11}}
\end{align*}
Thus \emph{if there exists an orthonormal basis for \(\V\) consisting of eigenvectors of \(\T\), then \(\T\T^* = \T^*\T\)}.
\end{remark}

\begin{definition} \label{def 6.8}
Let \(\V\) be an inner product space, and let \(\T\) be a linear operator on \(\V\).

\BLUE{(1)} We say that \(\T\) is \textbf{normal} if \(\T\T^* = \T^*\T\).

\BLUE{(2)} An \(n \X n\) real or complex matrix \(A\) is \textbf{normal} if \(AA^* = A^*A\).
\end{definition}

\begin{remark} \label{remark 6.4.2}
It follows immediately from \THM{6.10} that \(\T\) is normal if and only if \([\T]_{\beta}\) is normal for any orthonormal basis \(\beta\).

That is, given arbitrary orthonormal basis \(\beta\),
\begin{align*}
    & \T \text{ is normal} \iff \T\T^* = \T^*\T \\
    \iff & [\T\T^*]_{\beta} = [\T^*\T]_{\beta} \\
    \iff & [\T]_{\beta} [\T^*]_{\beta} = [\T^*]_{\beta} [\T]_{\beta} & \text{by \THM{2.11}} \\
    \iff & [\T]_{\beta} ([\T]_{\beta})^* = ([\T]_{\beta})^* [\T]_{\beta} & \text{by \THM{6.10}} \\
    \iff & [\T]_{\beta} \text{ is normal}
\end{align*}
\end{remark}

\begin{note}
\(\T\) 是正規算子,若且唯若存在一個標準正交基底\ \(\beta\) 使得\ \(\T\) 在\ \(\beta\) 的矩陣代表是正規矩陣。
\end{note}

\begin{example} \label{example 6.4.1}
Let \(\T: \SET{R}^2 \to \SET{R}^2\) be \textbf{rotation} by \(\theta\), where \(0 < \theta < \pi\).
The matrix representation of \(\T\) in the standard ordered basis is given by
\[
    A = \begin{pmatrix}
        \cos \theta & -\sin \theta \\
        \sin \theta & \cos \theta
    \end{pmatrix}.
\]
Note that \(A A^* = I = A^* A\); so \(A\) is normal, and hence (by \RMK{6.4.2}), \(\T\), is normal.
\end{example}

\begin{remark} \label{remark 6.4.3}
Geometrically, in \EXAMPLE{6.4.1}, \(A^*\) can be thought as a rotation matrix that rotates \emph{counterclockwise}.
That is,
\begin{align*}
    A^* & = \begin{pmatrix}
        \cos \theta & -\sin \theta \\
        \sin \theta & \cos \theta
    \end{pmatrix}^* \\
    & = \begin{pmatrix}
        \cos \theta & \sin \theta \\
        -\sin \theta & \cos \theta
    \end{pmatrix} & \text{by \DEF{6.2}} \\
    & = \begin{pmatrix}
        \cos (-\theta) & -\sin (-\theta) \\
        \sin (-\theta) & \cos (-\theta)
    \end{pmatrix} & \text{since \(\cos\), \(\sin\) are even and odd functions, respectively}
\end{align*}
\end{remark}

\begin{example} \label{example 6.4.2}
Suppose that \(A\) is a real \emph{skew-symmetric} matrix; that is, \(A^\top = -A\).
Then \(A\) is normal because both \(A A^\top\) and \(A^\top A\) are equal to \(-A^2\).
\end{example}

Clearly, the operator \(\T\) in \EXAMPLE{6.4.1} does \emph{not} even possess one eigenvector (if the field is considered to be \(\SET{R}\)).
So in the case of a \textbf{real} inner product space, we see that \emph{normality is not sufficient to guarantee an orthonormal basis of eigenvectors}.
All is not lost, however.
We show that normality \textbf{suffices} if \(\V\) is a \textbf{complex} inner product space.
Before we prove the promised result for normal operators, we need some general properties of normal operators.

\begin{theorem} \label{thm 6.15}
Let \(\V\) be an inner product space, and let \(\T\) be a \textbf{normal} operator on \(\V\).
(We do not say that \(\V\) is finite-dimensional.)
Then the Following statements are true.
\begin{enumerate}
\item \(\norm{\T(x)} = \norm{\T^*(x)}\) for all \(x \in \V\).
\item \(\T - c\ITRAN{}\) is normal for every \(c \in F\).
\item If \(x\) is an eigenvector of \(\T\) corresponding to eigenvalue \(\lambda\), then \(x\) is also an eigenvector of \(\T^*\) corresponding to eigenvalue \(\conjugatet{\lambda}\).
That is, if \(\T(x) = \lambda x\), then \(\T^*(x) = \conjugatet{\lambda}x\).
\item If \(\lambda_1\) and \(\lambda_2\) are \emph{distinct} eigenvalues of \(\T\) with corresponding eigenvectors \(x_1\) and \(x_2\), then \(x_1\) and \(x_2\) are orthogonal.
\end{enumerate}
\end{theorem}

\begin{proof} \ 

\begin{enumerate}
\item For any \(x \in \V\), we have
\begin{align*}
    \norm{\T(x)}^2 & = \LG \T(x), \T(x) \RG \\
        & = \LG \T^*(\T(x)), x \RG = \LG \T^*\T(x), x \RG & \text{by \THM{6.9}} \\
        & = \LG \T\T^*(x), x \RG = \LG \T(\T^*(x)), x \RG & \text{since \(\T\) is normal} \\
        & = \LG \T^*(x), \T^*(x) \RG & \text{by \THM{6.9}} \\
        & = \norm{\T^*(x)}^2.
\end{align*}

\item We have
\begin{align*}
    (\T - c\ITRAN{})(\T - c\ITRAN{})^*
        & = (\T - c\ITRAN{})(\T^* - \conjugatet{c}\ITRAN{}^*) & \text{by \THM{6.11}(a)(b)} \\
        & = (\T - c\ITRAN{})(\T^* - \conjugatet{c}\ITRAN{}) & \text{by \THM{6.11}(e)} \\
        & = \T\T^* - c\ITRAN{}\T^* - \T\conjugatet{c}\ITRAN{} + \conjugatet{c}c\ITRAN{}^2 & \text{expand} \\
        & = \T\T^* - c\T^* - \conjugatet{c}\T + \abs{c}^2\ITRAN{} \quad \quad \MAROON{(b.1)} & \text{simplify}
\end{align*}
and 
\begin{align*}
    (\T - c\ITRAN{})^*(\T - c\ITRAN{})
        & = (\T^* - \conjugatet{c}\ITRAN{})(\T - c\ITRAN{}) & \text{just use previous case} \\
        & = \T^*\T - \conjugatet{c}\ITRAN{}\T - \T^*c\ITRAN{} + \conjugatet{c}c\ITRAN{}^2 & \text{expand} \\
        & = \T\T^* - \MAROON{\conjugatet{c}\T} - \RED{c\T^*} + \abs{c}^2\ITRAN{} & \text{simplify} \\
        & = \T\T^* - \RED{c\T^*} - \MAROON{\conjugatet{c}\T} + \abs{c}^2\ITRAN{} = \MAROON{(b.1)} & \text{of course}
\end{align*}

\item Suppose that \(\T(x) = \lambda x\) for some \(x \in \V\).
Let \(\U = \T - \lambda \ITRAN{}\) \MAROON{(c.1)}.
Then \(\U(x) = \T(x) - \lambda \ITRAN{}(x) = \OV\) \MAROON{(c.2)}, and \(\U\) is normal by part (b).
Thus
\begin{align*}
    0 & = \norm{\OV} = \norm{\U(x)} & \text{by \MAROON{(c.2)}} \\
      & = \norm{\U^*(x)} & \text{since \(\U\) is normal, and by part(a)} \\
      & = \norm{(\T - \lambda \ITRAN{})^*(x)} & \text{by \MAROON{(c.1)}} \\
      & = \norm{(\T^* - \conjugatet{\lambda} \ITRAN{})(x)} & \text{just use \THM{6.11}} \\
      & = \norm{\T^*(x) - \conjugatet{\lambda}(x)}, & \text{of course}
\end{align*}
So \(\T^*(x) - \conjugatet{\lambda}(x) = \OV\) hence \(\T^*(x) = \conjugatet{\lambda}(x)\).
So in particular, if \(x\) is nonzero, that is, \(x\) is an eigenvector of \(\T\) corresponding to eigenvalue \(\lambda\), then \(x\) is an eigenvector of \(\T^*\) corresponding to eigenvalue \(\conjugatet{\lambda}\).

\item Let \(\lambda_1\) and \(\lambda_2\) be \emph{distinct} eigenvalues of \(\T\) with corresponding eigenvectors \(x_1\) and \(x_2\).
Then we have
\begin{align*}
    \lambda_1 \LG x_1, x_2 \RG & = \LG \lambda_1 x_1, x_2 \RG & \text{by \DEF{6.1}(a)} \\
        & = \LG \T(x_1), x_2 \RG & \text{by supposition} \\
        & = \LG x_1, \T^*(x_2) \RG & \text{by \THM{6.9}} \\
        & = \LG x_1, \conjugatet{\lambda_2} x_2 \RG & \text{by part(c), notice the conjugate} \\
        & = \conjugatet{\conjugatet{\lambda_2}} \LG x_1, x_2 \RG & \text{by \THM{6.1}(b)} \\
        & = \lambda_2 \LG x_1, x_2 \RG, & \text{by \THM{d.2}(a)}
\end{align*}
which implies \((\lambda_1 - \lambda_2) \LG x_1, x_2 \RG = 0\).
Since \(\lambda_1 \ne \lambda_2\), we conclude that \(\LG x_1, x_2 \RG = 0\), that is, \(x_1, x_2\) are orthogonal.
\end{enumerate}
\end{proof}

\begin{theorem} \label{thm 6.16}
Let \(\T\) be a linear operator on a \emph{finite}-dimensional \textbf{complex} inner product space \(\V\).
Then \(\T\) is \textbf{normal} if and only if there exists an
\textbf{orthonormal basis} for \(\V\) consisting of \textbf{eigenvectors} of \(\T\).
\end{theorem}

\begin{proof}
Suppose that \(\T\) is normal.
Since \(\V\) is over \(\SET{C}\), by the fundamental theorem of algebra (\THM{d.4}), the \CPOLY{} of \(\T\) splits.
So we may apply Schur's theorem \THM{6.14} to obtain an \emph{orthonormal} basis \(\beta = \{ v_1, v_2, ..., v_n \}\) for \(\V\) such that \([\T]_{\beta} = A\) is upper triangular.

Now We know that \(v_1\) is an eigenvector of \(\T\) because \(A\) is upper triangular.
Assume that \(v_1, v_2, ..., v_{k - 1}\) are eigenvectors of \(\T\) for some \(k > 1\).
We claim that \(v_k\) is also an eigenvector of \(\T\).
It then follows by mathematical induction on \(k\) that all of the \(v_i\)'s are eigenvectors of \(\T\).
Consider any \(j < k\), and let \(\lambda_j\) denote the eigenvalue of \(\T\) corresponding to \(v_j\).
Since \(\T\) is normal, by \THM{6.15}(c), \(\T^*(v_j) = \conjugatet{\lambda_j} v_j\). \MAROON{(1)}
Since \(A\) is upper triangular,
\begin{align*}
    \T(v_k) & = A_{1k} v_1 + A_{2k} v_2 + ... + A_{jk} v_j + ... + A_{kk}v_k + 0 \cdot v_{k + 1} + ... + 0 \cdot v_{n} \\
    & = A_{1k} v_1 + A_{2k} v_2 + ... + A_{jk} v_j + ... + A_{kk}v_k. \MAROON{(2)}
\end{align*}
Furthermore, by the \CORO{6.5.1}, since \(\beta\) is orthonormal,
\begin{align*}
    A_{jk} & = \LG \T(v_k), v_j \RG & \text{by \CORO{6.5.1}} \\
        & = \LG v_k, \T^*(v_j) \RG & \text{by \THM{6.9}} \\
        & = \LG v_k, \conjugatet{\lambda_j} v_j \RG & \text{by \MAROON{(1)}} \\
        & = \conjugatet{\conjugatet{\lambda_j}} \LG v_j, v_k \RG = \lambda_j \LG v_j, v_k \RG & \text{by \THM{6.1}(b)} \\
        & = \lambda_j \cdot 0 = 0. & \text{since \(v_j\)'s are orthonormal}
\end{align*}
So \(A_{jk} = 0\) for all \(j < k\), hence it follows from \MAROON{(2)} that \(\T(v_k) = A_{kk} v_k\), and hence \(v_k\) is an eigenvector of \(\T\).
So by induction, all the vectors in \(\beta\) are eigenvectors of \(\T\).

The converse was already proved in \RMK{6.4.1}.
\end{proof}

\begin{remark} \label{remark 6.4.4}
Interestingly, as the next example shows, \THM{6.16} does \textbf{not} extend to \textbf{infinite}-dimensional \textbf{complex} inner product spaces.
\end{remark}

\begin{example} \label{example 6.4.3}
Consider the inner product space \(\textsf{H}\) with the orthonormal set \(S\) from \EXAMPLE{6.1.9}.
Let \(\V = \spann(S)\), and let \(\T\) and \(\U\) be the linear operators on \(\V\) defined by \(\T(f) = f_1 f\) and \(\U(f) = f_{-1} f\).
Then by law of (complex) exponents,
\[
  \T(f_n)= f_{n + 1} \quad \text{ and } \quad \U(f_n) = f_{n-1} \quad \MAROON{(1)}
\]
for \emph{all} integers \(n\).
Thus
\begin{align*}
    \LG \T(f_m), f_n \RG & = \LG f_{m + 1}, f_n \RG & \text{by \MAROON{(1)}} \\
        & = \delta_{(m + 1), n} & \text{since \(S\) is orthonormal} \\
        & = \delta_{m, (n -1)} & \text{need to show, but trivial} \\
        & = \LG f_m, f_{n - 1} \RG & \text{again since \(S\) is orthonormal} \\
        & = \LG f_m, \U(f_n) \RG. & \text{by \MAROON{(1)}}
\end{align*}
It follows that \(\U = \T^*\).
Furthermore, by \MAROON{(1)}, it immediately follows that \(\T\T^* = \ITRAN{} = \T^*\T\); so \(\T\) is normal.

Now we show that \(\T\) has \textbf{no} eigenvectors.
For the sake of contradiction, suppose that \(f\) is an eigenvector of \(\T\), say, \(\T(f) = \lambda f\) for some \(\lambda\). \MAROON{(2)}
Since \(\V\) equals the span of \(S\), we may write \(f\) as a \emph{finite} linear combination of vectors in \(S\).
That is, without loss of generality, there exist integer \(n, m\) (as lower bound and upper bound for the indices of vectors we pick) such that \(n \le m\) and
\[
    f = \sum_{i = n}^m a_i f_i, \quad \text{ where } a_m \ne 0. \quad \quad \MAROON{(3)}
\]
In particular,
\begin{align*}
    \lambda \sum_{i = n}^m a_i f_i & = \lambda f & \text{by \MAROON{(3)}} \\
        & = \T(f) & \text{by \MAROON{(2)}} \\
        & = \T \left( \sum_{i = n}^m a_i f_i \right) & \text{by \MAROON{(3)}} \\
        & = \sum_{i = n}^m a_i \T(f_i) & \text{since \(\T\) is linear} \\
        & = \sum_{i = 1}^m a_i f_{i + 1} & \text{by \MAROON{(1)}}
\end{align*}
So \(\lambda \sum_{i = n}^m a_i f_i = \sum_{i = 1}^m a_i f_{i + 1}\).
Since \(a_m \ne 0\), from this equation, by moving \emph{all but the last} terms from the right side to the left side, we can write \(f_{m + 1}\) as a linear combination of \(f_n, f_{n + 1}, ..., f_m\).
But this is a \textbf{contradiction} because \(S\) is \LID{}.
\end{example}

\begin{remark} \label{remark 6.4.5}
\EXAMPLE{6.4.1} illustrates that \emph{normality is not sufficient} to guarantee the existence of an orthonormal basis of eigenvectors for \textbf{real} inner product spaces.
For real inner product spaces, we must \emph{replace normality by the stronger condition} \RED{that \(\T = \T^*\)} in order to guarantee such a basis.
\end{remark}

\begin{definition} \label{def 6.9}
Let \(\T\) be a linear operator on an inner product space \(\V\).
We say that \(\T\) is \textbf{self-adjoint} (or \textbf{Hermitian}) if \(\T = \T^*\).
An \(n \X n\) real or complex matrix \(A\) is \textbf{self-adjoint} (or \textbf{Hermitian}) if \(A = A^*\).
\end{definition}

\begin{remark} \label{remark 6.4.6}
It follows immediately that if \(\beta\) is an orthonormal basis, then (using \THM{6.10}) \(\T\) is self-adjoint if and only if \([\T]_{\beta}\) is self-adjoint.
For \textbf{real} matrices \(A\), this condition reduces to the requirement that \(A\) be \emph{symmetric}.
\end{remark}

\begin{remark} \label{remark 6.4.7}
Before we state our main result for self-adjoint operators, we need some preliminary work.
By definition, a linear operator on a real inner product space \emph{has only real eigenvalues}.
The lemma that follows shows that \emph{the same can be said} for self-adjoint operators on a \textbf{complex} inner product space.
Similarly, the \CPOLY{} of every linear operator on a \textbf{complex} inner product space splits, and \emph{the same is true} for \textbf{self-adjoint} operators on a \textbf{real} inner product space.
\end{remark}

\begin{note}
矩陣看成在複數,則特徵多項式一定\ split。
矩陣若看成在實數,但是矩陣本身\ self-adjoint,則特徵多項式也一定\ split,其中根據定義,分解後得到的根,也就是\ eigenvalues,也是實數。
\end{note}

\begin{lemma} \label{lem 6.5}
Let \(\T\) be a \textbf{self-adjoint} operator on a \emph{finite}-dimensional inner product space \(\V\).
Then
\begin{enumerate}
\item Every eigenvalue of \(\T\) is \textbf{real} (even though the inner product space is on complex field).
\item Suppose that \(\V\) is a \textbf{\RED{real}} inner product space.
Then the \CPOLY{} of \(\T\) splits (over \(\SET{R}\)).
\end{enumerate}
\end{lemma}

\begin{proof} \ 

\begin{enumerate}
\item Suppose that \(\T(x) = \lambda x\) for \(x \ne \OV\).
Because a self-adjoint operator is (of course by definition) also normal, we can apply \THM{6.15}(c) to obtain
\begin{align*}
    \lambda x & = \T(x) \\
        & = \T^*(x) & \text{since \(\T\) is self-adjoint} \\
        & = \conjugatet{\lambda} x, & \text{by \THM{6.15}(c)}
\end{align*}
which implies \(\lambda = \conjugatet{\lambda}\);
that is, (by \THM{d.2}(e)), \(\lambda\) is real.

\item Let \(n = \dim(\V)\), \(\beta\) be an \emph{orthonormal} basis for \(\V\), and \(A = [\T]_{\beta}\).
Then (by \RMK{6.4.6}) \(A\) is self-adjoint.
Let \(\T_A\) be the linear operator \RED{on \(\SET{C}^n\)} defined by \(\T_A(x) = Ax\) for all \(x \in \SET{C}^n\).
Because (by \THM{2.15}(a)) \([\T_A]_{\gamma} = A\), where \(\gamma\) is the standard ordered (orthonormal) basis for \(\SET{C}^n\), (by \RMK{6.4.6} again) \(\T_A\) is also self-adjoint.
So, by part(a), the eigenvalues of \(\T_A\) are \textbf{real}.
By the fundamental theorem of algebra, the
\CPOLY{} of \(\T_A\) splits (over \(\SET{C}\)) into factors of the form \(t - \lambda\).
\emph{But since each \(\lambda\) is real}, the \CPOLY{} splits \RED{over \(\SET{R}\)}.
But \(\T_A\) (by \DEF{5.4}) has the same \CPOLY{} as \(A\), which (again by \DEF{5.4}) has the same \CPOLY{} as \(\T\).
Therefore the \CPOLY{} of \(\T\) splits (over \(\SET{R}\)).
\end{enumerate}
\end{proof}

We are now able to establish one of the major results of this chapter.

\begin{theorem} \label{thm 6.17}
Let \(\T\) be a linear operator on a \emph{finite}-dimensional \textbf{real} inner product space \(\V\).
Then \(\T\) is \textbf{self-adjoint} if and only if there exists an \textbf{orthonormal} basis \(\beta\) for \(\V\) \textbf{consisting of eigenvectors} of \(\T\).
\end{theorem}

\begin{proof}
Suppose that \(\T\) is self-adjoint.
By \LEM{6.5}(b), the \CPOLY{} of \(\T\) \emph{splits} (over \(\SET{R}\)),
so we may apply Schur's theorem(\THM{6.14}) to obtain an \emph{orthonormal} basis \(\beta\) for \(\V\) such that the matrix \(A = [\T]_{\beta}\) is upper triangular.
But
\begin{align*}
    A^* & = ([\T]_{\beta})^* = [\T^*]_{\beta} & \text{by \THM{6.10}} \\
        & = [\T]_{\beta} & \text{since \(\T\) is self-adjoint} \\
        & = A,
\end{align*}
which implies \(A\) and \(A^*\) are both upper triangular.
But (trivially) that only happens when \(A\) is in fact a diagonal matrix.
Thus \(\beta\) must consist of eigenvectors of \(\T\).

Now suppose there exists a basis \(\beta\) for \(\V\) (over \(\SET{R}\)) such that it is orthonormal and consists of eigenvectors of \(\T\).
Then in particular, \(A = [\T]_{\beta}\) \MAROON{(1)} is a diagonal matrix, and the diagonal entries, by \CH{5}, are eigenvalues of \(\T\), which implies \emph{they must be real numbers}, hence we have \(A^* = A\). \MAROON{(2)}
But that implies
\begin{align*}
    [\T^*]_{\beta} & = ([\T]_{\beta})^* & \text{by \THM{6.10}} \\
        & = A^* & \text{by \MAROON{(1)}} \\
        & = A & \text{by \MAROON{(2)}} \\
        & = [\T]_{\beta}, & \text{by \MAROON{(1)} again}
\end{align*}
which implies \(\T^* = \T\), hence \(\T\) is self-adjoint.
\end{proof}

\begin{example} \label{example 6.4.4}
As we noted earlier, \emph{real} symmetric matrices are self-adjoint, and self-adjoint matrices are normal.
The following matrix \(A\) is \emph{complex} and symmetric:
\[
    A = \begin{pmatrix} \iu & \iu \\ \iu & 1 \end{pmatrix} \quad \text{ and } \quad A^* = \begin{pmatrix} -\iu & -\iu \\ -\iu & 1 \end{pmatrix}.
\]
But \(A\) is \emph{not} normal, because \((AA^*)_{12} = 1 + \iu\) and \((A^*A)_{12} = 1 - \iu\).
Therefore complex symmetric matrices need not be normal.
\end{example}
