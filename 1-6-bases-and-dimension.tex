\section{Bases and Dimension} \label{sec 1.6}

\begin{note}
注意,Bases 是\ Basis 的複數。
\end{note}

We saw in \SEC{1.5} that if \(S\) is a generating set for a subspace \(\W\) (i.e. \(\spann(S) = \W\)) and no \emph{proper} subset of \(S\) is a generating set for \(\W\), then \(S\) must be \LID{}.
A \LID{} generating set for \(\W\) possesses a very useful property - every vector in \(\W\) can be \emph{expressed in one and only one way} as a linear combination of the vectors in the generating set.
(This property is proved below in \THM{1.8}; also see some related \EXEC{1.4.16})
It is this property that makes \LID{}
generating sets the \emph{building blocks} of vector spaces.

\begin{definition} \label{def 1.8}
A \textbf{basis} \(\beta\) for a vector space \(\V\) is a \BLUE{(1)} \LID{} subset of \(\V\) \BLUE{(2)} that generates \(\V\).
If \(\beta\) is a basis for \(\V\), we also say that the
\emph{vectors of} \(\beta\) form a basis for \(\V\).
\end{definition}

\begin{example} \label{example 1.6.1}
Recalling that \(\spann(\emptyset) = \{ \OV \}\) (by \DEF{1.4}) and \(\emptyset\) is \LID{} (by \ATHM{1.15}(a)), we see that \(\emptyset\) is a basis for the zero vector space.
\end{example}

\begin{example} \label{example 1.6.2}
In \(F^n\), let \(e_1 = (1, 0, 0, ..., 0), e_2 = (0, 1, 0, ..., 0), ..., e_n = (0, 0, ..., 0, 1)\);

\(\{ e_1, e_2, ..., e_n \}\) is readily seen to be a basis(\LID{} by \EXEC{1.5.4}, generating set of \(F^n\) by \EXEC{1.4.7}) for \(F^n\) and is called the \textbf{standard basis} for \(F^n\).
\end{example}

\begin{example} \label{example 1.6.3}
In \(M_{m \X n}(F)\), let \(E_{ij}\) denote the matrix whose only nonzero entry is a \(1\) in the \(i\)th row and \(j\)th column.
Then \(\{ E_{ij} : 1 \le i \le m, 1 \le j \le n \}\) is a (standard?) basis for \(M_{m \X n}(F)\).
\end{example}

\begin{example} \label{example 1.6.4}
In \(\POLYNF\), the set \(\{ 1, x, x^2, ..., x^n \}\) is a basis.
We call this basis the \textbf{standard basis} for \(\POLYNF\).
\end{example}

\begin{additional definition} \label{adef 1.11}
A placeholder for the meaning of \textbf{standard basis} of various vector spaces, \(F^n\), \(M_{m \X n}(F)\), \(\POLYNF\).
\end{additional definition}

\begin{example} \label{example 1.6.5}
In \(\POLYF\), the set \(\{1, x, x^2, ... \}\) is a basis.
\end{example}

\begin{remark} \label{remark 1.6.1}
Observe that \EXAMPLE{1.6.5} shows that a basis (set) \emph{need not be finite}.
In fact, later in this section it is shown that no basis for \(\POLYF\) can be finite.
Hence \emph{not} every vector space has a finite basis.
\end{remark}

The next theorem, which is used frequently in \CH{2}, establishes the most significant property of a basis.

\begin{theorem} \label{thm 1.8}
Let \(\V\) be a vector space and \(u_1, u_2, ..., u_n\) be distinct vectors in \(\V\).
Then \(\beta = \{ u_1, u_2, ..., u_n \}\) is a basis for \(\V\) if and only if each \(v \in \V\) can be \textbf{uniquely} (i.e. one and only one) expressed as a linear combination of vectors of \(\beta\), that is, can be expressed in the form
\[
    v = a_1 u_1 + a_2 u_2 + ... + a_n u_n
\]
for unique scalars \(a_1, a_2, ..., a_n\).
\end{theorem}

\begin{proof}\ 

\(\Longrightarrow\):
Let \(\beta\) be a basis for \(\V\).
Then by \DEF{1.8}, \(\spann(\beta) = \V\) and \(\beta\) is \LID{}.
Then let arbitrary \(v \in \V\).
Thus \(v\) is a linear combination of the vectors of \(\beta\).
(Hence the existence part is proved.)
Now we prove the combination is unique.
Suppose that
\[
    v = a_1 u_1 + a_2 u_2 + ... + a_n u_n \text{ and } v = b_1 u_1 + b_2 u_2 + ... + b_n u_n. \MAROON{(1)}
\]
Then from \MAROON{(1)} we have
\begin{align*}
    \OV & = v - v \\
        & = a_1 u_1 + ... + a_n u_n - (b_1 u_1 + ... + b_n u_n) & \text{by \MAROON{(1)}} \\
        & = (a_1 - b_1) u_1 + ... + (a_n - b_n) u_n.
\end{align*}
But since \(\beta\) is \LID{}, we have \(a_1 - b_1 = a_2 - b_2 = ... = a_n - b_n = 0\).
Hence \(a_1 = b_1, a_2 = b_2, ..., a_n = b_n\), hence the linear combination is unique.

\(\Longleftarrow\):
Suppose any \(v \in \V\) can be uniquely represented by vectors in \(\beta\).
Then in particular, this means \(\V \subseteq \spann(\beta)\), hence \(\V = \spann(\beta)\) (another direction is automatically true).

Also in particular, \(\OV\) can be uniquely represented:
suppose \(\OV = a_1 u_1 + a_2 u_2 + ... + a_n u_n\).
But \(\OV\) already has a \textbf{trivial representation}, which implies
\[
    \OV = a_1 u_1 + a_2 u_2 + ... + a_n u_n = 0 u_1 + 0 u_2 + ... + 0 u_n,
\]
that is, \(a_1 = a_2 = ... = a_n = 0\).
Hence \(\beta\) is \LID{}.
So \(\beta\) is both \LID{} and a generating set of \(\V\), hence by \DEF{1.8} is a basis of \(\V\).
\end{proof}

\begin{note}
\THM{1.8} shows that if the vectors \(u_1, u_2, ..., u_n\) form a basis for a vector space \(\V\), then every vector in \(\V\) can be \emph{uniquely} expressed in the form
\[
    v = a_1 u_2 + a_2 u_2 + ... + a_n u_n
\]
for appropriately chosen scalars \(a_1, a_2, ..., a_n\).
Thus each \(v \in \V\) determines a unique \(n\)-tuple of scalars \((a_1, a_2, ..., a_n)\) and,
conversely, each \(n\)-tuple of scalars determines a unique vector \(v \in \V\) by using the entries of the \(n\)-tuple as the coefficients of a linear combination of \(u_1, u_2, ..., u_n\).
This fact suggests that \textbf{\(\V\) is like the vector space \(F^n\)}, where \(n\) is the number of vectors in the basis for \(\V\).
We see in \SEC{2.4} that this is indeed the case.
\end{note}

In this book, we are primarily interested in vector spaces \textbf{having finite bases}.
\THM{1.9} identifies a large class of vector spaces of this type.

\begin{theorem} \label{thm 1.9}
If a vector space \(\V\) is generated by a finite set \(S\), then some \(subset\) of \(S\) is a basis for \(\V\).
Hence \(\V\) has a finite basis.
\end{theorem}

\begin{proof}
Suppose \(\spann(S) = \V\) \BLUE{(1)} and \(S\) is a finite set.
We prove by cases of the elements of \(S\): \(S\) is empty set, \(S = \{ \OV \}\), and otherwise(that is, \(S\) has nonzero vector).

If \(S = \emptyset\), then by \DEF{1.4}, \(\spann(S) = \{ \OV \}\).
That is, by \BLUE{(1)}, \(\V = \{ \OV \}\).
Also by \ATHM{1.15}(a), \(S\) is \LID{}.
So with \BLUE{(1)} and \(S\) is \LID{}, by \DEF{1.8}, we have found a finite subset of \(S\), namely itself, which is a basis of \(\V\).

If \(S = \{ \OV \}\), then \(S\) is itself a subspace, hence \(\spann(S) = S\) by \ATHM{1.11}.
That is, by \BLUE{(1)}, \(\V = \spann(S) = S = \{ \OV \}\).
And by \DEF{1.4}, \(\spann(\emptyset) = \{ \OV \} = \V\);
by \ATHM{1.15}(a), \(\emptyset\) is \LID{}.
So we have found a finite subset of \(S\), namely \(\emptyset\), which is a basis of \(\V\).

Now suppose \(S\) has nonzero vector.
We pick it as \(u_1\).
Then by \ATHM{1.15}(b), \(\{ u_1 \}\) is \LID{}.
Continue, if possible, choosing vectors \(u_2, ..., u_k\) in \(S\) such that \(\{ u_1, u_2, ..., u_k \}\) is still a \LID{} set of \(k\) vectors. 

The process when we have \(k'\) elements is like: look \emph{each} remaining element that has \emph{not} been checked, and check whether the union of the \(k'\) elements and this elements is still \LID{};
if yes, we pick this element;
if no, we check next element;
if all the remaining unchecked elements cannot be added, then we ``break the loop''(end the process).

Since \(S\) is a finite set, this choosing process must end with a \LID{} set \(\beta = \{ u_1, u_2, ..., u_n \}\) having \(n\) elements.
There are two ways this could happen:
\begin{enumerate}
\item[(i)] The set \(\beta = S\).
    In this case, \(S\) is both a \LID{}(by the choosing process) set and a generating set for \(\V\) (by the supposition, \BLUE{(1)}).
    That is, \(S\) is itself a (finite) basis for \(\V\).

\item [(ii)] The set \(\beta\) is a \emph{proper} \LID{} subset of \(S\) such that if we add the remaining vectors in \(S\) which are not in \(\beta\), we will get a \LDP{} set (this happens when we ``break the loop''.
\end{enumerate}
In the second case, we claim that \(\beta\) is the desired subset of \(S\) that is a basis for \(\V\).
Because \(\beta\) is \LID{} by construction (of the choosing process), it suffices to show that \(\beta\) spans \(\V\) (so by \DEF{1.8} \(\beta\) is a basis of \(\V\)).

But by \THM{1.5}, we only need to show that \(S \subseteq \spann(\beta)\).
Why?
\begin{align*}
             & S \subseteq \spann(\beta) \\
    \implies & \spann(S) \subseteq \spann(\beta) & \text{by \THM{1.5}}(2) \\
    \implies & V \subseteq \spann(\beta) & \text{by the supposition, \BLUE{(1)}} \\
    \implies & V = \spann(\beta) & \text{the other direction is automatically true}
\end{align*}
So let arbitrary \(v \in S\), we have to show \(v \in \spann(\beta)\) to show \(S \subseteq \spann(\beta)\).
Then there are two cases, \(v \in \beta\) or \(v \notin \beta\).
If \(v \in \beta\), then clearly \(v \in \spann(\beta)\).
Otherwise, if \(v \notin \beta\), then (since \(v \in S\)) the preceding construction shows that \(\beta \cup \{ v \}\) is \LDP{}.
So by \THM{1.7}, since \(\beta\) is \LID{} by construction and \(\beta \cup \{ v \}\) is \LDP{}, \(v \in \spann(\beta)\).
So in all cases, \(v \in \spann(\beta)\).
Thus \(S \subseteq \spann(\beta)\), as desired.
\end{proof}

\begin{remark} \label{remark 1.6.2}
Because of the method by which the basis \(\beta\) was obtained in the proof of \THM{1.9}, this theorem is often remembered as saying that
\begin{center}
    A finite spanning set for \(\V\) can be \emph{reduced} to a basis for \(\V\).
\end{center}
This method is illustrated in the next example.
\end{remark}

\begin{example} \label{example 1.6.6}
Let
\[
    S = \{ (2, -3, 5), (8, -12, 20), (1, 0, -2), (0, 2, -1), (7, 2, 0) \}.
\]
It can be shown that \(S\) generates \(\SET{R}^3\).
We can select a basis for \(\SET{R}^3\) that is a \emph{subset} of \(S\) by the technique used in proving \THM{1.9}.

To start, select any \emph{nonzero} vector in \(S\), say \((2, -3, 5)\), to be a vector in the basis.
Since \(4(2, -3, 5) = (8 , -12, 20)\), the set \(\{ (2, -3, 5), (8 , -12, 20) \}\) is \LDP{} by \ATHM{1.16}.
Hence we do not include \((8, -12, 20)\) in our basis.
On the other hand, \((1, 0, -2)\) is not a multiple of \((2, -3, 5)\) and vice versa, so (again by \ATHM{1.16}) that the set \(\{ (2, -3, 5), (1, 0, -2) \}\) is \LID{}.
Thus we include \((1, 0, -2)\) as part of our basis.
Now we consider the set \(\{ (2, -3, 5), (1, 0, -2), (0, 2, -1) \}\) obtained by adjoining another vector \((0, 2, -1)\) in \(S\) to the two vectors that we have already included in our basis.
As before, we include \((0, 2, -1)\) in our basis or exclude it from the basis according to whether \(\{ (2, -3, 5), (1, 0, -2), (0, 2, -1) \}\) is \LID{} or \LDP{}.
An easy calculation shows that this set is \LID{}, so we include \((0, 2, -1)\) in our basis.
In a similar fashion the final vector in \(S\) is included or excluded from our basis according to whether the set
\[
    \{ (2, -3, 5), (1, 0, -2), (0, 2, -1), (7, 2, 0) \}
\]
is \LID{} or \LDP{}.
Because
\[
    2(2, -3, 5) + 3(1, 0, -2) + 4(0, 2, -1) - (7, 2, 0) = (0, 0, 0),
\]
we exclude \((7, 2, 0)\) from our basis.
We conclude (by \THM{1.9}) that
\[
    \{(2, -3, 5), (1, 0, -2), (0, 2, -1)\}
\]
is a \emph{subset} of \(S\) that is a basis for \(\SET{R}^3\).
\end{example}

The corollaries of the following theorem are perhaps the most significant results in \CH{1}.
(That is, \THM{1.10} is somewhat difficult to use, but \CORO{1.10.1}, \CORO{1.10.2}, \CORO{1.10.3} are more useful.)

\begin{theorem} [Replacement Theorem] \label{thm 1.10}
Let \(\V\) be a vector space that is generated by a set \(G\) containing exactly \(n\) vectors.
and let \(L\) be a \LID{} \emph{subset} of \(\V\) containing exactly \(m\) vectors.

Then \BLUE{(1)} \(m \le n\) \BLUE{(2)} and there \emph{exists} a \emph{subset} \(H\) of \(G\) containing exactly \(n - m\) vectors such that \(L \cup H\) generates \(\V\).
\end{theorem}

\begin{note}
Note that we just require \(L \cup H\) to generate \(\V\).
We do not require \(L \cup H\) to be \LID{} or something else.
\end{note}

\begin{note}
會叫做\ replacement theorem,是因為這邊實際上是把\ \(G\) 的部分\ elements 拔掉,拔到只剩\ \(H\);而被拔掉的那些\ elements 改用\ \(L\),然後\ \(L \cup H\) 一樣可以\ generate \(\V\)。
\end{note}

\begin{proof}

The proof is by mathematical induction on \(m\).
The induction begins with \(m = 0\); for in this case \(L\) is a subset of \(\V\) that contains zero vectors; that is, \(L = \emptyset\), so \(L\) is \LID{} by \ATHM{1.15}(a).
Then taking \(H = G\), we have \(L \cup H = L \cup G = \emptyset \cup G = G\), so \(\spann(L \cup H) = \spann(G) = \V\), as desired.
(In this case, the intuition is that we do not ``replace anything'' in \(G\) using \(L\), since \(L\) is empty.)

Now suppose inductively that the theorem is true for some integer \(m \ge 0\).
That is, if \(L\) is a \LID{} subset with \(m\) vectors, we have \(m \le n\) and there exists a subset \(H\) of \(G\) containing exactly \(n - m\) vectors such that \(L \cup H\) generates \(\V\).
We prove that the theorem is true for \(m + 1\).
That is, if \(L'\) is a \LID{} subset with \(m + 1\) vectors, we have \(m + 1 \le n\) and there exists a subset \(H'\) of \(G\) containing exactly \(n - (m + 1)\) vectors such that \(L' \cup H'\) generates \(\V\).

(BTW, note that if there is no \LID{} subset \(L\)(or \(L'\)) with \(m\)(or \(m + 1\)) vectors, then the hypothesis of these if-then statements is false, hence the whole if-then statement is \emph{vacuously} true).

So let \(L' = \{ v_1, v_2, ..., v_{m + 1} \}\) be a \LID{} subset of \(\V\) consisting of \(m + 1\) vectors \BLUE{(*)}.

By \CORO{1.6.1}, since \(\{ v_1, v_2, ..., v_m \} \subseteq \{ v_1, v_2, ..., v_{m + 1} \}\) and \(\{ v_1, v_2, ..., v_{m + 1} \}\) is \LID{}, \(\{ v_1, v_2, ..., v_m \}\) is \LID{} \textbf{having \(m\) elements}.
So we may apply the induction hypothesis to conclude that \(m \le n\) and that there is a subset \(\{ u_1, u_2, ..., u_{n - m} \}\) of \(G\) (containing exactly \(n - m\) vectors) such that \(\{ v_1, v_2, ..., v_m \} \cup \{u_1, u_2, ..., u_{n - m} \}\) generates \(\V\) \BLUE{(1)}.

And in particular, since \(v_{m + 1} \in \V\), by definition of span and \BLUE{(1)}, we have
\[
    a_1 v_1 + a_2 v_2 + ... + a_m v_m + b_1 u_1 + b_2 u_2 + ... + b_{n - m} u_{n - m} = v_{m + 1} \MAROON{(1)}
\]
for some scalar \(a_1, ..., a_m, b_1, ..., b_{n - m}\).

\textbf{Note that} \(n - m > 0\), for if \(n - m \le 0\), then the subset \(\{ u_1, u_2, ..., u_{n - m} \}\) of \(G\) will in fact be empty set(when \(n = m\)) or even not well-defined(when \(n < m\); in this case we have asserted some gibberish that there exists a set that have negative number of elements);
and if \(\{ u_1, u_2, ..., u_{n - m} \}\) is in fact empty, then from \MAROON{(1)} we conclude that \(v_{m + 1}\) is in fact in \(\spann(\{ v_1, v_2, ..., v_m \})\), and by \ATHM{1.17}(1) that \(\{ v_{m + 1}, v_1, v_2, ..., v_m \} = L'\) is \LDP{}, which is a contradiction to \BLUE{(*)}.
So it must be that \(n - m > 0\), or \(n > m\);
in particular, \(n \ge m + 1\).

Moreover, some \(b_i\), say \(b_1\), must be nonzero, for otherwise we obtain the same contradiction.
And solving \MAROON{(1)} gives
\begin{align*}
    u_1 = (-b_1^{-1} a_1) v_1 + (-b_1^{-1} a_2) v_2 + ... + (-b_1^{-1} a_m) v_m + (-b_1^{-1}) v_{m + 1} \\
        + (-b_1^{-1} b_{\RED{2}}) u_{\RED{2}} + ... + (-b^{-1}_1 b_{n - m}) u_{n - m}. \MAROON{(2)}
\end{align*}
Now let \(H' = \{ u_2, ..., u_{n - m} \}\).
Then from the nasty \MAROON{(2)} above we can find that \(\GREEN{u_1} \in \spann(L' \cup H')\).
Also, because \(v_1, v_2, ..., v_m, u_{\RED{2}}, ..., u_{m - n}\) are clearly in \(\spann(L' \cup H')\), it follows that
\[
    \{ v_1, v_2, ..., v_m, u_{\RED{2}}, ..., u_{m - n}\} \cup \{ \GREEN{u_1} \} \subseteq \spann(L' \cup H');
\]
that is,
\[
    \{ v_1, v_2, ..., v_m, \GREEN{u_1}, u_{\RED{2}}, ..., u_{m - n}\} \subseteq \spann(L' \cup H'). \MAROON{(3)}
\]
But because \(\{ v_l, v_2, ..., v_m, u_1, u_2, ..., u_{n - m} \}\) -- the LHS of the set inclusion \MAROON{(3)} -- generates \(\V\) (by \BLUE{(1)}),
\THM{1.5} implies that the RHS \(\spann(L' \cup H')\), which is a subspace of \(\V\), must contain the span of the LHS, that is, \(\V\).
So we have \(\spann(L' \cup H') = \V\).
Since we have shown that for the \LID{} set \(L'\) with \(m + 1\) elements,
\begin{itemize}
    \item \(m + 1 \le n\),
    \item \(H'\) is a subset of \(G\) that contains \((n - m) - 1 = n - (m + 1)\) vectors such that \(\spann(L' \cup H') = \V\),
\end{itemize}
the theorem is true for \(m + 1\).
This completes the induction.
\end{proof}

\begin{corollary} \label{corollary 1.10.1}
Let \(\V\) be a vector space having a \emph{finite} basis.
Then \emph{all bases} for \(\V\) are finite, and every basis for \(\V\) contains the \emph{same number of} vectors.
\end{corollary}

\begin{proof}
Let \(\beta\) be a finite basis for \(\V\), (the existence of \(\beta\) is guaranteed by supposition).
Let \(\gamma\) be an arbitrary basis for \(\V\).
Then in particular by \DEF{1.8}, \(\gamma\) is also a \LID{} subset, and \(\beta\) generates \(\V\), so by \THM{1.10}, we have \(\#(\gamma) \le \#(\beta)\).

But similarly, since \(\beta\) is also a \LID{} subset, and \(\gamma\) generates \(\V\), so again by \THM{1.10}, we have \(\#(\beta) \le \#(\gamma)\).

Together we have \(\#(\beta) = \#(\gamma)\).
Since \(\gamma\) is arbitrary, any basis of \(\V\) has the same number of elements of \(\beta\) and hence finite.
\end{proof}

\begin{remark} \label{remark 1.6.3}
If a vector space has a finite basis, \CORO{1.10.1} asserts that the \emph{number of vectors in any basis} for \(\V\) is an \emph{intrinsic property} of \(\V\).
This fact makes possible the following important (well-defined) definitions.
\end{remark}

\begin{definition} \label{def 1.9}
A vector space is called \textbf{finite-dimensional} if it has a basis consisting of a finite number of vectors.
The \emph{unique}(by \CORO{1.10.1}) integer \(n\) such that every basis for \(\V\) contains exactly \(n\) elements is called the \textbf{dimension} of \(\V\) and is denoted by \(\dim(\V)\).
A vector space that is \emph{not} finite-dimensional is called \textbf{infinite-dimensional}.
\end{definition}

\begin{corollary} \label{corollary 1.10.2}
An immediate conclusion in the replacement theorem states that if \(\V\) is a finite-dimensional vector space, then no \LID{} subset of \V can contain more than \(\dim(\V)\) vectors.
\end{corollary}

\begin{example} \label{example 1.6.7}
The vector space \(\{ \OV \}\) has dimension zero, since it has a basis \(\emptyset\), which has zero elements.
\end{example}

\begin{example} \label{example 1.6.8}
The vector space \(F^n\) has dimension \(n\), since \(\{ e_1, e_2, ..., e_n \}\), which has \(n\) elements, is a basis of \(F^n\).
\end{example}

\begin{example} \label{example 1.6.9}
The vector space \(M_{m \X n}(F)\) has dimension \(mn\), since by \EXAMPLE{1.6.3}, \(\{ E_{ij} : 1 \le i \le m, 1 \le j \le n \}\), which has \(mn\) elements, is a basis for \(M_{m \X n}(F)\).
\end{example}

\begin{example} \label{example 1.6.10}
The vector space \(\POLYNF\) has dimension \(n + 1\), since by \EXAMPLE{1.6.4}, \\
\(\{ 1, x, x^2, ..., x^n \}\), which has \(n + 1\) elements, is a basis for \(\POLYNF\).
\end{example}

The following examples show that \textbf{the dimension} of a vector space \textbf{depends on its field of scalars}.

\begin{example} \label{example 1.6.11}
Over the field of complex numbers, the vector space of complex numbers has dimension \(1\).
(A basis is \(\{ 1 \}\); given any complex number \(c\), \(c = c \X 1\), where the second \(c\) means a scalar from the complex number field \(\SET{C}\).)
\end{example}

\begin{example} \label{example 1.6.12}
Over the field of \textit{real numbers}, the vector space of complex numbers has dimension \(2\).
(A basis is \(\{ 1, i \}\).)
\end{example}

\begin{remark} \label{remark 1.6.4}
Another mind=blown example: the vector space of real number over \emph{rational number}.
It's infinite dimensional, and even uncountable dimensional.
\end{remark}

\begin{example} \label{example 1.6.13}
The vector space \(\POLYF\) is infinite-dimensional because, by \EXAMPLE{1.6.5}, it has an infinite \LID{} set, namely \(\{ 1, x, x^2, ... \}\).
Hence by (contrapositive of) \CORO{1.10.2}, \(\POLYF\) is infinite-dimensional.
\end{example}

In \EXAMPLE{1.6.13}, the \emph{infinite} \LID{} set \(\{1, x, x^2, ... \}\) is, in fact, a basis for \(\POLYF\).
Yet \textbf{nothing} that we have proved in this section guarantees an infinite-dimensional vector space must have a basis.
In \SEC{1.7} it is shown, however, that \textbf{every} vector space has basis. (It in fact depends on \textbf{Axiom of Choice}.)

Just as no \LID{} subset of a finite-dimensional vector space \(\V\) can contain more than \(\dim(\V)\) vectors, a corresponding statement can be
made about the size of a generating set.

\begin{corollary} \label{corollary 1.10.3}
Let V be a vector space with dimension n.
\begin{enumerate}
\item Any finite generating set for \(\V\) contains \emph{at least} \(n\) vectors, and a generating set for \(\V\) that contains \emph{exactly} \(n\) vectors is a basis for \(\V\).
\item Any \LID{} subset of \(\V\) that contains exactly \(n\) vectors is a basis for \(\V\).
\item Every \LID{} subset of \(\V\) can be \textit{extended to} a basis for \(\V\),
    that is, if \(L\) is a \LID{} subset of \(\V\), then there is a basis \(\beta\) of \(\V\) such that \(L \subseteq \beta\).
\end{enumerate}
\end{corollary}

\begin{proof}\ 

Let \(\beta\) be a basis for \(\V\).
\begin{enumerate}
\item Let \(G\) be a finite generating set for \(\V\) (\(\spann(G) = \V\)).
    By \THM{1.9} some subset \(H\) of \(G\) is a basis for \(\V\). \CORO{1.10.1} implies that \(H\) contains exactly \(n\) vectors.
    Since a subset of \(G\) contains \(n\) vectors, \(G\) must contain at least \(n\) vectors.
    
    Moreover, if \(G\) contains exactly \(n\) vectors, then we must have \(H = G\) (if a (finite) set's subset has the same element as the set, then the subset is in fact equal to the set), so that \(G\) is a basis for \(\V\).
\item Let \(L\) be a \LID{} subset of \(\V\) containing exactly \(\BLUE{n}\) vectors.
    It follows from the replacement theorem \THM{1.10} that there is a subset \(H\) of \(\beta\) containing \(n - \BLUE{n} = 0\) vectors such that \(L \cup H\) generates \(\V\).
    Thus \(H = \emptyset\), and \(L \cup H = L \cup \emptyset = L\) generates \(\V\). Since \(L\) is also \LID{}, \(L\) is a basis for \(\V\).
\item If \(L\) is a \LID{} subset of \(\V\) containing \(m\) vectors, then the replacement theorem \THM{1.10} asserts that there is a subset \(H\) of \(\beta\) containing exactly \(n - m\) vectors such that \(L \cup H\) generates \(\V\).
    Now \(L \cup H\) contains \emph{at most} \(m + (n - m) = n\) vectors (sine \(L\) and \(H\) may contain same elements);
    but part(a) also implies that the generating set \(L \cup H\) contains \emph{at least} \(n\) elements, so in fact \(L \cup H\) contains \emph{exactly} \(n\) elements, and again by (second statement of) part(a), \(L \cup H\) is a basis for \(\V\).
    So we have found a basis \(L \cup H\) that contains \(L\), as desired.
\end{enumerate}
\end{proof}

\begin{example} \label{1.6.14}
It follows from \EXAMPLE{1.4.4} that
\[
    \{ x^2 + 3x - 2, 2x^2 + 5x - 3, -x^2 -4x + 4 \}
\]
generates \(\POLYRR\) and has \(3\) elements, same as the dimension of \(\POLYRR\).
Hence by \CORO{1.10.3}(a) the set is a basis for \(\POLYRR\).
\end{example}

\begin{example} \label{example 1.6.15}
(Similarly to the previous example) It follows from \EXAMPLE{1.4.5} of and \CORO{1.10.3}(a) that
\[
\bigg\{
    \begin{pmatrix}
      1 & 1 \\
      1 & 0
    \end{pmatrix},
    \begin{pmatrix}
      1 & 1 \\
      0 & 1
    \end{pmatrix},
    \begin{pmatrix}
      1 & 0 \\
      1 & 1
    \end{pmatrix},
    \begin{pmatrix}
      0 & 1 \\
      1 & 1
    \end{pmatrix}
\bigg\}
\]
is a basis for \(M_{2 \X 2}(\SET{R})\).
\end{example}

\begin{example} \label{example 1.6.16}
It follows from \EXAMPLE{1.5.3} that
\[
    \{ (1, 0, 0, -1), (0, 1, 0, -1), (0, 0, 1, -1), (0, 0, 0, 1) \}
\]
is \LID{}.
But it has \(4\) elements, same as the dimension of \(\SET{R}^4\), hence by \CORO{1.10.3}(b) is a basis for \(\SET{R}^4\).
Note that currently we do not prove the set actually generates \(\SET{R}^4\) to show it's a basis, but instead use the fact it the number of elements is equal to the dimension of the vector space to derive it is a basis for \(\SET{R}^4\).
\end{example}

\begin{example} \label{example 1.6.17}
(Similarly to the previous example) For \(k = 0, 1, ..., n\), let \(p_k(x) = x^k + x^{k + 1} + ... + x^n\).
It follows from \EXAMPLE{1.5.4} and \CORO{1.10.3}(b) that
\(
    \{ p_0(x), p_1(x), ..., p_n(x) \}
\)
is a basis for \(\POLYNF\).
\end{example}

\begin{remark} \label{remark 1.6.5}
A procedure for reducing a generating set to a basis was illustrated in \EXAMPLE{1.6.6}.
\TODOREF{} In \SEC{3.4}, when we have learned more about solving systems of linear equations, we will discover a simpler method for reducing a generating set to a basis.
This procedure also can be used to \emph{extend} a linearly
independent set to a basis, which is always possible by the assertion of \CORO{1.10.3}(c).
\end{remark}

\subsection{An Overview of Dimension and Its Consequences}
\THM{1.9} as well as the replacement theorem \THM{1.10} and its corollaries \CORO{1.10.1}, \CORO{1.10.2}, \CORO{1.10.3}
contain a wealth of information about the relationships among \LID{} sets, bases, and generating sets.
For this reason, we summarize here the main results of this section in order to put them into better perspective.
\begin{itemize}
\item A basis for a vector space \(\V\) is a \LID{} subset of \(\V\) that generates \(\V\).
\item If \(\V\) has a finite basis, then every basis for \(\V\) contains \emph{the same} number of vectors.
    This number is called the dimension of \(\V\), and \(\V\) is said to be finite-dimensional.
    Thus if the dimension of \(\V\) is \(n\), \emph{every} basis for \(\V\) contains exactly \(n\) vectors.
\item Moreover, every \emph{\LID{}} subset of \(\V\) contains no more than \(n\) vectors and \emph{can be extended to} a basis for \(\V\) by including appropriately chosen vectors.
\item Also, each \emph{generating set} for \(\V\) contains at least \(n\) vectors and \emph{can be reduced to} a basis for \(\V\) by excluding appropriately chosen vectors.
\end{itemize}

\subsection{The Dimension of \emph{Subspaces}}
Our next result relates the dimension of a \emph{subspace} to the dimension of the vector space that contains it.

\begin{theorem} \label{thm 1.11}
Let \(\W\) be a subspace of a \emph{finite}-dimensional vector space \(\V\).
Then \BLUE{(1)} \(\W\) is finite-dimensional and \(\dim(\W) \le \dim(\V)\).
\BLUE{(2)} Moreover, if \(\dim(\W) = \dim(\V)\), then \(\V = \W\).
\end{theorem}

\begin{proof}
Let \(\W\) be a subspace of a finite-dimensional vector space \(\V\).
Let \(\dim(\V) = n\).
If \(\W = \{ \OV \}\), then \(\W\) is finite-dimensional
(since it has \(\emptyset\) as a basis having zero, i.e. finite, elements)
and \(\dim(\W) = 0 \le n\).
Otherwise, \(\W\) contains a nonzero vector \(x_1\);
so \(\{ x_l \}\) is a \LID{} set. (by \ATHM{1.15}(b).)
Continue choosing vectors, \(x_1, x_2 , ..., x_k \in \W\) such that \(\{ x_1, x_2, ..., x_k \}\) is \LID{}.
(the process is identical to the process in \THM{1.9})
Since no \LID{} subset of \(\V\) can contain more than \(n\) vectors, this process must stop at a stage where \(k \le n\) \MAROON{(1)} and \(\beta = \{ x_1, x_2, ..., x_k \}\) is \LID{} but adjoining \emph{any other vector \(w\) from} \(\W\)(i.e. \(w \in \W \setminus \beta\)) produces a linearly \emph{dependent} set.
So by \THM{1.7} we have \(w \in \spann(\beta)\).
Since \(w \in \W \setminus \beta\) is arbitrary element, we have \(\W \setminus \beta \in \spann(\beta)\).
But since of course \(\beta \in \spann(\beta)\) we have \((\W \setminus \beta) \cup \beta \in \spann(\beta)\).
That is, \(\W \in \spann(\beta)\).
But since \(\beta \subseteq \W\) (because \(x_1, x_2, ... x_k\) are chosen from \(\W\)), by \THM{1.5} we have \(\spann(\beta) \subseteq \W\).
So all in all we have \(\W = \spann(\beta)\).
Therefore, \(\beta\) is a \LID{} set that generates \(\W\), hence is a basis of \(\W\), so \(\dim(\W) = k\) and by \MAROON{(1)}, \(\dim(\W) = k \le n = \dim(\V)\).
If \(\dim(\W) = n\), then a basis \(\beta\) for \(\W\) is \(a\) \LID{} subset of \(\V\) containing \(n\) vectors.
But \CORO{1.10.3}(b) implies that the basis \(\beta\) for \(\W\) is also a basis for \(\V\);
So (by \DEF{1.8}) we have \(\spann(\beta) = \W\) and \(\spann(\beta) = \V\);
that is, \(\W = \V\).
\end{proof}

\begin{example} \label{example 1.6.18}
Let
\[
    \W = \{ ( a_1, a_2, a_3, a_4, a_5) \in F^5 : a_1 + a_3 + a_5 = 0, a_2 = a_4 \}.
\]
\sloppy It is easily shown(easy if you use \CH{3}) that \(\W\) is a subspace of \(F^5\) having \( (-1, 0, 1, 0, 0), (-1, 0, 0, 0, 1), (0, 1, 0,1, 0)\}\) as a basis.
Thus \(\dim(\W) = 3\).
\end{example}

\begin{example} \label{example 1.6.19}
The set of diagonal \(n \X n\) matrices is a subspace \(\W\) of \(M_{n \X n}(F)\) (See \EXAMPLE{1.3.3}).
A basis for \(\W\) is
\[
    \{ E_{11}, E_{22}, ..., E_{nn} \},
\]
having \(n\) elements where \(E_{ij}\) is the matrix in which the only \emph{nonzero} entry is a \(1\) in the \(i\)th row and \(j\)th column.
Thus \(\dim(\W) = n\).
\end{example}

\begin{example} \label{example 1.6.20}
We saw in \ATHM{1.1} that the set of symmetric \(n \X n\) matrices is a subspace \(\W\) of \(M_{n \X n}(F)\).
A basis for \(\W\) is
\[
    \{ A_{ij} : 1 \le i \le j \le n \},
\]
where \textbf{\(A_{ij}\) is the \(n \X n\) matrix having \(1\) in the \(i\)th row and \(j\)th column, \(1\) in the \(j\)th row and \(i\)th column}, and \(0\) elsewhere.
It follows that
\[
    \dim(\W) = n + (n - 1) + ... + 1 = \frac{1}{2}n(n + 1). 
\]
\end{example}

\begin{corollary} \label{corollary 1.11.1}
If \(\W\) is a subspace of a finite-dimensional vector space \(\V\), then any basis for \(\W\) can be extended to a basis for \(\V\).
\end{corollary}

\begin{proof}
Let \(S\) be a basis for \(\W\).
Because \(S\) is a \LID{} subset of \(\V\), \CORO{1.10.3}(c) guarantees that \(S\) can be extended to a basis for \(\V\).
\end{proof}

\begin{example} \label{example 1.6.21}
The set of all polynomials of the form
\[
    a_{18} x^{18} + a_{16} x^{16} + ... + a_2 x^2 + a_0,
\]
where \(a_{18}, a_{16}, ..., a_2, a_0 \in F\), is a subspace \(\W\) of \(\mathcal{P}_{18}(F)\).
A basis for \(\W\) is \(\{ 1, x^2, ..., x^{16}, x^{18} \}\), which is a subset of the \emph{standard basis} for \(\mathcal{P}_{18}(F)\).
\end{example}

\subsection{The Lagrange Interpolation Formula}
In many applications, we have a collection of data obtained from experiments or samples;
e.g. we may know the locations of an airplane at certain (\textbf{distinct}) times:
\begin{align*}
    t_1 = l_1,
    t_2 = l_2,
    ...,
    t_n = l_n,
\end{align*}
where \(t_i\) is a particular time point and \(l_i\) is the corresponding location for the time point.
And we want to \textbf{approximate} the locations of the plane at one or more \textbf{intermediate} times.
The process of estimating intermediate values of a variable from known values is called \href{https://www.wikiwand.com/en/Interpolation}{interpolation}.
In this section we find a polynomial that satisfies the given data points.

Let \(c_0, c_1, ..., c_n\) be \textbf{distinct} scalars in an infinite field \(F\).
(This \(c_i\) is like \(t_i\) of the example above.)
The polynomials \(f_0(x), f_1(x), ..., f_n(x)\) defined by
\[
    f_i(x) = \frac{(x - c_0) ... (x - c_{\BLUE{i - 1}})(x - c_{\BLUE{i + 1}}) ... (x - c_n)}{(\RED{c_i} - c_0) ... (\RED{c_i} - c_{\BLUE{i - 1}})(\RED{c_i} - c_{\BLUE{i + 1}}) ... (\RED{c_i} - c_n)}
    = \prod_{\substack{k = 0 \\
                       \RED{k \ne j}}}^n
             \frac{x - c_k}{c_i - c_k}
\]
are called the \textbf{Lagrange polynomials} (associated with \(c_0, c_1, ..., c_n\)).
Note that
\begin{itemize}
\item Each \(f_i(x)\) is a polynomial of degree \(n\) and hence is in \(\POLYNF\).
\item If we regard \(f_i(x)\) as a polynomial function \(f_i : F \to F\), we see that
    \begin{equation*}
        f_{\RED{i}}(c_{\RED{j}}) =
        \begin{cases}
            0 \text{ if } i \ne j \\
            1 \text{ if } i = j \\
        \end{cases}
        \MAROON{(1)}
    \end{equation*}
\end{itemize}


This property of Lagrange polynomials can be used to show that \(\beta = \{ f_0, f_1, ..., f_n \}\) is a \emph{\LID{}} subset of \(\POLYNF\):

Suppose that
\[
    \sum_{i = 0}^n a_i f_i = 0_{zero} \text{ for some scalars } a_0, a_1, ..., a_n \MAROON{(2)},
\]
where \(0_{zero}\) denote the zero function.
Then in particular, given any input, say \(c_j\) for \(j = 1, ..., n\), the output of the \(0_{zero}(c_j)\) is \(\OF\) (the zero of field \(F\)).
That is, (by the definition of equality of functions,) from \MAROON{(2)}, we have
\[
    \sum_{i = 0}^n a_i f_i(c_j) = \OF \text{ for } j = 0, 1, ..., n. \MAROON{(3)}
\]
But also, by \MAROON{(3)}, for \(j = 0, 1, ..., n\),
\begin{align*}
    \OF & = \sum_{i = 0}^n a_i f_i(c_{\RED{j}}) \\
        & = a_0 f_0(c_j) + a_1 f_1(c_j) + ... + \RED{a_j f_j(c_j)} + ... + a_n f_n(c_j) \\
        & = a_0 \OF + a_1 \OF + ... + \RED{a_j 1} + ... + a_n \OF & \text{by \MAROON{(1)}} \\
        & = a_j.
\end{align*}
Hence \(a_0 = a_1 = ... = a_j = 0\).
Hence by \MAROON{(2)} we see that \(\beta = \{ f_0, f_1, ..., f_n \}\) is \LID{}.

Since \(\beta\) is \LID{} and has \(n + 1\) elements, which equals to the dimension of the parent vector space \(\POLYNF\), by \CORO{1.10.3}(b), \(\beta\) is a basis for \(\POLYNF\).

Because \(\beta\) is a basis for \(\POLYNF\), every polynomial function \(g\) in \(\POLYNF\) is a linear combination of polynomial functions of \(\beta\), say,
\[
    g = \sum_{i = 0}^n b_i f_i \MAROON{(4)}
\]
Again by definition of equality of functions, given any input \(c_j\) for \(j = 0, 1, ..., n\), we have
\[
    g(c_j) = \sum_{i = 0}^n b_i f_i(c_j).
\]
But using similar derivation like above, we have \(RHS = \sum_{i = 0}^n b_i f_i(c_j) = b_j\).
So \(g(c_j) = b_j\) for all \(j = 0, 1, ..., n\).
(Or renaming index, we have \(g(c_i) = b_i\) for all \(i = 0, 1, ..., n\).)
So in fact from \MAROON{(4)} we have
\[
    g = \sum_{i = 0}^n b_i f_i = \sum_{i = 0}^n g(c_i) f_i,
\]
which is the \emph{unique} representation of \(g\) is a linear combination of elements of \(\beta\).
This representation is called the \textbf{Lagrange interpolation formula}.

Notice that if you use any \(n + 1\) scalars in \(F\), \(b_0, b_1, ..., b_n\) (not necessarily distinct) to represent the corresponding values of the ``input'' \(c_0, c_1, ..., c_n\),
and use \(c_0, c_1, ..., c_n\) to construct \(f_0, f_1, ..., f_n\) using the above method,
then using \(b_0, b_1, ..., b_n\) as the coefficients of linear combination of vectors in \(\beta = \{ f_0, f_1, ..., f_n \}\), the polynomial function
\[
    h = \sum_{i = 0}^n b_i f_i
\]
is a polynomial in \(\POLYNF\) such that \(h(c_i) = b_i\) for all \(i = 0, 1, ..., n\).
Also \(h\) is \emph{the unique} polynomial in \(\POLYNF\) satisfying this condition.
(Suppose there exists another degree \(n\) polynomial \(p\) satisfying these points and show it is equal to \(h\).
Then you can let \(q = p - h\), then we have \(q(c_i) = 0\) for \(i = 0, 1, ..., n\).
But using \RMK{1.6.6} \RED{below}, \(q\) is in fact the zero function, which implies \(p = h\).)

Thus we have found the unique polynomial of degree not exceeding \(n\) that has specified values \(b_j\) at given points \(c_j\) in its domain (for \(j = 0, 1, ..., n\)).

For example, let us construct the real polynomial \(g\) of degree \emph{at most} \(2\) whose graph contains the points \((1, 8), (2, 5)\), and \((3, -4)\).
(Thus, in the notation above, \(c_0 = 1, c_1 = 2, c_2 = 3, b_0 = 8, b_1 = 5\), and \(b_2 = -4\).)
The Lagrange polynomials associated with \(c_0, c_1\), and \(c_2\) are
\begin{align*}
    f_0(x) & = \frac{(x - c_1)(x - c_2)}{(c_0 - c_1)(c_0 - c_2)} = \frac{(x - 2)(x - 3)}{(1 - 2)(1 - 3)} = \frac{1}{2}(x^2 - 5x + 6). \\
    f_1(x) & = \frac{(x - c_0)(x - c_2)}{(c_1 - c_0)(c_1 - c_2)} = \frac{(x - 1)(x - 3)}{(2 - 1)(2 - 3)} = -1(x^2 - 4x + 3). \\
    f_2(x) & = \frac{(x - c_0)(x - c_1)}{(c_2 - c_0)(c_2 - c_1)} = \frac{(x - 1)(x - 2)}{(3 - 1)(3 - 2)} = \frac{1}{2}(x^2 - 3x + 2).
\end{align*}
Hence the desired polynomial is
\begin{align*}
    h(x) & = \sum_{i = 0}^2 b_i f_i(x) = b_0 f_0(x) + b_1 f_1(x) + b_2 f_2(x) \\
         & = 8 (\frac{1}{2}(x^2 - 5x + 6)) + 5 (-1(x^2 - 4x + 3)) - 4 (\frac{1}{2}(x^2 - 3x + 2)) \\
         & = -3x^2 + 6x + 5.
\end{align*}

\begin{remark} \label{remark 1.6.6}
An important consequence of the Lagrange interpolation formula is the following result:
If \(f \in \POLYNF\) and \(f(c_i) = 0\) for \(n + 1\) distinct scalars \(c_0, c_1, ..., c_n\) in \(F\),
then \(f\) is the zero function.
Why? Since
\begin{align*}
    f & = \sum_{i = 0}^n b_i f_i & \text{by the argument above} \\
      & = \sum_{i = 0}^n 0 f_i, & \text{since \(b_0 = b_1 = ... b_n = 0\)} \\
\end{align*}
which is the trivial representation of the zero vector(zero function) of \(\POLYNF\).
\end{remark}