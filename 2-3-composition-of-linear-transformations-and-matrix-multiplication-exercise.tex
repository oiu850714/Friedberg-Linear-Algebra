\exercisesection

\begin{exercise} \label{exercise 2.3.1}
Label the following statements as true or false.
In each part, \(\V, \W\), and \(Z\) denote vector spaces with ordered (finite) bases \(\alpha, \beta\), and \(\gamma\), respectively;
\(\T : \V \to \W\) and \(\U : \W \to Z\) denote linear transformations;
and \(A\) and \(B\) denote matrices.

\begin{enumerate}
\item \([\U\T]_{\alpha}^{\gamma} = [\T]_{\alpha}^{\beta}[\U]_{\beta}^{\gamma}\).
\item \([\T(v)]_{\beta} = [\T]_{\alpha}^{\beta}[v]_{\alpha}\) for all \(v \in \V\).
\item \([\U(w)]_{\beta} = [\U]_{\alpha}^{\beta}[w]_{\beta}\) for all \(w \in \W\).
\item \([\ITRANV]_{\alpha} = I\).
    Notice that LHS is the matrix representation (under the basis \(\alpha\)) of the identity transformation,
    and RHS is identity matrix.
\item \([\T^2]_{\alpha}^{\beta} = ([\T]_{\alpha}^{\beta})^2\).
\item \(A^2 = I\) implies that \(A = I\) or \(A = -I\).
\item \(\T = \LMTRAN_A\) for some matrix \(A\).
\item \(A^2 = O\) implies that \(A = O\), where \(O\) denotes the zero matrix.
\item \(\LMTRAN_{A + B} = \LMTRAN_A + \LMTRAN_B\).
\item If \(A\) is square and \(A_{ij} = \delta_{ij}\) for all \(i\) and \(j\), then \(A = I\).
\end{enumerate}
\end{exercise}

\begin{proof} \ 
\begin{enumerate}
\item False, RHS is not necessarily defined, and by \THM{2.11}, \([\U\T]_{\alpha}^{\gamma} = [\U]_{\beta}^{\gamma}[\T]_{\alpha}^{\beta}\).
\item True by \THM{2.14}.
\item False, both sides of the equation are not necessarily defined.
\item True, need to prove, but trivial.
\item False, \(\T^2\) is not necessarily defined, hence \([\T^2]_{\alpha}^{\beta}\) is not necessarily defined;
    similarly for RHS.
\item False.
    Counter example: \(A = \begin{pmatrix} 1 & 0 \\ 0 & -1 \end{pmatrix}\), then \(A \ne I\) and \(A \ne -I\), but \(A^2 = I\).
\item False. \(\T\)'s domain and codomain are not necessarily equal to \(F^m\) for some \(m\), hence \(\T\) is not necessarily equal to some left-multiplication transformation.
\item False, see \RMK{2.3.3}.
\item True by \THM{2.15}(c), but of course we need that \(A, B\) have same dimensions.
\item True by \DEF{2.7}.
\end{enumerate}
\end{proof}

\begin{exercise} \label{exercise 2.3.2} \ 

\begin{enumerate}
\item Let
\[
    \begin{array}{l}
        A=\left(\begin{array}{rr}
            1 & 3 \\
            2 & -1
        \end{array}\right),
        B=\left(\begin{array}{rrr}
            1 & 0 & -3 \\
            4 & 1 & 2
        \end{array}\right) \\
        C=\left(\begin{array}{rrr}
            1 & 1 & 4 \\
            -1 & -2 & 0
        \end{array}\right),
        \text { and }
        D=\left(\begin{array}{r}
            2 \\
            -2 \\
            3
        \end{array}\right).
    \end{array}
\]
Compute  \(A(2B + 3C), (AB)D, and A(BD)\).

\item Let
\[
    A=\left(\begin{array}{rr}
        2 & 5 \\
        -3 & 1 \\
        4 & 2
    \end{array}\right),
    B=\left(\begin{array}{rrr}
        3 & -2 & 0 \\
        1 & -1 & 4 \\
        5 & 5 & 3
    \end{array}\right),
    \text { and }
    C=\left(\begin{array}{lll}
        4 & 0 & 3
    \end{array}\right)
\]
Compute \(A^\top, A^\top B, B C^\top, CB,\) and \(CA\).
\end{enumerate}
\end{exercise}

\begin{proof}
Pure matrix calculation problem, skip.
\end{proof}

\begin{exercise} \label{exercise 2.3.3}
Let \(g(x) = 3 + x\).
Let \(\T: \mathcal{P}_{2}(\SET{R}) \to \mathcal{P}_{2}(\SET{R})\)  and \(\U: \mathcal{P}_{2}(\SET{R}) \to \SET{R}^3\) be the linear transformations respectively defined by
\[
    \T(f(x)) = f'(x) g(x) + 2 f(x) \text{ and } \U(a + bx + cx^2) = (a + b, c, a - b). 
\]
Let  \(\beta\) and \(\gamma\) be the standard ordered bases of  \(\mathcal{P}_{2}(\SET{R})\) and \(\SET{R}^3\), respectively.
\begin{enumerate}
\item Compute \([\U]_{\beta}^{\gamma}, [\T]_{\beta}\), and \([\U\T]_{\beta}^{\gamma}\) \emph{directly}.
    Then use \THM{2.11} to verify your result.
\item Let \(h(x) = 3 - 2x + x^2\).
    Compute \([h(x)]_{\beta}\) and \([\U(h(x))]_{\gamma}\).
    Then use \([\U]_{\beta}^{\gamma}\) from (a) and \THM{2.14} to verify your result.
\end{enumerate}
\end{exercise}

\begin{proof} \ 
\begin{enumerate}
\item
First we calculate the formula for \(\U\T\).
Given \(f(x) = a + bx + cx^2 \in \POLYRR\),
\begin{align*}
    (\U\T)(f(x)) & = \U(\T(a + bx + cx^2)) \\
                 & = \U((a + bx + cx^2)'(3 + x) + 2(a + bx + cx^2)) \\
                 & = \U((b + 2cx)(3 + x) + 2a + 2bx + 2cx^2) \\
                 & = \U(3b + 6cx + bx + 2cx^2 + 2a + 2bx + 2cx^2) \\
                 & = \U((3b + 2a) + (3b + 6c)x + 4cx^2) \\
                 & = ((3b + 2a) + (3b + 6c), 4c, (3b + 2a) - (3b + 6c)) \\
                 & = (2a + 6b + 6c, 4c, 2a - 6c)
\end{align*}
Then
\begin{align*}
    \U\T(1 + 0x + 0x^2) & = (2 + 0 + 0, 0, 2 - 0) = (2, 0, 2) \\
    \U\T(0 + 1x + 0x^2) & = (0 + 6 + 0, 0, 0 - 0) = (6, 0, 0) \\
    \U\T(0 + 0x + 1x^2) & = (0 + 0 + 6, 4, 0 - 6) = (6, 4, -6)
\end{align*}
So
\[
    [\U\T]_{\beta}^{\gamma} = \begin{pmatrix} 2 & 6 & 6 \\ 0 & 0 & 4 \\ 2 & 0 & -6 \end{pmatrix}.
\]
And
\begin{align*}
    \U(1 + 0x + 0x^2) & = (1 + 0, 0, 1 - 0) = (1, 0, 1) \\
    \U(0 + 1x + 0x^2) & = (0 + 1, 0, 0 - 1) = (1, 0, -1) \\
    \U(0 + 0x + 1x^2) & = (0 + 0, 1, 0 - 0) = (0, 1, 0)
\end{align*}
so
\[
    [\U]_{\beta}^{\gamma} = \begin{pmatrix} 1 & 1 & 0 \\ 0 & 0 & 1 \\ 1 & -1 & 0 \end{pmatrix},
\]
and
\begin{align*}
    \T(1 + 0x + 0x^2) & = 0(3 + x) + 2(1) = 2 + 0x + 0x^2 \\
    \T(0 + 1x + 0x^2) & = 1(3 + x) + 2(x) = 3 + 3x + 0x^2 \\
    \T(0 + 0x + 1x^2) & = (2x)(3 + x) + 2(x^2) = 0 + 6x + 4x^2
\end{align*}
so
\[
    [\T]_{\beta} = \begin{pmatrix} 2 & 3 & 0 \\ 0 & 3 & 6 \\ 0 & 0 & 4 \end{pmatrix}
\]
So
\[
    [\U]_{\beta}^{\gamma} [\T]_{\beta}
    = \begin{pmatrix} 1 & 1 & 0 \\ 0 & 0 & 1 \\ 1 & -1 & 0 \end{pmatrix}
      \begin{pmatrix} 2 & 3 & 0 \\ 0 & 3 & 6 \\ 0 & 0 & 4 \end{pmatrix}
    = \begin{pmatrix} 2 & 6 & 6 \\ 0 & 0 & 4 \\ 2 & 0 & -6 \end{pmatrix}
\]
which is equal to \([\U\T]_{\beta}^{\gamma}\).

\item
\[
    [h(x)]_{\beta} = \begin{pmatrix} 3 \\ -2 \\ 1 \end{pmatrix}
\]
And \(\U(h(x)) = (3 + (-2), 1, 3 - (-2)) = (1, 1, 5)\), hence
\[
    [\U(h(x))]_{\gamma} = \begin{pmatrix} 1 \\ 1 \\ 5 \end{pmatrix}
\]
Now
\[
    [\U]_{\beta}^{\gamma} [h(x)]_{\beta} =
    \begin{pmatrix} 1 & 1 & 0 \\ 0 & 0 & 1 \\ 1 & -1 & 0 \end{pmatrix} \begin{pmatrix} 3 \\ -2 \\ 1 \end{pmatrix}
    = \begin{pmatrix} 1 \\ 1 \\ 5 \end{pmatrix}
\]
which is equal to \([\U(h(x))]_{\gamma}\).
\end{enumerate}
\end{proof}

\begin{exercise} \label{exercise 2.3.4}
For each of the following parts, let \(\T\) be the \LTRAN{} defined in the corresponding part of \EXEC{2.2.5}.
Use \THM{2.14} to compute the following vectors:

\begin{enumerate}
\item \([\T(A)]_{\alpha}\), where \(A = \begin{pmatrix} 1 & 4 \\ -1 & 6 \end{pmatrix}\).
\item \([\T(f(x))]_{\alpha}\), where \(f(x) = 4 - 6x + 3x^2\).
\item \([\T(A)]_{\gamma}\), where \(A = \begin{pmatrix} 1 & 3 \\ 2 & 4 \end{pmatrix}\).
\item \([\T(f(x))]_{\gamma}\), where \(f(x) = 6 - x + 2x^2\).
\end{enumerate}

\end{exercise}

\begin{note}
Preventing from flipping pages back and forth, we list \EXEC{2.2.5} below:

Let
\[
    \alpha = \bigg\{
                \begin{pmatrix} 1 & 0 \\ 0 & 0 \end{pmatrix},
                \begin{pmatrix} 0 & 1 \\ 0 & 0 \end{pmatrix},
                \begin{pmatrix} 0 & 0 \\ 1 & 0 \end{pmatrix},
                \begin{pmatrix} 0 & 0 \\ 0 & 1 \end{pmatrix}
            \bigg\},
    \beta = \{ 1, x, x^{2} \},
    \text { and }
    \gamma = \{ 1 \}.
\]
\begin{enumerate}
\item Define \(\T : M_{2 \X 2}(F) \to M_{2 \X 2}(F)\) by \(\T(A) = A^\top\).
Compute \([\T]_{\alpha}\).
\item Define
\[
    \T: \POLYRR \to M_{2 \X 2}(\SET{R}) \text{ by }
    \T(f) = \begin{pmatrix} f'(0) & 2f(1) \\ 0 & f''(3) \end{pmatrix},
\]
where \('\) denotes differentiation.
Compute \([\T]_{\beta}^{\alpha}\).
\item Define \(\T: M_{2 \X 2}(F) \to F\) by \(\T(A) = \TRACE(A)\).
Compute \([\T]_{\alpha}^{\gamma}\).
\item Define \(\T: \POLYRR \to \SET{R}\) by \(\T(f) = f(2)\). 
Compute \([\T]_{\beta}^{\gamma}\).
\end{enumerate}
\end{note}

\begin{proof} \ 
\begin{enumerate}
\item
\[
    [\T]_{\alpha}
    = \begin{pmatrix}
        1 & 0 & 0 & 0 \\
        0 & 0 & 1 & 0 \\
        0 & 1 & 0 & 0 \\
        0 & 0 & 0 & 1
    \end{pmatrix}
    \text{ and }
    [A]_{\alpha} = \begin{pmatrix} 1 \\ 4 \\ -1 \\ 6 \end{pmatrix}
    \text{, hence by \THM{2.14},}
\]
\[
    [\T(A)]_{\alpha} = [\T]_{\alpha} [A]_{\alpha}
    = \begin{pmatrix}
        1 & 0 & 0 & 0 \\
        0 & 0 & 1 & 0 \\
        0 & 1 & 0 & 0 \\
        0 & 0 & 0 & 1
    \end{pmatrix}
    \begin{pmatrix} 1 \\ 4 \\ -1 \\ 6 \end{pmatrix}
    = \begin{pmatrix} 1 \\ -1 \\ 4 \\ 6 \end{pmatrix}
\]

\item
\[
    [\T]_{\beta}^{\alpha} = \begin{pmatrix} 0 & 1 & 0 \\ 2 & 2 & 2 \\ 0 & 0 & 0 \\ 0 & 0 & 2 \end{pmatrix}
    \text{ and  }
    [f(x)]_{\beta} = \begin{pmatrix} 4 \\ -6 \\ 3 \end{pmatrix}
    \text{, hence by \THM{2.14},}
\]
\[
    [\T(f(x))]_{\alpha} = [\T]_{\beta}^{\alpha} [f(x)]_{\beta}
    = \begin{pmatrix} 0 & 1 & 0 \\ 2 & 2 & 2 \\ 0 & 0 & 0 \\ 0 & 0 & 2 \end{pmatrix}
    \begin{pmatrix} 4 \\ -6 \\ 3 \end{pmatrix}
    = \begin{pmatrix} -6 \\ 2 \\ 0 \\ 6 \end{pmatrix}
\]

\item
\[
    [\T]_{\alpha}^{\gamma} = \begin{pmatrix} 1 & 0 & 0 & 1 \end{pmatrix}
    \text{ and  }
    [A]_{\alpha} = \begin{pmatrix} 1 \\ 2 \\ 3 \\ 4 \end{pmatrix}
    \text{, hence by \THM{2.14},}
\]
\[
    [\T(A)]_{\gamma} = [\T]_{\alpha}^{\gamma} [A]_{\alpha}
    = \begin{pmatrix} 1 & 0 & 0 & 1 \end{pmatrix}
    \begin{pmatrix} 1 \\ 2 \\ 3 \\ 4 \end{pmatrix}
    = \begin{pmatrix} 5 \end{pmatrix}
\]

\item
\[
    [\T]_{\beta}^{\gamma} = \begin{pmatrix} 1 & 2 & 4 \end{pmatrix}
    \text{ and  }
    [f(x)]_{\beta} = \begin{pmatrix} 6 \\ -1 \\ 2 \end{pmatrix}
    \text{, hence by \THM{2.14},}
\]
\[
    [\T(f(x))]_{\gamma} = [\T]_{\beta}^{\gamma} [f]_{\beta}
    = \begin{pmatrix} 1 & 2 & 4 \end{pmatrix}
    \begin{pmatrix} 6 \\ -1 \\ 2 \end{pmatrix}
    = \begin{pmatrix} 12 \end{pmatrix}
\]
\end{enumerate}
\end{proof}

\begin{exercise} \label{exercise 2.3.5}
Complete the proof of \THM{2.12} and \CORO{2.12.1}.
\end{exercise}

\begin{proof}
See \THM{2.12} and \CORO{2.12.1}.
\end{proof}

\begin{exercise} \label{exercise 2.3.6}
Prove \THM{2.13}(b).
\end{exercise}

\begin{proof}
See \THM{2.13}.
\end{proof}

\begin{exercise} \label{exercise 2.3.7}
Prove \THM{2.15}(c)(f).
\end{exercise}

\begin{proof}
See \THM{2.15}.
\end{proof}

\begin{exercise} \label{exercise 2.3.8}
Prove \THM{2.10}.
Now state and prove a more general result involving linear transformations with domains unequal to their codomains.
\end{exercise}

\begin{proof}
See \THM{2.10}.

Now for the general result, they are immediately true since by the ``nature'' of the proof, the structure of the proof in \THM{2.10} does not depend on the domain and codomain, so the proof can also be used in the general case.
But you need to specify appropriate domain and codomain to make the function composition well defined.
\end{proof}

\begin{exercise} \label{exercise 2.3.9}
Find linear transformations \(\U, \T: F^2 \to F^2\) such that \(\U\T = \TZERO\) (the zero transformation) but \(\T\U \ne \TZERO\).
Use your answer to find matrices \(A\) and \(B\) such that \(AB = O_{2 \X 2}\) but \(BA \ne O_{2 \X 2}\).
\end{exercise}

\begin{proof}
Let \(\U(a, b) = (a, a), \T(a, b) = (0, b)\).
Then \(\U\T(a, b) = \U(\T(a, b)) = \U(0, b) = (0, 0)\), hence \(\U\T = \TZERO\).
But \(\T\U(a, b) = \T(\U(a, b)) = \T(a, a) = (0, a)\), hence \(\T\U \ne \TZERO\).
Now from \THM{2.15}(d), we know \(\U = \LMTRAN_A\) and \(\T = \LMTRAN_B\) where \(A = [\U]_{\beta}\), \(B = [\T]_{\beta}\) and \(\beta\) is the standard basis of \(F^2\).
And (trivially) \(A = \begin{pmatrix} 1 & 0 \\ 1 & 0 \end{pmatrix}, B = \begin{pmatrix} 0 & 0 \\ 0 & 1 \end{pmatrix}\), and \(AB = O_{2 \X 2}\) but \(BA \ne O_{2 \X 2}\).
\end{proof}

\begin{exercise} \label{exercise 2.3.10}
Let \(A\) be an \(n \X n\) matrix.
Prove that \(A\) is a diagonal matrix if and only if \(A_{ij} = \delta_{ij} A_{ij}\) for all \(i\) and \(j\).
\end{exercise}

\begin{proof} \ 

\(\Longrightarrow\):
Suppose \(A\) is diagonal, then
\begin{equation*}
    A_{ij} = \begin{cases}
        A_{ij} & \text{(of course) if } i = j \\
        0 & \text{ if } i \ne j
    \end{cases}
\end{equation*}
But in particular,
\begin{equation*}
    A_{ij} = \begin{cases}
        A_{ij} = 1 A_{ij} = \delta_{ij} A_{ij} & \text{ if } i = j \\
        0 = 0 A_{ij} = \delta_{ij} A_{ij} & \text{ if } i \ne j
    \end{cases}
\end{equation*}
So for all \(i\) and \(j\), \(A_{ij} = \delta_{ij} A_{ij}\).

\(\Longleftarrow\):
For all \(i\) and \(j\),
\begin{equation*}
    A_{ij} = \begin{cases}
        \delta_{ij} A_{ij} = 1 A_{ij} = A_{ij} & \text{ if } i = j \\
        \delta_{ij} A_{ij} = 0 A_{ij} = 0 & \text{ if } i \ne j
    \end{cases}
\end{equation*}
Hence \(A\) has zero on non-diagonal entries, hence by definition \(A\) is a diagonal matrix.
\end{proof}

\begin{exercise} \label{exercise 2.3.11}
Let \(\V\) be a vector space, and let \(\T : \V \to \V\) be linear.
Prove that \(\T^2 = \TZERO\) if and only if \(\RANGET \subseteq \NULLT\).
\end{exercise}

\begin{proof} \ 
\(\Longrightarrow\): Suppose \(\T^2 = \TZERO\), we have to show \(\RANGET \subseteq \NULLT\).
So suppose arbitrary \(x \in \RANGET\).
Then by definition of range, we can find \(v \in \V\) such that \(\T(v) = x\).
And in particular, \(\T(\T(v)) = \T(x)\).
But LHS \(= \T(\T(v)) = \T^2(v) = \OV\) by supposition, so we have RHS \(= \T(x) = \OV\).
So by definition of null space, \(x \in \NULLT\).
Since \(x\) is arbitrary, we have \(\RANGET \in \NULLT\).

\(\Longleftarrow\): Suppose \(\RANGET \subseteq \NULLT\), we have to show \(\T^2 = \TZERO\).
Then for all \(x \in \V\), of course \(\BLUE{\T(x)} \in \RANGET\). but since \(\RANGET \subseteq \NULLT\), we have \(\BLUE{\T(x)} \in \NULLT\).
Then by definition, \(\T(\BLUE{\T(x)}) = \OV\), that is, \(\T^2(x) = \OV\).

So for all \(x \in \V\), \(\T^2(x) = \OV\), hence \(\T^2 = \TZERO\).
\end{proof}

\begin{exercise} \label{exercise 2.3.12}
Let \(\V, \W\), and \(Z\) be vector spaces, and let \(\T : \V \to \W\) and \(\U: \W \to Z\) be linear.
\begin{enumerate}
\item Prove that if \(\U\T\) is one-to-one. then \(\T\) is one-to-one.
    Must \(\U\) also be one-to-one?
\item Prove that if \(\U\T\) is onto, then \(\U\) is onto.
    Must \(\T\) also be onto?
\item Prove that if \(\U\) and \(\T\) are one-to-one and onto, then \(\U\T\) is also.
\end{enumerate}
\end{exercise}

\begin{note}
All these statements \textbf{do not require linearity} and are true for normal functions.
\end{note}

\begin{proof} \ 
\begin{enumerate}
\item Suppose \(\U\T\) is one-to-one, we have to show \(\T\) is also.
So suppose \(x, y \in \T\) such that \(\T(x) = \T(y)\) we have to show \(x = y\).
Then
\begin{align*}
             & \T(x) = \T(y) \\
    \implies & \U(\T(x)) = \U(\T(y)) & \text{of course} \\
    \implies & (\U\T)(x) = (\U\T)(y) & \text{by def of function composition} \\
    \implies & x = y, & \text{since \(\U\T\) is one-to-one}
\end{align*}
as desired.

Note that \(\U\) is not necessarily one-to-one.
Let \(\T : \SET{R}^2 \to \SET{R}^3\) by \(\T(a, b) = (a, b, 0)\), \(\U : \SET{R}^3 \to \SET{R}^3\) be projection on \(xy\)-plane along \(z\)-axis, that is \(\U(a, b, c) = (a, b, 0)\).
Then \((\U\T)(a, b) = \U(a, b, 0) = (a, b, 0)\), trivially \(\U\T\) is one-to-one, but \(\U(a, b, 1) = \U(a, b, 2) = (a, b, 0)\) and \((a, b, 1) \ne (a, b, 2)\), so \(\U\) is not one-to-one.

\item
Suppose \(\U\T\) is onto, we have to show \(\U\) is onto.
So let arbitrary \(z \in Z\), we have to find a vector \(w \in \W\) such that \(\U(w) = z\).

But in particular, since \(\U\T\) is onto, we can find a vector \(v \in \V\) such that \((\U\T)(v) = z\).
By definition of function composition, \(\U(\T(v)) = z\).
But \(\T(v)\) is in \(\W\), so we have found a vector \(w = \T(v) \in \W\) such that \(\U(w) = z\).
Since \(z\) is arbitrary, we have \(\U\) is onto.

Note that \(\T\) is not necessarily onto.
Let \(\T : \SET{R}^2 \to \SET{R}^3\) by \(\T(a, b) = (a, b, 0)\), \(\U : \SET{R}^3 \to \SET{R}^2\) by \(\U(a, b, c) = (a, b)\).
Then \((\U\T)(a, b) = \U(a, b, 0) = (a, b)\), trivially \(\U\T\) is onto, but \((0, 0, 1)\) cannot by mapped by \(\T\) since the output of \(\T\) must have nonzero in the third component, hence \(\T\) is not onto.

\item Suppose \(\U, \T\) are one-to-one and onto.
First we show \(\U\T\) is one-to-one.
So suppose \((\U\T)(x) = (\U\T)(y)\), we have to show \(x = y\).
Then
\begin{align*}
             & (\U\T)(x) = (\U\T)(y) \\
    \implies & \U(\T(x)) = \U(\T(y)) & \text{by definition of function composition} \\
    \implies & \T(x) = \T(y) & \text{since \(\U\) is one-to-one} \\
    \implies & x = y & \text{since \(\T\) is one-to-one}
\end{align*}
So \(\U\T\) is one-to-one.

Now we show \(\U\T\) is onto.
So suppose \(z \in Z\), we have to find \(v \in \V\) such that \((\U\T)(v) = z\).
But since \(\U\) is onto, we can find \(w \in \W\) such that \(\U(w) = z\).
And since \(\T\) is onto, we can find \(v \in \V\) such that \(\T(v) = w\).
Then \((\U\T)(v) = \U(\T(v)) = \U(w) = z\),
hence we have found \(v \in \V\) such that \((\U\T)(v) = z\).
So \(\U\T\) is onto.
\end{enumerate}
\end{proof}

\begin{exercise} \label{exercise 2.3.13}
Let \(A\) and \(B\) be \(n \X n\) matrices.
Prove that \(\TRACE(AB) = \TRACE(BA)\) and \(\TRACE(A) = \TRACE(A^\top)\).
\end{exercise}

\begin{proof}
\begin{align*}
    \TRACE(AB) & = \sum_{i = 1}^n (AB)_{ii} & \text{by def of trace} \\
               & = \sum_{i = 1}^n \bigg( \sum_{k = 1}^n A_{ik} B_{ki} \bigg) & \text{by \DEF{2.10}} \\
               & = \sum_{k = 1}^n \bigg( \sum_{i = 1}^n A_{ik} B_{ki} \bigg) & \text{finite summation can change order} \\
               & = \sum_{k = 1}^n \bigg( \sum_{i = 1}^n B_{ki} A_{ik} \bigg) & \text{of course} \\
               & = \sum_{k = 1}^n (BA)_{kk} & \text{by \DEF{2.10}} \\
               & = \TRACE(BA) & \text{by def of trace}
\end{align*}

And
\begin{align*}
    \TRACE(A) & = \sum_{i = 1}^n A_{ii} & \text{by def of trace} \\
              & = \sum_{i = 1}^n (A^\top)_{ii} & \text{since \((A^\top)_{ij} = A_{ji}\), and in particular \((A^\top)_{ii} = A_{ii}\)} \\
              & = \TRACE(A^\top).
\end{align*}
\end{proof}

\begin{exercise} \label{exercise 2.3.14}
Assume the notation in \THM{2.13}, that is,
let \(A\) be an \(m \X n\) matrix and \(B\) be an \(n \X p\) matrix.
For each \(j\), \(j = 1, 2, ..., p\), let \(u_j\) and \(v_j\) denote \emph{the \(j\)th columns of} \(AB\) and \(B\), respectively.
\begin{enumerate}
\item Suppose that \(z\) is a (column) vector in \(F^P\).
    Use \THM{2.13}(b) to prove that \(Bz\) is a \textbf{linear combination of the columns of} \(B\).
    In particular, if \(z = (a_1, a_2, ..., a_p)^\top\), then show that
    \[
        Bz = \sum_{j = 1}^p a_j v_j.
    \]

\item Extend (a) to prove that column \(j\) of \(AB\) is a linear combination of the columns of \(A\) with the coefficients being the entries of column \(j\) of \(B\).

\item For any row vector \(w \in F^m\), prove that \(wA\) is a linear combination of the \textbf{rows of} \(A\) with the coefficients being the coordinates of \(w\).
    Hint: Use properties of the \textit{transpose operation} applied to (a).
\item Prove the analogous result to (b) about rows:
    Row \(i\) of \(AB\) is a linear combination of the rows of \(B\) with the coefficients being the entries of row \(i\) of \(A\).
\end{enumerate}
\end{exercise}

\begin{proof} \ 
\begin{enumerate}
\item First given \(z = (a_1, a_2, ..., a_p) \in F^p\), \(z = a_1 e_1 + a_2 e_2 + ... + a_p e_p\) where \(\{ e_1, e_2, ..., e_p \}\) is the standard basis for \(F^p\).
And
\begin{align*}
    Bz & = B(a_1 e_1 + a_2 e_2 + ... + a_p e_p) \\
       & = a_1 B e_1 + a_2 B e_2 + ... + a_p B e_p & \text{by \THM{2.12}(a)(b)} \\
       & = a_1 v_1 + a_2 v_2 + ... + a_p v_p, & \text{by \THM{2.13}(b)}
\end{align*}
which is a linear combination of columns of \(B\) with the coefficients being the components of \(z\).

\item for the column \(j\), \(u_j\), of \(AB\),
\begin{align*}
    u_j & = A v_j & \text{by \THM{2.13}(a)} \\
        & = A \begin{pmatrix}
              B_{1j} \\ B_{2j} \\ \vdots \\ B_{nj}
          \end{pmatrix} \\
        & = A (B_{1j} e_1 + B_{2j} e_2 + ... + B_{nj} e_n) & \text{where \(e_1, ..., e_n\) is the standard basis of \(F^n\)} \\
        & = B_{1j} A e_1 + B_{2j} A e_2 + ... + B_{nj} A e_n \MAROON{\ \ \ (b.1)} & \text{by \THM{2.12}(a)(b)}.
\end{align*}
But similar to \THM{2.13}(b), \(A e_j\) is equal to the \(j\)th column of \(A\) for \(j = 1, ..., n\), hence by \MAROON{(b.1)}, \(u_j\), the column \(j\) of \(AB\), is a linear combination of columns of \(A\) with the coefficients from the column \(j\) of \(B\).

\item
Let \(w = (w_1, w_2, ..., w_m) \in F^m\).
First, by \ATHM{2.24}, \((wA)^\top = A^\top w^\top\), with \(A^\top\) being \(n \X m\) and \(w^\top\) being \(m \X 1\).
And by part(a), let \(q_j\) be the \(j\)th column of \(A^\top\), then
\[
    A^\top w^\top = \sum_{j = 1}^m w_j q_j
\]
So,
\begin{align*}
             & (wA)^\top = w_1 q_1 + w_2 q_2 + ... + w_m q_m \\
    \implies & ((wA)^\top)^\top = (w_1 q_1 + w_2 q_2 + ... + w_m q_m)^\top \\
    \implies & wA = (w_1 q_1 + w_2 q_2 + ... + w_m q_m)^\top & \text{by \ATHM{1.2}(2)} \\
    \implies & wA = w_1 q_1^\top + w_2 q_2^\top + ... + w_m q_m^\top, \MAROON{(c.1)} & \text{by \ATHM{1.2}(1)}
\end{align*}
But since \(q_j\) is the \(j\)th column of \(A^\top\), that means \(q_j^\top\) is the \(j\)th row of \(A\).
Hence by \MAROON{(c.1)}, \(wA\) is the linear combination of rows of \(A\) with the coefficients being the coordinates of \(w\).

\item Again, \((AB)^\top = B^\top A^\top\).
And by part(b), column \(i\) of \(B^\top A^\top\) is a linear combination of columns of \(B^\top\) with the coefficients being the entries of column \(i\) of \(A^\top\).

But that means \RED{row} \(i\) of \((B^\top A^\top)^{\RED{\top}}\) (which is equal to \(((AB)^\top)^\top = AB\)) is a linear combinations of \RED{rows} of \((B^\top)^{\RED{\top}}\) with the coefficients being the entries of \RED{row} \(i\) of \((A^\top)^{\RED{\top}}\).

Simplifying, we have row \(i\) of \(AB\) is a linear combinations of rows of \(B\) with the coefficients being the entries of row \(i\) of \(A\), as desired.

\end{enumerate}
\end{proof}

\begin{exercise} \label{exercise 2.3.15}
Let \(A\) and \(B\) be matrices for which the product matrix \(AB\) is defined, and let \(u_j\) and \(v_j\) denote the \(j\)th columns of \(AB\) and \(B\), respectively.
If \(v_p = c_1 v_{j_{1}} + c_2 v_{j_{2}} + ... + c_k v_{j_{k}}\) for some scalars \(c_1, c_2, ..., c_k\), prove that \(u_p = c_1 u_{j_{1}} + c_2 u_{j_{2}} + ... + c_k u_{j_{k}}\).
\end{exercise}

\begin{note}
That is, if the \(p\)th column of \(B\), \(v_p\), is the linear combination of \(v_{j_{1}}, v_{j_{2}}, ..., v_{j_{k}}\), where
\[
    v_{j_{1}}, v_{j_{2}}, ..., v_{j_{k}}, \text{ are the \(j_1\)th, \(j_2\)th, ..., \(j_k\)th columns of \(B\)}, 
\]
with the coefficients \(c_1, c_2, ..., c_k\),
then the \(p\)th column of \(AB\), \(u_p\), is the linear combination of \(u_{j_{1}}, u_{j_{2}}, ..., u_{j_{k}}\), where
\[
    u_{j_{1}}, u_{j_{2}}, ..., u_{j_{k}} \text{ are (also) the \(j_1\)th, \(j_2\)th, ..., \(j_k\)th columns of \(AB\)}
\]
and \textbf{also with the same} coefficients \(c_1, c_2, ..., c_k\).
\end{note}

\begin{proof}
\begin{align*}
             & v_p = c_1 v_{j_{1}} + c_2 v_{j_{2}} + ... + c_k v_{j_{k}} \\
    \implies & A v_p = A(c_1 v_{j_{1}} + c_2 v_{j_{2}} + ... + c_k v_{j_{k}}) & \text{in particular} \\
    \implies & A v_p = c_1 A v_{j_{1}} + c_2 A v_{j_{2}} + ... + c_k A v_{j_{k}} & \text{by \THM{2.12}(a)(b)} \\
    \implies & u_p = c_1 u_{j_1} + c_2 u_{j_2} + ... + c_k u_{j_k} & \text{by \THM{2.13}(a), \(A v_p = u_p\), and so on}
\end{align*}
\end{proof}

\begin{note}
注意,雖然我們有說\ \(v_p\) 可用\ \(v_{j_1}, ..., v_{j_k}\) 加上\ \(c_1, ..., c_k\) 這些係數組出來,\textbf{但我們沒有說\ \(c_1, ..., c_k\) 是唯一的}。我們只有說\ \(u_p\) 也可以用\ \(u_{j_1}, ..., u_{j_k}\) 加上\ \(c_1, ..., c_k\) 這些係數組出來。
\(v_{j_1}, ..., v_{j_k}\) 是線性獨立才會保證\ \(c_1, ..., c_k\) 是唯一的。
這種例子會用在\ \EXEC{3.4.15}。
\end{note}

\begin{exercise} \label{exercise 2.3.16}
Let \(\V\) be a \emph{finite}-dimensional vector space, and let \(\T : \V \to \V\) be linear.
\begin{enumerate}
\item If \(\rankT = \rank(\T^2)\), prove that \(\RANGET \cap \NULLT = \{ \OV \}\).
    Deduce that \(\V = \RANGET \oplus \NULLT\).
\item Prove that \(\V = \RANGE(\T^k) \oplus \NULL(\T^k)\) for some positive integer \(k\).
\end{enumerate}
\end{exercise}

\begin{proof} \ 

\begin{enumerate}
\item Suppose \(\rankT = \rank(\T^2)\).
That is, \(\dim(\RANGET) = \dim(\RANGE(\T^2))\) \MAROON{(a.1)}.

We first do some pre-work.
We first show that \(\NULLT = \NULL(\T^2)\) by showing that (by \THM{1.11},) \(\NULLT \subseteq \NULL(\T^2)\) and \(\dim(\NULLT) = \dim(\NULL(\T^2))\).

So suppose arbitrary \(v \in \NULLT\).
Then
\begin{align*}
             & \BLUE{\T(v)} = \OV & \text{since \(v \in \NULLT\)} \\
    \implies & \T(\BLUE{\T(v)}) = \T(\OV) & \text{same as above} \\
             & = \OV & \text{since \(\T\) is linear} \\
    \implies & (\T^2)(v) = \OV & \text{by \ADEF{2.6}} \\
    \implies & v \in \NULL(\T^2) & \text{by def of null space}
\end{align*}
Since \(v\) is arbitrary, \(\NULLT \subseteq \NULL(\T^2)\).

Now we show that \(\dim(\NULLT) = \dim(\NULL(\T^2))\).
And
\begin{align*}
    \dim(\NULLT) & = \dim(\V) - \dim(\RANGET) & \text{since \(\V\) is finite, and by \THM{2.3}} \\
                 & = \dim(\V) - \dim(\RANGE(\T^2)) & \text{by \MAROON{(a.1)}} \\
                 & = \dim(\NULL(\T^2)) & \text{by \THM{2.3} again}
\end{align*}
Hence (by \THM{1.11},) we have shown that \(\NULLT = \NULL(\T^2)\) \MAROON{(a.2)}.

Now we first show that \(\RANGET \cap \NULLT = \{ \OV \}\).
So let arbitrary \(w \in \RANGET \cap \NULLT\), we have to show that \(w = \OV\).
So in particular, \(w \in \RANGET\) and \(w \in \NULLT\).
Since \(w \in \RANGET\), we have some \(v \in \V\) such that \(\T(v) = w\) \MAROON{(a.3)}.
Since \(w \in \NULLT\), we have \(\T(w) = \OV\).
Together, we have \(\T(\T(v)) = \T(w) = \OV\), hence \((\T^2)(v) = \OV\), hence \(v \in \NULL(\T^2)\).
But by \MAROON{(a.2)}, \(v\) is also in \(\NULLT\), so \(\T(v) = \OV\), hence by \MAROON{(a.3)} we have \(w = \OV\), as desired.

Now we show \(\V = \RANGET \oplus \NULLT\).
Since we have shown \(\RANGET \cap \NULLT = \{ \OV \}\), by \ATHM{2.11}(1.2), \(\V = \RANGET \oplus \NULLT\) is true as desired.

\item
First we show that given any linear \(\T : \V \to \V\), \(\RANGE(\T^{k + 1}) \subseteq \RANGE(\T^k)\) for some nonnegative integer \(k\).
Suppose arbitrary \(v \in \RANGE(\T^{k + 1})\), then exists \(x \in \V\) such that \(\T^{k + 1}(x) = v\);
that is, \(\T^k(\BLUE{\T(x)}) = v\);
so we have found \(\BLUE{\T(x)} \in \V\) such that \(\T^k(\BLUE{\T(x)}) = v\), so \(v \in \RANGE(\T^k)\).
Hence \(\RANGE(\T^{k + 1}) \subseteq \RANGE(\T^k)\).

Now given arbitrary nonnegative integer \(k\), we have two cases:
\begin{itemize}
    \item \(\RANGE(\T^{k + 1}) = \RANGE(\T^k)\): Then by definition we have \(\T^{k + 1}(\V) = \T^k(\V)\) \MAROON{(b.1)}.
    
    And furthermore,
    \begin{align*}
        \T^{k + 2}(\V) & = (\T \T^{k+1})(\V) & \text{of course} \\
                      & = \T(\T^{k + 1}(\V)) & \text{of course} \\
                      & = \T(\T^k(\V)) & \text{by \MAROON{(b.1)}} \\
                      & = \T^{k + 1}(\V) & \text{of course} \\
                      & = \T^k(\V) & \text{by \MAROON{(b.1)}}
    \end{align*}
    and (informally) inductively, we have \(\T^{2k}(\V) = \T(\V)\).
    That is, \(\RANGE(\T^{2k}) = \RANGE(\T^k)\), and equivalently, \(\RANGE((\T^k)^2) = \RANGE(\T^k)\).
    Hence by part(a), we have \(\V = \RANGE(\T^k) \oplus \NULL(\T^k)\), as desired.
    
    \item \(\RANGE(\T^{k + 1}) \subset \RANGE(\T^k)\) (proper subset):

    Then (by \THM{1.11}(2)), \(\dim(\RANGE(\T^{k + 1}) < \dim(\RANGE)(\T^k))\).
    And (informally by well-ordering principle) since the (finite) dimension of \(\RANGE(\T^{k + 1})\) can only be \(0\) to \(\dim(\RANGE(\T^k)) - 1\), we must have some integer \(k'\) such that \(\dim(\RANGE(\T^{k + \RED{k'} + 1}) = \dim(\RANGE(\T^{k + \RED{k'}}))\).
    But then we have found \(k'' = k + k'\) such that \(\dim(\RANGE(\T^{k'' + 1}) = \dim(\RANGE(\T^{''}))\), and this falls back to the previous case.
\end{itemize}
\end{enumerate}
\end{proof}

\begin{exercise} \label{exercise 2.3.17}
(Refer to \ADEF{2.2}.)
Let \(\V\) be a vector space and \(\T : \V \to \V\) be a linear transformation.
Prove that \(\T = \T^2\) if and only if \(\T\) is a \textbf{projection} on \(W_1 = \{ y : \T(y) = y \}\)
along \(\NULLT\).
\end{exercise}

\begin{proof} \ 

\(\Longrightarrow\): Suppose \(\T = \T^2\), we have to show \(\T\) is a projection on \(W_1\) along \(\NULLT\).
But by \ADEF{2.2} we first need to show \(\V = W_1 \oplus \NULLT\).

Now (\RED{tricky part}) given arbitrary \(x \in \V\), we have \(x = \T(x) + (x - \T(x))\) \MAROON{(1)}.
And
\begin{align*}
             & \T(x) = \T^2(x) & \text{by supposition} \\
    \implies & \T(x) = \T(\T(x)) & \text{of course}
\end{align*}
Hence we have \(\T(\BLUE{\T(x)}) = \BLUE{\T(x)}\) \MAROON{(2)}, so by definition of \(W_1\), \(\T(x) \in W_1\) \MAROON{(3)}.
And
\begin{align*}
        \T(x - \T(x)) & = \T^2(x - \T(x)) & \text{by supposition} \\
                      & = \T(\T(x - \T(x)) & \text{of course} \\
                      & = \T(\T(x) - \T(\T(x))) & \text{(nasty, but) since \(\T\) is linear} \\
                      & = \T(\T(x)) - \T\Big( \T \big( \T(x) \big) \Big) & \text{since \(\T\) is linear!} \\
                      & = \T(x) - \T\Big( \T(x) \Big) & \text{by \MAROON{(2)}} \\
                      & = \T(x) - \T(x) & \text{by \MAROON{(2)}} \\
                      & = \OV.
\end{align*}
Hence \(x - \T(x) \in \NULLT\) \MAROON{(4)}.
So by \MAROON{(3)} and \MAROON{(4)}, we have \(x = \T(x) + (x - \T(x))\) where \(\T(x) \in W_1\) and \(x - \T(x) \in \NULLT\).
Hence by definition, \(x \in W_1 + \NULLT\), so \(\V \subseteq W_1 + \NULLT\), which implies \(\V = W_1 + \NULLT\) since other direction is automatically true.

Now, if \(x \in W_1 \cap \NULLT\), then by \(x \in W_1\) we have \(\T(x) = x\),
and by \(x \in \NULLT\) we have \(\T(x) = \OV\), hence \(x = \OV\), hence \(W_1 \cap \NULLT = \{ \OV \}\).

So together we have \(\V = W_1 \oplus \NULLT\).

Finally, for all \(x \in \V\),
\begin{align*}
    \T(x) & = \T(\BLUE{\T(x)} + (x - \T(x))) & \text{where by \MAROON{(3)} \(\BLUE{\T(x)} \in W_1\) and by \MAROON{(4)} \(x - \T(x) \in \NULLT\)} \\
          & = \T(\T(x)) + \T(x - \T(x)) & \text{since \(\T\) is linear} \\
          & = \T(x) + \T(x - \T(x)) & \text{by supposition that \(\T = \T^2\)} \\
          & = \T(x) + \OV & \text{by \MAROON{(4)}} \\
          & = \BLUE{\T(x)},
\end{align*}
hence by \ADEF{2.2}, \(\T\) is a projection on \(W_1\) along \(\NULLT\).

\(\Longleftarrow\): Suppose \(\T\) is a projection on \(W_1\) along \(\NULLT\), we have to show \(\T = \T^2\).
So by definition of projection, for all \(x \in \V\), we have \(x = w_1 + w_2\) where \(w_1 \in W_1\) and \(w_2 \in \NULLT\), and \(\T(x) = w_1\).
And
\begin{align*}
    \T^2(x) & = \T(\T(x)) \\
            & = \T(\T(w_1 + w_2)) \\
            & = \T(\T(w_1) + \T(w_2)) & \text{since \(\T\) is linear} \\
            & = \T(w_1 + \OV) & \text{since \(w_1 \in W_1\) and \(w_2 \in \OV\)} \\
            & = \T(w_1) \\
            & = w_1 & \text{since \(w_1 \in W_1\)}
\end{align*}
Hence \(\T^2(x) = \T(x) = w_1\).
So we have \(\T^2 = \T\).
\end{proof}

\begin{exercise} \label{exercise 2.3.18}
Let \(\beta\) be an ordered basis for a finite-dimensional vector space \(\V\), and let \(\T : \V \to \V\) be linear.
Prove that, for any nonnegative integer \(k\),
\([\T^k]_{\beta} = ([\T]_{\beta})^k\).
\end{exercise}

\begin{proof}
This is of course by \THM{2.11} and induction.
Anyway, we prove by induction on \(k\) with base case \(0\).
For \(k = 0\),
\begin{align*}
    [\T^0]_{\beta} & = [\ITRANV]_{\beta} & \text{by \ADEF{2.6}} \\
                   & = I & \text{by \EXEC{2.3.1}(d)} \\
                   & = ([\T]_{\beta})^0 & \text{by \ADEF{2.7}}
\end{align*}
So the base case is true.

Suppose \([\T^k]_{\beta} = ([\T]_{\beta})^k\) for some integer \(k \ge 0\).
Then
\begin{align*}
    [\T^{k + 1}]_{\beta} & = [\T^k \T]_{\beta} & \text{by \ADEF{2.6}} \\
                         & = [\T^k]_{\beta} [\T]_{\beta} & \text{by \THM{2.11}} \\
                         & & \text{(or \CORO{2.11.1})} \\
                         & = ([\T]_{\beta})^k [\T]_{\beta} & \text{by inductive hypothesis} \\
                         & = ([\T]_{\beta})^{k + 1} & \text{by \ADEF{2.7}}
\end{align*}
This closes the induction.
\end{proof}

\begin{exercise} \label{exercise 2.3.19}
Using only the definition of matrix multiplication, prove that, multiplication of matrices is associative (\THM{2.16}).
\end{exercise}

\begin{proof}
Let \(A, B, C\) be \(m \X n, n \X p, p \X q\) matrices respectively such that \(A(BC)\) is defined.
Then for all \(1 \le i \le m\), \(1 \le j \le q\),
\begin{align*}
    [A(BC)]_{ij} & = \sum_{k = 1}^n A_{ik} (BC)_{kj} & \text{by \DEF{2.10}} \\
                 & = \sum_{k = 1}^n A_{ik} \bigg( \sum_{l = 1}^p B_{kl} C_{lj} \bigg) & \text{by \DEF{2.10}} \\
                 & = \sum_{k = 1}^n \Bigg( \sum_{l = 1}^p A_{ik} \Big(B_{kl} C_{lj} \Big) \Bigg) & \text{move constant into summation} \\
                 & = \sum_{k = 1}^n \Bigg( \sum_{l = 1}^p \Big( A_{ik} B_{kl} \Big) C_{lj} \Bigg) & \text{of course} \\
                 & = \sum_{l = 1}^p \Bigg( \sum_{k = 1}^n \Big( A_{ik} B_{kl} \Big) C_{lj} \Bigg) & \text{finite summation can change order} \\
                 & = \sum_{l = 1}^p C_{lj} \Bigg( \sum_{k = 1}^n \Big( A_{ik} B_{kl} \Big) \Bigg) & \text{move constant out of summation} \\
                 & = \sum_{l = 1}^p C_{lj} (AB)_{il} & \text{by \DEF{2.10}} \\
                 & = \sum_{l = 1}^p (AB)_{il} C_{lj} & \text{of course} \\
                 & = [(AB)C]_{ij} & \text{by \DEF{2.10}}
\end{align*}
Hence \(A(BC) = (AB)C\).
\end{proof}

\begin{exercise} \label{exercise 2.3.20}
For an incidence matrix(see \ADEF{2.8}) with related matrix \(B\) defined by \(B_{ij} = 1\) if \(i\) is related to \(j\) and \(j\) is related to \(i\), and \(B_{ij} = 0\) otherwise,
prove that \(i\) belongs to a clique if and only if \((B^3)_{ii} > 0\).
\end{exercise}

\begin{proof} \ 

\(\Longrightarrow\): Suppose \(i\) belongs to a clique.
Then
\begin{align*}
    (B^3)_{ii} & = (BB^2)_{ii} & \text{by \ADEF{2.7} (and \THM{2.16})}\\
               & = \sum_{k = 1}^n B_{ik} (B^2)_{ki} & \text{by \THM{2.10}} \\
               & = \sum_{k = 1}^n B_{ik} \sum_{l = 1}^n B_{kl} B_{lk} & \text{by \DEF{2.10}} \\
               & = \sum_{k = 1}^n \sum_{l = 1}^n B_{ik} B_{kl} B_{li};
\end{align*}
So by definition of \(B\), \((B^3)_{ii}\) means all the combination of \(k, l\) where \(1 \le k \le n\) and \(1 \le l \le n\) such that \(i, k\) have relation with each other, and \(k, l\) have relation with each other, and \(l, i\) have relation with each other.
But since \(i\) belongs to a clique, we really can find another two members \(p, q\) where \(1 \le p \le n\) and \(1 \le q \le n\), and \(i, p, q\) are distinct and related to each other.
That is, (by permutation law,)
\begin{itemize}
    \item \(i, p\) have relation with each other, that is, \(B_{ip} = 1\);
    \item \(p, q\) have relation with each other, that is, \(B_{pq} = 1\);
    \item \(q, i\) have relation with each other, that is, \(B_{qi} = 1\).
\end{itemize}
So we have \(B_{ip} B_{pq} B_{qi} = 1 \X 1 \X 1 = 1\), hence \(\sum_{k = 1}^n \sum_{l = 1}^n B_{ik} B_{kl} B_{li}\), which contains \(B_{ip} B_{pq} B_{qi}\) and all terms are nonnegative, is greater than \(0\).

\(\Longleftarrow\): Suppose \((B^3)_{ii} > 0\) for some \(1 \le i \le n\).
Then by the calculation in the previous case, some terms in \(\sum_{k = 1}^n \sum_{l = 1}^n B_{ik} B_{kl} B_{li}\) is equal to \(1\).
That is, we can find three member \(i, p, q\) where \(1 \le p \le n\) and \(1 \le q \le n\) and \(i, p, q\) are distinct such that \(B_{ip} B_{pq} B_{qi} = 1\).
And (by zero product property) this means all of \(B_{ip}, B_{pq}, B_{qi}\) are nonzero, and by definition of \(B\), are equal to \(1\).
But by definition of \(B\), that implies \(i, p\) have relation with each other, \(p, q\) have relation with each other, and \(q, i\) have relation with each other;
so (by permutations law,) \(i, p, q\) form a clique.
so we have found a clique containing \(i\).
\end{proof}

\begin{exercise} \label{exercise 2.3.21}
Use \EXEC{2.3.20} to determine the cliques in the relations corresponding to the following incidence matrices.
\begin{enumerate}
\item
\[
    \begin{pmatrix}
        0 & 1 & 0 & 1 \\
        1 & 0 & 0 & 0 \\
        0 & 1 & 0 & 1 \\
        1 & 0 & 1 & 0
    \end{pmatrix}
\]
\item
\[
    \begin{pmatrix}
        0 & 0 & 1 & 1 \\
        1 & 0 & 0 & 1 \\
        1 & 0 & 0 & 1 \\
        1 & 0 & 1 & 0
    \end{pmatrix}
\]
\end{enumerate}
\end{exercise}

\begin{proof} \ 

\begin{enumerate}
\item The corresponding \(B\) is
\[
    \begin{pmatrix}
        0 & 1 & 0 & 1 \\
        1 & 0 & 0 & 0 \\
        0 & \RED{0} & 0 & 1 \\
        1 & 0 & 1 & 0
    \end{pmatrix}
\]
and \(B^3\) is
\[
    \begin{pmatrix}
        0 & 3 & 0 & 3 \\
        2 & 0 & 1 & 0 \\
        0 & 3 & 0 & 3 \\
        4 & 0 & 2 & 0
    \end{pmatrix}
\]
hence by \EXEC{2.3.20}, since all \((B^3)_{ii} = 0\), \(A\) has no clique.

\item
The corresponding \(B\) is
\[
    \begin{pmatrix}
        0 & 0 & 1 & 1 \\
        \RED{0} & 0 & 0 & \RED{0} \\
        1 & 0 & 0 & 1 \\
        1 & 0 & 1 & 0
    \end{pmatrix}
\]
and \(B^3\) is
\[
    \begin{pmatrix}
        2 & 0 & 3 & 3 \\
        3 & 0 & 2 & 3 \\
        3 & 0 & 2 & 3 \\
        3 & 0 & 3 & 2
    \end{pmatrix}
\]
hence by \EXEC{2.3.20}, \(1, 3, 4\) belong to some clique(s).
\end{enumerate}
\end{proof}

\begin{exercise} \label{exercise 2.3.22}
Let \(A\) be an incidence matrix that is associated with a dominance relation(see \ADEF{2.9}).
Prove that the matrix \(A + A^2\) has a row [column] in which each entry is positive except for the diagonal entry.
\end{exercise}

\begin{proof}
We go back to the equivalent statements using graph theory: Given any positive integer \(n\), every graph corresponding a dominant incidence \(n \X n\) matrix has a vertex that can reach all other vertices within two steps.
(and has a vertex that can be reached by all other vertices within two steps, but we just skip this statement.)

BTW, we use \(v \to u\) to represent there is an edge from \(v\) to \(u\).

We prove this by induction on the number of vertices, starting from \(2\).

But the base case is really trivial.

So inductively, suppose for some positive integer \(k \ge 2\), \begin{center}
    every graph corresponding to a dominant incidence \(k \X k\) matrix has a vertex that can reach all other vertices within two steps.
\end{center}
We have to show
\begin{center}
    every graph corresponding to a dominant incidence \(k + 1 \X k + 1\) matrix has a vertex that can reach all other vertices within two steps.
\end{center}

So let \(A\) be an arbitrary \(k + 1 \X k + 1\) dominant incidence matrix, and let \(G\) be corresponding graph.
If we consider a new \(k \X k \) matrix \(A'\) by removing \(k+1\)-th row and column of \(A\), it is of course a dominant incidence \(k \X k\) matrix, and the corresponding graph \(G'\) is of course a \emph{subgraph} of \(G\).

So, by inductive hypothesis, (since \(G'\) corresponds to a \(k \X k\) dominant incidence matrix,) we can find a vertex, say \(v_1\), in \(G'\) such that \(v_1\) can reach all other vertices within two steps.

Now we consider the whole graph \(G\), we have to find a vertex in \(G\) such that it can reach all other vertices within two steps.
Note that there is only one vertex that is in \(G\) but not in \(G'\), so we name it \(v_{k + 1}\).
And we name the remaining vertices \(v_1, v_2, ..., v_k\) (where \(v_1\) satisfies inductive hypothesis).

Now we consider the cases of the connection between \(v_1\) and \(v_{k + 1}\).
Since \(A\) is a dominant coincidence matrix, exactly one of the following happens:
\begin{itemize}
\item \(v_1 \to v_{k + 1}\):
    Then \(v_1\) can reach \(v_{k + 1}\) using one step, and (by inductive hypothesis) \(v_1\) can reach other vertices within two steps, so \(v_1\) can reach all other vertices in \(G\) within two steps.
\item \(v_{k + 1} \to v_1\) \MAROON{(1)}:
    For this case we further consider the vertices that \(v_1\) connects to.
    (BTW these vertices are not empty, otherwise \(v_1\) cannot even reach any other vertex, which contradicts the inductive hypothesis.)
    \begin{itemize}
    \item There exists \(v_{a_t}\) (for some index \(a_t\)) (such that \(v_1 \to v_{a_t}\) \textbf{and}) \(v_{a_t} \to v_{k + 1}\):
        In this case, \(v_1\) can reach \(v_{k + 1}\) within two steps by \(v_1 \to v_{a_t}\) and \(v_{a_t} \to v_{k + 1}\).
        So again \(v_1\) can reach all other vertices within two steps.
    \item The negation of the first case; that is, every vertex that is connected by \(v_1\) does not connect to \(v_{k + 1}\).
        \textbf{But by the property of dominant coincidence relation}, that means \(v_{k + 1}\) connects to all vertices that is connected to \(v_1\) \MAROON{(2)}.
        
        We will show \(v_{k + 1}\) can reach all other vertices within two steps.
        Apart from \(v_{k + 1}\), the remaining vertices can be split into two types: \(v_{a_1}, v_{a_2}, ..., v_{a_i}\) that \(v_1\) connects to, and \(v_{b_1}, v_{b_2}, ..., v_{b_j}\) that \(v_1\) does not connect to.
        
        For \(v_{a_1}, ..., v_{a_i}\), since we have \(v_{k + 1} \to v_1\) (from \MAROON{(1)}), \(v_{k + 1}\) can reach these vertices within two steps by \(v_{k + 1} \to v_1\) and \(v_1\) to \(v_{a_1}, ..., v_{a_i}\).
        
        For \(v_{b_1}, ..., v_{b_j}\), since these nodes are in (subgraph) \(G'\), and \(v_1\) can reach this nodes within two steps, but \(v_1\) does not connect to these nodes, this implies that \(v_1\) \textbf{must} reach these nodes by exactly two steps.
        That is, for each \(v_b\) in \(v_{b_1}, ..., v_{b_k}\), we can find a vertices \(v'\) such that \(v_1 \to v'\) and \(v' \to v_b\) \MAROON{(3)}.
        But in particular, \(v_1\) connects to \(v'\), then by \MAROON{(2)}, \(v_{k + 1} \to v'\).
        Then we have \(v_{k + 1} \to v'\) and (by \MAROON{(3)}) \(v' \to v_b\).
        So \(v_{k + 1}\) can reach all \(v_{b_1}, ... v_{b_j}\) within two steps.
        
        So \(v_{k + 1}\) can reach all other vertices within two steps.
    \end{itemize}
\end{itemize}
Hence in all cases we can find a vertex that can reach all other vertices within two steps.
This closes this induction.
\end{proof}

\begin{exercise} \label{exercise 2.3.23}
Prove that the matrix
\[
    A = \begin{pmatrix}
        0 & 1 & 0 \\
        0 & 0 & 1 \\
        1 & 0 & 0
    \end{pmatrix}
\]
corresponds to a dominance relation.
Use \EXEC{2.3.22} to determine which persons dominate [are dominated by) each of the others within two stages.
\end{exercise}

\begin{proof}
\(A_{11} = A_{22} = A_{33} = 0\), and \(A_{12} = 0, A_{21} = 0, A_{13} = 0, A_{31} = 1, A_{23} = 1, A_{32} = 0\), hence by \ADEF{2.9}, \(A\) is an incidence matrix that is associated with a dominance relation.

And
\[
    A + A^2
    = \begin{pmatrix}
        0 & 1 & 0 \\
        0 & 0 & 1 \\
        1 & 0 & 0
    \end{pmatrix}
    + \begin{pmatrix}
        0 & 1 & 0 \\
        0 & 0 & 1 \\
        1 & 0 & 0
    \end{pmatrix}^2
    = + \begin{pmatrix}
        0 & 1 & 1 \\
        1 & 0 & 1 \\
        1 & 1 & 0
    \end{pmatrix}^2
\]
That is, everyone can reach others within two steps, and everyone can be reached by others within two steps.
\end{proof}

\begin{exercise} \label{exercise 2.3.24}
Let \(A\) be an \(n \X n\) incidence matrix that corresponds to a dominance relation.
Determine the number of nonzero entries of \(A\).
\end{exercise}

\begin{proof}
It has \(\frac{n(n-1)}{2}\) nonzero entries.
We can also consider the corresponding graph of \(A\), which is in fact a \emph{directed complete graph} with \(n\) vertices, so it has \(\frac{n(n-1)}{2}\) edges.
\end{proof}

\begin{additional theorem} \label{athm 2.25}
This is the placeholder theorem for \EXEC{2.3.11}:

Given linear \(\T : \V \to \V\), then \(\T^2 = \TZERO\) if and only if \(\RANGET \subseteq \NULLT\).
\end{additional theorem}

\begin{additional theorem} \label{athm 2.26}
This is the placeholder theorem for \EXEC{2.3.12}:

\BLUE{(1)}: \(\U\T\) is one-to-one implies \(\T\) is one-to-one, but \(\U\) is not necessarily.

\BLUE{(2)}: \(\U\T\) is onto implies \(\U\) is onto, but \(\T\) is not necessarily.

\BLUE{(3)}: \(\U, \T\) both one-to-one and onto implies \(\U\T\) is also one-to-one and onto.

And these facts does \textbf{not} depends on linearity.
\end{additional theorem}

\begin{additional theorem} \label{athm 2.27}
This is the placeholder theorem for \EXEC{2.3.13}:

Let \(A, B\) be \(n \X n\) matrices\(AB\) is defined, then \(\TRACE(AB) = \TRACE(BA)\) and \(\TRACE(A) = \TRACE(A^\top)\).
Some trivial properties about trace.
\end{additional theorem}

\begin{additional theorem} \label{athm 2.28}
This is the placeholder theorem for \EXEC{2.3.14}:

\(Ax\) is a linear combination of \(A\)'s columns with coefficients being \(x\)'s components.
And the column \(j\) of \(AB\) a linear linear combination of the columns of \(A\) with the coefficients being the entries of column \(j\) of \(B\).

(And similar for rows' version).
\end{additional theorem}

\begin{additional theorem} \label{athm 2.29}
This is the placeholder theorem for \EXEC{2.3.15}, which is just nasty;
refer to the note in that exercise.
\end{additional theorem}

\begin{additional theorem} \label{athm 2.30}
This is the placeholder theorem for \EXEC{2.3.16}:

Let \(\V\) be finite, and let \(\T : \V \to \V\) be linear.
\begin{enumerate}
\item If \(\rankT = \rank(\T^2)\), then \(\RANGET \cap \NULLT = \{ \OV \}\) and (by \ATHM{2.11}(1.2)) \(\V = \RANGET \oplus \NULLT\).
\item \(\V = \RANGE(\T^k) \oplus \NULL(\T^k)\) for some positive integer \(k\).
\end{enumerate}
\end{additional theorem}

\begin{additional theorem} \label{athm 2.31}
This is the placeholder theorem for \EXEC{2.3.17}:

Let \(\V\) be a vector space and \(\T : \V \to \V\) be a linear transformation.
Then \(\T = \T^2\) if and only if \(\T\) is a \textbf{projection} on \(W_1 = \{ y : \T(y) = y \}\)
along \(\NULLT\).
\end{additional theorem}

\begin{additional theorem} \label{athm 2.32}
This is the placeholder theorem for \EXEC{2.3.18}:

This is \THM{2.11} with induction:
\[
    [\T^k]_{\beta} = ([\T]_{\beta})^k.
\]
\end{additional theorem}

\begin{additional theorem} \label{athm 2.33}
This is the placeholder theorem for \EXEC{2.3.20}, which determines whether a member belongs to a clique.
\end{additional theorem}

\begin{additional theorem} \label{athm 2.34}
This is the placeholder theorem for \EXEC{2.3.22}, which says every dominant coincidence relation has a member which can reach (can be reached by) all the other members within two steps.
\end{additional theorem}
