\section{The Matrix Representation of a Linear Transformation} \label{sec 2.2}

Until now, we have studied \LTRAN{}s by examining their ranges and null spaces.
In this section, we embark on one of the most useful approaches to the analysis of a \LTRAN{} on a \emph{finite}-dimensional vector space:
the representation of a \LTRAN{} \emph{by a matrix}.
In fact, we develop a \textbf{one-to-one correspondence between matrices and \LTRAN{}s} that allows us to utilize properties of one to study properties of the other.
We first need the concept of an \emph{ordered basis} for a vector space.

\begin{definition} \label{def 2.4}
Let \(\V\) be a \emph{finite}-dimensional vector space.
An \textbf{ordered basis} for \(\V\) is a basis for \(\V\) endowed \emph{with a specific order};
that is, an ordered basis for \(\V\) is a \emph{finite sequence} of \LID{} vectors in \(\V\) that generates \(\V\).
\end{definition}

\begin{example} \label{example 2.2.1}
In \(F^3\), \(\beta = \{ e_1, e_2, e_3 \}\) can be considered an ordered basis.
Also \(\gamma = \{ e_2, e_1, e_3 \}\) is an ordered basis, but \(\beta \ne \gamma\) as ordered bases.
\end{example}

\begin{additional definition} \label{adef 2.4}
For the vector space \(F^n\), we call \(\{ e_1, e_2, ..., e_n \}\) the \textbf{standard ordered basis} for \(F^n\).
Similarly, for the vector space \(\POLYNF\), we call \(\{ 1, x, ..., x^n \}\) the \textbf{standard ordered basis} for \(\POLYNF\).
\end{additional definition}

\begin{remark} \label{remark 2.2.1}
Now that we have the concept of ordered basis, we can \emph{identify} abstract vectors in an \(n\)-dimensional vector space \emph{with \(n\)-tuples}.
This identification is provided through the use of \emph{coordinate vectors}, as introduced next.
\end{remark}

\begin{definition} \label{def 2.5}
Let \(\beta = \{ u_1, u_2, ..., u_n \}\) be an ordered basis for a \emph{finite}-dimensional vector space \(\V\).
For \(x \in \V\), let \(a_1, a_2, ..., a_n\) be the \emph{unique} scalars such that
\[
    x = \sum_{i = 1}^n a_i u_i.
\]
(Uniqueness is guaranteed by \THM{1.8}.)
We define the \textbf{coordinate vector of \(x\) relative to \(\beta\)}, denoted \([x]_{\beta}\), by
\[
    [x]_{\beta} = \begin{pmatrix} a_1 \\ a_2 \\ \vdots \\ a_n \end{pmatrix}.
\]
\end{definition}

\begin{note}
Since \(u_i = 0 u_1 + ... + 0 u_{i - 1} + 1 u_i + 0 u_{i + 1} + ... + 0 u_n\), \([u_i]_{\beta} = e_i\), where \(e_i\) is the \(i\)th vector in the standard ordered basis of \(F^{n}\).
It is left as an exercise to show that the \emph{correspondence} \(x \to [x]_{\beta}\) provides us with a \LTRAN{} from \(\V\) to \(F^n\).
(Also notice that it's a \emph{bijection} from \(\V\) to \(F^n\).)
We study this transformation in \SEC{2.4} in more
detail.
\end{note}

\begin{example} \label{example 2.2.2}
Let \(\V = \POLYRR\), and let \(\beta = \{ 1, x , x^2 \}\) be the standard ordered basis for \(\V\).
If \(f(x) = 4 + 6x - 7x^2\), then
\[
    [f]_{\beta} = \begin{pmatrix} 4 \\ 6 \\ 7 \end{pmatrix}.
\]
\end{example}

Let us now proceed with the promised \emph{matrix representation} of a \LTRAN{}.
Suppose that \(\V\) and \(\W\) are \emph{finite}-dimensional vector spaces with ordered bases \(\beta = \{ v_1, v_2, ..., v_n \}\) and \(\gamma = \{ w_1, w_2, ..., w_m \}\), respectively.
Let \(\T : \V \to \W\) be linear.
Then for each \(j = 1, 2, ..., n\), the output of \(\T(v_j)\) can be represented as a linear combination using \(\gamma = \{ w_1, w_2, ..., w_m \}\), with the \emph{unique} scalars \(a_{1j}, a_{2j}, ..., a_{mj}\);
that is,
\[
    \T(v_j) = a_{1j} w_1 + a_{2j} w_2 + ... + a_{mj} w_m = \sum_{i = 1}^m a_{ij} w_i, \text{ for } j = 1, 2, ..., n.
\]
And we can use these \(a_{ij}\) for \(1 \le i \le m\), \(1 \le j \le n\), to define a matrix:

\begin{definition} \label{def 2.6}
Continuing from the discussion above, we call the \(m \X n\) matrix \(A\) defined by \(A_{ij} = a_{ij}\) the \textbf{matrix representation of \(\T\) in the ordered bases \(\beta\)
and \(\gamma\)} and write \(A = [\T]_{\beta}^{\gamma}\).
If \(\V = \W\) and \(\beta = \gamma\), then we write \(A = [\T]_{\beta}\).
\end{definition}

\begin{remark} \label{remark 2.2.2}
Notice that the \(j\)th column of \(A\) is
\[
    \begin{pmatrix} a_{1j} \\ a_{2j} \\ \vdots \\ a_{mj} \end{pmatrix},
\]
by the discussion above we know \(\T(v_j) = a_{1j} w_1 + a_{2j} w_2 + ... + a_{mj} w_m\),
so indeed the \(j\)th columns of \(A\), by \DEF{2.5}, is equal to \([\T(v_j)]_{\gamma}\).

\RED{Also observe that} if \(\U: \V \to \W\) is a \LTRAN{} such that \([\U]_{\beta}^{\gamma} = [\T]_{\beta}^{\gamma}\),
then in particular for \(j = 1, 2, ..., n\), the \(j\)th column of \([\U]_{\beta}^{\gamma}\) is equal to the \(j\)th column of \([\T]_{\beta}^{\gamma}\),
which (by what we have shown) implies \([\U(v_j)]_{\gamma} = [\T(v_j)]_{\gamma}\),
which implies \(\U(v_j) = \T(v_j)\).
Then by \CORO{2.6.1}, \(\U = \T\).
\end{remark}

\begin{example} \label{example 2.2.3}
Let \(\T : \SET{R}^2 \to \SET{R}^3\) be the \LTRAN{} defined by
\[
    \T(a_1, a_2) = (a_1 + 3a_2, 0, 2a_1 - 4a_2).
\]
Let \(\beta\) and \(\gamma\) be the standard ordered bases for \(\SET{R}^2\) and \(\SET{R}^3\), respectively.
Now
\[
    \T(1, 0) = (1, 0, 2) = 1(1, 0, 0) + 0(0, 1, 0) + 2(0, 0, 1)
\]
and
\[
    \T(0, 1) = (3, 0, -4) = 3(1, 0, 0) + 0(0, 1, 0) + -4(0, 0, 1)
\]
Hence
\[
    [\T(1, 0)]_{\gamma} = \begin{pmatrix} 1 \\ 0 \\ 2 \end{pmatrix},
    [\T(0, 1)]_{\gamma} = \begin{pmatrix} 3 \\ 0 \\ -4 \end{pmatrix},
\]
hence
\[
    [\T]_{\beta}^{\gamma} = \begin{pmatrix} [\T(1, 0)]_{\gamma} & [\T(0, 1)]_{\gamma} \end{pmatrix}
    = \begin{pmatrix} 1 & 3 \\ 0 & 0 \\ 2 & -4 \end{pmatrix}.
\]
If we let \(\MAROON{\gamma'} = \{ e_3, e_2, e_1 \}\) be another ordered basis for \(\SET{R}^3\), then similarly,
\[
    [\T]_{\beta}^{\MAROON{\gamma'}} = \begin{pmatrix} 2 & -4 \\ 0 & 0 \\ 1 & 3 \end{pmatrix}.
\]
This implies the matrix representation of \(\T\) under \emph{different} ordered basis can be distinct.
\end{example}

\begin{example} \label{example 2.2.4}
Let \(\T : \POLYRRR \to \POLYRR\) be the \LTRAN{} defined by \(\T(f) = f'\).
Let \(\beta = \{ 1, x, x^2, x^3 \}\) and \(\gamma = \{ 1, x, x^2 \}\) be the standard ordered bases for \(\POLYRRR\) and \(\POLYRR\)
respectively.
Then
\begin{align*}
         \T(1) = 0 & = \BLUE{0} \cdot 1 + \BLUE{0} \cdot x + \BLUE{0} \cdot x^2 \\
         \T(x) = 1 & = \GREEN{1} \cdot 1 + \GREEN{0} \cdot x + \GREEN{0} \cdot x^2 \\
      \T(x^2) = 2x & = \MAROON{0} \cdot 1 + \MAROON{2} \cdot x + \MAROON{0} \cdot x^2 \\
    \T(x^3) = 3x^2 & = \RED{0} \cdot 1 + \RED{0} \cdot x + \RED{3} \cdot x^2
\end{align*}
So
\[
    [\T]_{\beta}^{\gamma}
    = \begin{pmatrix}
     [\T(1)]_{\gamma} & [\T(x)]_{\gamma} & [\T(x^2)]_{\gamma} & [\T(x^3)]_{\gamma}
    \end{pmatrix}
    = \begin{pmatrix}
        \BLUE{0} & \GREEN{1} & \MAROON{0} & \RED{0} \\
        \BLUE{0} & \GREEN{0} & \MAROON{2} & \RED{0} \\
        \BLUE{0} & \GREEN{0} & \MAROON{0} & \RED{3}
    \end{pmatrix}
\]
\end{example}

\begin{additional definition} \label{adef}
Let \(\V\) and \(\W\) be finite-dimensional vector spaces with ordered bases \(\beta = \{ v_1, v_2, ..., v_n \}\) and \(\gamma = \{ w_1, w_2, ..., w_m \}\), respectively.
Then for the \emph{zero transformation},
\[
    \TZERO(v_j) = \OW = 0 w_1 + 0 w_2 + ... + 0 w_m.
\]
Hence (clearly) \([\T]_{\beta}^{\gamma} = O_{m \X n}\), the \(m \X n\) zero matrix.
Also, for the \emph{identity transformation},
\[
\ITRANV(v_j) = v_j = 0v_1 + 0v_2 + ... + 0v_{j - 1} + 1 v_j + 0 v_{j + 1} + ... + 0v_n.
\]
Hence (clearly) the \(j\)th column of \([\ITRANV]_{\beta}\) is \(e_j\), that is,
\[
    [\ITRANV]_{\beta} = \begin{pmatrix}
        1 & 0 & ... & 0 \\
        0 & 1 & ... & 0 \\
        \vdots & \vdots & & \vdots \\
        0 & 0 & ... & 1
    \end{pmatrix}.
\]
The preceding matrix is called the \(n \X n\) \emph{identity matrix} and is defined next, along with a very useful notation, the \emph{Kronecker delta}.
\end{additional definition}

\begin{definition} \label{def 2.7}
We define the \textbf{Kronecker delta} \(\delta_{ij}\) by \(\delta_{ij} = 1\) if \(i = j\) and \(\delta_{ij} = 0\) if \(i \ne j\).
The \(n \X n\) \textbf{identity matrix} \(I_n\) is defined by \((I_n)_{ij} = \delta_{ij}\).
When the context is clear, we sometimes omit the subscript \(n\) from \(I_n\).

For example,
\[
    I_1 = (1), I_2 = \begin{pmatrix} 1 & 0 \\ 0 & 1 \end{pmatrix}, 
    \text{ and } I_3 = \begin{pmatrix} 1 & 0 & 0 \\ 0 & 1 & 0 \\ 0 & 0 & 1 \end{pmatrix}.
\]
\end{definition}

Now that we have defined a procedure for \emph{associating} matrices with \LTRAN{}s, we show in \THM{2.8} that \emph{this association ``preserves'' addition and scalar multiplication}.
To make this more explicit, we need some preliminary discussion about the \emph{addition and scalar multiplication of \LTRAN{}s}.

\begin{definition} \label{def 2.8}
Let \(\T, \U: \V \to \W\) be arbitrary \emph{functions}, where \(\V\) and \(\W\) are vector spaces over \(F\), and let \(a \in F\).
We define

\BLUE{(1)} \(\T + \U : \V \to \W\) by \((\T + \U)(x) = \T(x) + \U(x)\) for all \(x \in \V\), and

\BLUE{(2)} \(a\T: \V \to \W\) by \((a\T)(x) = a\T(x)\) for all \(x \in \V\).
\end{definition}

\begin{note}
Of course, these are \emph{just the usual definitions of addition and scalar multiplication of functions}.
We are fortunate, however, to have the result that \emph{both sums and scalar multiples of \LTRAN{}s are also linear}.
\end{note}

\begin{theorem} \label{thm 2.7}
Let \(\V\) and \(\W\) be vector spaces over a field \(F\), and let \(\T, \U: \V \to \W\) be \emph{linear}.
\begin{enumerate}
\item For all \(a \in F\), \(a\T + \U\) is linear.
\item Using the operations of addition and scalar multiplication in the preceding definition, \textbf{the collection of all \LTRAN{}s from \(\V\) to \(\W\) is a vector space over F}.
\end{enumerate}
\end{theorem}

\begin{proof} \ 
\begin{enumerate}
\item We need to show \(a\T + \U\) is still a \LTRAN{}.
So let \(x, y \in \V\) and \(a \in F\).
Then
\begin{align*}
    (a\T + \U)(cx + y) & = (a\T)(cx + y) + \U(cx + y) & \text{by \DEF{2.8}(1)} \\
                       & = a\T(cx + y) + \U(cx + y) & \text{by \DEF{2.8}(2)} \\
                       & = a\Big( c\T(x) + \T(y) \Big) + \Big( c\U(x) + \U(y) \Big) & \text{since \(\T, \U\) are linear} \\
                       & = ac\T(x) + a\T(y) + c\U(x) + \U(y) & \text{by algebra on \(\W\)} \\
                       & = ca\T(x) + c\U(x) + a\T(y) + \U(y) & \text{same as above} \\
                       & = c(a\T(x) + \U(x)) + (a\T(y) + \U(y)) &
                       \text{same as above} \\
                       & = c((a\T)(x) + \U(x)) + ((a\T)(y) + \U(y)) & \text{by \DEF{2.8}(2)} \\
                       & = c((a\T + \U)(x)) + (a\T + \U)(y), & \text{by \DEF{2.8}(1)}
\end{align*}
Hence by \ATHM{2.1}(b), \(a\T + \U\) is linear.

\item
Noting that \(\TZERO\), the zero transformation, plays the role of the zero vector, it is easy to verify that the axioms of a vector space are satisfied, and hence that the collection of all \LTRAN{}s from \(\V\) into \(\W\) is a vector space over \(F\).
(See \EXEC{2.1.6}.)
\end{enumerate}
\end{proof}

\begin{definition} \label{def 2.9}
Let \(\V\) and \(\W\) be vector spaces over \(F\).
We denote the vector space of all \LTRAN{}s from \(\V\) into \(\W\) by \(\mathcal{L}(\V, \W)\).
In the case that \(\V = \W\), we write \(\mathcal{L}(\V)\) instead of \(\mathcal{L}(\V, \V)\).
\end{definition}

\begin{remark} \label{remark 2.2.3}
In \SEC{2.4}, we see a \emph{complete identification} of \(\mathcal{L}(\V, \W)\) with the vector space \(M_{m \X n}(F)\), where \(n\) and \(m\) are the dimensions of \(\V\) and \(\W\), respectively.
(``identification'' just means a one-to-one correspondence, that is, any \(m \X n\) matrix corresponds to one and only one \LTRAN{} from \(\V\) to \(\W\).)
This identification is easily established by the use of the next theorem.
\end{remark}

\begin{theorem} \label{thm 2.8}
Let \(\V\) and \(\W\) be \emph{finite}-dimensional vector spaces with ordered bases \(\beta\) and \(\gamma\), respectively, and let \(\T, \U: \V \to \W\) be \LTRAN{}s.
Then
\begin{enumerate}
\item \([\T + \U]_{\beta}^{\gamma} = [\T]_{\beta}^{\gamma} + [\U]_{\beta}^{\gamma}\)
\item \([a\T]_{\beta}^{\gamma} = a[\T]_{\beta}^{\gamma}\) for all scalars \(a\).
\end{enumerate}
\end{theorem}

\begin{proof}
Let \(\beta = \{ v_1, v_2, ..., v_n \}\) and \(\gamma = \{ w_1, w_2, ..., w_m \}\).
Then (from the discussion before \DEF{2.6},) there exist unique scalars \(a_{ij}\) and \(b_{ij}\) for \(1 \le i \le m, 1 \le j \le n\), such that
\[
    \T(v_j) = \sum_{i = 1}^m a_{ij} w_i \text{ and } \U(v_j) = \sum_{i = 1}^m b_{ij} w_i \MAROON{(1)}
\]
Hence (by \DEF{2.6},) \(([\T]_{\beta}^{\gamma})_{ij} = a_{ij}\) and \(([\U]_{\beta}^{\gamma})_{ij} = b_{ij}\) \MAROON{(2)}.

And
\begin{align*}
    (\T + \U)(v_j) & = \T(v_j) + \U(v_j) & \text{by \DEF{2.8}(1)} \\
                   & = \sum_{i = 1}^m a_{ij} w_i + \sum_{i = 1}^m b_{ij} w_i & \text{by \MAROON{(1)}} \\
                   & = \sum_{i = 1}^m (a_{ij} + b_{ij}) w_i \MAROON{(3)} & \text{by rules of finite summation}
\end{align*}
Thus for all \(i, j\) such that \(1 \le i \le m\) and \(1 \le j \le n\),
\begin{align*}
    ([\T + \U]_{\beta}^{\gamma})_{ij} & = a_{ij} + b_{ij} & \text{by \MAROON{(3)}} \\
                                      & = ([\T]_{\beta}^{\gamma})_{ij} + ([\U]_{\beta}^{\gamma})_{ij} & \text{by \MAROON{(2)}} \\
                                      & = ([\T]_{\beta}^{\gamma} + [\U]_{\beta}^{\gamma})_{ij} & \text{by \(+\) of matrices, in \EXAMPLE{1.2.2}}
\end{align*}
So (by \ADEF{1.3}(6),) \([\T + \U]_{\beta}^{\gamma} = [\T]_{\beta}^{\gamma} + [\U]_{\beta}^{\gamma}\).

Now given arbitrary scalar \(a\),
\begin{align*}
    (a\T)(v_j) & = a\T(v_j) & \text{by \DEF{2.8}(1)} \\
                   & = a\sum_{i = 1}^m a_{ij} w_i & \text{by \MAROON{(1)}} \\
                   & = \sum_{i = 1}^m (a a_{ij}) w_i \MAROON{(4)} & \text{by rules of finite summation}
\end{align*}
Thus for all \(i, j\) such that \(1 \le i \le m\) and \(1 \le j \le n\),
\begin{align*}
    ([a\T]_{\beta}^{\gamma})_{ij} & = a a_{ij} & \text{by \MAROON{(4)}} \\
                                  & = a([\T]_{\beta}^{\gamma})_{ij} & \text{by \MAROON{(2)}} \\
                                      & = (a[\T]_{\beta}^{\gamma})_{ij} & \text{by scalar \(\cdot\) of matrices, in \EXAMPLE{1.2.2}}
\end{align*}
So (by \ADEF{1.3}(6),) \([a\T]_{\beta}^{\gamma} = a[\T]_{\beta}^{\gamma}\).
\end{proof}

\begin{note}
注意\ \THM{2.8} 內一堆符號的意思,把它們唸出來就會是

(a): \(\T + \U\) 這個線性變換的(以\ \(\beta\) 跟\ \(\gamma\) 為基底的)矩陣表示法,等於\ \(\T\) 這個線性變換的(以\ \(\beta\) 跟\ \(\gamma\) 為基底的)矩陣表示法,加上\ \(\U\) 這個線性變換的(以\ \(\beta\) 跟\ \(\gamma\) 為基底的)矩陣表示法。

(b): \(a\T\) 這個線性變換的(以\ \(\beta\) 跟\ \(\gamma\) 為基底的)矩陣表示法,等於\ scalar \(a\),乘以\ \(\T\) 這個線性變換的(以\ \(\beta\) 跟\ \(\gamma\) 為基底的)矩陣表示法。

實際上就跟線性變換這個概念(也就是\ \DEF{2.1}) 本身有一樣的感覺,先相加再取矩陣代表,等於先取完矩陣代表後再相加;先乘\ scalar 再取矩陣代表,等於先舉完矩陣代表後再乘以\ scalar。
\end{note}

\begin{example} \label{example 2.2.5}
Let \(\T : \SET{R}^2 \to \SET{R}^3\) and \(\U : \SET{R}^2 \to \SET{R}^3\) be the \LTRAN{}s respectively defined by
\[
    \T(a_1, a_2) = (a_1 + 3a_2, 0, 2a_1 - 4a_2) \text{ and } \U(a_1, a_2) = (a_1 - a_2, 2a_1, 3a_1 + 2a_2).
\]
Let \(\beta\) and \(\gamma\) be the standard ordered bases of \(\SET{R}^2\) and \(\SET{R}^3\), respectively.
Then
\[
    [\T]_{\beta}^{\gamma} = \begin{pmatrix} 1 & 3 \\ 0 & 0 \\ 2 & -4 \end{pmatrix},
\]
(as computed in \EXAMPLE{2.2.3}), and
\[
    [\U]_{\beta}^{\gamma} = \begin{pmatrix} 1 & -1 \\ 2 & 0 \\ 3 & 2 \end{pmatrix},
\]
If we compute \(\T + \U\) using the preceding definitions, we obtain
\[
    (\T + \U)(a_1, a_2) = (2a_1 + 2a_2, 2a_1, 5a_1 - a_2).
\]
So
\[
    [\T + \U]_{\beta}^{\gamma} = \begin{pmatrix} 2 & 2 \\ 2 & 0 \\ 5 & -2 \end{pmatrix},
\]
and by \THM{2.8}, this is simply equal to \([\T]_{\beta}^{\gamma} + [\U]_{\beta}^{\gamma}\).
\end{example}

\exercisesection

\begin{exercise} \label{exercise 2.2.1}
Label the following statements as true or false.
Assume that \(\V\) and \(\W\) are \emph{finite}-dimensional vector spaces with ordered bases \(\beta\) and \(\gamma\), respectively, and \(\T, \U : \V \to \W\) are \LTRAN{}s.
\begin{enumerate}
\item For any scalar \(a\), \(a\T + \U\) is a \LTRAN{} from \(\V\) to \(\W\).
\item \([\T]_{\beta}^{\gamma} = [\U]_{\beta}^{\gamma}\) implies that \(\T = \U\).
\item If \(m = \dim(\V)\) and \(n = \dim(\W)\), then \([\T]_{\beta}^{\gamma}\) is an \(m \X n\) matrix.
\item \([\T + \U]_{\beta}^{\gamma} = [\T]_{\beta}^{\gamma} + [\U]_{\beta}^{\gamma}\).
\item \(\mathcal{L}(\V, \W)\) is a vector space.
\item \(\mathcal{L}(\V, \W) = \mathcal{L}(\W, \V)\).
\end{enumerate}
\end{exercise}

\begin{proof} \ 
\begin{enumerate}
\item True by \THM{2.7}(a).
\item True by \RMK{2.2.2}.
\item True by \DEF{2.6}.
\item True by \THM{2.8}(a).
\item True by \THM{2.7}(b).
\item False, the elements in \(\mathcal{L}(\V, \W)\) is \LTRAN{}s from \(\V\) into \(\W\), but the elements in \(\mathcal{L}(\W, \V)\) is \LTRAN{}s from \(\W\) into \(\V\).
\end{enumerate}
\end{proof}

\begin{exercise} \label{exercise 2.2.2}
Let \(\beta\) and \(\gamma\) be the standard ordered bases for \(\SET{R}^n\) and \(\SET{R}^m\), respectively.
For each \LTRAN{} \(\T: \SET{R}^n \to \SET{R}^m\), compute \([\T]_{\beta}^{\gamma}\).
\end{exercise}

\begin{proof} \ 

\begin{enumerate}
\item[(a)] \(\T: \SET{R}^2 \to \SET{R}^3\) defined by \(\T(a_1, a_2) = (2a_1 - a_2, 3a_1 + 4a_2, a_1)\).

\(\T(1, 0) = (2, 3, 1), \T(0, 1) = (-1, 4, 0)\), hence
\[
    [\T]_{\beta}^{\gamma} = \begin{pmatrix} 2 & -1 \\ 3 & 4 \\ 1 & 0 \end{pmatrix}
\]

\item[(e)] \(\T: \SET{R}^n \to \SET{R}^m\) defined by \(\T(a_1, a_2, ..., a_n) = (a_1, a_1, ..., a_1)\).

\(\T(e_1) = (1, 1, ..., 1)\),
\(\T(e_2) = \T(e_3) = ... = \T(e_n) = (0, 0, ..., 0)\), hence
\[
    [\T]_{\beta}^{\gamma} = \begin{pmatrix}
        1 & 0 & 0 & ... & 0 \\
        1 & 0 & 0 & ... & 0 \\
        \vdots & \vdots & \vdots & & \vdots \\
        1 & 0 & 0 & ... & 0
    \end{pmatrix}
\]

\item[(f)] \(\T: \SET{R}^n \to \SET{R}^m\) defined by \(\T(a_1, a_2, ..., a_n) = (a_n, a_{n - 1}, ..., a_1)\).

\(\T(e_1) = (0, 0, ..., 0, 1)\), \(\T(e_2) = (0, 0, ..., 1, 0)\), ..., \(\T(e_n) = (1, 0, 0, ..., 0)\)
hence
\[
    [\T]_{\beta}^{\gamma} = \begin{pmatrix}
        0 & 0 & 0 & ... & 0 & 1 \\
        0 & 0 & 0 & ... & 1 & 0 \\
        \vdots & \vdots & \vdots & & \vdots & \vdots \\
        0 & 1 & 0 & ... & 0 & 0 \\
        1 & 0 & 0 & ... & 0 & 0
    \end{pmatrix}
\]
which is an \href{https://www.wikiwand.com/en/Anti-diagonal_matrix}{anti-diagonal} matrix having \(1\) on the anti-diagonal and \(0\) otherwise.

\item[(g)] \(\T: \SET{R}^n \to \SET{R}\) defined by \(\T(a_1, a_2, ..., a_n) = a_1 + a_n\).

\(\T(e_1) = 1 + 0 = 1, \T(e_2) = 0 + 0 = 0, \T(e_3) = 0 + 0 = 0, ..., \T(e_{n - 1}) = 0 + 0 = 0, \T(e_n) = 0 + 1 = 1\),
hence
\[
    [\T]_{\beta}^{\gamma} = \begin{pmatrix} 1 & 0 & 0 & ... & 0 & 1 \\ \end{pmatrix}
\]
\end{enumerate}
Skip others.
\end{proof}

\begin{exercise} \label{exercise 2.2.3}
Let \(\T : \SET{R}^2 \to \SET{R}^3\) be defined by \(\T(a_1, a_2) = (a_1 - a_2, a_1, 2a_1 + a_2)\).
Let \(\beta\) be the standard ordered basis for \(\SET{R}^2\) and \(\gamma = \{ (1, 1, 0), (0, 1, 1) , (2, 2, 3) \}\).
Compute \([\T]_{\beta}^{\gamma}\).
If \(\alpha = \{ (1, 2), (2, 3) \}\), compute \([\T]_{\alpha}^{\gamma}\).
\end{exercise}

\begin{note}
In this exercise we give a matrix representation of \LTRAN{} in \textbf{non-standard} ordered basis.
\end{note}

\begin{proof} \ 

(By calculation,)
\begin{align*}
    \T(1, 0) & = (1, 1, 2) = -\frac{1}{3}(1, 1, 0) + 0(0, 1, 1) + \frac{2}{3}(2, 2, 3) \\
    \T(0, 1) & = (-1, 0, 1) = -1(1, 1, 0) + 1(0, 1, 1) + 0(2, 2, 3).
\end{align*}
Hence
\[
    [\T]_{\beta}^{\gamma} = \begin{pmatrix} -\frac{1}{3} & -1 \\ 0 & 1 \\ \frac{2}{3} & 0 \end{pmatrix}
\]

For ordered basis \(\alpha\),
\begin{align*}
    \T(1, 2) & = (-1, 1, 4) = -\frac{7}{3}(1, 1, 0) + 2(0, 1, 1) + \frac{2}{3}(2, 2, 3) \\
    \T(2, 3) & = (-1, 2, 7) = -\frac{11}{3}(1, 1, 0) + 3(0, 1, 1) + \frac{4}{3}(2, 2, 3) \\
\end{align*}

Hence
\[
    [\T]_{\alpha}^{\gamma} = \begin{pmatrix} -\frac{7}{3} & -\frac{11}{3} \\ 2 & 3 \\ \frac{2}{3} & \frac{4}{3} \end{pmatrix}
\]
\end{proof}

\begin{exercise} \label{exercise 2.2.4}
Define
\[
\T: M_{2 \X 2}(\SET{R}) \to \mathcal{P}_{2}(\SET{R}) \text { by }
    \T \begin{pmatrix} a & b \\ c & d \end{pmatrix} = (a + b) + (2d)x + bx^{2}.
\]
Let
\[
    \beta = \bigg\{
                \begin{pmatrix} 1 & 0 \\ 0 & 0 \end{pmatrix},
                \begin{pmatrix} 0 & 1 \\ 0 & 0 \end{pmatrix},
                \begin{pmatrix} 0 & 0 \\ 1 & 0 \end{pmatrix},
                \begin{pmatrix} 0 & 0 \\ 0 & 1 \end{pmatrix}
            \bigg\}
    \text { and }
    \gamma = \{ 1, x, x^{2} \}.
\]
Compute \([\T]_{\beta}^{\gamma}\).
\end{exercise}

\begin{proof}
\begin{align*}
    \T \begin{pmatrix} 1 & 0 \\ 0 & 0 \end{pmatrix} = (1 + 0) + (2 \X 0)x + 0x^{2} = 1 + 0x + 0x^2 \\
    \T \begin{pmatrix} 0 & 1 \\ 0 & 0 \end{pmatrix} = (0 + 1) + (2 \X 0)x + 1x^{2} = 1 + 0x + 1x^2 \\
    \T \begin{pmatrix} 0 & 0 \\ 1 & 0 \end{pmatrix} = (0 + 0) + (2 \X 0)x + 0x^{2} = 0 + 0x + 0x^2 \\
    \T \begin{pmatrix} 0 & 0 \\ 0 & 1 \end{pmatrix} = (0 + 0) + (2 \X 1)x + 0x^{2} = 0 + 2x + 0x^2
\end{align*}
Hence
\[
    [\T]_{\beta}^{\gamma} = \begin{pmatrix}
        1 & 1 & 0 & 0 \\
        0 & 0 & 0 & 2 \\
        0 & 1 & 0 & 0
    \end{pmatrix}
\]
\end{proof}

\begin{exercise} \label{exercise 2.2.5}
Let
\[
    \alpha = \bigg\{
                \begin{pmatrix} 1 & 0 \\ 0 & 0 \end{pmatrix},
                \begin{pmatrix} 0 & 1 \\ 0 & 0 \end{pmatrix},
                \begin{pmatrix} 0 & 0 \\ 1 & 0 \end{pmatrix},
                \begin{pmatrix} 0 & 0 \\ 0 & 1 \end{pmatrix}
            \bigg\},
    \beta = \{ 1, x, x^{2} \},
    \text { and }
    \gamma = \{ 1 \}.
\]
\begin{enumerate}
\item Define \(\T : M_{2 \X 2}(F) \to M_{2 \X 2}(F)\) by \(\T(A) = A^\top\).
Compute \([\T]_{\alpha}\).
\item Define
\[
    \T: \POLYRR \to M_{2 \X 2}(\SET{R}) \text{ by }
    \T(f) = \begin{pmatrix} f'(0) & 2f(1) \\ 0 & f''(3) \end{pmatrix},
\]
where \('\) denotes differentiation.
Compute \([\T]_{\beta}^{\alpha}\).
\item Define \(\T: M_{2 \X 2}(F) \to F\) by \(\T(A) = \TRACE(A)\).
Compute \([\T]_{\alpha}^{\gamma}\).
\item Define \(\T: \POLYRR \to \SET{R}\) by \(\T(f) = f(2)\). 
Compute \([\T]_{\beta}^{\gamma}\).
\item If
\[
    A = \begin{pmatrix} 1 & -2 \\ 0 & 4 \end{pmatrix}.
\]
compute \([A]_{\alpha}\)·
\item If \(f(x) = 3 - 6x + x^2\) , compute \([f]_{\beta}\).
\item For \(a \in F\), compute \([a]_{\gamma}\).
\end{enumerate}
\end{exercise}

\begin{proof} \ 
\begin{enumerate}
\item
\begin{align*}
    \T \begin{pmatrix} 1 & 0 \\ 0 & 0 \end{pmatrix}
        = \begin{pmatrix} 1 & 0 \\ 0 & 0 \end{pmatrix}^\top
        = \begin{pmatrix} 1 & 0 \\ 0 & 0 \end{pmatrix}
        = \MAROON{1} \begin{pmatrix} 1 & 0 \\ 0 & 0 \end{pmatrix}
        + \MAROON{0} \begin{pmatrix} 0 & 1 \\ 0 & 0 \end{pmatrix}
        + \MAROON{0} \begin{pmatrix} 0 & 0 \\ 1 & 0 \end{pmatrix}
        + \MAROON{0} \begin{pmatrix} 0 & 0 \\ 0 & 1 \end{pmatrix} \\
    \T \begin{pmatrix} 0 & 1 \\ 0 & 0 \end{pmatrix}
        = \begin{pmatrix} 0 & 1 \\ 0 & 0 \end{pmatrix}^\top
        = \begin{pmatrix} 0 & 0 \\ 1 & 0 \end{pmatrix}
        = \BLUE{0} \begin{pmatrix} 1 & 0 \\ 0 & 0 \end{pmatrix}
        + \BLUE{0} \begin{pmatrix} 0 & 1 \\ 0 & 0 \end{pmatrix}
        + \BLUE{1} \begin{pmatrix} 0 & 0 \\ 1 & 0 \end{pmatrix}
        + \BLUE{0} \begin{pmatrix} 0 & 0 \\ 0 & 1 \end{pmatrix} \\
    \T \begin{pmatrix} 0 & 0 \\ 1 & 0 \end{pmatrix}
        = \begin{pmatrix} 0 & 0 \\ 1 & 0 \end{pmatrix}^\top
        = \begin{pmatrix} 0 & 1 \\ 0 & 0 \end{pmatrix}
        = \RED{0} \begin{pmatrix} 1 & 0 \\ 0 & 0 \end{pmatrix}
        + \RED{1} \begin{pmatrix} 0 & 1 \\ 0 & 0 \end{pmatrix}
        + \RED{0} \begin{pmatrix} 0 & 0 \\ 1 & 0 \end{pmatrix}
        + \RED{0} \begin{pmatrix} 0 & 0 \\ 0 & 1 \end{pmatrix} \\
    \T \begin{pmatrix} 0 & 0 \\ 0 & 1 \end{pmatrix}
        = \begin{pmatrix} 0 & 0 \\ 0 & 1 \end{pmatrix}^\top
        = \begin{pmatrix} 0 & 0 \\ 0 & 1 \end{pmatrix}
        = \GREEN{0} \begin{pmatrix} 1 & 0 \\ 0 & 0 \end{pmatrix}
        + \GREEN{0} \begin{pmatrix} 0 & 1 \\ 0 & 0 \end{pmatrix}
        + \GREEN{0} \begin{pmatrix} 0 & 0 \\ 1 & 0 \end{pmatrix}
        + \GREEN{1} \begin{pmatrix} 0 & 0 \\ 0 & 1 \end{pmatrix} \\
\end{align*}
Hence
\[
    [\T]_{\alpha}
    = \begin{pmatrix}
        \MAROON{1} & \BLUE{0} & \RED{0} & \GREEN{0} \\
        \MAROON{0} & \BLUE{0} & \RED{1} & \GREEN{0} \\
        \MAROON{0} & \BLUE{1} & \RED{0} & \GREEN{0} \\
        \MAROON{0} & \BLUE{0} & \RED{0} & \GREEN{1}
    \end{pmatrix}
\]

\item (By calculation,)
\begin{align*}
    \T(1) = \begin{pmatrix} 0 & 2 \\ 0 & 0 \end{pmatrix}
          = 0 \begin{pmatrix} 1 & 0 \\ 0 & 0 \end{pmatrix}
          + 2 \begin{pmatrix} 0 & 1 \\ 0 & 0 \end{pmatrix}
          + 0 \begin{pmatrix} 0 & 0 \\ 1 & 0 \end{pmatrix}
          + 0 \begin{pmatrix} 0 & 0 \\ 0 & 1 \end{pmatrix} \\
    \T(x) = \begin{pmatrix} 1 & 2 \\ 0 & 0 \end{pmatrix}
          = 1 \begin{pmatrix} 1 & 0 \\ 0 & 0 \end{pmatrix}
          + 2 \begin{pmatrix} 0 & 1 \\ 0 & 0 \end{pmatrix}
          + 0 \begin{pmatrix} 0 & 0 \\ 1 & 0 \end{pmatrix}
          + 0 \begin{pmatrix} 0 & 0 \\ 0 & 1 \end{pmatrix} \\
    \T(x^2) = \begin{pmatrix} 0 & 2 \\ 0 & 2 \end{pmatrix}
          = 0 \begin{pmatrix} 1 & 0 \\ 0 & 0 \end{pmatrix}
          + 2 \begin{pmatrix} 0 & 1 \\ 0 & 0 \end{pmatrix}
          + 0 \begin{pmatrix} 0 & 0 \\ 1 & 0 \end{pmatrix}
          + 2 \begin{pmatrix} 0 & 0 \\ 0 & 1 \end{pmatrix} \\
\end{align*}

Hence
\[
    [\T]_{\beta}^{\alpha}
    = \begin{pmatrix} 0 & 1 & 0 \\ 2 & 2 & 2 \\ 0 & 0 & 0 \\ 0 & 0 & 2 \end{pmatrix}
\]

\item
\begin{align*}
    \T \begin{pmatrix} 1 & 0 \\ 0 & 0 \end{pmatrix}
        = \TRACE \begin{pmatrix} 1 & 0 \\ 0 & 0 \end{pmatrix} = 1 \\
    \T \begin{pmatrix} 0 & 1 \\ 0 & 0 \end{pmatrix}
        = \TRACE \begin{pmatrix} 0 & 1 \\ 0 & 0 \end{pmatrix} = 0 \\
    \T \begin{pmatrix} 0 & 0 \\ 1 & 0 \end{pmatrix}
        = \TRACE \begin{pmatrix} 0 & 0 \\ 1 & 0 \end{pmatrix} = 0 \\
    \T \begin{pmatrix} 0 & 0 \\ 0 & 1 \end{pmatrix}
        = \TRACE \begin{pmatrix} 0 & 0 \\ 0 & 1 \end{pmatrix} = 1
\end{align*}
Hence
\[
    [\T]_{\alpha}^{\gamma} = \begin{pmatrix} 1 & 0 & 0 & 1 \end{pmatrix}
\]

\item
\begin{align*}
    \T(1) = 1 \\
    \T(x) = 2 \\
    \T(x^2 = 4
\end{align*}
Hence
\[
    [\T]_{\beta}^{\gamma} = \begin{pmatrix} 1 & 2 & 4 \end{pmatrix}
\]

\item
\[
    A = \begin{pmatrix} 1 & -2 \\ 0 & 4 \end{pmatrix}
          = 1 \begin{pmatrix} 1 & 0 \\ 0 & 0 \end{pmatrix}
          + -2 \begin{pmatrix} 0 & 1 \\ 0 & 0 \end{pmatrix}
          + 0 \begin{pmatrix} 0 & 0 \\ 1 & 0 \end{pmatrix}
          + 4 \begin{pmatrix} 0 & 0 \\ 0 & 1 \end{pmatrix} \\
\]
Hence
\[
    [A]_{\alpha} = \begin{pmatrix} 1 \\ -2 \\ 0 \\ 4 \end{pmatrix}
\]

\item
\begin{align*}
    f(x) = 3 - 6x + x^2 = 3(1) + (-6)(x) + 1(x^2)
\end{align*}
Hence
\[
    [f]_{\beta} = \begin{pmatrix}
        3 \\ -6 \\ 1
    \end{pmatrix}
\]

\item
\begin{align*}
    a = a \cdot 1 \implies [a]_{\gamma} = (a),
\end{align*}
which is a coordinate vector with only one component.

\end{enumerate}
\end{proof}

\begin{exercise} \label{exercise 2.2.6}
Complete the proof of part (b) of \THM{2.7}.
\end{exercise}

\begin{proof}
By \THM{2.7}(a), the addition and scalar multiplication in \DEF{2.8} is closed.

Now let \(\T, \U, \U' \in \mathcal{L}(\V, \W)\) where \(\V\) and \(\W\) are \(n\)-dimensional and \(m\)-dimensional vector spaces, respectively.

(VS 1): For all \(x \in \V\),
\begin{align*}
    (\T + \U)(x) & = \T(x) + \U(x) & \text{by \DEF{2.8}(1)} \\
                 & = \U(x) + \T(x) & \text{by (VS 1) of \(\W\)} \\
                 & = (\U + \T)(x), & \text{by \DEF{2.8}(1)}
\end{align*}
hence \(\T + \U = \U + \T\).

(VS 2): For all \(x \in \V\),
\begin{align*}
    ((\T + \U) + \U')(x) & = (\T + \U)(x) + \U(x') & \text{by \DEF{2.8}(1)} \\
                         & = (\T(x) + \U(x)) + \U'(x) & \text{by \DEF{2.8}(1)} \\
                         & = \T(x) + (\U(x) + \U'(x)) & \text{by (VS 2) of \(\W\)} \\
                         & = \T(x) + (\U +\U')(x) & \text{by \DEF{2.8}(1)} \\
                         & = (\T + (\U +\U'))(x) & \text{by \DEF{2.8}(1)}
\end{align*}
hence \((\T + \U) + \U' = \T + (\U + \U')\).

(VS 3): We claim the zero transformation \(\TZERO\) is the zero vector of \(\mathcal{L}(\V, \W)\) since for all \(x \in \V\),
\begin{align*}
    (\T + \TZERO)(x) & = \T(x) + \TZERO(x) & \text{by \DEF{2.8}(1)} \\
                     & = \T(x), & \text{by (VS 3) of \(\W\)}
\end{align*}
Hence \(\T = \T + \TZERO\).

(VS 4): Given \(\T \in \mathcal{L}(\V, \W)\) we claim that \(-1 \cdot \T\) is the inverse of \(\T\) since for all \(x \in \V\),
\begin{align*}
    (\T + (-1 \cdot \T))(x) & = \T(x) + (-1 \cdot \T)(x) & \text{by \DEF{2.8}(1)} \\
                            & = \T(x) + (-1)\T(x) & \text{by \DEF{2.8}(2)} \\
                            & = \OW & \text{by (VS 4) of \(\W\)}
\end{align*}
Hence \(\T + (-1 \cdot \T) = \TZERO\).

(VS 5) For all \(x \in \V\),
\begin{align*}
    (1 \cdot \T)(x) & = 1 \T(x) & \text{by \DEF{2.8}(2)} \\
                    & = \T(x) & \text{by (VS 5) of \(\W\)}
\end{align*}
Hence \(1 \cdot \T = \T\).

(VS 6) For all \(x \in \V\),
\begin{align*}
    ((ab)\T)(x) & = (ab)\T(x) & \text{by \DEF{2.8}(2)} \\
                & = a(b\T(x)) & \text{by (VS 6) of \(\W\)} \\
                & = a\Big( (b\T)(x) \Big) & \text{by \DEF{2.8}(2)} \\
                & = \Big( a(b\T) \Big)(x) & \text{by \DEF{2.8}(2)}
\end{align*}
Hence \((ab)\T = a(b\T)\).

(VS 7) For all \(x \in \V\),
\begin{align*}
    (a(\T + \U))(x) & = a((\T + \U)(x)) & \text{by \DEF{2.8}(2)} \\
                    & = a(\T(x) + \U(x)) & \text{by \DEF{2.8}(1)} \\
                    & = a\T(x) + a\U(x) & \text{by (VS 7) of \(\W\)} \\
                    & = (a\T)(x) + (a\U)(x) & \text{by \DEF{2.8}(2)} \\
                    & = (a\T + a\U)(x) & \text{by \DEF{2.8}(1)}
\end{align*}
Hence \(a(\T + \U) = a\T + a\U\).

(VS 8) For all \(x \in \V\),
\begin{align*}
    ((a + b)\T)(x) & = (a + b)\big( (\T)(x) \big) & \text{by \DEF{2.8}(2)} \\
                    & = a\T(x) + b\T(x) & \text{by (VS 8) of \(\W\)} \\
                    & = (a\T)(x) + (b\T)(x) & \text{by \DEF{2.8}(2)} \\
                    & = (a\T + b\T)(x) & \text{by \DEF{2.8}(1)}
\end{align*}
Hence \((a + b)\T = a\T + b\T\).

Hence \(\mathcal{L}(\V, \W)\) with addition and scalar multiplication in \DEF{2.8} is a vector space.
\end{proof}

\begin{exercise} \label{exercise 2.2.7}
Prove part (b) of \THM{2.8}.
\end{exercise}

\begin{proof}
See \THM{2.8}.
\end{proof}

\begin{exercise} \label{exercise 2.2.8}
Let \(\V\) be an \(n\)-dimensional vector space with an ordered basis \(\beta\).
Define \(\T : \V \to F^n\) by \(\T(x) = [x]_{\beta}\).
Prove that \(\T\) is linear.
\end{exercise}

\begin{proof}
USEDINSEC2.4.

Let arbitrary \(x, y \in \V\), \(\beta = \{ v_1, v_2, ..., v_n \}\) be a basis for \(\V\), and scalar \(c\).
Then (by discussion before \DEF{2.6}) there exist unique scalars \(a_1, a_2, ..., a_n\), \(b_1, b_2, ..., b_n\) such that
\[
    x = \sum_{i = 1}^n a_i v_i \text{ and } y = \sum_{i = 1}^n b_i v_i.
\]
And
\begin{align*}
    cx + y & = c\sum_{i = 1}^n a_i v_i + \sum_{i = 1}^n b_i v_i \\
           & = \sum_{i = 1}^n (ca_i + b_i) v_i \MAROON{(1)} & \text{by rule of summation}.
\end{align*}
So
\begin{align*}
    \T(cx + y) & = [cx + y]_{\beta} & \text{by def of \(\T\)} \\
               & = \begin{pmatrix} ca_1 + b_1 \\ ca_2 + b_2 \\ \vdots \\ ca_n + b_n \end{pmatrix} & \text{by \MAROON{(1)}}
\end{align*}
And
\begin{align*}
    c\T(x) + \T(y) & = c[x]_{\beta} + [y]_{\beta} & \text{by def of \(\T\)} \\
                   & = c \begin{pmatrix}
                        a_1 \\ a_2 \\ \vdots \\ a_n
                    \end{pmatrix}
                    + \begin{pmatrix}
                        b_1 \\ b_2 \\ \vdots \\ b_n
                    \end{pmatrix} \\
                   & = \begin{pmatrix} ca_1 + b_1 \\ ca_2 + b_2 \\ \vdots \\ ca_n + b_n \end{pmatrix} & \text{by \(+\) and scalar \(\cdot\) of coordinate vectors}
\end{align*}
Hence \(\T(cx + y) = c\T(x) + \T(y)\), so by \ATHM{2.1}(b), \(\T\) is linear.
\end{proof}

\begin{exercise} \label{exercise 2.2.9}
Let \(\V\) be the vector space of complex numbers \textbf{over the field} \(\SET{R}\).
Define \(\T : \V \to \V\) by \(\T(z) = \conjugatet{z}\), where \(\conjugatet{z}\) is the complex conjugate of \(z\).
Prove that \(\T\) \emph{is} linear, and compute \([\T]_{\beta}\), where \(\beta = \{ 1, \iu \}\).
(Recall by \EXEC{2.1.39} that \(\T\) is \textbf{not linear} if \(\V\) is regarded as a vector space \textbf{over the field} \(\SET{C}\).)
\end{exercise}

\begin{proof}
From \EXEC{2.1.39}, we only have to prove \(\T\) also satisfies \DEF{2.1}(b).
So for any \emph{real number} scalar \(c\), given arbitrary complex number \(z \in \SET{C}\), let \(z = a + b\iu\) where \(a, b \in \SET{R}\), then
\begin{align*}
    \T(cz) & = \T(c(a + b\iu)) \\
           & = \T(ca + cb\iu) & \text{by rule of operations of complex number} \\
           & = \conjugatet{ca + cb\iu} & \text{by def of \(\T\)} \\
           & = ca - cb\iu & \text{by def of conjugate} \\
           & = c(a - b\iu) & \text{by rule of operations of complex number} \\
           & = c(\conjugatet{a + b\iu}) & \text{by def of conjugate} \\
           & = c\T(a + b\iu) & \text{by def of \(\T\)} \\
           & = c\T(z).
\end{align*}
Hence \DEF{2.1}(b) is satisfied, hence \(\T\) is linear.

Now
\begin{align*}
    \T(1) & = \conjugatet{1} = 1 = \MAROON{1} \cdot 1 + \MAROON{0} \cdot \iu \\
    \T(\iu) & = \conjugatet{\iu} = -\iu = \BLUE{0} \cdot 1 + \BLUE{(-1)} \cdot \iu \\
\end{align*}
Hence
\[
    [\T]_{\beta} = \begin{pmatrix}
        \MAROON{1} & \BLUE{0} \\
        \MAROON{0} & \BLUE{-1} \\
    \end{pmatrix}
\]
\end{proof}

\begin{exercise} \label{exercise 2.2.10}
Let \(\V\) be a vector space with the ordered basis \(\beta = \{ v_1, v_2, ..., v_n \}\).
Define \(v_0 = \OV\).
By \THM{2.6}, there exists a \LTRAN{} \(\T: \V \to \V\) such that \(\T(v_j) = v_j + v_{j - 1}\) for \(j = 1, 2, ..., n\).
Compute \([\T]_{\beta}\).
\end{exercise}

\begin{proof}
\begin{align*}
    \T(v_1) & = v_1 + v_0 = v_1 + \OV = v_1 = \MAROON{1} v_1 + \MAROON{0} v_2 + ... + \MAROON{0} v_n \\
    \T(v_2) & = v_2 + v_1 = 1 v_1 + 1 v_2 + 0 v_3 + ... + 0 v_n \\
    \T(v_3) & = v_3 + v_2 = 0 v_1 + 1 v_2 + 1 v_3 + 0 v_4 + ... + 0 v_n \\
    ... \\
    \T(v_n) & = v_n + v_{n - 1} = \BLUE{0} v_1 + \BLUE{0} v_2 + ... + \BLUE{0} v_{n - 2} + \BLUE{1} v_{n - 1} + \BLUE{1} v_n \\
\end{align*}
Hence
\[
    [\T]_{\beta} = \begin{pmatrix}
        \MAROON{1} & 1 & 0 & 0 & 0 & ... & 0 & 0 & \BLUE{0}\\
        \MAROON{0} & 1 & 1 & 0 & 0 & ... & 0 & 0 & \BLUE{0} \\
        \MAROON{0} & 0 & 1 & 1 & 0 & ... & 0 & 0 & \BLUE{0} \\
        \MAROON{\vdots} & \vdots & \vdots & \vdots & \vdots & & \vdots & \vdots & \BLUE{\vdots} \\
        \MAROON{0} & 0 & 0 & 0 & 0 & ... & 1 & 1 & \BLUE{0} \\
        \MAROON{0} & 0 & 0 & 0 & 0 & ... & 0 & 1 & \BLUE{1} \\
        \MAROON{0} & 0 & 0 & 0 & 0 & ... & 0 & 0 & \BLUE{1}
    \end{pmatrix}
\]
\end{proof}

\begin{exercise} \label{exercise 2.2.11}
Let \(\V\) be an \(n\)-dimensional vector space, and let \(\T : \V \to \V\) be a \LTRAN{}.
Suppose that \(\W\) is a \(\T\)-invariant subspace of \(\V\) (see \ADEF{2.3}) having dimension \(k\).
Show that there is an ordered basis \(\beta\) for \(\V\) such that \([\T]_{\beta}\) has the form
\[
    \begin{pmatrix} A & B \\ O & C \end{pmatrix},
\]
where \(A\) is a \(k \X k\) matrix and \(O\) is the \((n - k) \X k\) zero matrix.
\end{exercise}

\begin{proof}
Let \(\beta' = \{ v_1, v_2, ..., v_k \}\) be an ordered basis for \(\W\).
Then (by \CORO{1.11.1}) we can extend \(\beta'\) to get \(\beta = \{ v_1, v_2, ..., v_k, v_{k + 1}, ..., v_n \}\) to get a basis for \(\V\).
Since \(\W\) is \(\T\)-invariant, we have \(\T(v_1), \T(v_2), ..., \T(v_k) \in \W\).
So for \(j = 1, ..., k\), \(\T(v_j)\) can be represented using \(\beta'\), that is,
\[
    \T(v_j) = \sum_{i = 1}^k a_{ij} v_i.
\]
We also can of course using \(\beta\) to represent \(\T(v_j)\) for \(j = 1, ..., k\), but
\begin{align*}
    \T(v_j) & = \sum_{i = 1}^k a_{ij} v_i \\
            & = \sum_{i = 1}^k a_{ij} v_i + \OV \\
            & = \sum_{i = 1}^k a_{ij} v_i + \sum_{i = k + 1}^n \RED{0} \cdot v_i.
\end{align*}
And for \(j = k + 1, ..., n\), we just compute \([\T(v_j)]_{\beta}\) as usual.
Then
\begin{align*}
    [\T]_{\beta} & = \bigg[ [\T(v_1)]_{\beta} \ [\T(v_2)]_{\beta}\ ...\ [\T(v_k)]_{\beta}\ [\T(v_{k + 1})]_{\beta}\ ...\ [\T(v_n)]_{\beta} \bigg] \\
                 & =
    \left[\begin{array}{cccccccc}
        a_{11} & a_{12} & \ldots & a_{1k} & a_{1,k+1} & a_{1,k+2} & \ldots & a_{1n} \\
        a_{21} & a_{22} & \ldots & a_{2k} & a_{2,k+1} & a_{2,k+2} & \ldots & a_{2n} \\
        \vdots & \vdots & \vdots & \vdots & \vdots & \vdots & \vdots & \vdots \\
        a_{k 1} & a_{k 2} & \ldots & a_{kk} & a_{k,k+1} & a_{k,k+2} & \ldots & a_{kn} \\
        \RED{0} & \RED{0} & \RED{\ldots} & \RED{0} & a_{k+1,k+1} & a_{k+1,k+2} & \ldots & a_{k+1,n} \\
        \RED{0} & \RED{0} & \RED{\ldots} & \RED{0} & a_{k+2,k+1} & a_{k+2,k+2} & \ldots & a_{k+2,n} \\
        \RED{\vdots} & \RED{\vdots} & \RED{\vdots} & \RED{\vdots} & \vdots & \vdots & \vdots & \vdots \\
        \RED{0} & \RED{0} & \RED{\ldots} & \RED{0} & a_{n,k+1} & a_{n,k+2} & \ldots & a_{nn}
    \end{array}\right]
\end{align*}
which has the form
\(\left[\begin{array}{cc}
    A & B \\
    O & C
\end{array}\right]\), where \(A\) is a \(k \X k\) matrix and \(O\) is the \((n - k) \X k\) zero matrix.
\end{proof}

\begin{exercise} \label{exercise 2.2.12}
Let \(\beta = \{ v_1, v_2, ..., v_n \}\) be a basis for a vector space \(\V\) and \(\T : \V \to \V\) be a \LTRAN{}.
Prove that \([\T]_{\beta}\) is \textbf{upper triangular} if and only if \(\T(v_j) \in \spann(\{ v_1, v_2, ..., v_j \})\) for \(j = 1, 2, ..., n\).
\end{exercise}

\begin{proof} \ 

\(\Longleftarrow\): Suppose \(\T(v_j) \in \spann(v_1, v_2, ..., v_j)\) for \(j = 1, 2, ..., n\).
Then clearly we have
\begin{align*}
    \T(v_1) & = \sum_{i = 1}^1 a_{i1} v_i = a_{11} v_1. \\
    \T(v_2) & = \sum_{i = 1}^2 a_{i2} v_i = a_{12} v_1 + a_{22} v_2. \\
    \T(v_3) & = \sum_{i = 1}^3 a_{i3} v_i = a_{13} v_1 + a_{23} v_2 + a_{33} v_3 . \\
    ... \\
    \T(v_n) & = \sum_{i = 1}^n a_{in} v_i,
\end{align*}
for unique scalars \(a_{ij}\) where \(1 \le i \le j \le n\).
But that also implies
\begin{align*}
    \T(v_1) & = a_{11} v_1 + \sum_{i = 2}^n \RED{0} v_i. \\
    \T(v_2) & = a_{12} v_1 + a_{22} v_2 + \sum_{i = 3}^n \RED{0} v_i. \\
    \T(v_3) & = a_{13} v_1 + a_{23} v_2 + a_{33} v_3 + \sum_{i = 4}^n \RED{0} v_i. \\
    ... \\
    \T(v_n) & = a_{1n} v_1 + a_{2n} v_2 + a_{3n} v_3 + ... + a_{nn} v_n.
\end{align*}
So
\begin{align*}
    [\T]_{\beta} & = \bigg[ [\T(v_1)]_{\beta}\ [\T(v_2)]_{\beta}\ [\T(v_3)]_{\beta}\ ...\ [\T(v_n)]_{\beta} \bigg] \\
    & = \begin{pmatrix}
        a_{11} & a_{12} & a_{13} & ... & a_{1n} \\
        0      & a_{22} & a_{23} & ... & a_{2n} \\
        0      & 0      & a_{33} & ... & a_{3n} \\
        \vdots & \vdots & \vdots &     & \vdots \\
        0      &      0 &      0 & ... & a_{n-1,n} \\
        0      &      0 &      0 & ... & a_{nn}
    \end{pmatrix} \MAROON{(1)}
\end{align*}
which is an upper triangular matrix.

\(\Longrightarrow\): Suppose \([\T]_{\beta}\) is upper triangular.
Then \([\T]_{\beta}\) has the form like \MAROON{(1)}, and that representation just implies \(\T(v_j) \in \spann(\{ v_1, v_2, ..., v_j \})\) for \(j = 1, 2, ..., n\), as desired.
\end{proof}

\begin{exercise} \label{exercise 2.2.13}
Let \(\V\) be a finite-dimensional vector space and \(\T\) be the \textbf{projection} on \(\W\) along \(\W'\), where \(\W\) and \(\W'\) are subspaces of \(\V\). (See \ADEF{2.2}.)
Find an ordered basis \(\beta\) for \(\V\) such that \([\T]_{\beta}\) is a \emph{diagonal} matrix.
\end{exercise}

\begin{proof}
\(\T\) is a projection on \(\W\) along \(\W'\) implies \(\V = \W \oplus \W'\).
Now let \(\alpha = \{ v_1, v_2, ..., v_k \}\) be a basis for \(\W\), \(\beta = \{ v_{k + 1}, v_{k + 2}, ..., v_n \}\) be a basis for \(\W'\).
Then from \ATHM{1.27}(3.1), \(\alpha \cup \beta\) is a basis for \(\V\).

Now for \(i = \RED{1, ..., k}\),
\begin{align*}
    \T(v_i) & = \T(v_i + \OV) & \text{where \(v_i \in \W, \OV \in \W'\)} \\
            & = v_i & \text{by def of projection} \\
            & = \RED{0} v_1 + ... + \RED{0} v_{i - 1} + \RED{1} v_i + \RED{0} v_{i + 1} + ... + \RED{0} v_n
\end{align*}
and for \(i = k + 1, ..., n\),
\begin{align*}
    \T(v_i) & = \T(\OV + v_i) & \text{where \(\OV \in \W, v_i \in \W'\)} \\
            & = \OV & \text{by def of projection} \\
            & = 0 v_1 + ... + 0 v_i + ... + 0 v_n
\end{align*}
Then clearly,
\[
    [\T]_{\beta} = \begin{pmatrix}
    \RED{1} & \RED{0} & \RED{...} & \RED{0} & ... & 0 \\
    \RED{0} & \RED{1} & \RED{...} & \RED{0} & ... & 0 \\
    \RED{\vdots} & \RED{\vdots} & \RED{\ddots} & \RED{\vdots} & & \vdots \\
    \RED{0} & \RED{0} & \RED{...} & \RED{1} & ... & 0 \\
    0 & 0 & ... & 0 & ... & 0 \\
    \vdots & \vdots & & \vdots & & \vdots \\
    0 & 0 & ... & 0 & ... & 0
    \end{pmatrix}
\]
where the red part is \(k \X k\) identity matrix, and other entries are zero;
in particular \([\T]_{\beta}\) is diagonal.
\end{proof}

\begin{exercise} \label{exercise 2.2.14}
Let \(\V\) and \(\W\) be vector spaces, and let \(\T\) and \(\U\) be \emph{nonzero} \LTRAN{}s from \(\V\) into \(\W\).
If \(\RANGET \cap \RANGE(\U) = \{ \OW \}\), prove that
\(\{ \T, \U \}\) is a \emph{linearly independent} subset of \(\mathcal{L}(\V, \W)\).
\end{exercise}

\begin{proof}
For the sake of contradiction, suppose \(\T, \U\) are not \LID{}.
Then (by \ATHM{1.16}) we have \(\T = c\U\) for \(c \in F\).
Since both \(\T, \U\) are not equal to \(\TZERO\), that implies \(c \ne 0\), and there exists \(v \in \V\) such that \(\U(v) \ne \OW\).
Let \(w = \U(v)\).
Then
\begin{align*}
    \T(v) & = (c\U)(v) & \text{since \(\T = c\U\)} \\
          & = c\U(v) & \text{by \DEF{2.8}(2)} \\
          & = cw \\
    \implies & \T(\frac{v}{c}) = w & \text{since \(\T\) is linear}
\end{align*}
So the nonzero vector \(w\) is both in the range of \(\T\) and \(\U\), hence \(\RANGET \cap \RANGE(\U) \ne \{ \OW \}\), which is a contradiction.

So \(\T, \U\) are \LID{}.
\end{proof}

\begin{exercise} \label{exercise 2.2.15}
Let \(\V = \mathcal{P}(\SET{R})\), and for \(j \ge 1\) define \(\T_j(f) = f^{(j)}\), where \(f^{(j)}\) is the \(j\)th derivative of \(f\).
Prove that the set \(\{ \T_1, \T_2, ..., \T_n \}\) is a
linearly independent subset of \(\mathcal{L}(\V)\) for any positive integer \(n\).
\end{exercise}

\begin{proof}
Given positive integer \(n\), suppose
\[
    a_1\T_1 + a_2\T_2 + ... + a_n\T_n = \TZERO \MAROON{(1)},
\]
where \(\TZERO\) is the zero transformation (from \(\V\) to \(\V\)).
we have to show \(a_1 = a_2 = ... = a_n\) to show \(\T_1, \T_2, ..., \T_n\) are \LID{}.

So in particular given a polynomial \(x^n \in \mathcal{P}(\SET{R})\), from \MAROON{(1)}
\begin{align*}
      & a_1 \T_1(x^n) + a_2 \T_2(x^n) + ... + a_n \T_n(x^n) \\
    = & \TZERO(x^n) \\
    = & \RED{0}, \MAROON{\ \ \ \ \ \ \ (2)} & \text{by def of zero transformation}
\end{align*}
where \(\RED{0} \in \mathcal{P}(\SET{R})\) is the zero polynomial.
But,
\begin{align*}
    \T_i(x^n) & = (x^n)^{(i)} & \text{by definition of \(\T_i\)} \\
              & = \frac{n!}{(n-i)!}x^{n - i} & \text{by Calculus}
\end{align*}
So from \MAROON{(2)},
\begin{align*}
    \RED{0} & = a_1 \T_1(x^n) + a_2 \T_2(x^n) + ... + a_n \T_n(x^n) \\
            & = a_1 n x^{n - 1} + a_2 n(n-1) x^{n - 2} + ... + a_n n! 1.
\end{align*}
But by \EXEC{1.5.18}, since no two of \(x^{n - 1}, x^{n - 2}, ... x, 1\) have the same degree, they are \LID{} subset of \(\mathcal{P}(\SET{R})\),
which implies \(a_1 n = a_2 n(n - 1) = ... = a_n n! = 0\),
which trivially implies \(a_1 = a_2 = ... = a_n = 0\) since any factorial is not equal to zero.

Hence from \MAROON{(1)}, \(\T_1, \T_2, ..., \T_n\) is \LID{}.
\end{proof}

\begin{exercise} \label{exercise 2.2.16}
Let \(\V\) and \(\W\) be vector spaces, and let \(S\) be a \emph{subset} of \(\V\).
Define
\[
    S^0 = \{ \T \in \mathcal{L}(V, W) : \T(x) = \OW \text{ for all } x \in S \}.
\]
Prove the following statements.
\begin{enumerate}
\item \(S^0\) is a subspace of \(\mathcal{L}(\V, \W)\).
\item If \(S_1\) and \(S_2\) are \emph{subsets} of \(\V\) and \(S_1 \subseteq S_2\), then \(S_2^0 \subseteq S_1^0\).
\item If \(\V_1\) and \(\V_2\) are \emph{subspaces} of \(\V\), then \((\V_1 \cup \V_2)^0 = (\V_1 + \V_2)^0 = \V_1^0 \cap \V_2^0\).
\end{enumerate}
\end{exercise}

\begin{proof} \ 
\begin{enumerate}
\item Since \(\TZERO(x) = \OW\) for all \(x \in \V\), in particular \(\TZERO(x) = \OW\) for all \(x \in S \subseteq \V\), so \(\TZERO \in S^0\).
    Now suppose \(\T, \U \in S^0\).
    So \(\T(x) = \OW\) and \(\U(x) = \OW\) for all \(x \in S\).
    And for all \(x \in S\),
    \begin{align*}
        (\T + \U)(x) & = \T(x) + \U(x) & \text{by \DEF{2.8}(1)} \\
                     & = \OW + \OW \\
                     & = \OW,
    \end{align*}
    hence \(\T + \U \in S^0\).
    And given arbitrary scalar \(c\), for all \(x \in S\),
    \begin{align*}
        (c\T)(x) & = c\T(x) & \text{by \DEF{2.8}(2)} \\
                     & = c\OW
                     & = \OW,
    \end{align*}
    hence \(c\T\in S^0\).
    So by \THM{1.3}, \(S^0\) is a subspace of \(\mathcal{L}(\V, \W)\).

\item Suppose arbitrary \(\T \in S^0_2\), we have to show \(\T \in S^0_1\) to show \(S^0_2 \subseteq S^0_1\).
Then by definition of \(S^0_2\), for all \(x \in S_2\), \(\T(x) = \OW\).
But since \(S_1 \subseteq S_2\), in particular for all \(x \in S_1\), \(\T(x) = \OW\).
So by definition of \(S^0_1\), \(\T \in S^0_1\), as desired.

\item Note that now \(\V_1, \V_2\) are not only subsets but also subspaces of \(\V\).
So we have some set inclusion:
\begin{itemize}
    \item \(\V_1 \cup \V_2 \subseteq \V_1 + \V_2\).
    \item \(\V_1 \subseteq \V_1 + \V_2\).
    \item \(\V_2 \subseteq \V_1 + \V_2\).
\end{itemize}

Now we prove the equation by showing four set inclusions:
\begin{itemize}
    \item \((\V_1 \cup \V_2)^0 \subseteq (\V_1 + \V_2)^0\):
        Suppose arbitrary \(\T \in (\V_1 \cup \V_2)^0\),
        Then for all \(x \in \V_1 \cup \V_2\) \(\T(x) = \OW\) \BLUE{(1)}.
        
        Now we show that for all \(v \in \V_1 + \V_2\), \(\T(v) = \OW\), to conclude that \(\T \in (\V_1 + \V_2)^0\).
        So let arbitrary \(v \in \V_1 + \V_2\).
        Then \(v = v_1 + v_2\) for some \(v_1 \in \V_1\) and \(v_2 \in \V_2\).
        Since \(v_1 \in \V_1\), in particular \(v_1 \in \V_1 \cup \V_2\), so by \BLUE{(1)}, \(\T(v_1) = \OW\).
        Similarly, since \(v_2 \in \V_2\), in particular \(v_2 \in \V_1 \cup \V_2\), so by \BLUE{(1)}, \(\T(v_2) = \OW\).
        Hence
        \begin{align*}
                     & \T(v_1) + \T(v_2) = \OW + \OW = \OW \\
            \implies & \T(v_1 + v_2) = \OW & \text{since \(\T\) is linear} \\
            \implies & \T(v) = \OW.
        \end{align*}
        Since \(v\) is arbitrary, for all \(v \in \V_1 + \V_2\), \(\T(v) = \OW\), so \(\T \in (\V_1 + \V_2)^0\);
        and since \(\T\) is also arbitrary, \((\V_1 \cup \V_2)^0 \subseteq (\V_1 + \V_2)^0\).
    \item \((\V_1 + \V_2)^0 \subseteq (\V_1 \cup \V_2)^0\):
        Since \(\V_1 \cup \V_2 \subseteq \V_1 + \V_2\), by part(b), \((\V_1 + \V_2)^0 \subseteq (\V_1 \cup \V_2)^0\).
    \item \((\V_1 + \V_2)^0 \subseteq \V_1^0 \cap \V_2^0\):
        Since \(\V_1 \subseteq \V_1 + \V_2\), by part(b), \((\V_1 + \V_2)^0 \subseteq \V_1^0\).
        Similarly since \(\V_2 \subseteq \V_1 + \V_2\), by part(b), \((\V_1 + \V_2)^0 \subseteq \V_2^0\).
        So we have \((\V_1 + \V_2)^0 \subseteq \V_1^0\) and \((\V_1 + \V_2)^0 \subseteq \V_2^0\);
        in particular by set theory, \((\V_1 + \V_2)^0 \subseteq \V_1^0 \cap \V_2^0\).
    \item \(\V_1^0 \cap \V_2^0 \subseteq (\V_1 + \V_2)^0\):
        Now suppose arbitrary \(\T \in \V_1^0 \cap \V_2^0\).
        Then in particular \(\T \in \V_1^0\) and \(\T \in \V_2^0\).
        So by definition, for all \(x \in \V_1\), \(\T(x) = \OW\), and for all \(y \in \V_2\), \(\T(y) = \OW\).
        But then
        \begin{align*}
                     & \forall x \in V_1, y \in V_2, \T(x) = \OW \land \T(y) = \OW \\
            \implies & \forall x \in V_1, y \in V_2, \T(x) + \T(y) = \OW + \OW = \OW \\
            \implies & \forall x \in \V_1, y \in \V_2, \T(x + y) = \OW & \text{since \(\T\) is linear}
        \end{align*}
        which implies for all \(v \in \V_1 + \V_2\) where \(v = x + y\) with \(x \in \V_1\) and \(y \in \V_2\), \(\T(v) = \T(x + y) = \OW\).
        Then by definition, \(\T \in (\V_1 + \V_2)^0\).
        Since \(\T\) is arbitrary, \(\V_1^0 \cap \V_2^0 \subseteq (\V_1 + \V_2)^0\).
\end{itemize}

With these four set inclusions, we have \((\V_1 \cup \V_2)^0 = (\V_1 + \V_2)^0 = \V_1^0 \cap \V_2^0\), as desired.
\end{enumerate}
\end{proof}

\begin{exercise} \label{exercise 2.2.17}
Let \(\V\) and \(\W\) be vector spaces such that \(\dim(\V) = \dim(\W)\), and let \(\T : \V \to \W\) be linear.
(And by \DEF{1.9}, using notation \(\dim\) implies \(\V, \W\) have \emph{finite} dimensions.)
Show that there exist ordered bases \(\beta\) and \(\gamma\) for
\(\V\) and \(\W\), respectively, such that \([T]_{\beta}^{\gamma}\) is a \emph{diagonal} matrix.
\end{exercise}

\begin{proof}
Let \(\{ v_1, ..., v_n \}\) be a basis of \(\NULLT\) in \(\V\).
By \CORO{1.11.1}, we extend it to an ordered basis \(\beta = \{ v_1, ..., v_n, w_1, ..., w_m \}\) for \(\V\).

By the proof in \THM{2.3} we have shown that \(\{ \T(w_1), ..., \T(w_m) \}\) is an (ordered) basis for \(\RANGET\) in \(\W\).
Again we extend it to an ordered basis \(\gamma = \{ u_1, ..., u_n, \T(w_1), ..., \T(w_m) \}\) for \(\W\).

Then for the vectors \(v_1, ..., v_n, w_1, ..., w_m\) in the ordered basis \(\beta\) for \(\V\), we have
\begin{align*}
         \T(v_1) = \OW & = 0 u_1 + ... + 0 u_n + 0 \T(w_1) + 0 \T(w_2) + ... + 0 \T(w_m) \\
         \T(v_2) = \OW & = 0 u_1 + ... + 0 u_n + 0 \T(w_1) + 0 \T(w_2) + ... + 0 \T(w_m) \\
                       & \vdots \\
         \T(v_n) = \OW & = 0 u_1 + ... + 0 u_n + 0 \T(w_1) + 0 \T(w_2) + ... + 0 \T(w_m) \\
    \T(w_1) = 1\T(w_1) & = 0 u_1 + ... + 0 u_n + 1 \T(w_1) + 0 \T(w_2) + ... + 0 \T(w_m) \\
    \T(w_2) = 1\T(w_2) & = 0 u_1 + ... + 0 u_n + 0 \T(w_1) + 1 \T(w_2) + ... + 0 \T(w_m) \\
                       & \vdots \\
    \T(w_m) = 1\T(w_m) & = 0 u_1 + ... + 0 u_n + 0 \T(w_1) + 0 \T(w_2) + ... + 1 \T(w_m) \\
\end{align*}

Then from the equations above it's clear that \([\T]_{\beta}^{\gamma}\) is diagonal, and in fact simply has lower right block \(I_m\), and \(0\) elsewhere.

In particular, if \(\T\) is one-to-one, then \(m\) is equal to \(\dim(\V) = \dim(\W)\), and \([\T]_{\beta}^{\gamma}\) is equal to \(I_m\);
that is, \([\T]_{\beta}^{\gamma}\) is the identity matrix.
\end{proof}

\begin{note}
\EXEC{2.2.17} in fact has concept related to \emph{diagonalization}, see \CH{5}.
\end{note}

\begin{additional theorem} \label{athm 2.16}
This is the placeholder theorem for \EXEC{2.2.8}: Given \(n\)-dimensional \(\V\), \(\T : \V \to F^n\) by \(\T(v) = [v]_{\beta}\), where \(\beta\) is a basis for \(\V\), is linear.
\end{additional theorem}

\begin{additional theorem} \label{athm 2.17}
This is the placeholder theorem for \EXEC{2.2.9}:
If \(\V\) be the vector space of complex numbers \textbf{over the field} \(\SET{R}\).
Then \(\T : \V \to \V\) by \(\T(z) = \conjugatet{z}\), then \(\T\) \emph{is} linear.
But by \EXEC{2.1.39}, that \(\T\) is \textbf{not linear} if \(\V\) is regarded as a vector space \textbf{over the field} \(\SET{C}\).

\end{additional theorem}

\begin{additional theorem} \label{athm 2.18}
This is the placeholder theorem for \EXEC{2.2.11}:

Let \(\V\) be an \(n\)-dimensional vector space, and let \(\T : \V \to \V\) be a \LTRAN{}.
If \(\W\) is a \(\T\)-invariant subspace of \(\V\) (see \ADEF{2.3}) having dimension \(k\), then there is an ordered basis \(\beta\) for \(\V\) such that \([\T]_{\beta}\) has the form
\[
    \begin{pmatrix} A & B \\ O & C \end{pmatrix},
\]
where \(A\) is a \(k \X k\) matrix and \(O\) is the \((n - k) \X k\) zero matrix.
\end{additional theorem}

\begin{additional theorem} \label{athm 2.19}
This is the placeholder theorem for

\BLUE{(1)}: \EXEC{2.2.12}: 
Let \(\beta = \{ v_1, v_2, ..., v_n \}\) be a basis for a vector space \(\V\) and \(\T : \V \to \V\) be a \LTRAN{}.
Then \([\T]_{\beta}\) is \textbf{upper triangular} if and only if \(\T(v_j) \in \spann(\{ v_1, v_2, ..., v_j \})\) for \(j = 1, 2, ..., n\).

\BLUE{(2)}: \EXEC{2.2.13}:
Let \(\V\) be a finite-dimensional vector space and \(\T\) be the \textbf{projection} on \(\W\) along \(\W'\), where \(\W\) and \(\W'\) are subspaces of \(\V\).
Then we can find an ordered basis \(\beta\) for \(\V\) such that \([\T]_{\beta}\) is a \emph{diagonal} matrix.
\end{additional theorem}

\begin{additional theorem} \label{athm 2.20}
This is the placeholder theorem for \EXEC{2.2.14}:

Let \(\V\) and \(\W\) be vector spaces, and let \(\T\) and \(\U\) be \emph{nonzero} \LTRAN{}s from \(\V\) into \(\W\).
If \(\RANGET \cap \RANGE(\U) = \{ \OW \}\), then
\(\{ \T, \U \}\) is a \emph{linearly independent} subset of \(\mathcal{L}(\V, \W)\).
\end{additional theorem}

\begin{additional theorem} \label{athm 2.21}
This is the placeholder theorem for \EXEC{2.2.15}:

Let \(\V = \mathcal{P}(\SET{R})\), and for \(j \ge 1\) define \(\T_j(f) = f^{(j)}\), where \(f^{(j)}\) is the \(j\)th derivative of \(f\).
Then \(\{ \T_1, \T_2, ..., \T_n \}\) is a
linearly independent subset of \(\mathcal{L}(\V)\) for any positive integer \(n\).
\end{additional theorem}

\begin{additional theorem} \label{athm 2.22}
This is the placeholder theorem for \EXEC{2.2.16}:

Some properties of \(S^0\) of \(\mathcal{L}(\V, \W)\).
\end{additional theorem}

\begin{additional theorem} \label{athm 2.23}
This is the placeholder theorem for \EXEC{2.2.17}:

Given \(\V, \W\) having same finite-dimensions, and given any linear \(\T : \V \to \W\), we can find ordered bases \(\beta, \gamma\) of \(\V, \W\) respectively, such that any \([\T]_{\beta}^{\gamma}\) is diagonal. 
\end{additional theorem}