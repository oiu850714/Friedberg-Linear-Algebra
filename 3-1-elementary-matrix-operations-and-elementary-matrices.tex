\section{Elementary Matrix Operations and Elementary Matrices} \label{sec 3.1}

In this section, we define the elementary operations that are used throughout the chapter.
In subsequent sections, we use these operations to obtain simple computational methods for determining the rank of a linear transformation and the solution of a system of linear equations.
There are two types of elementary matrix operations -- row operations and column operations.
As we will see, the \emph{row} operations are more useful;
they \emph{arise from} the three operations that can be \emph{used to eliminate variables} in a system of linear equations.

\begin{definition} \label{def 3.1}
Let \(A\) be an \(m \X n\) matrix.
Any one of the following three operations on the rows [columns] of \(A\) is called an \textbf{elementary row [column] operation}:
\begin{enumerate}
\item[(1)] interchanging any two rows [columns] of \(A\);
\item[(2)] multiplying any row [column] of \(A\) by a \emph{nonzero} scalar;
\item[(3)] adding any scalar multiple of a row [column] of \(A\) to \emph{other} row [column].
\end{enumerate}

Any of these three operations is called an \textbf{elementary operation}.
Elementary operations are of \textbf{type 1}, \textbf{type 2} or \textbf{type 3} depending on whether they are obtained by (1), (2), or (3).
\end{definition}

\begin{note}
I use abbreviations e.r.o. and e.c.o. to represent elementary row operation and elementary column operation, respectively.
\end{note}

\begin{example} \label{example 3.1.1}
Let
\[
    A = \begin{pmatrix}
        1 & 2 & 3 & 4 \\
        2 & 1 & -1 & 3 \\
        4 & 0 & 1 & 2
    \end{pmatrix}.
\]
Interchanging the second row of \(A\) with the first row is an example of an e.r.o. of type 1. 
The resulting matrix is
\[
    B = \begin{pmatrix}
        2 & 1 & -1 & 3 \\
        1 & 2 & 3 & 4 \\
        4 & 0 & 1 & 2
    \end{pmatrix}.
\]
Multiplying the second column of \(A\) by \(3\) is an example of an e.c.o. of type 2.
The resulting matrix is
\[
    C = \begin{pmatrix}
        1 & 6 & 3 & 4 \\
        2 & 3 & -1 & 3 \\
        4 & 0 & 1 & 2
    \end{pmatrix}.
\]
Adding \(4\) times the third row of \(A\) to the first row is an example of an e.r.o. of type 3.
In this case, the resulting matrix is
\[
    M = \begin{pmatrix}
        17 & 2 & 7 & 12 \\
        2 & 1 & -1 & 3 \\
        4 & 0 & 1 & 2
    \end{pmatrix}.
\]
\end{example}

\begin{remark} \label{remark 3.1.1}
Notice that if a matrix \(Q\) can be obtained from a matrix \(P\) by means of an e.r.o., 
then \(P\) can be obtained from \(Q\) by an e.r.o. \textbf{of the same type}.
(See \EXEC{3.1.8}.)
So, in \EXAMPLE{3.1.1}, \(A\) can be obtained from \(M\) by adding \(-4\) times the third row of \(M\) to the first row of \(M\).
\end{remark}

\begin{definition} \label{def 3.2}
An \(n \X n\) \textbf{elementary matrix} is a matrix obtained by performing an(one) elementary operation on \(I_n\).
The elementary matrix is said to be of \textbf{type 1}, \textbf{2}, or \textbf{3} according to whether the elementary operation
performed on \(I_n\) is a type 1, 2, or 3 operation, respectively.
\end{definition}

For example, interchanging the first two \emph{rows} of \(I_3\) produces the elementary matrix
\[
    E = \begin{pmatrix}
        0 & 1 & 0 \\
        1 & 0 & 0 \\
        0 & 0 & 1
    \end{pmatrix}
\]

\begin{remark} \label{remark 3.1.2}
Note that \(E\) can also be obtained by interchanging the first two \emph{columns} of \(I_3\).
In fact, any elementary matrix can be obtained in \emph{at least two ways} -- either by performing an elementary \emph{row} operation on \(I_n\) or by performing an elementary \emph{column} operation on \(I_n\).
(See \EXEC{3.1.4}.)
Similarly,
\[
    \begin{pmatrix}
        1 & 0 & -2 \\
        0 & 1 & 0 \\
        0 & 0 & 1
    \end{pmatrix}
\]
is an elementary matrix since it can be obtained from \(I_3\) by an e.c.o. of type 3
(adding \(-2\) times the first \emph{column} of \(I_3\) to the third \emph{column})
or by an e.r.o. of type 3 (adding \(-2\) times the third \emph{row} to the first \emph{row}).
\end{remark}

Our first theorem shows that \emph{performing an e.r.o. on a matrix is equivalent to multiplying the matrix by an elementary matrix}.

\begin{theorem} \label{thm 3.1}
Let \(A \in M_{m \X n}(F)\), and suppose that \(B\) is obtained from \(A\) by performing an elementary row [column] operation.
Then there exists an \(m \X m\) [\(n \X n\)] elementary matrix \(E\) such that \(B = EA\) [\(B = AE\)].
In fact, \(E\) is obtained from \(I_m\) [\(I_n\)] by \textbf{performing the same} elementary row [column] operation \textbf{as} that which was performed on \(A\) to obtain \(B\).

Conversely, if \(E\) is an elementary \(m \X m\) [\(n \X n\)] matrix, then \(EA\) [\(AE\)] is the matrix obtained from \(A\) by \textbf{performing the same} elementary row [column] operation \textbf{as} that which produces \(E\) from \(I_m\) [\(I_n\)].
\end{theorem}

\begin{proof}
See \EXEC{3.1.7}.
\end{proof}

\begin{example} \label{example 3.1.2}
Consider the matrices \(A\) and \(B\) in \EXAMPLE{3.1.1}.
In this case, \(B\) is obtained from \(A\) by interchanging the first two rows of \(A\).
Performing this \emph{same} operation on \(I_{3}\), we obtain the elementary matrix
\[
    E = \begin{pmatrix}
        0 & 1 & 0 \\
        1 & 0 & 0 \\
        0 & 0 & 1
    \end{pmatrix}
\]
Note that \(EA = B\).

In the second part of \EXAMPLE{3.1.1}, \(C\) is obtained from \(A\) by multiplying the second column of \(A\) by \(3\).
Performing this same operation on \(I_{4}\), we obtain the elementary matrix

\[
    E = \begin{pmatrix}
        1 & 0 & 0 & 0 \\
        0 & 3 & 0 & 0 \\
        0 & 0 & 1 & 0 \\
        0 & 0 & 0 & 1
    \end{pmatrix}
\]
Observe that \(AE = C\).
\end{example}

It is a useful fact that the \emph{inverse} of an elementary matrix is \emph{also} an elementary matrix.

\begin{theorem} \label{theorem} \label{thm 3.2}
Elementary matrices are \emph{invertible}, and the inverse of an elementary matrix is an elementary matrix \emph{of the same type}.
\end{theorem}

\begin{proof}
Let \(E\) be an elementary \(n \X n\) matrix.
Then (by \DEF{3.1}) \(E\) can be obtained by an e.r.o. on \(I_n\).
By \emph{reversing the steps} used to transform \(I_n\) into \(E\), we can transform \(E\) back into \(I_n\).
The result is that \(I_n\) can be obtained from \(E\) by an e.r.o. of the same type.
By \THM{3.1}, there is an elementary matrix \(\overline{E}\) such that \(\overline{E}E = I_n\).
Therefore, by \ATHM{2.38}(``one-sided'' inverse is a ``two-sided'' inverse), \(E\) is invertible and \(E^{-1} = \overline{E}\).
\end{proof}

\exercisesection

\begin{exercise} \label{exercise 3.1.1}
Label the following statements as true or false.
\begin{enumerate}
\item An elementary matrix is always square.
\item The only entries of an elementary matrix are zeros and ones. \item The \(n \X n\) identity matrix is an elementary matrix.
\item The product of two \(n \X n\) elementary matrices is an elementary matrix.
\item The inverse of an elementary matrix is an elementary matrix.
\item The sum of two \(n \X n\) elementary matrices is an elementary matrix.
\item The transpose of an elementary matrix is an elementary matrix.
\item If \(B\) is a matrix that can be obtained by performing an e.r.o. on a matrix \(A\), then \(B\) can also be obtained by performing an e.\textbf{\RED{c}}.o. on \(A\).
\item If \(B\) is a matrix that can be obtained by performing an e.r.o. on a matrix \(A\), then \(A\) can be obtained by performing an e.r.o. on \(B\).
\end{enumerate}
\end{exercise}

\begin{proof} \ 

\begin{enumerate}
\item True. Since every elementary matrix is obtained from \(I_n\) by an elementary operation that does not change the size of the matrix, and \(I_n\) is square.
\item False. \(\begin{pmatrix} 2 & 0 & 0 \\ 0 & 1 & 0 \\ 0 & 0 & 1\end{pmatrix}\) is elementary of type 2.
\item True. \(I_n\) can be considered as a type 2 elementary matrix with the first row multiplied by \(1\).
\item False. \(\begin{pmatrix} 2 & 0 \\ 0 & 1 \end{pmatrix} \begin{pmatrix} 1 & 0 \\ 0 & 2 \end{pmatrix} = \begin{pmatrix} 2 & 0 \\ 0 & 2 \end{pmatrix}\), which needs ``two'' elementary operations to obtain from \(I_2\).
\item True by \THM{3.2}.
\item False. \(\begin{pmatrix} 1 & 0 \\ 0 & 1 \end{pmatrix} + \begin{pmatrix} 1 & 0 \\ 0 & 1 \end{pmatrix} = \begin{pmatrix} 2 & 0 \\ 0 & 2 \end{pmatrix}\), which is not elementary by part(d).
\item True by \EXEC{3.1.5}.
\item False. The matrix \(\begin{pmatrix} 5 & 5 \\ 1 & 1 \end{pmatrix}\) can be obtained from \(\begin{pmatrix} 1 & 1 \\ 1 & 1 \end{pmatrix}\) by a single e.r.o. of type 3, but not by any number of \emph{column} operations, since any e.c.o. on \(\begin{pmatrix} 1 & 1 \\ 1 & 1 \end{pmatrix}\) will still leave columns that are multiple of \(\begin{pmatrix} 1 \\ 1 \end{pmatrix}\).
\item True by \EXEC{3.1.8}.
\end{enumerate}
\end{proof}

\begin{exercise} \label{exercise 3.1.2}
Let
\[
    A = \left(\begin{array}{rrr}
        1 & 2 & 3 \\
        1 & 0 & 1 \\
        1 & -1 & 1
    \end{array}\right),
    B = \left(\begin{array}{rrr}
        1 & 0 & 3 \\
        1 & -2 & 1 \\
        1 & -3 & 1
    \end{array}\right), \text { and }
    C = \left(\begin{array}{rrr}
        1 & 0 & 3 \\
        0 & -2 & -2 \\
        1 & -3 & 1
    \end{array}\right)
\]
Find an elementary operation that transforms \(A\) into \(B\) and an elementary operation that transforms \(B\) into \(C\).
\emph{By means of several additional operations}, transform \(C\) into \(I_3\).
\end{exercise}

\begin{proof}
Using type 3 e.c.o., we add column \(1\) of \(A\) times \(-2\) to column \(2\) of \(A\) to obtain \(B\);
and using type 3. e.r.o., we add row \(1\) of \(B\) times \(-1\) to row \(2\) of \(B\) to obtain \(C\);

Finally, the process of obtaining \(I_3\) from \(C\) is:
\[
    \begin{aligned}
    C & =\left(\begin{array}{ccc}
        1 & 0 & 3 \\
        0 & -2 & -2 \\
        1 & -3 & 1
    \end{array}\right)
    \rightsquigarrow
    \left(\begin{array}{ccc}
        1 & 0 & 3 \\
        0 & \RED{1} & \RED{1} \\
        1 & -3 & 1
    \end{array}\right) \\
      & \rightsquigarrow
    \left(\begin{array}{ccc}
        1 & 0 & 3 \\
        0 & 1 & 1 \\
        \RED{0} & -3 & \RED{-2}
    \end{array}\right)
    \rightsquigarrow
    \left(\begin{array}{ccc}
        1 & 0 & 3 \\
        0 & 1 & 1 \\
        0 & \RED{0} & \RED{1}
    \end{array}\right) \\
    & \rightsquigarrow
    \left(\begin{array}{ccc}
        1 & 0 & \RED{0} \\
        0 & 1 & 1 \\
        0 & 0 & 1
    \end{array}\right)
    \rightsquigarrow\left(\begin{array}{ccc}
        1 & 0 & 0 \\
        0 & 1 & \RED{0} \\
        0 & 0 & 1
    \end{array}\right)
\end{aligned}
\]
\end{proof}

\begin{exercise} \label{exercise 3.1.3}
Use the proof of \THM{3.2} to obtain the \emph{inverse} of each of the following elementary matrices.
\[
    \text{(a)} \left(\begin{array}{lll}0 & 0 & 1 \\ 0 & 1 & 0 \\ 1 & 0 & 0\end{array}\right)
    \text{ (b) } \left(\begin{array}{lll}1 & 0 & 0 \\ 0 & 3 & 0 \\ 0 & 0 & 1\end{array}\right)
    \text{ (c) } \left(\begin{array}{rrr}1 & 0 & 0 \\ 0 & 1 & 0 \\ -2 & 0 & 1\end{array}\right)
\]
\end{exercise}

\begin{proof} \ 

\begin{enumerate}
\item This is type 1 e.r.o. by changing row \(1, 3\).
So the inverse is also type 1 e.r.o. by changing row \(1, 3\) back;
that is, the inverse is \(\begin{pmatrix} 0 & 0 & 1 \\ 0 & 1 & 0 \\ 1 & 0 & 0 \end{pmatrix}\) itself.

\item This is type 2 e.r.o. by multiplying row \(2\) by \(3\).
So the inverse is type 2 e.r.o. by multiplying row \(2\) by \(\frac1{3}\);
that is, \(\begin{pmatrix} 1 & 0 & 0 \\ 0 & \frac1{3} & 0 \\ 0 & 0 & 1 \end{pmatrix}\).

\item This is type 3 e.r.o. by adding row \(1\) times \(-2\) to row \(3\).
So the inverse is type 3 e.r.o. by add row \(1\) time \(2\) to row \(3\).
that is,\(\begin{pmatrix} 1 & 0 & 0 \\ 0 & 1 & 0 \\ 2 & 0 & 1 \end{pmatrix}\).
\end{enumerate}
\end{proof}

\begin{exercise} \label{exercise 3.1.4}
Prove the assertion made on \RMK{3.1.2}:
Any elementary \(n \X n\) matrix can be obtained in at least two ways -- either by performing an e.r.o. on \(I_n\) or by performing an e.c.o. on \(I_n\).
\end{exercise}

\begin{proof}
Let \(E\) be an \(n \X n\) elementary matrix.
There are three types of elementary operations:
\begin{enumerate}
\item[1.] Interchange two columns or two rows.
\item[2.] Multiply any row or any column of \(E\) by a nonzero scalar.
\item[3.] Add any scalar multiple of a row of \(E\) to another row, or add any scalar multiple of a column of \(E\) to another column.
\end{enumerate}

We say that an elementary matrix is type (1), (2) or (3) if it is obtained performing an elementary operation type 1, 2 or 3, respectively, on \(I_n\).

If we suppose that \(E\) is type (1) and suppose that an elementary operation is performed on rows \(i\) and \(j\) of  \(I_n\),
then by interchanging \emph{columns} \(i\) and \(j\) of \(I_n\) we get the same matrix.

If we suppose that \(E\) is type (2) by multiplying row \(i\) of \(I_n\) using a nonzero scalar \(a\), than \(E\) is a matrix where all entries are same as in \(I_n\), except an item on a position \((i, i)\) in \(E\), which is equal \(a\).
If we multiply a \emph{column} \(i\) by a nonzero scalar \(a\) on \(I_n\), we can get the same matrix \(E\).

Finally, suppose that \(E\) is type (3) by multiplying row \(i\) of \(I_n\) by a nonzero scalar \(a\) and adding to row \(j\) of \(I_n\),
then \(E\) is a matrix where all entries are same as in \(I_n\), except an item on a position \((j, i)\), which is equal to \(a\).
If we multiply the \emph{column} \(j\) of \(I_n\) by the scalar \(a\) and add to the column \(i\) of \(I_n\) we can get the same matrix \(E\).

In all cases, we can find the second way to create matrix \(E\).
\end{proof}

\begin{exercise} \label{exercise 3.1.5}
Prove that \(E\) is an elementary matrix if and only if \(E^\top\) is.
\end{exercise}

\begin{proof}
We can easily check that type 1 and type 2 elementary matrices are symmetric, hence the transpose of them are also elementary matrices.

For type 3, suppose \(E\) is obtained from adding row \(i\) of \(I_n\) times scalar \(c\) to row \(j\) of \(I_n\).
Then \(E\) can be considered as \(I_n\) except that the position \((j, i)\) is equal to \(c\).
In particular, \(E^\top\) can be considered as \(I_n\) except that the position \((i, j)\) is equal to \(c\).
Then it's obvious that \(E^\top\) is obtained from adding row \(j\) of \(I_n\) times scalar \(c\) to row \(i\) of \(I_n\), so \(E^\top\) is also an elementary matrix.
\end{proof}

\begin{exercise} \label{exercise 3.1.6}
Let \(A\) be an \(m \X n\) matrix.
Prove that if \(B\) can be obtained from \(A\) by an elementary row [column] operation, then \(B^\top\) can be obtained from
\(A^\top\) by the corresponding elementary \emph{column} [row] operation.
\end{exercise}

\begin{proof}
Let
\[
    A = \begin{pmatrix} a_1 \\ a_2 \\ \vdots \\ a_m \end{pmatrix}
\]
where \(a_i\) are rows of \(A\).
Then in particular
\[
    A^\top = \begin{pmatrix} a_1^\top a_2^\top ... a_m^\top \end{pmatrix}
\]

Suppose \(B\) is obtain from \(A\) with a type 1 e.r.o., that is, changing row \(i\) and row \(j\) of \(A\) (and WLOG where \(i \le j\)).
Then \(B\) can be considered as
\[
    B = \begin{pmatrix} a_1 \\ a_2 \\ \vdots \\ \RED{a_j} \\ \vdots \\ \RED{a_i} \\ \vdots \\ a_m \end{pmatrix}
\]
Then in particular
\[
    B^\top = \begin{pmatrix} a_1^\top a_2^\top ... \RED{a_j^\top} ... \RED{a_i^\top} ... a_m^\top \end{pmatrix}
\]
which can be obtain from \(A^\top\) with a type 1 e.\RED{c}.o., that is, changing column \(i\) and column \(i\) of \(A^\top\).

Suppose \(B\) is obtain from \(A\) with a type 2 e.r.o., that is, multiplying row \(i\) of \(A\) by a nonzero scalar \(c\).
Then \(B\) can be considered as
\[
    B = \begin{pmatrix} a_1 \\ a_2 \\ \vdots \\ \RED{ca_i} \\ \vdots \\ a_m \end{pmatrix}
\]
Then in particular
\[
    B^\top = \begin{pmatrix} a_1^\top a_2^\top ... \RED{ca_i^\top} ... a_m^\top \end{pmatrix}
\]
which can be obtain from \(A^\top\) with a type 2 e.\RED{c}.o., that is, multiplying column \(i\) of \(A^\top\) by the nonzero scalar \(c\).

Finally, suppose \(B\) is obtain from \(A\) with a type 3 e.r.o., that is, adding row \(i\) of \(A\) times a scalar \(c\) to row \(j\) of \(A\).
Then \(B\) can be considered as
\[
    B = \begin{pmatrix} a_1 \\ a_2 \\ \vdots \\ \RED{ca_i + a_j} \\ \vdots \\ a_m \end{pmatrix}
\]
Then in particular
\[
    B^\top = \begin{pmatrix} a_1^\top a_2^\top ... \RED{(ca_i + a_j)^\top} ... a_m^\top \end{pmatrix} = \begin{pmatrix} a_1^\top a_2^\top ... \RED{(ca_i^\top + a_j^\top)} ... a_m^\top \end{pmatrix}
\]
(where the last equal sign is from \ATHM{1.2}(1)) which can be obtain from \(A^\top\) with a type 3 e.\RED{c}.o., that is, adding column \(i\) of \(A^\top\) times a scalar \(c\) to column \(j\) of \(A^\top\).

The case of e.c.o.s is similar to prove.
\end{proof}

\begin{exercise} \label{exercise 3.1.7}
Prove \THM{3.1}.
\end{exercise}

\begin{proof}
Let
\[
    A=\left[\begin{array}{cccc}
        a_{11} & a_{12} & \cdots & a_{1 n} \\
        a_{21} & a_{22} & \cdots & a_{2 n} \\
        \vdots & \vdots & \ddots & \vdots \\
        a_{m 1} & a_{m 2} & \cdots & a_{m n}
    \end{array}\right] \in M_{m \X n}(F)
\]

\begin{enumerate}
\item 
Suppose \(B\) is obtained from \(A\) with a type 1 e.r.o., that is, changing row \(i\) and row \(j\) of \(A\).
And if we let \(E\) be the same type of elementary matrix that changes row \(i\) and row \(j\) of \(I_n\);
that is,
\[
    E=\left[\begin{array}{cccccccc}
        1 & 0 & \cdots & 0 & \cdots & 0 & \cdots & 0 \\
        0 & 1 & \cdots & 0 & \cdots & 0 & \cdots & 0 \\
        \vdots & \vdots & \ddots & \vdots & & \vdots & & \vdots \\
        0 & 0 & \cdots & 0 & \cdots & \RED{1} & \cdots & 0 \\
        \vdots & \vdots & & \vdots & & \vdots & & \vdots \\
        0 & 0 & \cdots & \RED{1} & \cdots & 0 & \cdots & 0 \\
        \vdots & \vdots & & \vdots & & \vdots & & \vdots \\
        0 & 0 & \cdots & 0 & \cdots & 0 & \cdots & 1
    \end{array}\right] \in M_{m \X m}(F)
\]
(where the top-right \RED{\(1\)} appears in position \((i, j)\), the bottom-left \RED{\(1\)} appears in position \((j, i)\).)
Then
\[
    E A=\left[\begin{array}{cccccccc}
        a_{11} & a_{12} & \cdots & a_{1 i} & \cdots & a_{1 j} & \cdots & a_{1 n} \\
        a_{21} & a_{22} & \cdots & a_{2 i} & \cdots & a_{2 j} & \cdots & a_{2 n} \\
        \vdots & \vdots & & \vdots & & \vdots & & \vdots \\
        \RED{a_{j 1}} & \RED{a_{j 2}} & \RED{\cdots} & \RED{a_{j i}} & \RED{\cdots} & \RED{a_{j j}} & \RED{\cdots} & \RED{a_{j n}} \\
        \vdots & \vdots & & \vdots & & \vdots & & \vdots \\
        \RED{a_{i 1}} & \RED{a_{i 2}} & \RED{\cdots} & \RED{a_{i i}} & \RED{\cdots} & \RED{a_{i j}} & \RED{\cdots} & \RED{a_{i n}} \\
        \vdots & \vdots & & \vdots & & \vdots & & \vdots \\
        a_{m 1} & a_{m 2} & \cdots & a_{m i} & \cdots & a_{m j} & \cdots & a_{m n}
    \end{array}\right]
\]
is exactly the matrix obtained from changing row \(i\) and row \(j\) of \(A\); that is, \(B\).

\item
Suppose \(B\) is obtained from \(A\) with a type 2 e.r.o., that is, multiplying row \(i\) by scalar \(c\).
And if we let \(E\) be the same type of elementary matrix that multiplies row \(i\) by scalar \(c\),
that is,
\[
    E = \left[\begin{array}{ccccccc}
        1 & 0 & \cdots & 0 & 0 & \cdots & 0 \\
        0 & 1 & \cdots & 0 & 0 & \cdots & 0 \\
        \vdots & \vdots & \ddots & \vdots & \vdots & & \vdots \\
        0 & 0 & \cdots & \RED{c} & 0 & \cdots & 0 \\
        0 & 0 & \cdots & 0 & 1 & \cdots & 0 \\
        \vdots & \vdots & & \vdots & \vdots & \ddots & \vdots \\
        0 & 0 & \cdots & 0 & 0 & \cdots & 1
    \end{array}\right] \in M_{m \X m}(F)
\]
(where that red \(\RED{c}\) appears on position \((i, i)\).)
Then
\[
    E \cdot A=\left[\begin{array}{cccccc}
        a_{11} & a_{12} & \cdots & a_{1 i} & \cdots & a_{1 n} \\
        a_{21} & a_{22} & \cdots & a_{2 i} & \cdots & a_{2 n} \\
        \vdots & \vdots & & \vdots & & \vdots \\
        \RED{c a_{i 1}} &  \RED{c a_{i 2}} & \RED{\cdots} & \RED{c a_{i i}} & \RED{\cdots} & \RED{c a_{i n}} \\
        \vdots & \vdots & & \vdots & & \vdots \\
        a_{m 1} & a_{m 2} & \cdots & a_{m i} & \cdots & a_{m n}
    \end{array}\right]
\]
is exactly the matrix obtained from multiplying row \(i\) of \(A\) by scalar \(c\), that is, \(B\).

\item 
Suppose \(B\) is obtained from \(A\) with a type 3 e.r.o., that is, add row \(i\) of \(A\) times scalar \(c\) to row \(j\) of \(A\).
And if we let \(E\) be the same type of elementary matrix that adds row \(i\) of \(I_n\) times \(c\) to row \(j\) of \(I_n\);
that is,
\[
    E=\left[\begin{array}{cccccccc}
        1 & 0 & \cdots & 0 & \cdots & 0 & \cdots & 0 \\
        0 & 1 & \cdots & 0 & \cdots & 0 & \cdots & 0 \\
        \vdots & \vdots & \ddots & \vdots & & \vdots & & \vdots \\
        0 & 0 & \cdots & 1 & \cdots & 0 & \cdots & 0 \\
        \vdots & \vdots & & \vdots & & \vdots & & \vdots \\
        0 & 0 & \cdots & \RED{c} & \cdots & 1 & \cdots & 0 \\
        \vdots & \vdots & & \vdots & & \vdots & & \vdots \\
        0 & 0 & \cdots & 0 & \cdots & 0 & \cdots & 1
    \end{array}\right] \in M_{m \X m}(F)
\]
(where \RED{\(c\)} appears in position \((j, i)\).)
Then
\[
    E A= \left[\begin{array}{cccccccc}
        a_{11} & a_{12} & \cdots & a_{1 i} & \cdots & a_{1 j} & \cdots & a_{1 n} \\
        a_{21} & a_{22} & \cdots & a_{2 i} & \cdots & a_{2 j} & \cdots & a_{2 n} \\
        \vdots & \vdots & & \vdots & & \vdots & & \vdots \\
        a_{i 1} & a_{i 2} & \cdots & a_{i i} & \cdots & a_{i j} & \cdots & a_{i n} \\
        \vdots & \vdots & & \vdots & & \vdots & & \vdots \\
        \RED{a_{j 1} + c a_{i 1}} & \RED{a_{j 2} + c a_{i 2}} & \RED{\cdots} & \RED{a_{j i}+ c a_{i i}} & \RED{\cdots} & \RED{a_{j j} + c a_{i j}} & \RED{\cdots} & \RED{a_{j n}+ c a_{i n}} \\
        \vdots & \vdots & & \vdots & & \vdots & & \vdots \\
        a_{m 1} & a_{m 2} & \cdots & a_{m i} & \cdots & a_{m j} & \cdots & a_{m n}
\end{array}\right]
\]
is exactly the matrix obtained from add row \(i\) of \(A\) times scalar \(c\) to row \(j\) of \(A\), that is, \(B\).
\end{enumerate}

For elementary \emph{column} operations, if \(B \in M_{m \X n}\) is obtained from \(A \in M_{m \X n}\) by performing any type of e.c.o.,
then from \EXEC{3.1.6}, \(B^\top \in M_{n \X m}\) is obtained from \(A^\top \in M_{n \X m}\) by performing the corresponding type of e.\RED{r}.o..
Then by the previous case, we have \(B^\top = EA^\top\) \MAROON{(1)} for some elementary matrix \(E\), and
\begin{align*}
    B & = (B^\top)^\top & \text{by \ATHM{1.2}(2)} \\
      & = (EA^\top)^\top & \text{by \MAROON{(1)}} \\
      & = (A^\top)^\top E^\top & \text{by \ATHM{2.24}} \\
      & = A E^\top & \text{by \ATHM{1.2}(2)}
\end{align*}
And by \EXEC{3.1.5}, \(E^\top\) is also elementary, so we have found a elementary matrix \(E^\top\) s.t. \(B = A E^\top\), as desired.
\end{proof}

\begin{exercise} \label{exercise 3.1.8}
Prove that if a matrix \(Q\) can be obtained from a matrix \(P\) by an e.r.o., then \(P\) can be obtained from \(Q\) by an e.r.o. of the same type.
Hint: Treat each type of e.r.o. separately.
\end{exercise}

\begin{proof}
If we interchange the \(i\)th and \(j\)th rows of \(P\) to get \(Q\), we can change the \(i\)th and \(j\)th rows \(Q\) again to get \(P\), and the operation is type 1 e.r.o..

If we multiply the \(i\)th row of \(P\) by nonzero scalar \(c\) to get \(Q\), we can multiply the \(i\)th row of \(Q\) by nonzero scalar \(\frac1{c}\) to get \(P\), and the operation is type 2 e.r.o..

If we add the \(i\)th row of \(P\) times scalar \(c\), to the \(j\)th row of \(P\), to get \(Q\), then we can add the \(i\)th row of \(Q\) times scalar \(-c\), to the \(j\)th row of \(Q\), to get \(P\), and the operation is type 3 e.r.o..
\end{proof}

\begin{proof} [Another proof for \EXEC{3.1.8}]
By \THM{3.1}, we can write \(EP = Q\) where \(E\) corresponds to an e.r.o..
And by \THM{3.2}, \(E^{-1}\) is also elementary of the same type.
And
\[
    E^{-1} Q = E^{-1} E P = P.
\]
So by \THM{3.1}, \(P\) can be obtained from \(Q\) by an e.r.o. of the same type as \(E^{-1}\), which is the same type as \(E\).
\end{proof}

\begin{exercise} \label{exercise 3.1.9}
Prove that any elementary row [column] operation of type 1 can be obtained by a succession of three elementary row [column] operations of type 3 followed by one elementary row [column] operation of type 2.
\end{exercise}

\begin{proof}
Let \[A = \begin{pmatrix} a_1 \\ a_2 \\ \vdots \\ a_m \end{pmatrix}\].
Then if \(B\) can be obtain from \(A\) by performing a type 1 e.r.o., that is,
\[
    B = \begin{pmatrix} a_1 \\ a_2 \\ \vdots \\ \RED{a_j} \\ \vdots \\ \RED{a_i} \\ \vdots \\ a_m \end{pmatrix}
\]
Then
\[
    A = \begin{pmatrix} a_1 \\ a_2 \\ \vdots \\ \RED{a_i} \\ \vdots \\ \RED{a_j} \\ \vdots \\ a_m \end{pmatrix}
    \text{\RED{(*1)}} \to \begin{pmatrix} a_1 \\ a_2 \\ \vdots \\ \RED{a_i + a_j} \\ \vdots \\ \RED{a_j} \\ \vdots \\ a_m \end{pmatrix}
    \text{\RED{(*2)}} \to \begin{pmatrix} a_1 \\ a_2 \\ \vdots \\ \RED{a_i + a_j} \\ \vdots \\ \RED{-a_i} \\ \vdots \\ a_m \end{pmatrix}
    \text{\RED{(*3)}} \to \begin{pmatrix} a_1 \\ a_2 \\ \vdots \\ \RED{a_j} \\ \vdots \\ \RED{-a_i} \\ \vdots \\ a_m \end{pmatrix}
    \text{\RED{(*4)}} \to \begin{pmatrix} a_1 \\ a_2 \\ \vdots \\ \RED{a_j} \\ \vdots \\ \RED{a_i} \\ \vdots \\ a_m \end{pmatrix}
    = B
\]
where \RED{(*1)} is e.r.o. type 3 by adding row \(j\) to row \(i\);
\RED{(*2)} is e.r.o. type 3 by adding \(-1 \X\) row \(i\) to row \(j\);
\RED{(*3)} is e.r.o. type 3 by adding row \(j\) to row \(i\);
\RED{(*4)} is e.r.o. type 2 by multiplying row \(j\) by \(-1\).

The case of e.r.c. is very similar to prove.
\end{proof}

\begin{exercise} \label{exercise 3.1.10}
Prove that any elementary row [column] operation of type 2 can be obtained by dividing some row [column] by a nonzero scalar.
\end{exercise}

\begin{proof}
Well, multiplying nonzero scalar \(c\) is equal to dividing scalar \(\frac1{c}\).
\end{proof}

\begin{exercise} \label{exercise 3.1.11}
Prove that any elementary row [column] operation of type 3 can be obtained by subtracting a multiple of some row [column] from another row [column].
\end{exercise}

\begin{proof}
Well, adding a row times scalar \(c\) is equal to subtracting a row times \(c\).
\end{proof}

\begin{exercise} \label{exercise 3.1.12}
Let \(A\) be an \(m \X n\) matrix.
Prove that there exists a sequence of e.r.o.s of types 1 and 3 that transforms \(A\) into an upper triangular matrix.
\end{exercise}

\begin{proof}
The \emph{forward pass} of Gaussian elimination(see \THM{3.14}) give a sequence of e.r.o.s of type 1, 2, and 3 that transform any matrix into upper triangular.
In particular, the forward pass only uses type 2 e.r.o.s to transforms the leading term of each row into \(1\).
So if we remove type 2 e.r.o.s in the sequence, then we have found a sequence of e.r.o.s of type 1 and 3 that transforms \(A\) into an upper triangular matrix.
\end{proof}

\begin{additional theorem} \label{athm 3.1}
This is the placeholder theorem for \EXEC{3.1.5}: \(E\) is an elementary matrix if and only if \(E^\top\) is.
\end{additional theorem}

\begin{additional theorem} \label{athm 3.2}
This is the placeholder theorem for \EXEC{3.1.6}: Let \(A\) be an \(m \X n\) matrix.
If \(B\) can be obtained from \(A\) by an e.r.o. [e.c.o.], then \(B^\top\) can be obtained from \(A^\top\) by the corresponding e.c.o. [e.r.o.].
\end{additional theorem}

\begin{additional theorem} \label{athm 3.3}
This is the placeholder theorem for \EXEC{3.1.8}: If a matrix \(Q\) can be obtained from a matrix \(P\) by an e.r.o., then \(P\) can be obtained from \(Q\) by an e.r.o. of the same type.
\end{additional theorem}
