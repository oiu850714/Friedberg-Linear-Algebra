\exercisesection

\begin{exercise} \label{exercise 1.6.1}\ 
Label the following statements as true or false.
\begin{enumerate}
\item The zero vector space has no basis.
\item Every vector space that is generated by a finite set has a basis.
\item Every vector space has a finite basis.
\item A vector space cannot have more than one basis.
\item If a vector space has a finite basis, then the number of vectors in every basis is the same.
\item The dimension of \(\mathcal{P}_n(F)\) is \(n\).
\item The dimension of \(M_{m \X n}(F)\) is \(m + n\).
\item Suppose that \(V\) is a finite-dimensional vector space, that \(S_1\) is a \LID{} subset of \(V\), and that \(S_2\) is a subset of \(v\) that
generates \(V\).
    Then \(S_1\) cannot contain more vectors than \(S_2\).
\item If \(S\) generates the vector space \(V\), then every vector in \(V\) can be written as a linear combination of vectors in \(S\) in only one way. 
\item Every subpace of a finite-dimensional space is finite-dimensional.
\item If \(V\) is a vector space having dimension \(n\), then \(V\) bas exactly one subspace with dimension \(0\) and exactly one subspace with dimension \(n\).
\item If \(V\) is a vector space having dimension \(n\), and if \(S\) is a subset of \(V\) with \(n\) vectors, then \(S\) is \LID{} if and only if \(S\)
spans \(V\).
\end{enumerate}
\end{exercise}

\begin{proof}\ 
\begin{enumerate}
\item False, the zero vector has \(\emptyset\) as basis.
\item True, by \THM{1.9}.
\item False, \(\mathcal{P}(F)\) has a infinite basis \(\{ 1, x, x^2, ... \}\).
\item False, counterexample: \(\mathcal{P}_n(F)\) have \(\{ 1, x, x^2, ..., x^n \}\) and Lagrange polynomial as bases.
\item True, by \CORO{1.10.1}.
\item False, the dimension of \(\mathcal{P}_n(F)\) is \(n + 1\).
\item The dimension of \(M_{m \X n}(F)\) is \(mn\).
\item True by \THM{1.10}(1).
\item False; \(S\) must also be \LID{}.
\item True by \THM{1.11}(1).
\item True. \(V\) has exactly one subspace \(\{ \OV \}\) with dimension \(0\) and by \THM{1.11}(2) exactly one subspace \(V\) with dimension \(n\).
\item True.
    If \(S\) is \LID{} and has \(n\) vectors, then by \CORO{1.10.3}(b) \(S\) is a basis so \(S\) spans \(V\).
    If \(S\) spans \(V\) and has \(n\) vectors, then by \CORO{1.10.3}(a) \(S\) is a basis so \(S\) is \LID{}. 
\end{enumerate}
\end{proof}

\begin{exercise} \label{exercise 1.6.2}
Determine which of the following sets are bases for \(\SET{R}^3\).
\begin{enumerate}
\item \(\{ (1 , 0, -1), (2, 5, 1), (0, -4, 3) \}\)
\item \(\{ (2, -4, 1), (0, 3, -1), (6, 0, -1) \}\)
\item \(\{ (1, 2, -1), (1, 0, 2), (2, 1, 1) \}\)
\item \(\{ (-1, 3, 1),(2, -4, -3),(-3, 8 ,2) \}\)
\item \(\{ (1, -3, -2), (-3, 1, 3), (-2, -10, -2) \}\)
\end{enumerate}
\end{exercise}

\begin{proof}
Calculation problem, skip.
But since they all have \(3\) vectors which is equal to the dimension of \(\SET{R}^3\), it's enough to check whether each is \LID{} (by \EXEC{1.6.1}(l)).
\end{proof}

\begin{exercise} \label{exercise 1.6.3}
Determine which of the following sets are bases for \(\mathcal{P}_2(\SET{R})\).
\begin{enumerate}
\item \(\{ -1 - x + 2x^2, 2 + x - 2x^2, 1 - 2x + 4x^2 \}\)
\item \(\{ 1 + 2x + x^2, 3 + x^2, x + x^2 \}\)
\item \(\{ 1 - 2x -2x^2, -2 + 3x - x^2, 1 - x + 6x^2 \}\)
\item \(\{ -1 + 2x + 4x^2, 3 - 4x - 10x^2, -2 - 5x - 6x^2 \}\)
\item \(\{ 1 + 2x - x^2, 4 - 2x + x^2, -1 + 18x - 9x^2 \}\)
\end{enumerate}
\end{exercise}

\begin{proof}
Calculation problem, skip.
But since they all have \(3\) vectors which is equal to the dimension of \(\mathcal{P}_n(\SET{R})\), it's enough to check whether each is \LID{} (by \EXEC{1.6.1}(l)).
\end{proof}

\begin{exercise} \label{exercise 1.6.4}
Do the polynomials \(x^3 - 2x^2 + 1, 4x^2 - x + 3\), and \(3x - 2\) generate \(\mathcal{P}_3(\SET{R})\)?
Justify your answer.
\end{exercise}

\begin{proof}
No, \(\mathcal{P}_3(\SET{R})\) has dimension \(4\), a subset with \(3\) vectors cannot generate it, by \CORO{1.10.3}(a).
\end{proof}

\begin{exercise} \label{exercise 1.6.5}
Is \(\{ (1, 4, -6), (1, 5, 8), (2, 1, 1), (0, 1,0) \}\) a \LID{} subset of \(\SET{R}^3\)?
Justify your answer.
\end{exercise}

\begin{proof}
No, since the standard basis, which has \(3\) vectors, spans \(\SET{R}^3\), by \THM{1.10}(1), any \LID{} set can only have at most \(3\) vectors, so \(\{ (1, 4, -6), (1, 5, 8), (2, 1, 1), (0, 1,0) \}\), which has \(4\) elements, is not \LID{}.
\end{proof}

\begin{exercise} \label{exercise 1.6.6}
Give three different bases for \(F^2\) and for \(M_{2 \X 2}(F)\).
\end{exercise}

\begin{proof}\ 

\(F^2\): \(\{ e_1, e_2 \}\), \(\{ 2 e_1, 2 e_2 \}\) and \(\{ 3 e_1, 3 e_2 \}\).

\(M_{2 \X 2}(F)\): \(\{ E_{11}, E_{12}, E_{21}, E_{22} \}\), \(\{ 2 E_{11}, 2 E_{12}, 2 E_{21}, 2 E_{22} \}\) and \(\{ 3 E_{11}, 3 E_{12}, 3 E_{21}, 3 E_{22} \}\).
\end{proof}

\begin{exercise} \label{exercise 1.6.7}
The vectors \(u_1 = (2, -3, 1), u_2 = (1, 4, -2), u_3 = (-8, 12, -4), u_4 = (1, 37, -17)\), and \(u_5 = (-3, -5, 8)\) generate \(\SET{R}^3\).
Find a subset of the set \(\{ u_1, u_2, u_3, u_4, u_5 \}\) that is a basis for \(\SET{R}^3\).
\end{exercise}

\begin{proof}
Calculation problem, skip.
Note that we can use the same choosing process in the proof of \THM{1.9}.
\end{proof}

\begin{exercise} \label{exercise 1.6.8}
Let \(W\) denote the subspace of \(\SET{R}^5\) consisting of all the vectors \emph{having coordinates that sum to zero}.
The vectors
\begin{align*}
    u_1 & = (2, -3, 4, -5, 2), & u_2 & = (-6, 9, - 12, 15, -6), \\
    u_3 & = (3, -2, 7, -9, 1), & u_4 & = (2, -8, 2, -2, 6), \\
    u_5 & = (-1, 1, 2, 1, -3), & u_6 & = (0, -3, -18, 9, 12) , \\
    u_7 & = (1, 0, -2, 3, -2), & u_8 & = (2, -1, 1, -9, 7)
\end{align*}
\emph{generate} \(W\).
Find a subset of the set \(\{ u_1, u_2, ..., u_8 \}\) that is a basis for \(W\).
\end{exercise}

\begin{proof}
Since we have known(by the statement in exercise) \(\{ u_1, u_2, ..., u_8 \}\) generates \(W\), we can use the technique in the proof of \THM{1.9} to find a subset of \(\{ u_1, ..., u_8 \}\) that is a basis of \(W\).

So we first pick (nonzero vector) \(u_1\) into the resulting set \(\beta\).
So currently \(\beta = \{ u_1 \}\).
Now we check whether \(\beta \cup \{u_2\}\) is \LID{}, but of course it is not, since \(-3u_1 = u_2\), so we skip \(u_2\).
Now we check whether \(\beta \cup \{u_3\}\) is \LID{}, and in fact it is (skipping calculation), so we add \(u_3\) into \(\beta\).
So currently \(\beta = \{ u_1, u_3 \}\).
Now we check whether \(\beta \cup \{u_4\}\) is \LID{}, but it is not(skipping calculation), so we skip \(u_4\).
Similarly, \(\beta \cup u_5\) is \LID{}, so we add \(u_5\) into \(\beta\).
So currently \(\beta = \{ u_1, u_3, u_5 \}\);
and skip \(u_6\);
and pick \(u_7\) so \(\beta = \{ u_1, u_3, u_5, u_7 \}\);
and skip \(u_8\).

Now by \THM{1.9}, the resulting set \(\beta = \{ u_1, u_3, u_5, u_7 \}\) by construction is a basis of \(W\).
(And that also implies \(W\) has dimension \(4\)).
\end{proof}

\begin{exercise} \label{exercise 1.6.9}
The vectors \(u_1 = (1, 1, 1, 1), u_2 = (0, 1, 1, 1), u_3 = (0, 0, 1, 1)\), and \(u_4 = (0, 0, 0, 1)\) form a basis for \(F^4\).
Find the unique representation of an arbitrary vector \((a_1, a_2, a_3, a_4) \in F^4\) as a linear combination of \(u_1, u_2, u_3\), and \(u_4\).
\end{exercise}

\begin{proof}
It is in fact a calculation problem, we can solve the corresponding system of linear equations.
But anyway,
\[
    (a_1, a_2, a_3, a_4) = a_1 u_1 + (a_2 - a_1) u_2 + (a_3 - a_2) u_3 + (a_4 - a_3) u_4.
\]
\end{proof}

\begin{exercise} \label{exercise 1.6.10}
In each part, use the Lagrange interpolation formula to construct the polynomial of smallest degree whose graph contains the following points.
\begin{enumerate}
\item \((-2, -6), (-1 , 5), (1, 3)\)
\item \((-4, 24), (1, 9), (3, 3)\)
\item \((-2, 3), (-1, -6), (1, 0), (3, -2)\)
\item \((-3,-30), (-2, 7), (0, 15), (1, 10)\)
\end{enumerate}
\end{exercise}

\begin{proof}
\begin{enumerate}
\item We have \(c_0 = -2, c_1 = -1, c_2 = 1, b_0 = -6, b_1 = 5, b_2 = 3\).
    So \begin{align*}
        h(x) & = \sum_{i = 0}^n b_i f_i(x) = b_0 f_0(x) + b_1 f_1(x) + b_2 f_2(x) \\
             & = (-6) \frac{(x - (-1))(x - 1)}{(-2 - (-1))(-2 - 1)} \\
             & + (5) \frac{(x - (-2))(x - 1)}{(-1 - (-2))(-1 - 1)} \\
             & + (3) \frac{(x - (-2))(x - (-1))}{(1 - (-2))(1 - (-1))} \\
             & = ... (skip) \\
             & = -4x^2 - x + 8.
    \end{align*}
\end{enumerate}
Skip other items.
\end{proof}

\begin{exercise} \label{exercise 1.6.11}
Let \(u\) and \(v\) be distinct vectors of a vector space \(V\).
Show that if \(\{ u, v \}\) is a basis for \(V\) and \(a\) and \(b\) are \emph{nonzero} scalars,
then both \(\{ u + v, au \}\) and \(\{ au, bv \}\) are also bases for \(V\).
\end{exercise}

\begin{proof}
Since \{ u, v \} has \(2\) elements and is a basis for \(V\), \(V\) has dimension \(2\).
And both \(\{ u + v, au \}\) and \(\{ au, bv \}\) have \(2\) elements, by \CORO{1.10.3}(b) it suffices to show these sets are \LID{}.

So suppose \(c_1 (u + v) + c_2 (au) = \OV\), we have to show \(c_1 = c_2 = 0\).
But
\begin{align*}
             & c_1 (u + v) + c_2 (au) = \OV \\
    \implies & c_1 u + c_1 v + (c_2 a) u = \OV \\
    \implies & (c_1 + c_2 a) u + c_1 v = \OV.
\end{align*}
Since \(\{ u, v \}\) is \LID{}, we have \(c_1 + c_2 a = 0\) and \(c_1 = 0\), which trivially implies \(c_1 = c_2 = 0\), as desired. (Note that \(a \ne 0\).)

Now suppose \(c_1 (au) + c_2 (bv) = \OV\), we have to show \(c_1 = c_2 = 0\).
But
\begin{align*}
             & c_1 (au) + c_2 (bv) = \OV \\
    \implies & (c_1 a) u + (c_2 b) v = \OV.
\end{align*}
Since \(\{ u, v \}\) is \LID{}, we have \(c_1 a = 0\) and \(c_2 b = 0\), which trivially implies \(c_1 = c_2 = 0\), as desired. (Note that \(a \ne 0\) and \(b \ne 0\).)
\end{proof}

\begin{exercise} \label{exercise 1.6.12}
Let \(u, v\), and \(w\) be distinct vectors of a vector space \(V\).
Show that if \(\{ u, v, w \}\) is a basis for \(V\), then \(\{ u + v + w, v + w, w \}\) is also a basis for \(V\).
\end{exercise}

\begin{proof}
With similar argument as \EXEC{1.6.11}, it suffices to show \(\{ u + v + w, v + w, w \}\) is \LID{}.

So suppose \(c_1(u + v + w) + c_2(v + w) + c_3w = \OV\), we have to show \(c_1 = c_2 = c_3 = 0\).
Then
\begin{align*}
             & c_1(u + v + w) + c_2(v + w) + c_3w = \OV \\
    \implies & c_1 u + (c_1 + c_2) v + (c_1 + c_2 + c_3) w = \OV.
\end{align*}
Since \(\{ u, v, w \}\) is \LID{}, we have \(c_1 = 0\) and \(c_1 + c_2 = 0\) and \(c_1 + c_2 + c_3 = 0\), which trivially implies \(c_1 = c_2 = c_3 = 0\), as desired
\end{proof}

\begin{note}
\TODOREF{} Note that when we learn the \emph{elementary row operations} in \CH{3}, \EXEC{1.6.11}, \EXEC{1.6.12} are immediately true.
Similarly to \EXEC{1.5.13}.
\end{note}

\begin{exercise} \label{exercise 1.6.13}
The \emph{set of solutions} to the system of linear equations
\begin{align*}
       x_1 - 2x_2 + x_3 & = 0 \\
      2x_l - 3x_2 + x_3 & = 0
\end{align*}
is a subspace of \(\SET{R}^3\).
Find a basis for this subspace.
\end{exercise}

\begin{proof}
Solving the equation, if we let \(x_3\) as free variable \(t\), then all \((t, t, t)\) is the solution of the system.
So the corresponding subspace is \(W = \{ (t, t, t) : t \in \SET{R} \}\).
And it's trivial that \(\{ (1, 1, 1) \}\) is \LID{} and spans \(W\), so is a basis for \(W\).
And that also implies \(W\) has dimension \(1\).
\end{proof}

\begin{exercise} \label{exercise 1.6.14}
Similar to the previous exercise, skip.
\end{exercise}

\begin{exercise} \label{exercise 1.6.15}
The set of all \(n \X n\) matrices having \textbf{trace} equal to zero is a subspace \(W\) of \(M_{n \X n}(F)\) (see \EXAMPLE{1.3.4}).
Find a basis for \(W\).
What is the dimension of \(W\)?
\end{exercise}

\begin{proof}
We claim that the set \(S = \{ E_{ij} : 1 \le i \le n \land 1 \le j \le n \land \RED{i \ne j} \} \cup \{ E_{ii} - E_{11} : 2 \le i \le n \}\) is a basis for \(W\).
It's trivial that \(S\) is \LID{};
so we have to show \(S\) generates \(W\).

Now suppose \(A\) is an arbitrary matrix in \(W\), so we have \(\TRACE(A) = 0\): \(A_{11} + A_{22} + ... + A_{nn} = 0\);
that is, \(A_{11} = -(A_{22} + A_{33} + ... + A_{nn}\) \MAROON{(1)}.
\begin{align*}
    A & = \sum_{i = 1}^n \sum_{j = 1}^n A_{ij} E_{ij} & \text{of course} \\
      & = \sum_{i = 1}^n \sum_{\substack{j = 1 \\
                       \RED{j \ne i}}}^n A_{ij} E_{ij} + \sum_{i = 1}^n A_{ii} E_{ii} & \text{by splitting diagonal entries} \\
      & = \sum_{i = 1}^n \sum_{\substack{j = 1 \\
                       j \ne i}}^n A_{ij} E_{ij} + (\sum_{i = 2}^{n - 1} A_{ii} E_{ii}) + A_{11}E_{11} & \text{of course} \\
      & = \sum_{i = 1}^n \sum_{\substack{j = 1 \\
                       j \ne i}}^n A_{ij} E_{ij} + (\sum_{i = 2}^{n - 1} A_{ii} E_{ii}) + (-(A_{22} + A_{33} + ... + A_{nn}))E_{11} & \text{by \MAROON{(1)}} \\
      & = \sum_{i = 1}^n \sum_{\substack{j = 1 \\
                       j \ne i}}^n A_{ij} E_{ij} + \sum_{i = 2}^{n - 1} A_{ii} (E_{ii} - E_{11}),
\end{align*}
which has the first component as a linear combination of the first union set of \(S\), and the second component as a linear combination of the second union set of \(S\).
Hence \(A\) belongs to \(\spann(S)\).
Since \(A\) is arbitrary, we have \(W \subseteq \spann(S)\).

Also, observing the second union set of \(S\), it's trivial that any matrix in \(\spann(S)\) has zero trace, so by definition is in \(W\), so \(\spann(S) \subseteq W\).

So we have \(\spann(S) = W\).
So \(S\) is a \LID{} set that spans \(W\), hence is a basis for \(W\).
Since the first union set of \(S\) has \(n \X (n - 1)\) elements, the second union set of \(S\) has \(n - 1\) elements, \(S\) has \(n \X (n - 1) + (n - 1) = n^2 - 1\) elements.
So \(W\) has dimension \(n^2 - 1\).
\end{proof}

\begin{exercise} \label{exercise 1.6.16}
The set of all upper triangular \(n \X n\) matrices is a subspace \(W\) of \(M_{n \X n}(F)\) (see \EXEC{1.3.12}).
Find a basis for \(W\).
What is the dimension of \(W\)?
\end{exercise}

\begin{proof}
It's trivial that \(\{ E_{ij} : 1 \le i \le j \le n \}\) is a basis for \(W\), and has \(n + (n - 1) + ... + 1 = \frac{n^2 + n}{2}\) elements, so \(W\) has dimension \(\frac{n^2 + n}{2}\).
\end{proof}

\begin{exercise} \label{exercise 1.6.17}
The set of all skew-symmetric \(n \X n\) matrices is a subspace \(W\) of \(M_{n \X n}(F)\) (see \EXEC{1.3.28}).
Find a basis for \(W\).
What is the dimension of \(W\)?
\end{exercise}

\begin{proof}
First, WLOG, suppose \(n \ge 2\).
(If \(n = 1\), then it's trivial that \(W\) is in fact has only zero vector, a ``\(1 \X 1\)'' zero matrix).
Note that any skew-symmetric matrix \(A\) has the following properties:
\begin{enumerate}
    \item \(A_{ii} = 0\) for \(1 \le i \le n\), since \(A^\top = -A\) implies \((A^\top)_{ii} = -A_{ii}\), which implies \(A_{ii} = -A_{ii}\), which implies \(A_{ii} = 0\).
    \item Similarly, for \(i \ne j\), \(A_{ij} = -A_{ji}\), since \(A^\top = -A\) implies \((A^\top)_{ij} = -A_{ij}\), which implies \(A_{ji} = -A_{ij}\), that is, \(A_{ij} = -A_{ji}\).
\end{enumerate}
With these properties we claim that the set \(S = \{ E_{ij} - E_{ji} : 1 \le i\ \RED{<}\ j \le n \}\) is a basis for \(W\).

First (informally) it's trivial that \(S\) is \LID{} since each vector in \(S\) is a combination of \emph{distinct} standard vectors,, and standard vectors are \LID{}.
(\TODOREF{} This argument is immediately true when we learn the concept of elementary row operations in \CH{3}; or refer to \EXEC{1.6.11}, \EXEC{1.6.12}, which have the same spirit.)

Now we show \(\spann(S) = W\).
First, it's trivial that any matrix in \(\spann(S)\) is skew-symmetric since it satisfies the properties above, so \(\spann(S) \subseteq W\).
Also, given any skew-symmetric matrix \(A\),
\begin{align*}
    A & = \sum_{i = 1}^{n} \sum_{j = 1}^n A_{ij} E_{ij} & \text{of course} \\
      & = \sum_{i = 1}^{n - 1} \sum_{j = i + 1}^n A_{ij} E_{ij} + \sum_{i = 1}^n A_{ii} E_{ii} + \sum_{i = 2}^n \sum_{j = 1}^{i - 1} A_{ij} E_{ij} & \text{by splitting into three parts,} \\
      & & \text{upper-right, diagonal, lower-left} \\
      & = \sum_{i = 1}^{n - 1} \sum_{j = i + 1}^n A_{ij} E_{ij} + 0 + \sum_{i = 2}^n \sum_{j = 1}^{i - 1} A_{ij} E_{ij} & \text{by property(a)} \\
      & = \sum_{i = 1}^{n - 1} \sum_{j = i + 1}^n A_{ij} E_{ij} + \sum_{i = 2}^n \sum_{j = 1}^{i - 1} \MAROON{-A_{ji}} E_{ij} & \text{by property(b)} \\
      & = \sum_{i = 1}^{n - 1} \sum_{j = i + 1}^n A_{ij} E_{ij} + \RED{\sum_{j = 1}^{n - 1} \sum_{i = j + 1}^n} -A_{ji} E_{ij} & \text{nasty but true,} \\
      & & \text{just count column first instead of row first} \\
      & = \sum_{i = 1}^{n - 1} \sum_{j = i + 1}^n A_{ij} E_{ij} + \sum_{i = 1}^{n - 1} \sum_{j = i + 1}^n -A_{ij} E_{ji} & \text{by renaming index variable} \\
      & & \text{\(i\) to \(j\) and \(j\) to \(i\)} \\
      & = \sum_{i = 1}^{n - 1} \sum_{j = i + 1}^n A_{ij} (E_{ij} - E_{ji}), & \text{combining summation}
\end{align*}
which is a linear combination of the vectors in \(S\), so \(A \in \spann(S)\).
Since \(A\) is arbitrary, we have \(W \subseteq \spann(S)\).

So we have \(W = \spann(S)\) and \(S\) is \LID{}, hence \(S\) is a basis for \(W\).
And \(S\) has \((n - 1) + (n - 2) + ... + 1 = n(n - 1)/2\) elements, hence \(W\) has dimension \(n(n - 1)/2\).
\end{proof}

\begin{exercise} \label{exercise 1.6.18}
Let \(V\) denote the vector space of all sequences in \(F\), as defined in \EXAMPLE{1.2.5}.
Find a basis for the subspace \(W\) of \(V\) consisting of the sequences \((a_n)\) that have only a \emph{finite number of nonzero terms} \(a_n\)
Justify your answer.
\end{exercise}

\begin{proof}
\TODOREF \RED{SKIP}.
I feel the problem requires good understanding of sequence(or Analysis).
\end{proof}

\begin{exercise} \label{exercise 1.6.19}
Complete the proof of \THM{1.8}.
\end{exercise}

\begin{proof}
See \THM{1.8}.
\end{proof}

\begin{exercise} \label{exercise 1.6.20}
Let \(V\) be a vector space having (finite) dimension \(n\), and let \(S\) be a subset of \(V\) that generates \(V\).
Note that we do \emph{not} say \(S\) is a finite set.
\begin{enumerate}
\item Prove that there is a subset of \(S\) that is a basis for \(V\). (Be careful not to assume that \(S\) is finite.)
\item Prove that \(S\) contains at least \(n\) vectors.
\end{enumerate}
\end{exercise}

\begin{proof}
We first prove part(b).
If \(S\) is infinite set that generates \(V\), then in particular \(\#(S) > n\).
If \(S\) is finite, then by \CORO{1.10.3}(a), it has at least \(n\) elements.
So in all cases, \(S\) contains at least \(n\) vectors.

Now we prove part(a).
In fact, since at present we have known that the finite-dimensional vector space \(V\) has finite bases, we can use the similar process in \THM{1.9} s.t. we choose a subset of \(S\), \(\beta = \{ u_1, u_2, ..., u_k \}\) s.t. \(\beta\) is \LID{} and \textbf{\(k\) must not exceed \(n\)}, hence \(\beta\) is finite.
(otherwise we get a \LID{} subset with elements more than the dimension of the (finite) vector space, which is impossible.)
And using the remaining argument after the choosing process of \THM{1.9}, we conclude that the (finite) \LID{} subset \(\beta\) of \(S\) is really a basis for \(V\).
\end{proof}

\begin{exercise} \label{exercise 1.6.21}
Prove that a vector space \(V\) is infinite-dimensional if and only if it contains an infinite \LID{} subset.
\end{exercise}

\begin{proof} \ 

\(\Longleftarrow\): For the sake of contradiction, suppose \(V\) contains an infinite \LID{} subset \(S\) but \(V\) is finite-dimensional, say \(\dim(V) = n\).
In particular we can find a basis \(\beta\) of \(V\) with \(n\) elements.
And by \CORO{1.6.1}, given any finite subset \(S'\) of \(S\) s.t. \(S\) has \(n + 1\) elements, \(S'\) is also \LID{}.
By replacement theorem \THM{1.10}(1), since \(\beta\) generates \(V\), and \(S'\) is \LID{}, we have \(\#(S') \le \#(\beta)\);
that is, \(n + 1 \le n\), which is impossible.
So \(V\) must be infinite dimensional.

\(\Longrightarrow\):
Now suppose \(V\) is infinite-dimensional.
Then (the \emph{subset}) \(\BLUE{V}\) (of \(\GREEN{V}\)) of course generates \(\GREEN{V}\).
So we use the same choosing process in \EXEC{1.6.20}(a) to choose elements of the \emph{infinite} (subset) \(\BLUE{V}\) of \(\GREEN{V}\) to construct a \LID{} subset of \(\GREEN{V}\).
Then the resulting set \emph{must be infinite}, otherwise if the resulting set is finite, by \EXEC{1.6.20}(a) it is a \textbf{finite} basis for \(V\), so now by \DEF{1.9}, \(V\) is finite-dimensional, which contradicts \(V\) is infinite-dimensional.
\end{proof}

\begin{exercise} \label{exercise 1.6.22}
Let \(W_1\) and \(W_2\) be \emph{subspaces} of a finite-dimensional vector space \(V\).
Determine necessary and sufficient conditions on \(W_1\) and \(W_2\) so that \(\dim(W_1 \cap W_2) = \dim(W_1)\).
\end{exercise}

\begin{proof}
We claim \(W_1 \subseteq W_2\) if and only if \(\dim(W_1 \cap W_2) = \dim(W_1)\).

\(\Longrightarrow\): Suppose \(W_1 \subseteq W_2\).
Then of course \(\dim(W_1 \cap W_2) = \dim(W_1)\) since \(W_1 \cap W_2 = W_1\).

\(\Longleftarrow\):
Suppose \(\dim(W_1 \cap W_2) = \dim(W_1) = n\). \MAROON{(1)}.
Now let \(\beta = \{ v_1, v_2, ..., v_n \}\) be a bases for \(W_1 \cap W_2\); in particular \(\beta\) is of course a subset of \(W_1 \cap W_2\) \MAROON{(2)}, hence \(\beta \subseteq W_1\).
So \(\beta\) is a \LID{} subset of \(W_1\) and has \(n\) vectors, which equals to the dimension of \(W_1\), so by \CORO{1.10.3}(b) \(\beta\) is also a basis of \(W_1\).

Now finally, given arbitrary vector \(v \in W_1\), we have to show \(v \in W_2\) so that \(W_1 \subseteq W_2\).
Since \(v \in W_1\), we can represent \(v\) as linear combination of the basis \(\beta\) of \(W_1\); in particular, \(v \in \spann(\beta)\) \MAROON{(3)}.
In particular, from \MAROON{(2)}, we have \(\beta \subseteq W_2\), so by \THM{1.5}(2), \(\spann(\beta) \subseteq W_2\) \MAROON{(4)}.
So by \MAROON{(3)(4)} we have \(v \in W_2\), as desired.
\end{proof}

\begin{exercise} \label{exercise 1.6.23}
\sloppy Let \(v_1, v_2, ..., v_k, v\) be vectors in a vector space \(V\), and define \(W_1 = \spann(\{v_1, v_2, ..., v_k\})\), and \(W_2 = \spann(\{ v_1, v_2, ..., v_k, v \})\).
\begin{enumerate}
\item Find necessary and sufficient conditions \RED{on \(v\)} such that \(\dim(W_1) = \dim(W_2)\).
\item State and prove a relationship involving \(\dim(W_1)\) and \(\dim(W_2)\) in the case that \(\dim(W_1) \ne \dim(W_2)\).
\end{enumerate}
\end{exercise}

\begin{proof}\ 

\begin{enumerate}
\item We claim that \(v \in \spann(\{ v_1, v_2, ..., v_k \}\) if and only if \(\dim(W_1) = \dim(W_2)\).

\(\Longrightarrow\): Suppose \(v \in \spann(\{ v_1, v_2, ..., v_k \}\).
To show \(\dim(W_1) = \dim(W_2)\), it suffices to show \(W_1 = W_2\).
But since \(\{ v_1, v_2, ..., v_k \} \subseteq \{ v_1, v_2, ..., v_k, v \}\), by \THM{1.6}, \(\spann(\{ v_1, v_2, ..., v_k \}) \subseteq \spann(\{ v_1, v_2, ..., v_k, v \})\).
That is, \(W_1 \subseteq W_2\).
So it suffices to show \(W_2 \subseteq W_1\).
Now suppose arbitrary \(u \in W_2\).
Then
\begin{align*}
             & u \in W_2 \\
    \implies & u \in \spann(\{ v_1, v_2, ..., v_k, v \}) \\
    \implies & u = a_1 v_1 + a_2 v_2 + ... + a_k v_k + a v \\
    \implies & u = a_1 v_1 + a_2 v_2 + ... + a_k v_k + a (b_1 a_1 + b_2 a_2 + ... + b_k a_k), & \text{\(\because v \in \spann(\{ v_1, v_2, ..., v_k \})\)}
\end{align*}
So \(u\) is a linear combination of \(v_1, v_2, ..., v_k\).
So \(u \in \spann(\{ v_1, v_2, ..., v_k \})\).
That is, \(u \in W_1\).
So we have \(W_2 \subseteq W_1\), as desired.

\(\Longleftarrow\): Now suppose \(\dim(W_1) = \dim(W_2) = n\).

That is, \(\dim(\spann(\{ v_1, v_2, ..., v_k \})) = \dim(\spann(\{ v_1, v_2, ..., v_k, v \}))\) \MAROON{(1)}.
But that implies \(\{ v_1, v_2, ..., v_k, v \}\) is \LDP{}, otherwise if \(\{ v_1, v_2, ..., v_k, v \}\), then by \CORO{1.6.1} \(\{ v_1, v_2, ..., v_k \}\) is also \LID{}.
But the two sets have different elements, so the dimensions of span of them are different, which contradicts \MAROON{(1)}.
And by \ATHM{1.17}(2), we have \(v \in \spann(\{ v_1, v_2, ..., v_k \})\), as desired.

\item Equivalently, by taking the negation of both sides of the statement claimed in part(a), \(v \notin \spann(\{ v_1, v_2, ..., v_k \})\) if and only if \(\dim(W_1) \ne \dim(W_2)\).
\end{enumerate}
\end{proof}

\begin{exercise} \label{exercise 1.6.24}
Let \(f(x)\) be a polynomial of degree \(n\) in \(\mathcal{P}_n(\SET{R})\).
Prove that for any \(g(x) \in \mathcal{P}_n(\SET{R})\) there exist scalars \(c_0, c_1, ..., c_n\) such that
\[
    g(x) = c_0 f(x) + c_1 f'(x) + c_2 f''(x) + ... + c_n f^{(n)}(x),
\]
where \(f^{(n)}(x)\) denotes the \(n\)th derivative of \(f(x)\).
\end{exercise}

\begin{proof}
Let arbitrary \(f(x) = a_0 + a_1 x + a_2 x^2 + ... + a_n x^n\) be a polynomial of degree \(n\) in \(\mathcal{P}_n(\SET{R})\).
Now it suffices to show \(\beta = \{ f, f', f'', ..., f^{(n)} \}\) is a basis for \(\mathcal{P}_n(\SET{R})\), thus every polynomial with degree \(\le n\) can be represented by \(\beta\).

Also it suffices to show \(\beta\) is \LID{} to show it is a basis for \(\mathcal{P}_n(\SET{R})\), since \(\beta\) has \(n + 1\) elements, which equals to the dimension of \(\mathcal{P}_n(\SET{R})\).

Since we have known(by Calculus) that no two elements of \(\beta\) have the same degree, by \EXEC{1.5.18}, \(\beta\) is \LID{}, as desired.
\end{proof}

\begin{exercise} \label{exercise 1.6.25}
Let \(V\), \(W\), and \(Z\) be as in \EXEC{1.2.21}.
If \(V\) and \(W\) are vector spaces over \(F\) of dimensions \(m\) and \(n\), determine the dimension of \(Z\).
\end{exercise}

\begin{proof}
\sloppy Let \(\beta = \{ (v_1, \OW), (v_2, \OW), ..., (v_m, \OW) \} \cup \{ (\OV, w_1), (\OV, w_2), ..., (\OV, w_n) \}\),
where \(\{ v_1, ..., v_m \}\) is a basis for \(V\), \(\{ w_1, ..., w_n \}\) is a basis for \(W\), \(\OW, \OV\) are zero vectors of \(W, V\), respectively.
Then it's obvious that \(\beta\) is a basis for \(Z\), and has \(m + n\) elements, so \(Z\) has dimension \(m + n\).
\end{proof}

\begin{exercise} \label{exercise 1.6.26}
For a fixed \(a \in \SET{R}\), determine the dimension of the subspace of \(\mathcal{P}_n(\SET{R})\) defined by \(W = \{ f \in \mathcal{P}_n(\SET{R}) : f(a) = 0 \}\).

Related: \EXEC{1.3.13}.
\end{exercise}

\begin{proof}
Let \(a \in \SET{R}\) and \(W = \{ f \in \mathcal{P}_n(\SET{R}) : f(a) = 0 \}\).
Now let arbitrary \(f \in W\).
Then we have
\[
    f(x) = p_0 + p_1 x + p_2 x^2 + ... + p_n x^n \MAROON{(1)}.
\]
for some real numbers \(p_0, p_1, ..., p_n\).
In particular we have \(f(a) = 0\), that is,
\[
    f(a) = p_0 + p_1 a + p_2 a^2 + ... + p_n a^n = 0,
\]
which implies \(p_0 = -p_1 a - p_2 a^2 - ... - p_n a^n\) \MAROON{(2)}.
So from \MAROON{(1)(2)} we have
\begin{align*}
    f(x) & = (-p_1 a - p_2 a^2 - ... - p_n a^n) + p_1 x + p_2 x^2 + ... + p_n x^n \\
         & = p_1(x - a) + p_2(x^2 - a^2) + ... + p_n (x^n - a^n).
\end{align*}
Now we let \(\beta = \{ x - a, x^2 - a^2, ..., x^n - a^n \}\).
Then \(f \in \spann(\beta)\).
Since \(f\) is arbitrary, we have \(W \subseteq \spann(\beta)\).

Another direction \(\spann(\beta) \subseteq W\) is trivial since the output of any linear combination of \(\beta\) given \(x = a\) is \(0\).

So we have \(W = \spann(\beta)\).
Now to show \(\beta\) is a basis for \(W\), it suffices to show \(\beta\) is \LID{}.
But since no two elements in \(\beta\) has the same degree, it follows from \EXEC{1.5.18} that \(\beta\) is \LID{}, as desired.

So \(\beta\) is a basis for \(W\), and that implies \(W\) has dimension \(n\).
\end{proof}

\begin{exercise} \label{exercise 1.6.27}
Let \(W_1\) and \(W_2\) be the subspaces of \(\mathcal{P}(F)\) defined in \EXEC{1.3.25}.
Determine the dimensions of the subspaces \(W_1 \cap \mathcal{P}_n(F)\) and \(W_2 \cap \mathcal{P}_n(F)\).
\end{exercise}

\begin{proof}
\(W_1 \cap \mathcal{P}_n(F)\) is the set of all polynomials of degree \(\le n\) and having only even power terms.
\(W_2 \cap \mathcal{P}_n(F)\) is the set of all polynomials of degree \(\le n\) and having only odd power terms.

Now suppose \(n\) is even.

Then \(n = 2k\) for some natural number \(k\), and trivially \(W_1 \cap \mathcal{P}_n(F)\) has a basis \(\{ 1, x^2, ..., x^{2k} \} = \{ x^{2 \X 0}, x^{2 \X 1}, ..., x^{2 \X k} \}\), which has \(k + 1 = n/2 + 1\) elements.
So \(W_1 \cap \mathcal{P}_n(F)\) has dimension \(n/2 + 1\).

Similarly \(W_2 \cap \mathcal{P}_n(F)\) has a basis \(\{ x, x^3, ..., x^{2k - 1} \} = \{ x^{2 \X 0 + 1}, x^{2 \X 1 + 1}, ..., x^{2 \X (k - 1) + 1} \}\), which has \(k = n/2\) elements.
So \(W_2 \cap \mathcal{P}_n(F)\) has dimension \(n/2\).

Now suppose \(n\) is odd.

Then \(n = 2k + 1\) for some natural number \(k\), and trivially \(W_1 \cap \mathcal{P}_n(F)\) has a basis \(\{ 1, x^2, ..., x^{2k} \} = \{ x^{2 \X 0}, x^{2 \X 1}, ..., x^{2 \X k} \}\), which has \(k + 1 = (n - 1)/2 + 1\) elements.
So \(W_1 \cap \mathcal{P}_n(F)\) has dimension \((n - 1)/2 + 1\).

Similarly \(W_2 \cap \mathcal{P}_n(F)\) has a basis \(\{ x , x^3, ..., x^{2k + 1} \} = \{ x^{2 \X 0 + 1}, x^{2 \X 1 + 1}, ..., x^{2 \X k + 1} \}\), which also has \(k + 1 = (n - 1)/2 + 1\) elements.
So \(W_2 \cap \mathcal{P}_n(F)\) has dimension \((n - 1)/2 + 1\).
\end{proof}

\begin{exercise} \label{exercise 1.6.28}
Let \(V\) be a finite-dimensional vector space (of complex numbers) \emph{over \(\SET{C}\)} with dimension \(n\).
Prove that if \(V\) is now regarded as a vector space \emph{over \(\SET{R}\)}, then \(\dim(V) = 2n\).
(See \EXAMPLE{1.6.11} and \EXAMPLE{1.6.12}.)
\end{exercise}

\begin{proof}
\TODOREF \RED{SKIP}.
Go back when learning complex numbers rigorously.
\end{proof}

\begin{exercise} \label{exercise 1.6.29} \ 

\begin{enumerate}
\item Prove that if \(W_1\) and \(W_2\) are finite-dimensional subspaces of a vector space \(V\), then \BLUE{(1)} the subspace \(W_1 + W_2\) is finite-dimensional, \BLUE{(2)} and \(\dim(W_1 + W_2) = \dim(W_1) + \dim(W_2) - \dim(W_1 \cap W_2)\).
(Note that we do not say \(V\) is finite-dimensional.)
Hint: Start with a basis \(\{ u_1, u_2, ..., u_k \}\) for \(W_1 \cap W_2\) and extend this set to a basis \(\{ u_1, u_2, ..., u_k, v_1, v_2, ..., v_m \}\) for \(W_1\) and to a basis \(\{ u_1, u_2, ..., u_k, w_1, w_2, ..., w_p \}\) for \(W_2\).
\item Let \(W_1\) and \(W_2\) be finite-dimensional subspaces of a vector space \(V\), and let(suppose?) \(V = W_1 + W_2\). 
Deduce that \(V\) is the \emph{direct} sum of \(W_1\)
and \(W_2\) if and only if \(\dim(V) = \dim(W_1) + \dim(W_2)\).
\end{enumerate}
\end{exercise}

\begin{proof} \ 

\begin{enumerate}
\item
We prove \BLUE{(2)} first, which immediately implies \BLUE{(1)}.

First let \(\beta = \{ u_1, u_2, ..., u_k \}\) be a basis for \(W_1 \cap W_2\).
Then since \(W_1 \cap W_2\) is (clearly) a subspace of \(W_1\). (Why clear? Both \(W_1, W_2\) are subspace of \(V\), so by \THM{1.4}, \(W_1 \cap W_2\) are subspace of \(V\);
in particular \(W_1 \cap W_2\) is a vector space.
But \(W_1 \cap W_2\) is a \emph{subset} of \(W_1\).
So \(W_1 \cap W_2\) is a subset of \(W_1\) and is also a vector space, so by \DEF{1.2} \(W_1 \cap W_2\) is a subspace of \(W_1\).)
Similarly \(W_1 \cap W_2\) is a subspace of \(W_2\).
And \CORO{1.11.1} we can \emph{extend} the basis \(\beta\) of \(W_1 \cap W_2\) to get a \(W_1\)'s basis: \(\beta_1 = \{ u_1, ..., u_k, v_1, ..., v_m \}\);
similarly for \(W_2\)'s basis: \(\beta_2 = \{ u_1, ..., u_k, w_1, ..., w_n \}\).

And for \(W_1\)'s basis we can conclude that \(\dim(W_1) = \#\beta_1 = k + m\);
similarly \(\dim(W_2) = \#\beta_2 = k + n\).
So we have
\begin{align*}
    & \dim(W_1) + \dim(W_2) - \dim(W_1 \cap W_2) \\
    & = k + m + k + n - \#\beta \\
    & = k + m + k + n - k \\ 
    & = k + m + n \MAROON{(1)}.
\end{align*}

\sloppy To show \(\dim(W_1 + W_2) = \dim(W_1) + \dim(W_2) - \dim(W_1 \cap W_2) = k + m + n\), it suffices to show \(W_1 + W_2\) has a basis with \(k + m + n\) elements.
Now consider
\[
    T = \{ u_1, ..., u_k, v_1, ..., v_m, w_1, ..., w_n \},
\]
then it has \(k + m + n\) elements.
We will show that \(T\) is actually a basis for \(W_1 + W_2\).
So we first show \(T\) is \LID{}.

So suppose \(a_1 u_1 + ... + a_k u_k + b_1 v_1 + ... + b_m v_m + c_1 w_1 + ... + c_n w_n = \OV\) \MAROON{(2)}.
\begin{align*}
    \MAROON{(2)} \implies a_1 u_1 + ... + a_k u_k + b_1 v_1 + ... + b_m v_m = - c_1 w_1 - ... - c_n w_n \MAROON{(3)}
\end{align*}
Now the \(LHS\) of \MAROON{(3)} is a linear combination of \(\beta_1\), which is \(W_1\)'s basis, so \(LHS \in W_1\).
The \(RHS\) of \(\MAROON{(3)}\) is a linear combination of (subset of) \(\beta_2\), which is \(W_2\)'s basis, so \(RHS \in W_2\).
And (of course) \(LHS = RHS\), so both \(LHS\) and \(RHS\) belong to \(W_1\) and \(W_2\), so they belong to \(W_1 \cap W_2\).

In particular, \MAROON{(3)}'s \(RHS = -c_1 w_1 - c_2 w_2 - ... - c_n w_n \in W_1 \cap W_2\), so it can be expressed as a linear combination of the basis \(\beta\) of \(W_1 \cap W_2\);
that is,
\[
    -c_1 w_1 - c_2 w_2 - ... - c_n w_n = d_1 u_1 + d_2 u_2 + ... + d_k u_k
\]
for some scalars \(d_1, ..., d_k\).
But arranging equation, we have
\[
    \OV = c_1 w_1 + c_2 w_2 + ... + c_n w_n + d_1 u_1 + d_2 u_2 + ... + d_k u_k,
\]
But this is a linear combination using \(W_2\)'s basis, which is \LID{}, which implies \(c_1 = c_2 = ... = c_n = d_1 = d_2 = ... = d_k = 0\).

And in particular, \(c_1 = c_2 = ... = c_n = 0\), which implies the \(RHS\) of \MAROON{(3)} is \(\OV\), so \MAROON{(3)}'s \(LHS = \OV\).
But again, \(LHS\) is a linear combination using \(W_1\)'s basis, which is \LID{}, which implies \(a_1 = a_2 = ... = a_k = b_1 = b_2 = ... = b_m = 0\).
So now we have \(a_1 = ... = a_k = b_1 = ... = b_m = c_1 = ... c_n = 0\), and from \MAROON{(2)} we conclude that \(T\) is \LID{}.

Now we show \(\spann(T) = W_1 + W_2\).
So suppose arbitrary \(x \in W_1 + W_2\).
Then
\begin{align*}
             & x = x_1 + x_2 \\
             & \text{ (for some } x_1 \in W_1, x_2 \in W_2) \\
    \implies & x = (p_1 u_1 + ... + p_k u_k + q_1 v_1 + ... + q_m v_m) + (r_1 u_1 + ... + r_k u_k + s_1 w_1 + ... + s_n w_n) \\
             & \text{(represent as combinations)} \\
    \implies & x = (p_1 + r_1) u_1 + ... + (p_k + r_k) u_k + q_1 v_1 + ... + q_m v_m + s_1 w_1 + ... + s_n w_n,
\end{align*}
which is a linear combination of \(T\), so \(x \in \spann(T)\).
Since \(x\) is arbitrary, we have \(W_1 + W_2 \subseteq \spann(T)\).
Now suppose arbitrary \(x \in \spann(T)\).
Using the similar equation arranging above, it's trivial that \(x \in W_1 + W_2\), so we have \(\spann(T) \subseteq W_1 + W_2\).
So \(W_1 + W_2 = \spann(T)\).

So we have \(T\) spans \(W_1 + W_2\) and is \LID{}, hence \(T\) is a basis for \(W_1 + W_2\) and has \(k + m + n\) elements, as desired.

\item
We restate the result from part(a):
\[
    \dim(W_1 + W_2) = \dim(W_1) + \dim(W_2) - \dim(W_1 \cap W_2).
\]
And suppose \(V = W_1 + W_2\). \MAROON{(4)} (this is a supposition given by the exercise's description.)

\(\Longrightarrow\):
Suppose \(V = W_1 \oplus W_2\).
Then in particular \(W_1 \cap W_2 = \{ \OV \}\), so we have
\begin{align*}
    \dim(W_1 + W_2) & = \dim(W_1) + \dim(W_2) - \dim(W_1 \cap W_2) \\
                    & = \dim(W_1) + \dim(W_2) - \dim(\{ \OV \}) \\
                    & = \dim(W_1) + \dim(W_2) - 0 \\
                    & = \dim(W_1) + \dim(W_2).
\end{align*}

\(\Longleftarrow\):
Now suppose \(\dim(W_1 + W_2) = \dim(W_1) + \dim(W_2)\).
But we know that \(\dim(W_1 + W_2) = \dim(W_1) + \dim(W_2) - \dim(W_1 \cap W_2)\), so these equations imply \(\dim(W_1 \cap W_2) = 0\), which implies \(W_1 \cap W_2 = \{ \OV \}\).
With \MAROON{(4)}, we have \(V = W_1 \oplus W_2\).
\end{enumerate}
\end{proof}

\begin{exercise} \label{exercise 1.6.30}
Let
\[
    V = M_{2 \X 2}(F),
    W_1 = \bigg\{
        \begin{pmatrix}
            a & b \\
            c & a
        \end{pmatrix}
        \in V: a, b, c \in F
        \bigg\},
\]
and
\[
    W_2 = \bigg\{
        \begin{pmatrix}
            0 & a \\
            -a & b
        \end{pmatrix}
        \in V: a, b \in F
        \bigg\}.
\]
Prove that \(W_1\) and \(W_2\) are subspaces of \(V\), and find the dimensions of \(W_1\), \(W_2\), \(W_1 + W_2\), and \(W_1 \cap W_2\).
\end{exercise}

\begin{proof}
It's very trivial to show the three conditions of \THM{1.3} to show that \(W_1, W_2\) are subspaces of \(V\).

And it's trivial that \(\{ E_{11} + E_{22}, E_{12}, E_{21} \}\) is a basis for \(W_1\), so \(\dim(W_1) = 3\).
Similarly \(\{ E_{12} - E_{21}, E_{22} \}\) is a basis for \(W_2\), so \(\dim(W_1) = 2\).
Now (it's again trivial that) any matrix in \(W_1 \cap W_2\) can only have the form
\(\begin{pmatrix}
    0 & a \\
    -a & 0
\end{pmatrix}\)
and trivially \(\dim(W_1 \cap W_2) = 1\).

Finally, from \EXEC{1.6.29}(a) we have \(\dim(W_1 + W_2) = \dim(W_1) + \dim(W_2) - \dim(W_1 \cap W_2) = 3 + 2 - 1 = 4\).
(And that implies \(W_1 + W_2 = V\).)
\end{proof}

\begin{exercise} \label{exercise 1.6.31}
Let \(W_1\) and \(W_2\) be subspaces having (\emph{finite}) dimensions \(m\) and \(n\), respectively, of a vector space \(V\), where \(m \le n\).
\begin{enumerate}
\item Prove that \(\dim(W_1 \cap W_2) \le n\).
\item Prove that \(\dim(W_1 + W_2) \le m + n\).
\end{enumerate}
\end{exercise}

\begin{proof} \ 

\begin{enumerate}
\item Since \(W_1 \cap W_2\) is a subspace of \(W_2\), which has (finite) dimension \(n\), by \THM{1.11} we have \(\dim(W_1 \cap W_2) \le \dim(W_2) = n\).
\item From \EXEC{1.6.29}(a), we have
    \begin{align*}
        \dim(W_1 + W_2) & = \dim(W_1) + \dim(W_2) - \dim(W_1 \cap W_2) \\
                        & \le \dim(W_1) + \dim(W_2) & \text{since \(\dim(W_1 \cap W_2) \ge 0\)} \\
                        & = m + n.
    \end{align*}
\end{enumerate}
\end{proof}

\begin{exercise} \label{exercise 1.6.32}
Find examples of subspaces \(W_1\) and \(W_2\) of \(\SET{R}^3\) such that \(\dim(W_1) > \dim(W_2) > 0\) and
\begin{enumerate}
\item \(\dim(W_1 \cap W_2) = \dim(W_2)\);
\item \(\dim(W_1 + W_2) = \dim(W_1) + \dim(W_2)\);
\item \(\dim(W_1 + W_2) < \dim(W_1) + \dim(W_2)\).
\end{enumerate}
\end{exercise}

\begin{proof}
I will just informally given some geometric examples.
\begin{enumerate}
\item (The equation is related to \EXEC{1.6.22}.) Let \(W_1\) be \(xy\)-plane, which has dimension \(2\), \(W_2\) be just \(x\)-axis, which has dimension \(1\).
    Then \(W_1 \cap W_2\) is \(x\)-axis, which equals to \(W_2\), so \(\dim(W_1 \cap W_2) = \dim(W_2)\).
\item (The equation is related to \EXEC{1.6.29}(b).) By \EXEC{1.6.29}(b), the equality holds when \(W_1 \cap W_2 = \{ (0, 0, 0) \}\).
    So let \(W_1\) be \(xy\)-plane, which has dimension \(2\), \(W_2\) be just \(z\)-axis, which has dimension \(1\).
    And \(W_1 \cap W_2 = \{ (0, 0, 0) \}\).
    Then \(W_1 + W_2\) is the whole \(\SET{R}^3\), which has dimension \(3 = 2 + 1 = \dim(W_1) + \dim(W_2)\).
\item Let \(W_1\) be \(x\)-axis having dimension \(1\), \(W_2\) be \(\SET{R}^3\) having dimension \(3\), then \(\dim(W_1 + W_2) = \dim(\SET{R}^3) = 3 < 1 + 3 = \dim(W_1) + \dim(W_2)\).
\end{enumerate}
\end{proof}

\begin{exercise} \label{exercise 1.6.33} \ 

\begin{enumerate}
\item Let \(W_1\) and \(W_2\) be subspaces of a vector space \(V\) such that \(V = W_1 \oplus W_2\).
    If \(\beta_1\) and \(\beta_2\) are bases for \(W_1\) and \(W_2\), respectively, show that \(\beta_1 \cap \beta_2 = \emptyset\) and \(\beta_1 \cup \beta_2\) is a \emph{basis} for \(V\).
\item Conversely, let \(\beta_1\) and \(\beta_2\) be \emph{disjoint} bases for subspaces \(W_1\) and \(W_2\), respectively, of a vector space \(V\).
    Prove that if \(\beta_1 \cup \beta_2\) is a basis for \(V\), then \(V = W_1 \oplus W_2\).
\end{enumerate}
\end{exercise}

\begin{proof} \ 
\begin{enumerate}
\item
Let \(W_1\) and \(W_2\) be subspaces of a vector space \(V\) such that \(V = W_1 \oplus W_2\), and \(\beta_1\) and \(\beta_2\) are bases for \(W_1\) and \(W_2\).

We first show that \(\beta_1 \cap \beta_2 = \emptyset\).
For the sake of contradiction, suppose not.
Then there exists \(v\) s.t. \(v \in \beta_1\) and \(v \in \beta_2\).
Note that \(v\) must be nonzero vector since \(\beta_1, \beta_2\) are (basis, hence) \LID{}.
But in particular, \(v \in \spann(\beta_1) = W_1\) and \(v \in \spann(\beta_2) = W_2\).
So we have found a nonzero vector \(v\) s.t. \(v \in W_1 \cap W_2\), which contradicts that (\(V = W_1 \oplus W_2\), hence in particular we have) \(W_1 \cap W_2 = \{ \OV \}\).
Now we show that \(\beta_1 \cup \beta_2\) is a basis for \(V\).
So we first show that \(\spann(\beta_1 \cup \beta_2) = V\).
But it suffices to show this by showing \(V \subseteq \spann(\beta_1 \cup \beta_2)\) since the other direction is automatically true.
So suppose arbitrary \(v \in V\).
Since \(V = W_1 \oplus W_2\), in particular \(V = W_1 + W_2\).
So \(v = w_1 + w_2\) for some \(w_1 \in W_1, w_2 \in W_2\);
But \(w_1\) can be represented as a linear combination of the basis \(\beta_1\) of \(W_1\); similarly to \(w_2\) with \(\beta_2\).
So \(v\) can be represented as a sum of a linear combination of \(\beta_1\) and \(\beta_2\), which implies \(v\) is a linear combination of \(\beta_1 \cup \beta_2\).
Hence \(v \in \spann(\beta_1 \cup \beta_2)\), as desired.

Now we show \(\beta_1 \cup \beta_2\) is \LID{}.
So suppose
\[
    \OV = c_1 v_1 + c_2 v_2 + ... + c_m v_m + d_1 u_1 + d_2 u_2 + ... + d_n u_n \MAROON{(1)},
\]
where \(v_1, ..., v_m \in \beta_1\), \(u_1, ..., u_n \in \beta_2\). \(c_1, ..., c_m, d_1, ..., d_n\) are scalars.
Then from \MAROON{(1)} we have
\[
    c_1 v_1 + c_2 v_2 + ... + c_m v_m = - d_1 u_1 - d_2 u_2 - ... - d_n u_n \MAROON{(2)};
\]
But \(LHS \in \spann(\beta_1) = W_1\) and \(RHS \in \spann(\beta_2) = W_2\), which implies they both belong to \(W_1 \cap W_2\), which implies they equal to \(\OV\) (since \(W_1 \cap W_2 = \{ \OV \}\)).
So we have \(c_1 v_1 + c_2 v_2 + ... + c_m v_m = \OV\) and \(- d_1 u_1 - d_2 u_2 - ... - d_n u_n = \OV\).
But \(v_1, ..., v_m\) (belong to the basis of \(\beta_1\) hence) are \LID{} and \(u_1, ..., u_n\) (belong to the basis of \(\beta_2\) hence) are \LID{}, this means \(c_1 = c_2 = ... = c_m = d_1 = d_2 = ... = d_n = 0\).
From \MAROON{(1)}, we have \(\beta_1 \cup \beta_2\) is \LID{}.

So we have \(\spann(\beta_1 \cup \beta_2) = V\) and \(\beta_1 \cup \beta_2\) is \LID{}, hence is a basis for \(V\).

\item
Now let \(\beta_1 = \{ v_1, ..., v_m \}\) and \(\beta_2 = \{u_1, ..., u_n \}\) be \emph{disjoint} bases for subspaces \(W_1\) and \(W_2\), respectively, of a vector space \(V\).
Now suppose \(\beta_1 \cup \beta_2 = \{ v_1, ..., v_m, u_1, ..., u_n \}\) is a basis for \(V\).
We have to show \(V = W_1 \oplus W_2\).
That is, \(W_1 + W_2 = V\) and \(W_1 \cap W_2 = \{ \OV \}\).

We first show \(W_1 + W_2 = V\), and it suffices to show \(V \subseteq W_1 + W_2\) since the other direction is automatically true.
But for arbitrary \(v \in V\),
\begin{align*}
    v & = a_1 v_1 + a_2 v_2 + ... + a_m v_m + b_1 u_1 + b_2 u_2 + ... + b_n u_n & \text{as a l.c. of the basis \(\beta_1 \cup \beta_2\)} \\
      & = (a_1 v_1 + a_2 v_2 + ... + a_m v_m) + (b_1 u_1 + b_2 u_2 + ... + b_n u_n) & \text{in particular} \\
      & \in \spann(\beta_1) + \spann(\beta_2) & \text{by def of ``sum''} \\
      & = W_1 + W_2 & \text{since \(\beta_1, \beta_2\) are basis for \(W_1, W_2\)}
\end{align*}
So \(V \subseteq W_1 + W_2\).

Now we show \(W_1 \cap W_2 = \{ \OV \}\).
So suppose \(v \in W_1 \cap W_2\), we have to show \(v = \OV\).
Then
\begin{align*}
             & v \in W_1 \cap W_2 \\
    \implies & v \in W_1 \land v \in W_2 & \text{in particular} \\
    \implies & v = a_1 v_1 + a_2 v_2 + ... + a_m v_m \\
             & \land v = b_1 u_1 + b_2 u_2 + ... + b_n u_n & \text{for some scalars \(a_1, ..., a_m, b_1, ..., b_n\)} \\
    \implies & a_1 v_1 + ... + a_m v_m - b_1 u_1 - ... - b_n u_n = \OV,
\end{align*}
which implies \(a_1 = ... = a_m = b_1 = ... = b_n = 0\) since \(\{v_1, ..., v_m, u_1, ..., u_m\}\) (is a basis for \(V\) hence) is \LID{};
and that implies \(v = a_1 v_1 + ... + a_m v_m = \OV\), as desired.
\end{enumerate}
\end{proof}

\begin{exercise} \label{exercise 1.6.34}\ 

\begin{enumerate}
\item Prove that if \(W_1\) is any subspace of a finite-dimensional vector space \(V\), then there exists a subspace \(W_2\) of \(V\) such that \(V = W_1 \oplus W_2\).
\item Let \(V = \SET{R}^2\) and \(W_1 = \{ (a_1, 0) : a_1 \in \SET{R} \}\).
    Give examples of two different subspaces \(W_2\) and \(W'_2\) such that \(V = W_1 \oplus W_2\) and \(V = W_1 \oplus W'_2\).
\end{enumerate}
\end{exercise}

\begin{proof} \ 
\begin{enumerate}
\item Let \(\beta_1 = \{ v_1, v_2, ..., v_k \}\) be a basis for \(W_1\).
    Then by \CORO{1.11.1} we can extended \(\beta_1\) as \(\{v_1, ..., v_k, v_{k + 1}, ..., v_n \}\) to get a basis for \(V\).
    And of course \(\{ v_1, ..., v_k \} \cap \{ v_{k + 1}, ..., v_n \} = \emptyset\).
    Now let \(W_2 = \spann(\{ v_{k + 1}, ..., v_n \})\).
    Then by \EXEC{1.6.33}(b), we have \(V = W_1 \oplus W_2\), as desired.
\item Clearly, \(\beta_1 = \{ (1, 0) \}\) is a basis for \(W_1\).
    Now let \(\beta_2 = \{ (0, 1) \}\), \(\beta'_2 = \{ (1, 1) \}\), \(W_2 = \spann(\beta_2)\), \(W'_2 = \spann(\beta'_2)\).
    Since \(\beta_1 \cap \beta_2 = \emptyset\) and (clearly) \(\beta_1 \cup \beta_2\) is a basis for \(W\), by \EXEC{1.6.33}(b), \(V = W_1 \oplus W_2\).
    Similarly, \(V = W_1 \oplus W'_2\).
\end{enumerate}
\end{proof}

\begin{exercise} \label{exercise 1.6.35}
Let \(W\) be a subspace of a finite-dimensional vector space \(V\), and consider the basis \(\{ u_1, u_2, ..., u_k \}\) for \(W\).
Let \(\alpha = \{ u_1, u_2, ..., u_k, u_{k + 1}, ..., u_n \}\) be an extension of this basis to a basis for \(V\).
\begin{enumerate}
\item Prove that \(\beta = \{ u_{k + 1} + W, u_{k + 2} + W, ..., u_n + W \}\) is a basis for \(V / W\).
\item Derive a formula relating \(\dim(V), \dim(W)\), and \(\dim(V/W)\).
\end{enumerate}
\end{exercise}

\begin{note}
We call the zero vector \(\OV + W\) (to see why this is the zero vector, see \EXEC{1.3.31}(d)'s (VS 3)), of \(V/W\) as \(0_{_{V/W}}\), to prevent ambiguity of zero vectors of \(V\) and \(W\).
The definitions of addition and scalar multiplication of \(V/W\) are also in \EXEC{1.3.31}.
\end{note}

\begin{proof} \ 

\begin{enumerate}
\item
We have to show \(\beta\) is \LID{} and \(\spann(\beta) = V/W\).
So suppose \(0_{_{V/W}} = a_{k + 1} (u_{k + 1} + W) + a_{k + 2} (u_{k + 2} + W) + ... + a_n (u_n + W)\) \MAROON{(1)};
we have to show \(a_{k + 1} = a_{k + 2} = ... = a_n = 0\) to show \(\beta\) is \LID{} \MAROON{(2)}.
Now from \MAROON{(1)},
\begin{align*}
    0_{_{V/W}} & = \OV + W \\
               & = a_{k + 1} (u_{k + 1} + W) + a_{k + 2} (u_{k + 2} + W) + ... + a_n (u_n + W) & \text{by \MAROON{(1)}} \\
               & = (a_{k + 1} u_{k + 1} + W) + (a_{k + 2} u_{k + 2} + W) + ... + (a_n u_n + W) & \text{by def of \(\cdot\)} \\
               & = (a_{k + 1} u_{k + 1} + a_{k + 2} u_{k + 2} + ... + a_n u_n) + W, & \text{by def of \(+\)}
\end{align*}
which implies \(a_{k + 1} u_{k + 1} + a_{k + 2} u_{k + 2} + ... + a_n u_n = \OV\).
But since \(u_{k + 1}, u_{k + 2}, ..., u_n\) is \LID{} (of the vector space \(V\), not \(V/W\)), we have \(a_{k + 1} = a_{k + 2} = ... = a_n = 0\).
So \MAROON{(2)} is satisfied, so \(\beta\) is \LID{}, as desired.

Now we will show \(\spann(\beta) = V/W\).
But again, it suffices to show \(V/W \subseteq \spann(\beta)\) because the other direction is automatically true.
So suppose arbitrary ``vector'' \(v + W \in V/W\), we have to show \(v + W \in \spann(\beta)\).
And
\begin{align*}
    v + W & = (a_1 u_1 + a_2 u_2 + ... + a_n u_n) + W \\
          & \text{\ \ \ \ \ \ (since \(\alpha\) is a basis for \(V\))} \\
          & = ((a_1 u_1 + a_2 u_2 + ... + a_k u_k) + (a_{k + 1} u_{k + 1} + ... + a_n u_n)) + W \\
          & \text{\ \ \ \ \ \ (in particular)} \\
          & = ((a_1 u_1 + a_2 u_2 + ... + a_k u_k) + W) + ((a_{k + 1} u_{k + 1} + ... + a_n u_n) + W) \\
          & \text{\ \ \ \ \ \  (by def of \(+\))} \\
          & = (\OV + W) + ((a_{k + 1} u_{k + 1} + ... + a_n u_n) + W) \\
          & \text{\ \ \ \ \ \ (\RED{see below (*)})} \\
          & = ((a_{k + 1} u_{k + 1} + ... + a_n u_n) + W) & \\
          & \text{\ \ \ \ \ \ (by (VS 3) of \(V/W\))} \\
          & = a_{k + 1}(u_{k + 1} + W) + a_{k + 2}(u_{k + 2} + W) + ... + a_n(u_n + W) \\
          & \text{\ \ \ \ \ \ (by def of \(+\), \(\cdot\))} \\
          & \in \spann(\beta),
\end{align*}
as desired.

\RED{(*)}: We have to show \(((a_1 u_1 + a_2 u_2 + ... + a_k u_k) + W) = \OV + W\).
Note that any element in these sets is \emph{vector of \(V\)}, not vector of \(V/W\).
So we just show for any \(v\), \(v\) belongs to the former if and only if \(v\) belongs to the latter:
\begin{align*}
         & v \in (a_1 u_1 + a_2 u_2 + ... + a_k u_k) + W \\
    \iff & v = (a_1 u_1 + a_2 u_2 + ... + a_k u_k) + (b_1 u_1 + b_2 u_2 + ... + b_k u_k) \\
         & \text{(since \(\{ u_1, ..., u_k \}\) is a basis for \(W\))} \\
    \iff & v = (a_1 + b_1) u_1 + (a_2 + b_2) u_2 + ... + (a_k + b_k) u_k & \text{of course} \\
    \iff & v = \OV + ((a_1 + b_1) u_1 + (a_2 + b_2) u_2 + ... + (a_k + b_k) u_k) & \text{of course} \\
    \iff & v \in \OV + W \\
         & \text{(since \(\{ u_1, ..., u_k \}\) is a basis for \(W\))} \\
\end{align*}
So we have \((a_1 u_1 + a_2 u_2 + ... + a_k u_k) + W = \OV + W\).

\item
\(\alpha\) is a basis for \(V\), so \(\dim(V) = \#\alpha = n\).
\(\{ u_1, u_2, ..., u_k \}\) is a basis for \(W\), so \(\dim(W) = k\).
\(\beta\) is a basis for \(V/W\), so \(\dim(V/W) = \#\beta = n - k\).
So we have \(\dim(V) = \dim(W) + \dim(V/W)\).
(Or \(\dim(V/W) = \dim(V) - \dim(W)\) if now we care \(V/W\) instead of \(V\).)
\end{enumerate}
\end{proof}

\begin{additional theorem} \label{athm 1.19}
This is the placeholder theorem for the basis and dimension of some set of matrices having particular properties:

\BLUE{(1)} \EXEC{1.6.15}: The set of all \(n \X n\) matrices with trace equal to zero has dimension \(n^2 - 1\).

\BLUE{(2)} \EXEC{1.6.16}: The set of all \(n \X n\) upper triangular matrices has dimension \(n(n + 1)/2\).

\BLUE{(3)} \EXEC{1.6.17}: The set of all \(n \X n\) skew-symmetric matrices has dimension \(n(n - 1)/2\).
\end{additional theorem}

\begin{additional theorem} \label{athm 1.20}
This is the placeholder theorem for \EXEC{1.6.20}: Given any subset \(S\) of finite-dimensional \(V\) with dimension \(n\) where \(S\) is not necessarily finite:

\BLUE{(1))} We still can find a subset of \(S\) which is a basis for \(V\).

\BLUE{(2)} \(S\) still has at least \(n\) elements.
\end{additional theorem}

\begin{additional theorem} \label{athm 1.21}
This is the placeholder theorem for \EXEC{1.6.21}: \(V\) is infinite-dimensional if and only if it contains an \emph{infinite} \LID{} subset.
\end{additional theorem}

\begin{additional theorem} \label{athm 1.22}
This is the placeholder theorem for \EXEC{1.6.22}: Let \(W_1, W_2\) be subspaces of a finite-dimensional vector space \(V\), then \(W_1 \subseteq W_2\) if and only if \(\dim(W_1 \cap W_2) = \dim(W_1)\).
\end{additional theorem}

\begin{additional theorem} \label{athm 1.23}
This is the placeholder theorem for \EXEC{1.6.23}(a):

\(\dim(\spann(\{ v_1, v_2, ..., v_k \})) = \dim(\spann(\{ v_1, v_2, ..., v_k, v \}))\) if and only if \(v \in \spann(\{ v_1, v_2, ..., v_k \})\).
\end{additional theorem}

\begin{additional theorem} \label{athm 1.24}
This is the placeholder theorem for \EXEC{1.6.24}:

Given any polynomial \(f\) with degree \(n\), the corresponding set
\[
    \{ f, f', f'', ..., f^{(n)} \}
\]
is a basis for \(\mathcal{P}_n(F)\).
\end{additional theorem}

\begin{additional theorem} \label{athm 1.25}
This is the placeholder theorem for \EXEC{1.6.25}:
If \(V\) and \(W\) are vector spaces over \(F\) of dimensions \(m\) and \(n\), the dimension of \(Z\) defined in \EXEC{1.2.21} is \(m + n\).
\end{additional theorem}

\begin{additional theorem} \label{athm 1.26}
This is the placeholder theorem for \EXEC{1.6.25}:
The set of all polynomials with degree \(\le n\) and equal to \(0\) given a fixed input \(a \in \SET{R}\) has dimension \(n\).
\end{additional theorem}

\begin{additional theorem} \label{athm 1.27}
This is the placeholder theorem for \EXEC{1.6.29}, \EXEC{1.6.31}, \EXEC{1.6.33}, and \EXEC{1.6.34}:

\EXEC{1.6.29}: Given finite dimensional subspaces \(W_1, W_2\) of a vector space \(V\).

\BLUE{(1.1)}: \(\dim(W_1 + W_2) = \dim(W_1) + \dim(W_2) - \dim(W_1 \cap W_2)\).

\BLUE{(1.2)} IF \(V = W_1 + W_2\), then \(V = W_1 \oplus W_2\) if and only if \(\dim(W_1 + W_2) = \dim(W_1) + \dim(W_2)\).

\EXEC{1.6.31}: Let \(W_1, W_2\) be as like \BLUE{(1.1)} with dimension \(m, n\) respectively, where \(m \ge n\).

\BLUE{(2.1)}: \(\dim(W_1 \cap W_2) \le n\).

\BLUE{(2.2)}: \(\dim(W_1 + W_2) \le m + n\).

\EXEC{1.6.33}:

\BLUE{(3.1)} Let \(W_1\) and \(W_2\) be subspaces of a vector space \(V\) such that \(V = W_1 \oplus W_2\).
If \(\beta_1\) and \(\beta_2\) are bases for \(W_1\) and \(W_2\), respectively, then \(\beta_1 \cap \beta_2 = \emptyset\) and \(\beta_1 \cup \beta_2\) is a \emph{basis} for \(V\).

\BLUE{(3.2)} Conversely, let \(\beta_1\) and \(\beta_2\) be \emph{disjoint} bases for subspaces \(W_1\) and \(W_2\), respectively, of a vector space \(V\).
If \(\beta_1 \cup \beta_2\) is a basis for \(V\), then \(V = W_1 \oplus W_2\).
    
\EXEC{1.6.34}:

\BLUE{(4)} If \(W_1\) is any subspace of a finite-dimensional vector space \(V\), then there exists a subspace \(W_2\) of \(V\) such that \(V = W_1 \oplus W_2\).
\end{additional theorem}

\begin{additional theorem} \label{athm 1.28}
This is the placeholder theorem for \EXEC{1.6.35}:

Let \(W\) be a subspace of a finite-dimensional vector space \(V\), and \(\{ u_1, u_2, ..., u_k \}\) be a basis for \(W\).
Let \(\{ u_1, u_2, ..., u_k, u_{k + 1}, ..., u_n \}\) be an extension of this basis to a basis for \(V\).

\BLUE{(1)}: \(\beta = \{ u_{k + 1} + W, u_{k + 2} + W, ..., u_n + W \}\) is a basis for \(V / W\).

\BLUE{(2)}: \(\dim(V/W) = \dim(V) - \dim(W)\).
\end{additional theorem}