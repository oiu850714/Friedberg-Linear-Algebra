\section{Summary - Important Facts about Determinants} \label{sec 4.4}

This section is a summary of \SEC{4.1} to \SEC{4.3}.
Skip the text.

\exercisesection

\begin{exercise} \label{exercise 4.4.1}
Label the following statements as true or false.
\begin{enumerate}
\item The determinant of a square matrix may be computed by expanding the matrix along any row or column.
\item In evaluating the determinant of a matrix, it is wise to expand along a row or column containing the largest number of zero entries.
\item If two rows or columns of \(A\) are identical, then \(\det(A) = 0\).
\item If \(B\) is a matrix obtained by interchanging two rows or two columns of \(A\), then \(\det(B) = \det(A)\).
\item If \(B\) is a matrix obtained by multiplying each entry of some row or column of \(A\) by a scalar \(k\), then \(\det(B) = \det(A)\).
\item If \(B\) is a matrix obtained from \(A\) by adding a multiple of some row to a different row, then \(\det(B) = \det(A)\).
\item The determinant of an upper triangular \(n \X n\) matrix is the product of its diagonal entries.
\item For every \(A \in M_{n \X n}(F)\), \(\det(A^\top) = -\det(A)\).
\item If \(A, B \in M_{n \X n}(F)\), then \(\det(AB) = \det(A) \cdot \det(B)\).
\item If \(Q\) is an invertible matrix, then \(\det(Q^{-1}) = [\det(Q)]^{-1}\).
\item A matrix \(Q\) is invertible if and only if \(\det(Q) \ne 0\).
\end{enumerate}
\end{exercise}

\begin{proof} \ 

\begin{enumerate}
\item True.
\item True.
\item True.
\item False, \(\det(B) = -\det(A)\).
\item False. The scalar \(k\) need to be nonzero, and if that is true, then \(\det(B) = k \cdot \det(A)\).
\item True.
\item True.
\item False. \(\det(A^\top) = \det(A)\).
\item True.
\item True.
\item True.
\end{enumerate}
\end{proof}

Exercise 2 to Exercise 4 are calculation problems. Skip.

Exercise 5 and Exercise 6 are the same as \EXEC{4.3.20}, \EXEC{4.3.21}, respectively.