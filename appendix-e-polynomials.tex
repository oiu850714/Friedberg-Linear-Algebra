\section{Polynomials} \label{sec 8.e}

In this appendix, we discuss some useful properties of the polynomials \emph{with coefficients from a field}.
For the definition of a polynomial, refer to \SEC{1.2}.
(See \ADEF{1.4}.)
Throughout this appendix, we assume that all polynomials have coefficients from a fixed field \(F\).

\begin{appendix definition} \label{def e.1}
A polynomial \(f(x)\) \textbf{divides} a polynomial \(g(x)\) if there exists a polynomial \(q(x)\) such that \(g(x) = f(x)q(x)\).
\end{appendix definition}

Our first result shows that the familiar long division process for polynomials \emph{with real} coefficients \textbf{is valid for} polynomials with coefficients from an \textbf{arbitrary field}.

\begin{appendix theorem} [The Division Algorithm for Polynomials] \label{thm e.1}
Let \(f(x)\) be a polynomial of degree \(n\), and let \(g(x)\) be a polynomial of degree \(m \ge 0\).
Then there \textbf{exist unique} polynomials \(q(x)\) and \(r(x)\) such that
\[
    f(x) = q(x)g(x) + r(x), \quad \quad \quad \MAROON{(1)}
\]
where the degree of \(r(x)\) is \emph{less than} \(m\).
\end{appendix theorem}

\begin{proof}
We begin by establishing the \emph{existence} of \(q(x)\) and \(r(x)\) that satisfy \MAROON{(1)}.
There are two cases:
\begin{enumerate}
\item[\BLUE{Case I}] If \(n < m\), take \(q(x) = 0\), having degree \(-1 < m\) by definition, and \(r(x) = f(x)\), then \MAROON{(1)} is satisfied.
\item[Case II] Otherwise, if \(0 \le m \le n\), we apply mathematical induction on \(n\).
\end{enumerate}

First, for the base case, suppose that \(n = 0\).
Then (since \(0 \le m \le n = 0\),) \(m = 0\), and it follows that \(f(x)\) and \(g(x)\) are \emph{nonzero} constants.
Hence we may take \(q(x) = f(x)/g(x)\) and \(r(x) = 0\) to satisfy \MAROON{(1)}.
(Note that \(q(x)\) is also a nonzero constant, and \(r(x)\) has degree \(-1 < 0 = m\).)

Now suppose that the result is valid for all polynomials with degree \emph{less than or equal to} \(n - 1\) for some fixed \(n > 0\).
And suppose \(f(x)\) is an arbitrary polynomial with degree \(n\), and let \(g(x)\) be a given arbitrary polynomial with degree \(0 \le m \le n\).
Suppose that \(f(x), g(x)\) has the form
\[
    f(x) = a_n x^n + a_{n - 1} x^{n - 1} + ... + a_1 x + a_0
\]
and
\[
    g(x) = b_m x^m + b_{m - 1} x^{m - 1} + ... + b_1 x + b_0.
\]
Then we have
\begin{align*}
    & a_n b_m^{-1} x^{n - m} g(x) \\
    & = (a_n b_m^{-1}) x^{n - m} (b_m x^m + b_{m - 1} x^{m - 1} + ... + b_1 x + b_0) \\
    & = \RED{a_n} x^n + a_n b_m^{-1} b_{m - 1} x^{n - 1} + ... + a_n b_m^{-1} b_1 x^{n - m + 1} + a_n b_m^{-1} b_0 x^{n - m}.
\end{align*}
So let \(h(x)\) be the polynomial defined by
\[
    h(x) = f(x) - a_n b_m^{-1} x^{n - m} g(x). \quad \quad \quad \MAROON{(2)}
\]
Then from the nasty equation above, \(h(x)\) is a polynomial of degree \(k < n\), since the coefficient of \(x^n\) is cancelled.
Now, if \(k\) is in fact also less than \(m\), then the relation between \(h(x)\) and \(g(x)\) again satisfies the \BLUE{Case I} in the beginning, so we can just take \(q_1(x) = 0\) and \(r(x) = h(x)\) such that \(h(x) = q_1(x)g(x) + r(x)\).
Otherwise, since \(k < n\), we still can apply the induction hypothesis to obtain polynomials \(q_1(x)\) and \(r(x)\) such that \(r(x)\) has degree less than \(m\), and
\[
    h(x) = q_1(x)g(x) + r(x). \quad \quad \quad \MAROON{(3)}
\]
Combining \MAROON{(2)} and \MAROON{(3)} and solving for \(f(x)\) gives us
\begin{align*}
    f(x) & = h(x) + a_n b_m^{-1} x^{n - m} g(x) & \text{by \MAROON{(2)}} \\
         & = q_1(x)g(x) + r(x) + a_n b_m^{-1} x^{n - m} g(x) & \text{by \MAROON{(3)}} \\
         & = \left[ q_1(x) + a_n b_m^{-1} x^{n - m} \right] \cdot g(x) + r(x) & \text{of course} \\
         & = q(x) \cdot g(x) + r(x),
\end{align*}
where \(q(x) = q_1(x) + a_n b_m^{-1} x^{n - m}\), and \(r(x)\) has degree less than \(m\), hence the existence of \(q(x)\) and \(r(x)\) by mathematical induction.

We now show the \textbf{uniqueness} of \(q(x)\) and \(r(x)\).
Suppose that \(q_1(x), q_2(x), r_1(x)\), and \(r_2(x)\) exist such that \(r_1(x)\) and \(r_2(x)\) each has degree less than \(m\) and
\[
    f(x) = q_1(x)g(x) + r_1(x) = q_2(x)g(x) + r_2(x).
\]
Then
\[
    [q_1(x)- q_2(x)]g(x) = r_2(x) - r_1(x). \quad \quad \quad \MAROON{(4)}
\]
Since both \(r_1(x)\) and \(r_2(x)\) have degree less than \(n\), \(r_2(x) - r_1(x)\) must have degree less than \(n\).
But since \(g(x)\) has degree \(m\), it must follow that \(q_1(x) - q_2(x)\) is the zero polynomial.
(Otherwise the product of \([q_1(x)- q_2(x)]g(x)\) has degree \(\ge m\) and is equal to \(r_2(x) - r_1(x)\) having degree less than \(m\), which is impossible.)
Hence \(q_1(x) = q_2(x)\), and the LHS of \MAROON{(4)} is zero polynomial, hence the RHS of \MAROON{(4)} is also the zero polynomial, and thus \(r_1(x) = r_2(x)\).
\end{proof}

\begin{remark} \label{remark e.1}
In the context of \THM{e.1}, we call \(q(x)\) and \(r(x)\) the \textbf{quotient} and \textbf{remainder}, respectively, for the division of \(f(x)\) by \(g(x)\).
For example, suppose that \(F\) is the field of complex numbers.
Then the quotient and remainder for the division of
\[
    f(x) = (3 + \iu)x^5 - (1 - \iu)x^4 + 6x^3 + (-6 + 2\iu)x^2 + (2 + \iu)x + 1
\]
by
\[
    g(x) = (3 + \iu)x^2 - 2\iu x + 4
\]
are, respectively,
\[
    q(x) = x^3 + \iu x^2 - 2 \quad \text{ and } \quad r(x) = (2 - 3\iu)x + 9.
\]
\end{remark}

\begin{appendix corollary} \label{corollary e.1.1}
Let \(f(x)\) be a polynomial of positive degree, and let \(a \in F\).
Then \(f(a) = 0\) if and only if \(x - a\) divides \(f(x)\).
\end{appendix corollary}

\begin{proof}
Suppose that \(x - a\) divides \(f(x)\).
Then there exists a polynomial \(q(x)\) such that \(f(x) = (x - a)q(x)\).
Thus \(f(a) = (a - a)q(a) = 0 \cdot q(a) = 0\).

Conversely, suppose that \(f(a) = 0\).
Then let \(g(x) = (x - a)\), having degree \RED{\(1\)}.
By the division algorithm, \THM{e.1}, there exist polynomials \(q(x)\) and \(r(x)\) such that \(r(x)\) has degree less than \RED{\(1\)} and
\[
    f(x) = q(x)g(x) + r(x) = q(x)(x - a) + r(x).
\]
Substituting \(a\) for \(x\) in the equation above, we obtain
\[
    0 = f(a) = q(a)(a - a) + r(a) = q(a) \cdot 0 + r(a) = r(a).
\]
So \(r(a) = 0\).
Since \(r(x)\) has degree less than \(1\), it must be the \emph{constant polynomial} \(r(x) = 0\).
Thus \(f(x) = q(x)(x - a)\).
\end{proof}

\begin{remark} \label{remark e.2}
For any polynomial \(f(x)\) with coefficients from a field \(F\), an element \(a \in F\) is called a \textbf{zero} of \(f(x)\) if \(f(a) = 0\).
With this terminology, \CORO{e.1.1} states that \(a\) is a zero of \(f(x)\) if and only if \(x - a\) divides \(f(x)\).
\end{remark}

\begin{appendix corollary} \label{corollary e.1.2}
Any polynomial of degree \(n \ge 1\) has at most \(n\) distinct zeros.
\end{appendix corollary}

\begin{proof}
The proof is by mathematical induction on \(n\).
The result is obvious if \(n = 1\).
Now suppose that the result is true for some positive integer \(n\), and let \(f(x)\) be a polynomial of degree \(n + 1\).
If \(f(x)\) \emph{has no} zeros, then there is nothing to prove.
Otherwise, if \(a\) is a zero of \(f(x)\), then by \CORO{e.1.1} we may write \(f(x) = (x - a)q(x)\) for some polynomial \(q(x)\).
Note that \(q(x)\) must be of degree \(n\);
therefore, by the induction hypothesis, \(q(x)\) can have at most \(n\) distinct zeros.
Since any zero of \(f(x)\) distinct from \(a\) is also a zero of \(q(x)\), it follows that \(f(x)\) can have at most \(n + 1\) distinct zeros.
\end{proof}

\begin{remark} \label{remark e.3}
Polynomials \emph{having no common divisors} arise naturally in the study of \emph{canonical forms}.
(See \CH{7}.)
\end{remark}

\begin{appendix definition} \label{def e.2}
Polynomials \(f_1(x), f_2(x), ..., f_n(x)\) are called \textbf{relatively prime} if no polynomial of \textbf{positive} degree divides \textbf{all of} them.
\end{appendix definition}

\begin{remark} \label{remark e.4}
Note that if \(f_1(x), f_2(x), ..., f_n(x)\) are relatively prime, it \textbf{does not} imply that any \(n - 1\) polynomials of them are relatively prime.
Some trivial example: \(f_1(x) = x + 1, f_2(x) = (x + 1)(x + 3), f_3(x) = x + 2\), then all of them are relatively prime, but \(f_1(x)\) and \(f_2(x)\) are not.
\end{remark}

For example, the polynomials with \emph{real} coefficients \(f(x) = x^2(x - 1)\) and \(h(x) = (x - 1)(x - 2)\) are not relatively prime because \(x - 1\) divides each of them.
On the other hand, consider \(f(x)\) and \(g(x) = (x - 2)(x - 3)\), which do not appear to have common factors.
Could other factorizations of \(f(x)\) and \(g(x)\) reveal a hidden common factor?
We will soon see (\THM{e.9}) that the preceding factors are \emph{the only ones}.
Thus \(f(x)\) and \(g(x)\) are relatively prime because they have no common factors of positive degree.

\begin{lemma} \label{lem e.1}
If \(f_1(x)\) and \(f_2(x)\) are relatively prime polynomials, there exist polynomials \(q_1(x)\) and \(q_2(x)\) such that
\[
    q_1(x) f_1(x) + q_2(x) f_2(x) = 1,
\]
where \(1\) denotes the constant polynomial with value \(1\).
\end{lemma}

\begin{note}
這個\ Lemma 有一個數論版本的。若\ \(a, b\) 互質,則存在整數\ \(x, y\) 使得\ \(ax + by = 1\)。
See \href{https://www.wikiwand.com/en/Extended_Euclidean_algorithm}{Extended Euclidean algorithm}。
\end{note}

\begin{proof}
Without loss of generality, assume that the degree of \(f_1(x)\) is greater than or equal to the degree of \(f_2(x)\).
The proof is by mathematical induction on the \emph{degree of \(f_2(x)\)}.
If \(f_2(x)\) has degree \(0\), then \(f_2(x)\) is a nonzero constant \(c\).
In this case, we can take \(q_1(x) = 0\) and \(q_2(x) = 1/c\), then
\[
    f_1(x) q_1(x) + f_2(x) q_2(x) = f_1(x) \cdot 0 + c \cdot \frac{1}{c} = 1.
\]

Now suppose that the theorem holds whenever the polynomial of lesser degree has degree \emph{less than} \(n\) for some positive integer \(n\).
We need to show the theorem holds for the case \(n\).

So again without loss of generality, assume that the degree of \(f_1(x)\) is greater than or equal to the degree of \(f_2(x)\),
and suppose that \(f_2(x)\) has degree \(n\).
By the division algorithm, \THM{e.1}, for \(f_1(x)\) and \(f_2(x)\), there exist polynomials \(q(x)\) and \(r(x)\) such that \(r(x)\) has degree less than \(n\) and
\[
    f_1(x) = q(x)f_2(x) + r(x) \quad \quad \quad \MAROON{(1)}
\]
Since \(f_1(x)\) and \(f_2(x)\) are relatively prime, \(r(x)\) is \emph{not} the zero polynomial.
We claim that \(f_2(x)\) and \(r(x)\) are relatively prime.
Suppose otherwise; then (by \DEF{e.2}) there exists a polynomial \(g(x)\) of \emph{positive} degree that divides both \(f_2(x)\) and \(r(x)\).
Hence, by \MAROON{(1)}, \(g(x)\) \emph{also divides} \(f_1(x)\), contradicting the fact that \(f_1(x)\) and \(f_2(x)\) are relatively prime.

Since \(r(x)\) has degree less than \(n\) and \(r(x)\) and \(f_2(x)\) are relatively prime,
we may apply the induction hypothesis to \(f_2(x)\) and \(r(x)\).
Thus there exist polynomials \(g_1(x)\) and \(g_2(x)\) such that
\[
    g_1(x) f_2(x) + g_2(x) r(x) = 1. \quad \quad \quad \MAROON{(2)}
\]
Combining \MAROON{(1)} and \MAROON{(2)}, we have
\begin{align*}
    1 & = g_1(x)f_2(x) + g_2(x)r(x) & \text{by \MAROON{(2)}} \\
      & = g_1(x)f_2(x) + g_2(x)[f_1(x) - q(x)f_2(x)] & \text{by \MAROON{(1)}} \\
      & = g_2(x) f_1(x) + [g_1(x) - g_2(x)q(x)]f_2(x) & \text{of course}
\end{align*}
Thus, setting \(q_1(x) = g_2(x)\) and \(q_2(x) = g_1(x) - g_2(x)q(x)\), we obtain the desired result.
\end{proof}

\begin{appendix theorem} \label{thm e.2}
If \(f_1(x), f_2(x), ..., f_n(x)\) are relatively prime polynomials, there exist polynomials \(q_1(x), q_2(x), ..., q_n(x)\) such that
\[
    q_1(x)f_1(x) + q_2(x)f_2(x) + ... + q_n(x)f_n(x) = 1,
\]
where \(1\) denotes the constant polynomial with value \(1\).
\end{appendix theorem}

\begin{proof}
The proof is by mathematical induction on \(n\), the number of polynomials.
And \LEM{e.1} establishes the case \(n = 2\).

Now assume the result is true for fewer than \(n\) polynomials, for some \(n \ge 3\).

So suppose we have \(n\) relatively prime polynomials \(f_1(x), f_2(x), ..., f_n(x)\).
We will first consider the trivial case where the first \(n - 1\) polynomials are \emph{also} relatively prime.
(Note that this may not necessarily true, see \RMK{e.4}.)
By the induction hypothesis, there exist polynomials \(q_1(x), q_2(x), ..., q_{n - 1}(x)\) such that
\[
    q_1(x)f_1(x) + q_2(x)f_2(x) + ... + q_{n-1}(x)f_{n-1}(x) = 1.
\]
Then just setting \(q_n(x) = 0\), the zero polynomial, we have
\[
    q_1(x)f_1(x) + q_2(x)f_n(x) + ... + q_{n-1}(x)f_{n-1}(x) + \RED{q_n(x)}f_n(x) = 1.
\]
which proves the result if the first \(n - 1\) polynomials are relatively prime.

Now suppose that the first \(n - 1\) polynomials are \textbf{not} relatively prime, and let \(g(x)\) be the \emph{monic polynomial}(see \DEF{e.4}) of maximum positive degree that \emph{divides} each of these polynomials \(f_1(x), ..., f_{n - 1}(x)\).
For \(k = 1, 2, ..., n - 1\), let \(h_k(x)\) be the polynomial defined, such that,
\[
    f_k(x) = g(x)h_k(x). \quad \quad \quad \MAROON{(1)}
\]
Then the polynomials \(h_1(x), h_2(x), ..., h_{n - 1}(x)\) are relatively prime.
(Otherwise \(g(x)\) does \emph{not} have the maximum positive degree, a contradiction!)
So by the induction hypothesis, there exist polynomials \(\phi_1(x), \phi_2(x), ..., \phi_{n-1}(x)\) such that
\[
    \phi_1(x)h_1(x) + \phi_2(x)h_2(x) + ... + \phi_{n-1}(x)h_{n - 1}(x) = 1.
\]
Multiplying both sides of this equation by \(g(x)\), we obtain
\begin{align*}
             & \phi_1(x)h_1(x)g(x) + \phi_2(x)h_2(x)g(x) + ... + \phi_{n-1}(x)h_{n - 1}(x)g(x) = g(x). \\
    \implies & \phi_1(x)f_1(x) + \phi_2(x)f_2(x) + ... + \phi_{n-1}(x)f_{n-1}(x) = g(x) \quad \quad \MAROON{(2)} & \text{by \MAROON{(1)}}
\end{align*}
Note that \(g(x)\) and \(f_n(x)\) are relatively prime.
(If \(g(x)\) and \(f_n(x)\) are \emph{not} relatively prime, then all \(f_1(x), ..., f_{n-1}(x), f_n(x)\) have the factor \(g(x)\), which implies \(f_1, ..., f_n\) are not relatively prime, again a contradiction.)
Hence (by \LEM{e.1}) there exist polynomials \(p(x)\) and \(q_n(x)\) such that
\[
    p(x)g(x) + q_n(x)f_n(x) = 1. \quad \quad \quad \MAROON{(3)}
\]

So finally, let \(q_i(x) = p(x)\phi_i(x)\) for \(i= 1, 2, ..., n \RED{- 1}\).
Then from \MAROON{(2)} and \MAROON{(3)} we have
\begin{align*}
    & q_1(x)f_1(x) + q_2(x)f_2(x) + ... + q_{n-1}(x)f_{n-1}(x) + q_n(x)f_n(x) \\
    & = p(x)\phi_1(x)f_1(x) + p(x)\phi_2(x)f_2(x) + ... + p(x)\phi_{n\RED{-1}}(x)f_{n\RED{-1}}(x) + q_n(x)f_n(x) \\
    & = p(x) [\phi_1(x)f_1(x) + \phi_2(x)f_2(x) + ... + \phi_{n-1}(x)f_{n-1}(x)] + q_n(x)f_n(x) & \text{of course} \\
    & = p(x) g(x) + q_n(x)f_n(x) & \text{by \MAROON{(2)}} \\
    & = p(x) g(x) + 1 - p(x)g(x) & \text{by \MAROON{(3)}} \\
    & = 1
\end{align*}
which completes the induction argument.
\end{proof}

\begin{example} \label{example e.1}
Let \(f_1(x) = x - 1\), \(f_2(x) = x^2 + x - 2\), \(f_3(x) = x^2 - x - 1\), and \(f_4(x) = x + 1\).
As polynomials \emph{with real} coefficients, \(f_1(x), f_2(x), f_3(x)\) and \(f_4(x)\) are relatively prime.
It is easily verified that the polynomials \(q_1(x) = -x^2, q_2(x) = x, q_3(x) = -3\), and \(q_4(x) = x - 2\) satisfy
\[
    q_1(x)f_1(x) + q_2(x)f_2(x) + q_3(x)f_3(x) + q_4(x)f_4(x) = 1,
\]
and hence these polynomials satisfy the conclusion of \THM{e.2}.
\end{example}

Throughout \CH{5}, \CH{6}, and \CH{7}, we consider linear operators that are \emph{polynomials in a particular operator \(\T\)} and matrices that are \emph{polynomials in a particular matrix \(A\)}.
For these operators and matrices, the following notation is convenient.

\begin{appendix definition} \label{def e.3}
Let
\[
    f(t) = a_0 + a_1 x + ... + a_n x^n
\]
be a polynomial with coefficients from a field \(F\).
If \(\T\) is a linear operator on a vector space \(\V\) \emph{over} \(F\), we define
\[
    f(\T) = a_0 \ITRAN{} + a_1 \T + ... + a_n \T^n.
\]
Similarly, if \(A\) is an \(n \X n\) matrix with entries from \(F\), we define
\[
    f(A) = a_0 I + a_1 A + ... + a_n A^n.
\]
\end{appendix definition}

\begin{example} \label{example e.2}
Let \(\T\) be the linear operator on \(\SET{R}^2\) defined by \(\T(a, b) = (2a + b, a - b)\), and let \(f(x) = x^2 + 2x - 3\).
It is easily checked that \(\T^2(a, b) = (5a + b, a + 2b)\); so
\begin{align*}
    f(\T)(a, b) & = (\T^2 + 2\T - 3\ITRAN{})(a, b) \\
        & = (5a + b, a + 2b) + (4a + 2b, 2a - 2b) - 3(a, b) = (6a + 3b, 3a - 3b).
\end{align*}
Similarly, if
\[
    A = \begin{pmatrix} 2 & 1 \\ 1 & -1 \end{pmatrix},
\]
then
\[
    f(A) = A^2 + 2A - 3I = \begin{pmatrix} 5 & 1 \\ 1 & 2 \end{pmatrix} + 2 \begin{pmatrix} 2 & 1 \\ 1 & -1 \end{pmatrix} - 3 \begin{pmatrix} 1 & 0 \\ 0 & 1 \end{pmatrix} = \begin{pmatrix} 6 & 3 \\ 3 & -3 \end{pmatrix}.
\]
\end{example}

The next three results use this notation.

\begin{appendix theorem} \label{thm e.3}
Let \(f(x)\) be a polynomial with coefficients from a field \(F\), and let \(\T\) be a linear operator on a vector space \(\V\) over \(F\).
Then the following statements are true.
\begin{enumerate}
\item \(f(\T)\) is a liner operator on \(\V\).
\item If \(\beta\) is a \emph{finite} ordered basis for \(\V\) and \(A = [\T]_{\beta}\), then \([f(\T)]_{\beta} = f(A)\).
\end{enumerate}
\end{appendix theorem}

\begin{proof}
Let
\[
    f(x) = a_0 + a_1 x + ... + a_n x^n.
\]
\begin{enumerate}
\item We have \(f(\T) = a_0 \ITRAN{} + a_1 \T + ... + a_n \T^n\) by \DEF{e.3}.
But since \(\T^i\) is linear for nonnegative integer \(i\), (using induction and the fact that the composition of linear operators is linear), \(f(\T)\) is in fact a linear combination of linear functions hence is also linear.

\item
First, by \THM{2.11} and induction we have
\[
     [\T^n]_{\beta} = ([\T]_{\beta})^n. \quad \quad \quad \MAROON{(1)}
\]
And
\begin{align*}
    [f(\T)]_{\beta} & = [a_0 \ITRAN{} + a_1 \T + ... + a_n \T^n]_{\beta} & \text{by \DEF{e.3}} \\
        & = a_0 [\ITRAN{}]_{\beta} + a_1 [\T]_{\beta} + ... + a_n [\T^n]_{\beta} & \text{by \THM{2.8}(a)(b)} \\
        & = a_0 I + a_1 A + ... + a_n A^n & \text{by \MAROON{(1)}} \\
        & = f(A) & \text{by \DEF{e.3}}
\end{align*}
\end{enumerate}
\end{proof}

\begin{appendix theorem} \label{thm e.4}
Let \(\T\) be a linear operator on a vector space \(\V\) over a field \(F\), and let \(A\) be a square matrix with entries from \(F\).
Then, for any polynomials \(f_1(x)\) and \(f_2(x)\) with coefficients from \(F\),
\begin{enumerate}
\item \(f_1(\T)f_2(\T) = f_2(\T)f_1(\T)\).
\item \(f_1(A)f_2(A) = f_2(A)f_1(A)\).
\end{enumerate}
\end{appendix theorem}

\begin{proof}
Since \(f_1(x)f_2(x) = f_2(x)f_1(x)\), it's intuitively true that \DEF{e.3} is also commutative.
Let
\[
    f_1(x) = \sum_{i = 0}^n a_i x^i \quad \text{ and } \quad f_2(x) = \sum_{j = 0}^m b_j x^j.
\]
Then since \(A^0 = I\) and \(\T^0 = \ITRAN{}\) by \ADEF{2.6} and \ADEF{2.7}, by \DEF{e.3} and these facts,
\[
    f_1(\T) = \sum_{i = 0}^n a_i \T^i \quad \text{ and } \quad f_2(\T) = \sum_{j = 0}^m b_j \T^j.
\]
and
\[
    f_1(A) = \sum_{i = 0}^n a_i A^i \quad \text{ and } \quad f_2(\T) = \sum_{j = 0}^m b_j A^j.
\]
Then
\begin{align*}
    f_1(\T) f_2(\T) & = \sum_{i = 0}^n a_i \T^i \sum_{j = 0}^m b_j \T^j \\
                    & = \sum_{i = 0}^n \sum_{j = 0}^m (a_i \T^i) (b_j \T^j) & \text{move ``constant'' into summation} \\
                    & = \sum_{j = 0}^m \sum_{i = 0}^n (a_i \T^i) (b_j \T^j) & \text{change order of finite summation} \\
                    & = \sum_{j = 0}^m \sum_{i = 0}^n (b_j \T^j) (a_i \T^i) & \text{of course by \CH{2}} \\
                    & = \sum_{j = 0}^m (b_j \T^j) \sum_{i = 0}^n (a_i \T^i) & \text{move ``constant'' out of summation} \\
                    & = f_2(\T) f_1(\T),
\end{align*}
and
\begin{align*}
    f_1(A) f_2(A) & = \sum_{i = 0}^n a_i A^i \sum_{j = 0}^m b_j A^j \\
                    & = \sum_{i = 0}^n \sum_{j = 0}^m (a_i A^i) (b_j A^j) & \text{move ``constant'' into summation} \\
                    & = \sum_{j = 0}^m \sum_{i = 0}^n (a_i A^i) (b_j A^j) & \text{change order of finite summation} \\
                    & = \sum_{j = 0}^m \sum_{i = 0}^n (b_j A^j) (a_i A^i) & \text{of course by \CH{2}} \\
                    & = \sum_{j = 0}^m (b_j A^j) \sum_{i = 0}^n (a_i A^i) & \text{move ``constant'' out of summation} \\
                    & = f_2(A) f_1(A),
\end{align*}
as desired.
\end{proof}

\begin{appendix theorem} \label{thm e.5}
Let \(\T\) be a linear operator on a vector space \(\V\) over a field \(F\), and let \(A\) be an \(n \X n\) matrix with entries from \(F\).
If \(f_1(x)\) and \(f_2(x)\) are \emph{relatively prime} polynomials with entries from \(F\), then there exist polynomials \(q_1(x)\) and \(q_2(x)\) with entries from \(F\) such that
\begin{enumerate}
\item \(q_1(\T)f_1(\T) + q_2(\T)f_2(\T) = \ITRAN{}\).
\item \(q_1(A)f_1(A) + q_2(A)f_2(A) = I\).
\end{enumerate}
\end{appendix theorem}

\begin{proof}
By \LEM{e.1}, there exists polynomials \(q_1(x), q_2(x)\) such that
\[
    q_1(x)f_1(x) + q_2(x)f_2(x) = 1.
\]
Then by applying \DEF{e.3}, we have
\[
    q_1(\T)f_1(\T) + q_2(\T)f_2(\T) = 1 \cdot \ITRAN{} = \ITRAN{}
\]
and
\[
    q_1(A)f_1(A) + q_2(A)f_2(A) = 1 \cdot I = I,
\]
as desired.
\end{proof}

\begin{remark} \label{remark e.5}
In \CH{5} and \CH{7}, we are concerned with determining when a linear operator \(\T\) on a finite-dimensional vector space can be \emph{diagonalized} and with finding a \emph{simple (canonical)} representation of \(\T\).
Both of these problems are \emph{affected by the factorization of a certain polynomial determined by \(\T\)}
(the \emph{\CPOLY{}} of \(\T\)).
In this setting, particular types of polynomials play an important role.
\end{remark}

\begin{appendix definition} \label{def e.4}
A polynomial \(f(x)\) With coefficients from a field \(F\) is called \textbf{monic} if its leading coefficient is \(1\).
If \(f(x)\) has positive degree and \emph{cannot be expressed as a product of polynomials} with coefficients from \(F\) such that each \emph{has positive degree}, then \(f(x)\) is called \textbf{irreducible}.
\end{appendix definition}

\begin{note}
這個定義就有點對應到質數在數論的定義。
\end{note}

\begin{remark} \label{remark e.6}
Observe that whether a polynomial is irreducible \textbf{depends on the field \(F\)} from which its coefficients come.
For example, \(f(x) = x^2 + 1\) is irreducible over the field of real numbers, but it is \emph{not} irreducible over the field of complex numbers since \(x^2 + 1 = (x + \iu)(x - \iu)\).
\textbf{Clearly any polynomial of degree \(1\) is irreducible.}
Moreover, for polynomials with coefficients from an \textbf{algebraically closed} field, the polynomials of degree \(1\) are \emph{the only} irreducible polynomials.
The following facts are easily established.
\end{remark}

\begin{appendix theorem} \label{thm e.6}
Let \(\phi(x)\) and \(f(x)\) be polynomials.
If \(\phi(x)\) is irreducible and \(\phi(x)\) does not divide \(f(x)\), then \(\phi(x)\) and \(f(x)\) are relatively prime.
\end{appendix theorem}

\begin{proof}
For the sake of contradiction, suppose \(\phi(x)\) is irreducible and \(\phi(x)\) does not divide \(f(x)\),
\textbf{but} \(\phi(x)\) and \(f(x)\) are \emph{not} relatively prime.
Then by \DEF{e.2} there exists \(g(x)\) with \emph{positive degree} s.t. \(g(x)\) divides both \(\phi(x)\) and \(f(x)\).
That is,
\[
    \phi(x) = g(x)h_1(x) \quad \MAROON{(1)} \text{ and } \quad f(x) = g(x)h_2(x) \quad \MAROON{(2)}
\]
for some polynomials \(h_1(x), h_2(x), g(x)\), where \(g(x)\) has positive degree.
Then we show a contradiction by splitting the cases of the degree of \(h_1(x)\).

If \(h_1(x)\) has positive degree, than by \MAROON{(1)}, \(\phi(x)\) is equal to the product of polynomials having positive degree, hence by \DEF{e.4} is \emph{not} irreducible, a contradiction;

If \(h_1(x)\) is the zero polynomial, then by \MAROON{(1)}, \(\phi(x)\) is also equal to zero polynomial, so by \DEF{e.4} again is \emph{not} irreducible, a contradiction;

Finally, suppose \(h_1(x)\) has degree \(0\) and \(h_1(x) = c\) for some nonzero constant \(c\).
Then from \MAROON{(1)}, we have \(\phi(x) = g(x) \cdot c\), hence \(g(x) = \phi(x) \cdot c^{-1}\) \quad \MAROON{(3)}.
And
\begin{align*}
    f(x) & = g(x)h_2(x) & \text{by \MAROON{(2)}} \\
         & = \phi(x) \cdot c^{-1} \cdot h_2(x), & \text{by \MAROON{(3)}}
\end{align*}
So \(\phi(x)\) \emph{divides} \(f(x)\), which again is a contradiction.

So in all cases, we have a contradiction.
So \(f(x)\) and \(\phi(x)\) must be relatively prime.
\end{proof}

\begin{appendix theorem} \label{thm e.7}
Given any two irreducible \textbf{monic} polynomials, if they are distinct, then they are relatively prime.
\end{appendix theorem}

\begin{note}
We need the polynomial to be monic, since if \(f(x)\) is irreducible, then \(2f(x)\) is also irreducible, and one of them cannot be monic;
\emph{but} \(2f(x) = 2 \cdot f(x)\), where \(2\) is a (degree \(0\)) constant polynomial, so by \DEF{e.1}, \(f(x)\) divides \(2f(x)\).

But this case is trivial and useless since \(f(x)\) and \(2f(x)\) still have the same degree, so we exclude it.
\end{note}

\begin{proof}
Let \(f_1(x)\) and \(f_2(x)\) be two distinct monic irreducible polynomials.
For the sake of contradiction, suppose they are \emph{not} relatively prime.
Then there exists \(g(x), h_1(x), h_2(x)\), where \(g(x)\) has positive degree, such that
\[
    f_1(x) = g(x) h_1(x) \quad \text{ and } \quad f_2(x) = g(x) h_2(x).
\]
Then we split by cases on the degree of \(h_1(x)\) and \(h_2(x)\).
If one of them has positive degree, then \(f_1(x)\) or \(f_2(x)\) is equal to the product of polynomials having positive degree, and is by definition not irreducible, a contradiction.

Else, if one of them is zero polynomial, then \(f_1(x)\) or \(f_2(x)\) is also equal to the zero polynomial and is by definition not irreducible, again a contradiction.

Finally, the remaining case is both \(h_1(x)\) and \(h_2(x)\) are degree-\(0\) polynomial and equal to constant \(c_1, c_2\) respectively, 
So we have
\[
    f_1(x) = c_1 g(x) \quad \text{ and } \quad f_2(x) = c_2 g(x), \quad \quad \MAROON{(1)}
\]
But since \(f_1(x)\) and \(f_2(x)\) are \textbf{monic}, it must be the case that \(c_1 = c_2\), for if \(c_1 \ne c_2\), then from \MAROON{(1)}, the leading coefficient of \(f_1(x)\) [\(f_2(x)\)] is the product of \(c_1\) [\(c_2\)] and the leading coefficient of \(g(x)\), and one of \(f_1(x)\) and \(f_2(x)\) cannot be monic.
But \MAROON{(1)} and \(c_1 = c_2\) imply \(f_1(x) = f_2(x)\), which contradicts that \(f_1(x)\) and \(f_2(x)\) are distinct.

So in all cases we get a contradiction.
Hence \(f_1(x)\) and \(f_2(x)\) must be relatively prime.
\end{proof}

\begin{appendix theorem} \label{thm e.8}
Let \(f(x), g(x)\), and \(\phi(x)\) be polynomials.
If \(\phi(x)\) is irreducible and divides the \emph{product} \(f(x)g(x)\) \BLUE{(1)}, \quad then \(\phi(x)\) divides \(f(x)\) or \(\phi(x)\) divides \(g(x)\).
\end{appendix theorem}

\begin{proof}
Suppose that \(\phi(x)\) does \emph{not} divide \(f(x)\). \BLUE{(2)} \quad
Then it suffices to show \(\phi(x)\) divides \(g(x)\).

First since \(\phi(x)\) is irreducible, with the supposition \BLUE{(2)}, by \THM{e.6}, \(\phi(x)\) and \(f(x)\) are relatively prime, and so by \THM{e.2}, there exist polynomials \(q_1(x)\) and \(q_2(x)\) such that
\[
    1 = q_1(x)\phi(x) + q_2(x)f(x).
\]
Multiplying both sides of this equation by \(g(x)\) yields
\[
    g(x) = q_1(x)\phi(x)g(x) + q_2(x) \RED{f(x)g(x)}. \quad \quad \quad \MAROON{(1)}
\]
Since by supposition \BLUE{(1)}, \(\phi(x)\) divides \(f(x)g(x)\), there is a polynomial \(h(x)\) such that \(f(x)g(x) = \phi(x)h(x)\). \MAROON{(2)} \quad
Thus
\begin{align*}
    g(x) & = q_1(x)\phi(x)g(x) + q_2(x) \RED{f(x)g(x)} & \text{by \MAROON{(1)}} \\
         & = q_1(x)\phi(x)g(x) + q_2(x) \RED{\phi(x)h(x)} & \text{by \MAROON{(2)}} \\
         & = \phi(x) [q_1(x)g(x) + q_2(x)h(x)] & \text{of course}
\end{align*}
So \(\phi(x)\) divides \(g(x)\).
\end{proof}

\begin{appendix corollary} \label{corollary e.8.1}
Let \(\phi(x), \phi_1(x), \phi_2(x), ..., \phi_n(x)\) be irreducible monic polynomials.
If \(\phi(x)\) divides the product \(\phi_1(x)\phi_2(x) ... \phi_n(x)\), then \(\phi(x) = \phi_i(x)\) for some \(i (i = 1, 2, ..., n)\).
\end{appendix corollary}

\begin{note}
In this case we have \(\phi(x) \RED{=} \phi_i(x)\), not just \(\phi(x)\) divides \(\phi_i(x)\), this is because that all \(\phi_i(x)\) are irreducible.
\end{note}

\begin{proof}
We prove the corollary by mathematical induction on \(n\).
For \(n = 1\), we have two irreducible monic polynomials \(\phi(x)\) and \(\phi_1(x)\);
and since \(\phi(x)\) divides \(\phi_1(x)\), they are \emph{not} relatively prime, so by (the contrapositive of) \THM{e.7}, \(\phi(x) = \phi_1(x)\).

Suppose then that for some \(n > 1\), the corollary is true for any irreducible polynomial \(\phi(x)\) and any \(n - 1\) irreducible monic polynomials; now again let \(\phi(x)\) be arbitrary irreducible polynomial and \(\phi_1(x), \phi_2(x), ..., \phi_n(x)\) be any \(n\) irreducible polynomials.
If \(\phi(x)\) divides
\[
    \phi_1(x)\phi_2(x)...\phi_n(x) = [\phi_1(x)\phi_2(x)...\phi_{n-1}(x)]\phi_n(x),
\]
then by \THM{e.8}, \(\phi(x)\) divides the product \(\phi_1(x)\phi_2(x) ... \phi_{n - 1}(x)\) or \(\phi(x)\) divides \(\phi_n(x)\).
In the first case, \(\phi = \phi_i(x)\) for some \(i (i = 1, 2, ..., n - 1)\) by the induction hypothesis;
in the second case, \(\phi(x) = \phi_n(x)\) again by (the contrapositive of) \THM{e.7}.
So the induction is true for \(n\).
\end{proof}

We are now able to establish \emph{the unique factorization theorem}, which is used throughout \CH{5} and \CH{7}.
This result states that every polynomial of positive degree is \emph{uniquely expressible} as a constant times a product of irreducible monic polynomials.

\begin{note}
就對應到數論的質因數分解。
\end{note}

\begin{appendix theorem} [Unique Factorization Theorem for Polynomials] \label{thm e.9}
For any polynomial \(f(x)\) of positive degree, there exist a unique constant \(c\), unique distinct irreducible monic polynomials \(\phi_1(x), \phi_2(x), ..., \phi_l(x)\), and unique positive integers \(n_1, n_2, ..., n_k\) such that
\[
    f(x) = c[\phi_1(x)]^{n_1}[\phi_2(x)]^{n_2}...[\phi_k(x)]^{n_k}.
\]
\end{appendix theorem}

\begin{proof}
We begin by showing the \textbf{existence} of such a factorization using mathematical induction on the degree of \(f(x)\).
If \(f(x)\) is of degree \(1\), then \(f(x) = ax + b\) for some constants \(a\) and \(b\) with \(a \ne 0\).
Setting \(\phi(x) = x + b/a\), we have \(f(x) = a\phi(x)\).
Since \(\phi(x)\) is an irreducible monic polynomial, the result is proved in this case.

Now suppose that the conclusion is true for any polynomial with positive degree less than some integer \(n > 1\), and let \(f(x)\) be a polynomial of degree \(n\).
Then
\[
    f(x) = a_n x^n + ... + a_1 x + a_0
\]
for some constants \(a_i\) with \(a_n \ne 0\).
If \(f(x)\) is irreducible, then
\[
    f(x) = a_n \left( x^n + \frac{a_{n-1}}{a_n} x^{n - 1} + ... + \frac{a_1}{a_n} x +  \frac{a_0}{a_n} \right)
\]
is a representation of \(f(x)\) as a product of \(a_{n}\) and an irreducible \emph{monic} polynomial.
If \(f(x)\) is not irreducible, then \(f(x) = g(x)h(x)\) for some polynomials \(g(x)\) and \(h(x)\), each of \emph{positive} degree \emph{less than} \(n\).
The induction hypothesis guarantees that both \(g(x)\) and \(h(x)\) factor as products of a constant and powers of distinct irreducible monic polynomials.
Consequently \(f(x) = g(x)h(x)\) also factors in this way.
Thus, in either case, \(f(x)\) can be factored as a product of a constant and powers of distinct irreducible monic polynomials.

It remains to establish the \textbf{uniqueness} of such a factorization.
Suppose that
\begin{align*}
    f(x) & = c[\phi_1(x)]^{n_1}[\phi_2(x)]^{n_2}...[\phi_k(x)]^{n_k} \\
         & = d[\psi_1(x)]^{m_1}[\psi_2(x)]^{m_2}...[\psi_r(x)]^{m_r}, \quad \quad \MAROON{(1)}
\end{align*}
where \(c\) and \(d\) are constants,
\(\phi_i(x)\) and \(\psi_j(x)\) are \RED{distinct} (this will be used later) irreducible monic polynomials, respectively,
and \(n_i\) and \(m_j\) are positive integers for \(i = 1, 2, ..., k\) and \(j = 1, 2, ..., r\).
Clearly both \(c\) and \(d\) must be the leading coefficient of \(f(x)\); hence \(c = d\).
Dividing by \(c\), we find that \MAROON{(1)} becomes
\[
    [\phi_1(x)]^{n_1}[\phi_2(x)]^{n_2}...[\phi_k(x)]^{n_k} = [\psi_1(x)]^{m_1}[\psi_2(x)]^{m_2}...[\psi_r(x)]^{m_r}. \quad \quad \MAROON{(2)}
\]
So \(\phi_i(x)\) divides the right side of \MAROON{(2)} for \(i = 1, 2, ..., k\).
Consequently, by \CORO{e.8.1}, each \(\phi_i(x)\) equals some \(\psi_j(x)\), and similarly,
each \(\psi_j(x)\) equals some \(\phi_j(x)\).
We conclude that \(r = k\) and that, by \emph{renumbering} if necessary,
\(\phi_i(x) = \psi_{\RED{i}}(x)\) for \(i = 1, 2, ..., k\). \MAROON{(3)}
So now it suffices to show \(n_i = m_i\) for all \(i = 1, 2, ..., k\).
For the sake of contradiction, suppose that \(n_i \ne m_i\) for some \(i\).
Without loss of generality, we may suppose that \(i = 1\) and \(n_1 > m_1\).
Then by \emph{canceling} \(\phi_1(x)^{m_1}\) from both sides of \MAROON{(2)}, and the fact \MAROON{(3)} that \(\phi_i(x) = \psi_(x)\), we obtain
\begin{align*}
    [\phi_1(x)]^{n_1 - m_1}[\phi_2(x)]^{n_2}...[\phi_k(x)]^{n_k} & = [\phi_1(x)]^{\RED{m_1 - m_1}}[\phi_2(x)]^{m_2}...[\phi_k(x)]^{m_k} \\
    & = [\phi_{\RED{2}}(x)]^{m_2}...[\phi_{\RED{k}}(x)]^{m_k} \quad \quad \MAROON{(4)}
\end{align*}
Since \(n_1 - m_1 > 0\), \(\phi_1(x)\) (of course) divides the left side of \MAROON{(4)} and hence divides the right side also.
So again by \CORO{e.8.1} \(\phi_1(x)\) is equal to one of the irreducible monic polynomial of the right side;
that is \(\phi_1(x) = \phi_i(x)\) for some \(i = \RED{2, ... , k}\).
But this contradicts that \(\phi_1(x), \phi_2(x), ..., \phi_k(x)\) are \RED{distinct}.
Hence the factorizations of \(f(x)\) in \MAROON{(1)} are the same.
\end{proof}

\begin{remark} \label{remark e.7}
It is often useful to regard a polynomial \(f(x) = a_n x^n + ... + a_1 x + a_0\) with coefficients from a field \(F\) \textbf{as a function} \(f: F \to F\).
In this case, the value of \(f\) at \(c \in F\) is \(f(c) = a_n c^n + ... + a_1 c + a_0\).
\emph{Unfortunately}, for \emph{arbitrary fields} there is not a one-to-one correspondence between polynomials and polynomial functions.
For example, if \(f(x) = x^2\) and \(g(x) = x\) are two polynomials \RED{over the field \(Z_2\)} (defined in \EXAMPLE{c.4}),
then \(f(x)\) and \(g(x)\) have different degrees and hence are \RED{not equal as polynomials}.
But (trivially) \(f(a) = g(a)\) for all \(a \in Z_2\), so that \(f\) and \(g\) are \RED{equal polynomial \emph{functions}}.
Our final result shows that \textbf{this anomaly cannot occur over an infinite field}.
\end{remark}

\begin{appendix theorem} \label{thm e.10}
Let \(f(x)\) and \(g(x)\) be polynomials with coefficients from an \emph{infinite} field \(F\).
If \(f(a) = g(a)\) for all \(a \in F\), then the polynomial \emph{functions} \(f(x)\) and \(g(x)\) are equal.
\end{appendix theorem}

\begin{proof}
Suppose that \(f(a) = g(a)\) for all \(a \in F\).
Define \(h(x) = f(x) - g(x)\), and suppose for the sake of contradiction that \(h(x)\) is of degree \(n \ge 1\).
Since \(f(x) - g(x)\) can only be a polynomial with degree at most \(n\), it follows from \CORO{e.1.2} that \(h(x)\) can have at most \(n\) zeroes. \MAROON{(1)}.
But
\[
    h(a) = f(a) - g(a) = 0
\]
for every \(a \in F\), hence \(h(x)\) has infinitely many zeroes, which contradicts \MAROON{(1)}.
Thus the degree of \(h(x)\) \(<= 0\), so \(h(x)\) must be a constant polynomial.
And since \(h(a) = 0\) for each \(a \in F\), it follows that \(h(x)\) is the zero polynomial.
Hence \(f(x) = g(x)\).
\end{proof}
