%  Too long, split exercises into another file.
\exercisesection

\begin{exercise} \label{exercise 1.3.1}
Label the following statements as true or false.
\begin{enumerate}
    \item If \(V\) is a vector space and \(W\) is a subset of \(V\) that is a vector space, then \(W\) is a subspace of \(V\).
    \item The empty set is a subspace of every vector space.
    \item If \(V\) is a vector space other than the zero vector space, then \(V\) contains a subspace \(W\) such that \(W \ne V\).
    \item The intersection of any two \emph{subsets} of \(V\) is a subspace of \(V\).
    \item An \(n \X n\) diagonal matrix can never have more than \(n\) nonzero entries.
    \item The trace of a square matrix is the product of its diagonal entries.
    \item Let \(W\) be the \(xy\)-plane in \(\SET{R}^3\);
          that is, \(W = \{ (a_1, a_2, 0) : a_1, a_2 \in \SET{R}\}\).
          Then \(W = \SET{R}^2\).
\end{enumerate}
\end{exercise}

\begin{proof}\ 

\begin{enumerate}
    \item I think the statement is just not well-defined, because it does not say what \(+\),  \(\cdot\) of \(W\) actually are.
    \item False.
          Every vector space \(V\) needs to satisfy \DEF{1.1} (VS 3), which implies \(V\) contains at least a zero vector, so \(V\) is non-empty.
          Contrapositively, if \(V\) is empty, then \(V\) is not a vector space, hence cannot be a subspace of any vector spaces.
    \item True. Just let \(W = \{ 0 \}\). Since \(V \ne \{ 0 \}\) by supposition, we have \(W \ne V\).
    \item False.
          Counterexample, \(\{ (1, 0) \} \subseteq \SET{R}^2\), \(\{ (0, 1) \} \subseteq \SET{R}^2\), but \(\{ (1, 0) \} \cap \{ (0, 1) \} = \emptyset\), which by part(b) is not a subspace of \(\SET{R}^2\).
    \item True.
          Any diagonal matrix can only have \(n\) entries \(a_{11}, a_{22}, ..., a_{nn}\) to be nonzero.
    \item False. By definition in \EXAMPLE{1.3.4} the trace is the \emph{sum} of the square matrix's diagonal entries.
    \item False. There is a ``type mismatch''.
          \(\SET{R}^2 = \{ (a_1, a_2) : a_1, a_2 \in \SET{R}\}\), which does not equal to \(W\).
\end{enumerate}
\end{proof}

\begin{exercise} \label{exercise 1.3.2}
Skip
\end{exercise}

\begin{exercise} \label{exercise 1.3.3}
Prove that \((aA + bB)^\top = aA^\top + bB^\top\) for any \(A, B \in M_{m \X n}(F)\) and any \(a, b \in F\).
\end{exercise}

\begin{proof}
Given any \(1 \le i \le m\) and \(1 \le j \le n\),
\begin{align*}
    ((aA + bB)^\top)_{ij} & = (aA + bB)_{ji} & \text{by \ADEF{1.6}} \\
                       & = (aA)_{ji} + (bB)_{ji} & \text{by definition in \EXAMPLE{1.2.2}} \\
                       & = aA_{ji} + bB_{ji}, & \text{same as above}
\end{align*}
and
\begin{align*}
    (aA^\top + bB^\top)_{ij} & = (aA^\top)_{ij} + (bB^\top)_{ij} & \text{by def in \EXAMPLE{1.2.2}} \\
                       & = a(A^\top)_{ij} + b(B^\top)_{ij} & \text{same as above} \\
                       & = aA_{ji} + bB_{ji} & \text{by \ADEF{1.6}}
\end{align*}
So by \ADEF{1.3}(6) (the equality of matrix), \((aA + bB)^\top = aA^\top + bB^\top\), as desired.
\end{proof}

\begin{exercise} \label{exercise 1.3.4}
Prove that \((A^\top)^\top = A\) for each \(A \in M_{m \X n}(F)\).
\end{exercise}

\begin{proof}
Given any \(A \in M_{m \X n}(F)\), for all \(1 \le i \le m\) and \(1 \le j \le n\),
\begin{align*}
    ((A^\top)^\top)_{ij} & = (A^\top)_{ji} & \text{by \ADEF{1.6}} \\
                   & = A_{ij}. & \text{again by \ADEF{1.6}}
\end{align*}
So by definition of matrix equality, \((A^\top)^\top = A\).
\end{proof}

\begin{exercise} \label{exercise 1.3.5}
Prove that \(A + A^\top\) is symmetric for any square matrix \(A\).
\end{exercise}

\begin{proof}
By \ADEF{1.6}(2) we need to show \(A + A^\top = (A + A^\top)^\top\).
But
\begin{align*}
    RHS & = (A + A^\top)^\top \\
        & = A^\top + (A^\top)^\top & \text{by \EXEC{1.3.3}} \\
        & = A^\top + A & \text{by \EXEC{1.3.4}} \\
        & = A + A^\top & \text{by \DEF{1.1}(VS 1), commutative, of \(n \X n\) matrices} \\
        & = LHS,
\end{align*}
as desired.
\end{proof}

\begin{exercise} \label{exercise 1.3.6}
Prove that \(\TRACE(aA + bB) = a \TRACE(A) + b \TRACE(B)\) for any \(A, B \in M_{n \X n}(F)\).
\end{exercise}

\begin{proof}
\begin{align*}
    \TRACE(aA + bB) & = \sum_{i = 1}^{n} (aA + bB)_{ii} & \text{by def in \EXAMPLE{1.3.4}} \\
                    & = \sum_{i = 1}^{n} aA_{ii} + bB_{ii} & \text{by def of \(+\), \(\cdot\) of matrix} \\
                    & = \sum_{i = 1}^{n} aA_{ii} + \sum_{i = 1}^{n} bB_{ii} & \text{finite summation is distributive} \\
                    & = a \sum_{i = 1}^{n} A_{ii} + b \sum_{i = 1}^{n} B_{ii} & \text{extract constant out of summation} \\
                    & = a \TRACE(A) + b \TRACE(B) & \text{by def in \EXAMPLE{1.3.4}}
\end{align*}
\end{proof}

\begin{exercise} \label{exercise 1.3.7}
Prove that diagonal matrices are symmetric matrices.
\end{exercise}

\begin{proof}
Given any diagonal matrices \(D\), we need to show \(D^\top = D\).
Now let \(1 \le i \le n\) and \(1 \le j \le n\), where \(n\) is the number of rows and columns of \(D\).
There are two cases:
\begin{itemize}
    \item[\(i = j\)]: Then
        \begin{align*}
            (D^\top)_{ij} & = D_{ji} & \text{by \ADEF{1.6}(1)} \\
                       & = D_{ij} & \text{since \(i = j\)}
        \end{align*}
    \item[\(i \ne j\)]: Then
        \begin{align*}
            (D^\top)_{ij} & = D_{ji} & \text{by \ADEF{1.6}(1)} \\
                       & = 0 & \text{since \(i \ne j\)}
        \end{align*}
        Moreover since \(i \ne j\), we also have \(D_{ij} = 0\).
        So in this case we also have \((D^\top)_{ij} = D_{ij}\).
\end{itemize}
So in all cases we have \((D^\top)_{ij} = D_{ij}\), hence \(D^\top = D\), as desired.
\end{proof}

\begin{exercise} \label{exercise 1.3.8}
Determine whether the following sets are subspaces of \(\SET{R}^3\) under the operations of addition and scalar multiplication defined on \(\SET{R}^3\).
Justify your answers.
\begin{enumerate}
    \item \(W_1 = \{(a_1, a_2, a_3) \in \SET{R}^3 : a_1 = 3 a_2 \land a_3 = -a_2\}\)
    \item \(W_2 = \{(a_1, a_2, a_3) \in \SET{R}^3: a_1 = a_3 + 2\}\)
    \item \(W_3 = \{(a_1, a_2, a_3) \in \SET{R}^3 : 2a_1 - 7a_2 + a_3 = 0\}\)
    \item \(W_4 = \{(a_1, a_2, a_3) \in \SET{R}^3 : a_1 - 4a_2 - a_3 = 0\}\)
    \item \(W_5 = \{(a_1, a_2, a_3) \in \SET{R}^3 : a_1 + 2a_2 - 3a_3 = 1\}\)
    \item \(W_6 = \{(a_1, a_2, a_3) \in \SET{R}^3 : 5a_1^2 - 3a_2^2 + 6a_3^2 = 0\}\)
\end{enumerate}
\end{exercise}

\begin{proof}
I try not to write calculation questions because using \LaTeX\ for it is painful.

\begin{enumerate}
    \item It's an intersection of the two plane passing the origin, hence a subspace.
    \item \((0, 0, 0)\) does not belong to \(W_2\) since \(a_1\) can never be equal to \(a_3\).
          By \THM{1.3}(a), \(W_2\) is not a vector space.
    \item It's a plane passing the origin, hence a subspace.
    \item Similarly as part(c), it's a plane passing the origin, hence a subspace.
    \item It's a plane \emph{not} passing the origin, hence \((0, 0, 0)\) does not belong to \(W_5\), so by \THM{1.3}(a), \(W_5\) is not a vector space.
    \item The equation does not form a point/line/plane/space passing the origin, so the \(W_6\) is not a vector space.
\end{enumerate}
\end{proof}

\begin{exercise} \label{exercise 1.3.9}
Let \(W_1, W_3\), and \(W_4\) be as in \EXEC{1.3.8}.
Describe \(W_1 \cap W_3\), \(W_1 \cap W_4\), and \(W_3 \cap W_4\), and observe that each is a subspace of \(\SET{R}^3\).
\end{exercise}

\begin{proof}
Since \(W_1, W_3\), and \(W_4\) in \EXEC{1.3.8} are all subspaces, by \THM{1.4} their intersections are also subspaces.
And their intersections geometrically are just point/line/plane passing the origin.
\end{proof}

\begin{exercise} \label{exercise 1.3.10}
Prove that \(W_1 = \{ (a_l, a_2, ..., a_n) \in F^n: a_1 + a_2 + ... + a_n = 0 \} \) is a subspace of \(F^n\), but \(W_2 = \{ (a_l, a_2, ..., a_n) \in F^n : a_1 + a_2 + ... + a_n = 1\} \)
is not.
\end{exercise}

\begin{proof}
For \(W_2\) since the zero vector \((a_1, a_2, ..., a_n)\) where \(a_i = 0\) for all \(1 \le i \le n\) does not belong to \(W_2\) because \(\sum_{i = 1}^{n} a_i = 0 \ne 1\), by \THM{1.3}(a), \(W_2\) is not a vector space, hence is not a subspace of \(F^n\).

For \(W_1\), the zero vector \((a_1, a_2, ..., a_n)\) where \(a_i = 0\) for all \(1 \le i \le n\) belongs to \(W_1\) since \(\sum_{i = 1}^{n} a_i = 0\).
Also, given any \((a_1, a_2, ..., a_n)\) and \((b_1, b_2, ..., b_n)\) in \(W_1\), we have
\begin{align*}
             & \sum_{i = 1}^{n} a_i = 0 \land \sum_{i = 1}^{n} b_i = 0 & \text{by def of \(W_1\)} \\
    \implies & \sum_{i = 1}^{n} a_i + \sum_{i = 1}^{n} b_i = 0 + 0 = 0 & \text{of course} \\
    \implies & \sum_{i = 1}^{n} a_i + b_i = 0 & \text{combine finite summation} \\
    \implies & (a_1 + b_1, a_2 + b_2, ..., a_n + b_n) \in W_1, & \text{by def of \(W_1\)}
\end{align*}
hence the addition is closed under \(W_1\).
And
\begin{align*}
             & c \sum_{i = 1}^{n} a_i = c \X 0 = 0 \\
    \implies & \sum_{i = 1}^{n} c a_i = 0 & \text{put constant into summation} \\
    \implies & (c a_1, c a_2, ..., c a_n) \in W_1, & \text{by def of \(W_1\)}
\end{align*}
hence the scalar multiplication is closed under \(W_1\).
So by \THM{1.3}, \(W_1\) is a subspace of \(F^n\).
\end{proof}

\begin{exercise} \label{exercise 1.3.11}
Is the set \(W = \{f(x) \in \mathcal{P}(F): f(x) = 0 \lor f(x) \text{ has degree } n\}\) a subspace of \(\mathcal{P}(F)\) if \(n \ge 1\)?
Justify your answer.
\end{exercise}

\begin{proof}
No, since the sum of two degree \(n\) polynomial may have degree both not equal to \(-1\) and \(n\), so the result may not belong to \(W\).
\end{proof}

\begin{exercise} \label{exercise 1.3.12}
Prove that the set of \(m \X n\) upper triangular matrices is a subspace of \(M_{m \X n}(F)\).
\end{exercise}

\begin{proof}
It is trivial to verify that
\begin{enumerate}
    \item Zero matrix is an upper triangular matrix.
    \item Sum of two upper triangular matrices is also an upper triangular matrix.
    \item Scalar multiplication of an upper triangular matrix is also an upper triangular matrix.
\end{enumerate}
Hence by \THM{1.3}, the set of \(m \X n\) upper triangular matrices is a subspace of \(M_{m \X n}(F)\).
\end{proof}

\begin{exercise} \label{exercise 1.3.13}
Let \(S\) be a nonempty set and \(F\) a field.
Prove that for any \(s_0 \in S\), \(\{ f \in \mathcal{F}(S, F) : f(s_0) = 0\}\), is a subspace of \(\mathcal{F}(S, F)\).
\end{exercise}

\begin{proof}
This is the general case of \EXEC{1.2.20}, and the proof is similar.
\end{proof}

\begin{exercise} \label{exercise 1.3.14}
Let \(S\) be a nonempty set and \(F\) a field.
Let \(\mathcal{C}(S, F)\) denote the set of all functions \(f \in \mathcal{F}(S, F)\) such that \(f(s) = 0\) \emph{for all but a finite number of elements} of \(S\).
Prove that \(\mathcal{C}(S, F)\) is a subspace of \(\mathcal{F}(S, F)\).
\end{exercise}

\begin{proof}
First the zero function is in \(\mathcal{C}(S, F)\) since there is \emph{no elements} (or ``zero'' elements, finite elements) such that the output of the zero function is nonzero.

Now suppose \(f \in \mathcal{C}(S, F)\).
That is, \(f\) is a function s.t. \(f(s) \ne 0\) for all elements \(s\) in a \emph{finite} set \(\{ s_1, s_2, ..., s_n \} \subseteq S\).
Then if \(c \in F\) and \(c \ne 0\), \((cf)(s) = cf(s) \ne 0\) for all \(s \in \{ s_1, s_2, ..., s_n \} \).
(for \(c = 0\) \(cf\), is the zero function and fallbacks to the previous case).
By definition of \(\mathcal{C}(S, F)\), \(cf \in \mathcal{C}(S, F)\), so the scalar multiplication is closed  \(\mathcal{C}(S, F)\).

Now suppose \(f, g \in \mathcal{C}(S, F)\).
That is, \(f\) is a function s.t. \(f(s) \ne 0\) for all elements \(s\) in a \emph{finite} set \(S_n = \{ s_1, s_2, ..., s_n \} \subseteq S \) and \(g\) is a function s.t. \(g(t) \ne 0\) for all elements \(t\) in a \emph{finite} set \(T_m = \{ t_1, t_2, ..., t_m \} \subseteq S\).
Then first \(S_n \cup T_m\) is still finite set.
Also, only elements in \(u \in S_n \cup T_m\) may cause \((f + g)(u) = f(u) + g(u)\) to be \emph{not} equal to \(0\).
So there are still finite elements which make the output of \(f + g\) nonzero.
Hence by definition of \(\mathcal{C}(S, F)\), \(f + g \in \mathcal{C}(S, F)\), so the addition is closed under \(\mathcal{C}(S, F)\).

So by \THM{1.3}, \(\mathcal{C}(S, F)\) is a subspace of \(\mathcal{F}(S, F)\).
\end{proof}

\begin{note}
The vector space in the previous exercise is somewhat weird to me, because except for the zero function, each vector in the vector space is like ``zero function but with finite holes on \(x\)-axis''(and the corresponding function values to these holes are nonzero).
\end{note}

\begin{exercise} \label{exercise 1.3.15}
Is the set of all differentiable real-valued functions defined on \(\SET{R}\) a subspace of \(\mathcal{C}(\SET{R})\)?
Justify your answer.
\end{exercise}

\begin{proof}
By definition in \EXAMPLE{1.3.2}, \(\mathcal{C}(\SET{R})\) is the set of all continuous functions.
And by Calculus, differentiability implies continuity, so the set of all differentiable real-valued functions defined on \(\SET{R}\) is a subset of \(\mathcal{C}(\SET{R})\).
And we have shown in \EXEC{1.2.10} that the former is indeed a vector space, hence the former is a subspace of the latter.
\end{proof}

\begin{exercise} \label{exercise 1.3.16}
Let \(\mathcal{C}^n(\SET{R})\) denote the set of all real-valued functions defined on the real line that have \emph{a continuous \(n\)th} derivative.
Prove that \(\mathcal{C}^n(\SET{R})\) is a subspace of \(\FRR\).
\end{exercise}

\begin{proof}
Again the zero function belongs to the set \(\mathcal{C}^n(\SET{R})\) since the \(n\)th derivative of zero function is zero function, which is continuous.

Now given \(f, g \in \mathcal{C}^n(\SET{R})\), \(f^n, g^n\) is continuous by the property of the set.
Furthermore, by Calculus, \((f + g)^n = f^n + g^n\), which is the sum of two continuous functions, hence is continuous.
So by definition of \(\mathcal{C}^n(\SET{R})\), \(f + g \in \mathcal{C}^n(\SET{R})\), so addition is closed under \(\mathcal{C}^n(\SET{R})\).

Similarly, given \(c \in \SET{R}\), by Calculus, \((cf)^n = cf^n\), which is a multiplication of a scalar and a continuous function, hence is continuous.
So by definition of \(\mathcal{C}^n(\SET{R})\), \(cf \in \mathcal{C}^n(\SET{R})\), so scalar multiplication is closed under \(\mathcal{C}^n(\SET{R})\).

So by \THM{1.3}, \(\mathcal{C}^n(\SET{R})\) is a subspace of \(\FRR\).
\end{proof}

\begin{exercise} \label{exercise 1.3.17}
Prove that a \emph{subset} \(W\) of a vector space \(V\) is a \emph{subspace} of \(V\) if and only if \(W \ne \emptyset\), and, whenever \(a \in F\) and \(x, y \in W\), then \(ax \in W\) and \(x + y \in W\).
\end{exercise}

\begin{proof}
First suppose a subset \(W\) of \(V\) is a subspace of \(V\).
Then by \DEF{1.1}(VS 3), \(W\) contains zero vector and so is not empty.
Also, since \(W\) is a subspace, the addition and scalar multiplication are closed under \(W\), as desired.

Now suppose \(W \ne \emptyset\) and the addition and scalar multiplication are closed under \(W\).
Then in particular, we can pick a vector \(w \in W\) since \(W \ne \emptyset\).
And of course \(w \in V\). \MAROON{(1)}.
And by \MAROON{(1)} and \THM{1.2}(a)(of the \emph{vector space} \(V\)), given \(F\)'s additive identity \(\OF\), \(\OF w = \OV\). \MAROON{(2)}.
And since scalar multiplication is closed under \(W\), \(\OF w \in W\), that is, by \MAROON{(2)}, \(\OV \in W\).
So all conditions of \THM{1.3} are satisfied, hence \(W\) is a subspace of \(V\).
\end{proof}

\begin{exercise} \label{exercise 1.3.18}
Prove that a subset \(W\) of a vector space \(V\) is a subspace of \(V\) if and only if \(\OV \in W\) and \(a x + y \in W\) whenever \(a \in F\) and \(x, y \in W\).
\end{exercise}

\begin{proof}
The forward direction is trivial by \DEF{1.1} (VS 3) and closure of addition and scalar multiplication.

Now suppose \(\OV \in W\) and \(a x + y \in W\) whenever \(a \in F\) and \(x, y \in W\).
Then given arbitrary \(x, y \in W\), given \(\IF \in F\), \(\IF x\ +\ y \in W\).
But by (VS 5) of \(V\), \(\IF x = x\), so we have \(\IF x\ +\ y = x + y \in W\), hence addition is closed under \(W\).
Also, given arbitrary \(x \in W\) and \(a \in F\), since \(\OV \in W\), we have \(ax + \OV \in W\).
But by (VS 3) of \(V\), \(ax + \OV = ax\), so we have \(ax + \OV = ax \in W\), hence scalar multiplication is closed under \(W\).
So all conditions of \THM{1.3} are satisfied, hen \(W\) is a subspace of \(V\).
\end{proof}

\begin{exercise} \label{exercise 1.3.19}
Let \(W_1\) and \(W_2\) be subspaces of a vector space \(V\).
Prove that \(W_1 \cup W_2\) is a subspace of \(V\) if and only if \(W_1 \subseteq W_2\) or \(W_2 \subseteq W_1\).
\end{exercise}

\begin{proof}
We first prove the backward direction since it's trivial.
Suppose \(W_1 \subseteq W_2\) or \(W_2 \subseteq W_1\).
Then if \(W_1 \subseteq W_2\), \(W_1 \cup W_2 = W_2\), which is a subspace of \(V\).
If \(W_2 \subseteq W_1\), \(W_1 \cup W_2 = W_1\), which is a subspace of \(V\).
In all cases, \(W_1 \cup W_2\) is a subspace of \(V\).

Now suppose \(W_1 \cup W_2\) is a subspace of \(V\) \MAROON{(1)}.
For the sake of contradiction suppose \(W_1 \subsetneq W_2\) and \(W_2 \subsetneq W_1\) \MAROON{(2)}.

Then by definition of \(\subsetneq\), we can find \(u\) s.t. \(u \in W_1 \land u \not \in W_2\) \MAROON{(3)} and \(v\) s.t. \(v \in W_2 \land u \not \in W_1\) \MAROON{(4)}.
But in particular, both \(u, v\) are still belong to \(W_1 \cup W_2\).
And by \MAROON{(1)}, \(u + v \in W_1 \cup W_2\) by closed addition.
So in particular \(u + v \in W_1\) or \(u + v \in W_2\).
So there are two cases:
\begin{itemize}
    \item If \(u + v \in W_1\), then by \MAROON{(3)}, \(u \in W_1\), hence by closed scalar multiplication, \(-u \in W_1\), and we have \((u + v) + (-u) \in W_1\), that is, \(v \in W_1\), which contradicts \MAROON{(4)}.
    \item If \(u + v \in W_2\), then by \MAROON{(4)}, \(v \in W_2\), hence by closed scalar multiplication, \(-v \in W_2\), and we have \((u + v) + (-v) \in W_2\), that is, \(u \in W_2\), which contradicts \MAROON{(3)}.
\end{itemize}
So in all cases we get a contradiction.
So \MAROON{(2)} is false, hence \(W_1 \subseteq W_2\) or \(W_2 \subseteq W_1\), as desired.
\end{proof}

\begin{exercise} \label{exercise 1.3.20}
Prove that if \(W\) is a subspace of a vector space \(V\) and \(w_1, w_2, ..., w_n\) are in \(W\), then \(a_1 w_1 + a_2 w_2 + ... + a_n w_n \in W\) for any scalars \(a_1, a_2, ... a_n\).
\end{exercise}

\begin{proof}
We prove it by induction on the number of addition's operands, with base case \(= 2\).
For the base case it is of course true by the closed addition and scalar multiplication.

Now suppose given integer \(n \ge 2\), \(a_1 w_1 + a_2 w_2 + ... + a_n w_n \in W\) for any \(n\) scalars \(a_1, a_2, ... a_n\) and \(n\) vectors \(w_1, w_2, ..., w_n \in W\).
We have to show \(a_1 w_1 + a_2 w_2 + ... + a_n w_n + a_{n + 1} w_{n + 1} \in W\) for any \(n + 1\) scalars \(a_1, a_2, ... a_n, a_{n + 1}\),
and any \(n + 1\) vectors \(w_1, w_2, ..., w_n, w_{n + 1} \in W\).
By the induction hypothesis, we have \(v = a_1 w_1 + a_2 w_2 + ... + a_n w_n \in W\).
And again by closed addition and scalar multiplication, \(v + a_{n + 1} w_{n + 1} \in W\);
that is, \(a_1 w_1 + a_2 w_2 + ... + a_n w_n + a_{n + 1} w_{n + 1} \in W\), as desired.
So this closes the induction.
\end{proof}

\begin{exercise} \label{exercise 1.3.21}
Let \(V\) denote the vector space of sequences in \(\SET{R}\), as defined in \EXAMPLE{1.2.5}.
Show that the set of \emph{convergent sequences} \((a_n)\) (that is, those for which \(\lim_{n \toINF} a_n\) exists) is a subspace of \(V\).
\end{exercise}

\begin{proof}
First, by Calculus, the sequence of zeros is convergent hence belongs to the set of convergent sequences.
Also, by limit law of sequences, given two convergent sequences \((a_n)\) and \((b_n)\) and scalar \(c \in \SET{R}\),
\begin{align*}
    \lim_{n \toINF} a_n + b_n & = \lim_{n \toINF} a_n + \lim_{n \toINF} b_n \\
    \lim_{n \toINF} c a_n & = c \lim_{n \toINF} a_n,
\end{align*}
so both \((a_n + b_n)\) and \((c a_n)\) converge and hence belong to the set of convergent sequences, so addition and scalar multiplication is closed under the set of convergent sequences.
So by \THM{1.3}, the set of convergent sequences is a subspace of \(V\).
\end{proof}

\begin{exercise} \label{exercise 1.3.22}
Let \(F_1\) and \(F_2\) be fields.
A function \(g \in \mathcal{F}(F_1, F_2)\) is called an \textbf{even function} if \(g(-t) = g(t)\) for each \(t \in F_1\)
and is called an \textbf{odd function} if \(g(-t) = -g(t)\) for each \(t \in F_1\).
Prove that the set of all even functions in \(\mathcal{F}(F_1, F_2)\) and the set of all odd functions in \(\mathcal{F}(F_1, F_2)\) are subspaces of \(\mathcal{F}(F_1, F_2)\).
\end{exercise}

\begin{proof}
For the case of even function, this is the generalization of \EXEC{1.2.12}, and the proof is similar,
because \EXEC{1.2.12} is a case of real line function, and the structure of the proof only used the field property of the real line.

The case of odd function is also similar;
note that the zero function by definition is also an odd function.
And
\begin{align*}
    (f + g)(-t) & = f(-t) + g(-t) & \text{by definition in \EXAMPLE{1.2.3}} \\
                & = -f(t) + g(-t) & \text{since \(f\) is odd} \\
                & = -f(t) + -g(t) & \text{since \(g\) is odd} \\
                & = -(f(t) + g(t)) & \text{by field algebra} \\
                & = -((f + g)(t)) & \text{by definition in \EXAMPLE{1.2.3}}
\end{align*}
and
\begin{align*}
    (cf)(-t) & = cf(-t) & \text{by definition in \EXAMPLE{1.2.3}} \\
                & = c(-f(t)) & \text{since \(f\) is odd} \\
                & = -(cf(t)) & \text{by field algebra} \\
                & = -((cf)(t)) & \text{by definition in \EXAMPLE{1.2.3}}
\end{align*}
So these operations are closed, so by \THM{1.3}, the case of odd function is a subspace, so we are done.
\end{proof}

The following definitions are used in \EXEC{1.3.23} to \EXEC{1.3.30}.

\begin{additional definition} \label{adef 1.8}
If \(S_1\) and \(S_2\) are \emph{nonempty} \emph{subsets} (not necessarily subspaces) of a vector space \(V\), then the \textbf{sum} of \(S_1\) and \(S_2\), denoted \(S_1 + S_2\), is the set \( \{ x + y : x \in S_1 \land y \in S_2 \} \).
\end{additional definition}

\begin{additional definition} \label{adef 1.9}
A vector space \(V\) is called the \textbf{direct sum} of \(W_1\) and \(W_2\) if \(W_1\) and \(W_2\) are \emph{subspaces} of \(V\) such that \(W_1 \cap W_2 = \{\OV\}\) and \(W_1 + W_2 = V\).
We denote that \(V\) is the direct sum of \(W_1\) and \(W_2\) by writing \(V = W_1 \oplus W_2\).
\end{additional definition}

\begin{note}
A good introduction videos for these definitions:
\begin{itemize}
    \item \href{https://youtu.be/Rcj1-E3SAhs?t=901}{video 1}
    \item \href{https://www.youtube.com/watch?v=qs240Jhl6Rs}{video 2}
\end{itemize}
\end{note}

\begin{exercise} \label{exercise 1.3.23}
Let \(W_1\) and \(W_2\) be \emph{subspaces} of a vector space \(V\).
\begin{enumerate}
    \item Prove that \(W_1 + W_2\) is a subspace of \(V\) that contains both \(W_1\) and \(W_2\).
    \item Prove that any \emph{subspace} of \(V\) that contains both \(W_1\) and \(W_2\) must also contain \(W_1 + W_2\). That is, \(W_1 + W_2\) is the ``smallest'' subspace that contains \(W_1 \cup W_2\).
\end{enumerate}
\end{exercise}

\begin{proof}\ 

For part(a), we first prove that given subspaces \(W_1, W_2 \subseteq V\), \(W_1 + W_2\) is also a subspace.
First, since \(\OV = \BLUE{\OV} + \GREEN{\OV}\) by \DEF{1.1}(VS 3);
and \(\BLUE{\OV} \in W_1\) and \(\GREEN{\OV} \in W_2\) (since they are subspaces!), by definition of \(W_1 + W_2\), \(\OV = \BLUE{\OV} + \GREEN{\OV} \in W_1 + W_2\).

Now suppose \(u, v \in W_1 + W_2\).
Then by definition of \(W_1 + W_2\), \(u = u_1 + u_2\) where \(u_1 \in W_1, u_2 \in W_2\) and \(v = v_1 + v_2\) where \(v_1 \in W_1, v_2 \in W_2\).
And
\begin{align*}
    u + v & = (u_1 + u_2) + (v_1 + v_2) \\
          & = (u_1 + v_1) + (u_2 + v_2), & \text{by \DEF{1.1}(VS 1), (VS 2) of \(V\)}
\end{align*}
But since \(u_1, v_1 \in W_1\) and \(W_1\) is a subspace, we have \(u_1 + v_1 \in W_1\).
Similarly we have \(u_2 + v_2 \in W_2\).
And again by definition of \(W_1 + W_2\), we have \((u_1 + v_1) + (u_2 + v_2) \in W_1 + W_2\), that is, \(u + v \in W_1 + W_2\), hence the addition is closed under \(W_1 + W_2\).

Now given scalar \(c \in F\),
\begin{align*}
    c u & = c(u_1 + u_2) \\
        & = c u_1 + c u_2 & \text{by (VS 7)}
\end{align*}
But since \(u_1 \in W_1\) and \(W_1\) is a subspace, we have \(c u_1 \in W_1\).
Similarly we have \(c u_2 \in W_2\).
And again by definition of \(W_1 + W_2\), we have \(c u_1 + c u_2 \in W_1 + W_2\), that is, \(c u \in W_1 + W_2\), hence the scalar multiplication is closed under \(W_1 + W_2\).

So by \THM{1.3}, \(W_1 + W_2\) is a subspace of \(V\).

Now we show \(W_1 \cup W_2 \subseteq W_1 + W_2\).
So let arbitrary \(v \in W_1 \cup W_2\).
If \(v \in W_1\), then since by (VS 3) \(v = v + \OV\) and \(\OV \in W_2\) (sine \(W_2\) is a subspace!), by definition of \(W_1 + W_2\), \(v + \OV \in W_1 + W_2\), that is, \(v \in W_1 + W_2\).
If \(v \in W_2\), then since by (VS 3) (and (VS 1)) \(v = \OV + v\) and \(\OV \in W_1\) (since \(W_1\) is a subspace!), by definition of \(W_1 + W_2\), \(\OV + v \in W_1 + W_2\), that is, \(v \in W_1 + W_2\).
So in all cases, \(v \in W_1 + W_2\).
Since \(v\) is arbitrary, we have \(W_1 \cup W_2 \subseteq W_1 + W_2\).

Now we proved part(b).
Suppose \(W\) is a subspace of \(V\) and contains \(W_1 \cup W_2\).
We have to show \(W_1 + W_2 \subseteq W\).
So let arbitrary \(v \in W_1 + W_2\), we have to show \(v \in W\).
Then by definition of \(W_1 + W_2\), \(v = v_1 + v_2\) where \(v_1 \in W_1\) and \(v_2 \in W_2\).
Since \(W_1 \cup W_2 \subseteq W\), we have \(v_1 \in W\) and \(v_2 \in W\).
And since \(W\) is a subspace, we have \(v_1 + v_2 \in W\) by closed addition;
that is, \(v \in W\).
Since \(v\) is arbitrary, we have \(W_1 + W_2 \subseteq W\).
\end{proof}

\begin{exercise} \label{exercise 1.3.24}
Show that \(F^n\) is the direct sum of the subspaces
\[
    W_1 = \{(a_1, a_2, ..., a_n) \in F^n : a_n = 0 \}
\]
(only the last component must equal to \(0\)) and
\[
    W_2 = \{(a_1, a_2, ..., a_n) \in F^n : a_1 = a_2 = ... = a_{n - 1} = 0 \}
\]
(only the last component can be nonzero).
\end{exercise}

\begin{proof}
First it's clear that \(W_1 \cap W_2 = \{ \OV \}\).
And by \ADEF{1.9} we also need to show \(F^n = W_1 + W_2\).
That is, \(F^n \subseteq W_1 + W_2 \land W_1 + W_2 \subseteq F^n\).
So suppose \((a_1, a_2, ..., a_n) \in F^n\).
Then \((a_1, a_2, ..., a_n) = (a_1, a_2, ..., a_{n - 1}, 0) + (0, 0, ..., a_n)\), so by definition of \(W_1 + W_2\), \((a_1, a_2, ..., a_n) \in W_1 + W_2\).
So we have \(F^n \subseteq W_1 + W_2\).

The other direction is automatically true, since \(W_1 + W_2\) is a subspace of \(F^n\) (by \EXEC{1.3.23}(a)), hence a subset of \(F^n\).
\end{proof}

\begin{exercise} \label{exercise 1.3.25}
Let \(W_1\) denote the set of all polynomials \(f(x)\) in \(\mathcal{P}(F)\) such that in the representation
\[
    f(x) = a_n x^n + a_{n - 1} x^{n - 1} + ... + a_1 x + a_0,
\]
we have \(a_i = 0\) whenever \(i\) is \emph{even}.
Likewise let \(W_2\) denote the set of all polynomials \(g(x)\) in \(\mathcal{P}(F)\) such that in the representation
\[
    g(x) = b_m x_m + b_{m - 1} x_{m - 1} + ... + b_1 x + b_0,
\]
we have \(b_i = 0\) whenever \(i\) is \emph{odd}.
Prove that \(\mathcal{P}(F) = W_1 \oplus W_2\).
\end{exercise}

\begin{proof}
First, it's trivial that zero polynomial is the only polynomial satisfying the condition of \(W_1\) and \(W_2\) \MAROON{(1)},
and given any polynomial we can split it into two polynomial, one having only terms of odd degrees and the other have only terms of even degree;
and that implies \(\mathcal{P}(F) \subseteq W_1 + W_2\).
And like previous exercise, \(W_1 + W_2 \subseteq \mathcal{P}(F)\) is automatically true, so we have \(\mathcal{P}(F) = W_1 + W_2\) \MAROON{(2)}.
So by \ADEF{1.9} and \MAROON{(1)(2)}, \(\mathcal{P}(F) = W_1 \oplus W_2\).
\end{proof}

\begin{exercise} \label{exercise 1.3.26}
Skip, it's natural.
\end{exercise}

\begin{exercise} \label{exercise 1.3.27}
Skip, it's natural.
\end{exercise}

\begin{exercise} \label{exercise 1.3.28}
A matrix \(M\) is called \textbf{skew-symmetric} if \(M^\top = -M\).
Clearly, a skew-symmetric matrix is square.
Let \(F\) be a field.
Prove that the set \(W_1\) of all skew-symmetric \(n \X n\) matrices with entries from \(F\) is a subspace of \(M_{n \X n}(F)\).
Now assume that \(F\) is not of characteristic two (see textbook page 549), and let \(W_2\) be the subspace(it is a subspace by \ATHM{1.1}) of \(M_{n \X n}(F)\) consisting of all symmetric \(n \X n\) matrices.
Prove that \(M_{n \X n}(F) = W_1 \oplus W_2\).
\end{exercise}

\begin{proof}
We first show \(W_1\) is a subspace.
Clearly zero matrix \(O_{n \X n}\) belongs to \(W_1\) since \(O^\top_{n \X n} = O_{n \X n}\) \MAROON{(1)} and \(O_{n \X n} = -O_{n \X n}\).
Also given \(M_1, M_2 \in W_1\),
\begin{align*}
    (M_1 + M_2)^\top & = M_1^\top + M_2^\top & \text{by \ATHM{1.2}(1)} \\
                  & = -M_1 + -M_2 & \text{by def of skew-symmetric} \\
                  & = -(M_1 + M_2), & \text{by \(W_1\)'s (VS 7)}
\end{align*}
So addition is closed under \(W_1\).
And given \(c \in F\),
\begin{align*}
    (cM_1)^\top & = (cM_1 + O_{n \X n})^\top & \text{by (VS 3) of \(M_{n \X n}(F)\)} \\
             & = cM_1^\top + O_{n \X n}^\top & \text{by \ATHM{1.2}(1)} \\
             & = cM_1^\top + O_{n \X n} & \text{by \MAROON{(1)}} \\
             & = cM_1^\top & \text{by \(M_{n \X n}(F)\)'s (VS 3)} \\
             & = c(-M_1) & \text{by def of skew-symmetric} \\
             & = -(cM_1) & \text{by \(M_{n \X n}(F)\)'s (VS 6) and commutative property of \(F\)}
\end{align*}
So scalar multiplication is closed under \(W_1\).
Hence by \THM{1.3}, \(W_1\) is a subspace of \(M_{n \X n}(F)\).

Now we show that \(M_{n \X n}(F) = W_1 \oplus W_2\).
First we show that \(W_1 \cap W_2 = \{ O_{n \X n} \}\);
that is, if \(M \in W_1 \cap W_2\), then \(M = O_{n \X n}\).
So suppose \(M \in W_1 \cap W_2\).
By definition of \(W_1\) and \(W_2\), \(M\) is skew-symmetric and symmetric, so we have \(M^\top = -M\) and \(M^\top = M\).
That is, \(M = -M\).
But that would be true only if \(M = O_{n \X n}\)(otherwise we trivially get some contradiction).
So \(M = O_{n \X n}\).

Now we show \(M_{n \X n}(F) \subseteq W_1 + W_2\) (again, the other direction is automatically true since \(W_1 + W_2\) is a subspace of \(M_{n \X n}(F)\)).
So let \(A\) be arbitrary matrix in \(M_{n \X n}(F)\).
(\RED{Now the tricky part.})
In particular, there exists \(B\) in \(M_{n \X n}(F)\) s.t. \(B = \frac1{2} A\) \MAROON{(2)}.
Then
\begin{align*}
    A & = \frac1{2} A + \frac1{2} A & \text{of course} \\
      & = B + B & \text{by \MAROON{(2)}} \\
      & = B + B + 0 & \text{by (VS 3)} \\
      & = B + B + (B^\top + (-B^\top)) & \text{by (VS 4)} \\
      & = (B + (-B^\top)) + (B + B^\top) \MAROON{(3)}  & \text{by (VS 1) (VS 2)}
\end{align*}
Now we claim that \(B + (-B^\top)\) is skew-symmetric, since
\begin{align*}
    (B + (-B^\top))^\top & = B^\top + -((B^\top)^\top) & \text{by \ATHM{1.2}(1)} \\
                   & = B^\top + -B & \text{by \ATHM{1.2}(2)} \\
                   & = -(B + (-B^\top)), & \text{just trivial}
\end{align*}
and by \ATHM{1.2}(3), \(B + B^\top\) is symmetric.
So by definition of \(W_1 + W_2\), \((B + (-B^\top)) + (B + B^\top) \in W_1 + W_2\).
That is, by \MAROON{(3)}, \(A \in W_1 + W_2\).
Since \(A\) is arbitrary, we have \(M_{n \X n}(F) \subseteq W_1 + W_2\).

So we have \(W_1 \cap W_2 = \{ O_{n \X n} \}\) and \(M_{n \X n} = W_1 + W_2\), hence \(M_{n \X n} = W_1 \oplus W_2\), as desired.
\end{proof}

\begin{exercise} \label{exercise 1.3.29}
Let \(F\) be a field that is not of characteristic two(textbook page 549).
Define
\[
    W_1 = \{ A \in M_{n \X n}(F) : A_{ij} = 0 \text{ whenever } i \le j \}
\]
and \(W_2\) to be the set of all symmetric \(n \X n\) matrices with entries from \(F\).
(It's trivial and already shown that) Both \(W_1\) and \(W_2\) are subspaces of \(M_{n \X n}(F)\).
Prove that \(M_{n \X n}(F) = W_1 \oplus W_2\).
Compare this exercise with \EXEC{1.3.28}.
\end{exercise}

\begin{proof}
First we show \(W_1 \cap W_2 = \{ O_{n \X n} \}\).
Suppose \(A \in W_1 \cap W_2\) but \(A \ne O_{n \X n}\).
Then there exists \(1 \le i \le n\) and \(1 \le j \le n\) s.t. \(A_{ij} \ne 0\).
Also by definition of \(W_1\) and \(W_2\) we have \(A_{ij} = 0\) whenever \(i \le j\) \MAROON{(1)} and \(A^\top = A\) \MAROON{(2)}.
So by \MAROON{(1)} we must have \(i > j\) \MAROON{(3)}.
And by definition of transpose we have \((A^\top)_{ij} = A_{ji}\);
but by \MAROON{(2)} that implies \(A_{ij} = A_{ji}\), so we have \(A_{ji} \ne 0\).
But by \MAROON{(3)} we have \(j < i\), which implies \(A\) violates the condition of \(W_1\) so does not belong to \(W_1\), which is a contradiction.
Hence \(A\) must be equal to \(O_{n \X n}\).

Now we show that \(M_{n \X n} \subseteq W_1 + W_2\).
(Again the other direction is automatically satisfied since \(W_1 + W_2\) is a subspace).
Suppose \(A\) is an arbitrary matrix in \(M_{n \X n}\).
Then let \(B\) be a symmetric matrix such that \(B_{ij} = B_{ji} = A_{ij}\) when \(i \le j\).
Then \(A - B\) is a matrix satisfying \(W_1\)'s condition, and \(B\) is a symmetric matrix.
So by definition of \(W_1 + W_2\), \((A - B) + B \in W_1 + W_2\).
That is, \(A \in W_1 + W_2\).
Since \(A\) is arbitrary, we have \(M_{n \X n} \subseteq W_1 + W_2\).

So we have \(W_1 \cap W_2 = \{ O_{n \X n} \}\) and \(M_{n \X n} = W_1 + W_2\), hence \(M_{n \X n} = W_1 \oplus W_2\), as desired.

Compared to \EXEC{1.3.28}, we can use \emph{different} pairs of subspaces to ``directly sum'' to the whole space.
In general, the pair \((W_1, W_2)\) is \emph{not} necessarily unique.
\end{proof}

\begin{exercise} \label{exercise 1.3.30}
Let \(W_1\) and \(W_2\) be subspaces of a vector space \(V\).
Prove that \(V\) is the direct sum of \(W_1\) and \(W_2\) if and only if each vector in \(V\) can be uniquely written as \(x_1 + x_2\), where \(x_1 \in W_1\) and \(x_2 \in W_2\).
\end{exercise}

\begin{proof}\ 

\(\Longrightarrow\): Suppose \(V = W_1 \oplus W_2\).
Now suppose there is \(v \in V\) s.t. \(v = u_1 + u_2 = v_1 + v_2\) \MAROON{(1)} where \(u_1, v_1 \in W_1\) and \(u_2, v_2 \in W_2\).
We will show that \(u_1 = v_1\) and \(u_2 = v_2\).
Then by rearranging equation in \MAROON{(1)}, we have \(u_1 - v_1 = u_2 - v_2\) \MAROON{(2)}.
Since \(u_1, v_1 \in W_1\), we have \(u_1 - v_1 \in W_1\); similarly we get \(u_2 - v_2 \in W_2\).
But by \MAROON{(2)}, we further get \(u_1 - v_1 \in W_2\) and \(u_2 - v_2 \in W_1\), hence we have \(u_1 - v_1 \in W_1 \cap W_2 = \{ \OV \}\) and \(u_2 - v_2 \in W_1 \cap W_2 = \{ \OV \}\).
That is, both \(u_1 - v_1 = \OV\) and \(u_2 - v_2 = \OV\) \MAROON{(3)}.
So again by rearranging \MAROON{(3)}, we have \(u_1 = v_1\) and \(u_2 = v_2\), as desired.

\(\Longleftarrow\): Now suppose any \(v \in V\) can be uniquely written as \(x_1 + x_2\), where \(x_1 \in W_1\) and \(x_2 \in W_2\).
Will will show that \(V = W_1 \oplus W_2\).
But the supposition directly implies \(V \subseteq W_1 + W_2\) (and the other direction is automatically implied since \(W_1 + W_2\) is a subspace) so we have \(V = W_1 + W_2\).
So we only need to prove \(W_1 \cap W_2 = \{ \OV \}\).
For the sake of contradiction, suppose \(v \in W_1 \cap W_2\) but \(v \ne \OV\).
So in particular, \(v \in W_1\) and \(v \in W_2\).
And by (VS 3) (VS 1), we have \(v = v + \OV = \OV + v\).
But that implies \(v = v + \OV\) where \(v \in W_1\) and \(\OV \in W_2\)(since \(W_2\) is a subspace), and \(v = \OV + v\) where \(\OV \in W_1\)(since \(W_1\) is a subspace) and \(v \in W_2\).
So \(v\) \emph{cannot} be uniquely written as \(x + y\) where \(x \in W_1\) and \(y \in W_2\), which is a contradiction.
So \(W_1 \cap W_2 = \{ \OV \}\), so we have \(V = W_1 \oplus W_2\), as desired.
\end{proof}

\begin{exercise} \label{exercise 1.3.31}
Let \(W\) be a subspace of a vector space \(V\) over a field \(F\).
For any \(v \in V\) the set \(\{v\} + W = \{v + w : w \in W\}\) is called the \textbf{coset} of \(W\) \textbf{containing} \(v\).
It is customary to \emph{denote} this coset by \(v + W\) rather than \(\{v\} + W\).
\begin{enumerate}
    \item Prove that \(v + W\) is a \(subspace\) of \(V\) if and only if \(v \in W\).
    \item Prove that \(v_1 + W = v_2 + W\) if and only if \(v_1 - v_2 \in W\).
\end{enumerate}
Addition and scalar multiplication by scalars of \(F\) can be defined in the collection \(S = \{v + W: v \in V\}\) of all cosets of \(W\) as follows:
\[
    (v_1 + W) + (v_2 + W) = (v_1 + v_2) + W
\]
for all \(v_1, v_2 \in V\), and
\[
    a(v + W) = av + W
\]
for all \(v \in V\) and \(a \in F\).
\begin{enumerate}
    \setcounter{enumi}{2}
    \item Prove that the preceding operations are \textbf{well defined};
        that is, show that if \(v_1 + W = v_1' + W\) and \(v_2 + W = v_2' + W\), then
        \[
            (v_1 + W) + (v_2 + W) = (v_1' + W) + (v_2' + W)
        \]
        and
        \[
            a(v_1 + W) = a(v_1' + W)
        \]
        for all \(a \in F\).
    \item Prove that the set \(S\) is \emph{a vector space} with the operations defined in (c).
    This vector space is called the \textbf{quotient space \(V\) modulo \(W\)} and is denoted by \(V / W\).
\end{enumerate}
\end{exercise}

\begin{note}
在證明\ part(c) 前面定義的\ \(+\) and \(\cdot\) are well-defined 之前,我們先討論什麼叫做\ operation well-defined。
一個\ operation \(*\) 要\ well-defined 的意思是,給定(可當作\ \(*\) 的\ operands 的) \(a, a', b, b'\),若\ \(a = a'\),則\ \(a * b = a' * b\);類似地,若\ \(b = b'\),則\ \(a * b = a * b'\)。
兩個組合起來就變成,若\ \(a = a'\) 且\ \(b = b'\),則\ \(a * b = a' * b'\)。
注意,這實際上相依於「兩個\ ``object'' 相等」的定義。

這整段論述叫做\ \textbf{axiom of substitution},可參考\ \href{https://link.springer.com/book/10.1007/978-981-10-1789-6}{Terence Tao 的\ Analysis I, Appendix A.7}。
\end{note}

\begin{note}
因為我們最終是要證明\ \(S\) 是一個\ vector space (with \(+, \cdot\) defined in part(c)),換句話說我們要「想像」 \(S\) 的\ element 是一個 vector,而\ \(S\) 的\ element 長的就像\ \(v_1 + W\)、\(v_2 + W\) 等等。
所以根據前面提到的\ axiom of substitution,part(c) 的意思就是如果\ \(v_1 + W\) 「這個 vector」相等於\ \(v_1' + W\) 「這個\ vector」,且\ \(v_2 + W\)「這個\ vector」相等於\ \(v_2' + W\) 「這個\ vector」,則\ \((v_1 + W) + (v_2 + W)\)「這個\ vector」相等於 \((v_1' + W) + (v_2' + W)\) 「這個\ vector」。
注意,\(v_1 + W = v_1' + W\) 不能得出\ \(v_1 = v_1'\) (可用\ part(a)(b) 舉出反例);
實際上在推導\ \(S\) 是個\ vector space 時,對於\ S 來說,\(v_1, v_1'\) 根本不是\ vector!
\end{note}

\begin{proof}\ 

\begin{enumerate}
\item
\(\Longrightarrow\): Suppose \(v + W\) is a subspace of \(V\), we have to show \(v \in W\).
Since \(v + W\) is a subspace of \(V\), we have \(\OV \in v + W\).
But by (VS 4) of \(V\), \(\OV = v + (-v)\), so \(v + (-v)\) is also in \(v + W\).
And by definition of \(v + W\) (or \(\{ v \} + W\), precisely), since we can only get \(v\) in the singleton set \(\{ v \}\), \(-v\) must be in \(W\).
But since \(W\) is also a subspace of \(V\), by scalar multiplication we have \(-(-v) = v\) is also in \(W\), as desired.

\(\Longleftarrow\): Now suppose \(v \in W\), we have to show \(v + W\) is a subspace of \(V\).
Since \(v \in W\) and \(W\) is a subspace, by (VS 4) of \(W\) we have \(-v \in W\).
So we have \(v \in \{ v \}\) and \(-v \in W\), then by definition of \(v + W\), we have \(v + (-v) \in v + W\);
that is, \(\OV = v + -v \in v + W\).

Now let arbitrary \(v_1, v_2 \in v + W\), we have to show \(v_1 + v_2 \in v + W\).
By definition of \(v + W\), we have \(v_1 = v + w_1\) and \(v_2 = v + w_2\) where \(v\) is of course belongs to \(\{ v \}\) and \(w_1, w_2 \in W\).
And trivially we have \(v_1 + v_2 = (v + w_1) + (v + w_2) = v + (v + w_1 + w_2)\) \MAROON{(1)}.
But since \(v\) is also in \(W\) by supposition, by closed addition of the subspace \(W\), we have \(v + w_1 + w_2 \in W\).
So all in all we have \(v \in \{ v \}\) and \(v + w_1 + w_2 \in W\), so by definition of \(v + W\) we have \(v + (v + w_1 + w_2) \in v + W\);
that is, by \MAROON{(1)}, \(v_1 + v_2 \in v + W\), as desired.

Now let \(w \in v + W\) and \(c\) be a scalar of \(F\), we have to show \(cw \in v + W\).
Again similarly we have \(w = v + w'\) where \(v\) is of course in \(\{ v \}\) and \(w' \in W\).

And \(c w = c (v + w') = cv + cw' = (1 + (c - 1))v + cw' = v + ((c - 1)v + cw')\) \MAROON{(2)}.

But since \(v \in W\), by closed scalar multiplication of \(W\) we have \((c - 1)v \in W\) and \(cw' \in W\), and by closed addition of \(W\) we have \((c - 1)v + cw' \in W\).
So we have \(v \in \{ v \}\) and \((c - 1)v + cw' \in W\), by definition of \(v + W\), \(v + ((c - 1)v + cw') \in v + W\).
That is, by \MAROON{(2)}, \(cw \in v + W\), as desired.

So by \THM{1.3}, \(v + W\) is a subspace of \(V\).

\item
\(\Longrightarrow\): Suppose \(v_1 + W = v_2 + W\), we have to show \(v_1 - v_2 \in W\).
So let \(w \in W\), by definition of \(v_1 + W\), \(v_1 + w \in v_1 + W\).
Also, since \(v_1 + W = v_2 + W\), we have \(v_1 + w \in v_2 + W\).
And by definition of \(v_2 + W\), \(v_1 + w = v_2 + w'\) \MAROON{(3)} where \(w' \in W\).
And from \MAROON{(3)} we can derive \(v_1 - v_2 = w' - w\) \MAROON{(4)}.
But since \(w \in W\) and \(w' \in W\), by closed addition and scalar multiplication, \(w' - w \in W\), and by \MAROON{(4)} that means \(v_1 - v_2 \in W\), as desired.

\(\Longleftarrow\): Now suppose \(v_1 - v_2 \in W\), we have to show \(v_1 + W = v_2 + W\).
So we show this by showing \(v_1 + W \subseteq v_2 + W\) and \(v_2 + W \subseteq v_1 + W\).

So let \(v \in v_1 + W\).
Then by definition of \(v_1 + W\), \(v = v_1 + w\) where \(w \in W\), and in particular \(w = v - v_1\) \MAROON{(5)}.
Also, since \(v_1 - v_2 \in W\), by closed addition, \((v_1 - v_2) + w \in W\).
That is, by \MAROON{(5)}, \((v_1 - v_2) + (v - v_1) \in W\);
that is, \(v - v_2 \in W\) \MAROON{(6)}.
But since \(v_2\) of course belongs to \(\{ v_2 \}\), then by \MAROON{(6)} and definition of \(v_2 + W\), \(v_2 + (v - v_2) \in v_2 + W\), which implies \(v \in v_2 + W\).

Now let \(v \in v_2 + W\).
Then by definition of \(v_2 + W\), \(v = v_2 + w\) where \(w \in W\), and in particular \(w = v - v_2\) \MAROON{(7)}.
Also, since \(v_1 - v_2 \in W\), by closed scalar multiplication, \(-(v_1 - v_2) = v_2 - v_1 \in W\);
and by closed addition, \((v_2 - v_1) + w \in W\).
That is, by \MAROON{(7)}, \((v_2 - v_1) + (v - v_2) \in W\);
that is, \(v - v_1 \in W\) \MAROON{(8)}.
But since \(v_1\) of course belongs to \(\{ v_1 \}\), then by \MAROON{(8)} and definition of \(v_1 + W\), \(v_1 + (v - v_1) \in v_1 + W\), which implies \(v \in v_1 + W\).

So we have shown \(v_1 + W = v_2 + W\), as desired.

\item
Suppose \(v_1 + W = v_1' + W\) \MAROON{(9)} and \(v_2 + W = v_2' + W\) \MAROON{(10)}.
We have to show
\[
    (v_1 + W) + (v_2 + W) = (v_1' + W) + (v_2' + W)
\]
for all \(v_1, v_2 \in V\), and
\[
    a(v_1 + W) = a(v_1' + W)
\]
for all \(a \in F\).
That is, by definition of these operations, we have to show
\[
    (v_1 + v_2) + W = (v_1' + v_2' + W)
\]
and
\[
    av_1 + W = av_1' + W.
\]

We first show addition is well-defined by showing \((v_1 + v_2) + W \subseteq (v_1' + v_2' + W)\) and \((v_1' + v_2') + W \subseteq (v_1 + v_2 + W)\).

So suppose \(v \in (v_1 + v_2) + W\);
by definition of \((v_1 + v_2) + W\), \(v = (v_1 + v_2) + w\) where \(w \in W\).
In particular, \(w = v - (v_1 + v_2)\) \MAROON{(11)}.

Also, by \MAROON{(9)} and part(b), \(v_1 - v_1' \in W\) \MAROON{(12)}, and by \MAROON{(10)} and part(b), \(v_2 - v_2' \in W\) \MAROON{(13)}.
With \(w \in W\) and \(\MAROON{(12)}\) and closed addition of \(W\), we have \(w + (v_1 - v_1') \in W\).
That is, by \MAROON{(11)}, \((v - (v_1 + v_2)) + (v_1 - v_1') \in W\);
that is, \(v - (v_1' + v_2) \in W\). \MAROON{(14)}.

Similarly With \MAROON{(14)} and \(\MAROON{(13)}\) and closed addition of \(W\), we have \((v - (v_1' + v_2)) + (v_2 - v_2') \in W\);
that is, \(v - (v_1' + v_2') \in W\) \MAROON{(15)}.

So, since \(v_1' + v_2'\) of course belongs to \(\{ v_1' + v_2' \}\), and \MAROON{(15)}, by definition of \((v_1' + v_2') + W\), we have \((v_1' + v_2') + (v - (v_1' + v_2')) \in (v_1' + v_2') + W\).
That is, \(v \in (v_1' + v_2') + W\), as desired.

Now suppose \(v \in (v_1' + v_2') + W\);
then by definition of \((v_1' + v_2') + W\), \(v = (v_1' + v_2') + w\) where \(w \in W\).
In particular, \(w = v - (v_1' + v_2')\) \MAROON{(16)}.

Also, by \MAROON{(9)} and part(b), \(v_1 - v_1' \in W\), and by \MAROON{(10)} and part(b), \(v_2 - v_2' \in W\).
In particular, by closed scalar multiplication of \(W\), \(-(v_1 - v_1') = v_1' - v_1 \in W\) \MAROON{(17)} and \(-(v_2 - v_2') = v_2' - v_2 \in W\) \MAROON{(18)}.
With \(w \in W\) and \(\MAROON{(17)}\) and closed addition of \(W\), we have \(w + (v_1' - v_1) \in W\).
That is, by \MAROON{(16)}, \((v - (v_1' + v_2')) + (v_1' - v_1) \in W\);
that is, \(v - (v_1 + v_2') \in W\). \MAROON{(19)}.

Similarly With \MAROON{(19)} and \(\MAROON{(18)}\) and closed addition of \(W\), we have \((v - (v_1 + v_2')) + (v_2' - v_2) \in W\);
that is, \(v - (v_1 + v_2) \in W\) \MAROON{(20)}.

So, since \(v_1 + v_2\) of course belongs to \(\{ v_1 + v_2 \}\), and \MAROON{(20)}, by definition of \((v_1 + v_2) + W\), we have \((v_1 + v_2) + (v - (v_1 + v_2)) \in (v_1 + v_2) + W\).
That is, \(v \in (v_1 + v_2) + W\), as desired.

So we have shown \((v_1 + v_2) + W = (v_1' + v_2' + W)\), that is, \((v_1 + W) + (v_2 + W) = (v_1' + W) + (v_2' + W)\), so the definition of addition is well defined.

Now We show scalar multiplication is well-defined by showing \(av_1 + W \subseteq av_1' + W\) and \(av_1' + W \subseteq av_1 + W\) (given arbitrary \(a \in F\)).

So suppose \(v \in av_1 + W\).
Then \(v = av_1 + w\) where \(w \in W\).
In particular \(w = v - av_1\).
Again by part(b) we have \(v_1 - v_1' \in W\).
And by closed scalar multiplication of \(W\), we have \((a(v_1 - v_1') = av_1 - av_1' \in W\).
so we have \(w + (av_1 - av_1') \in W\);
that is, \((v - av_1) + (av_1 - av_1') \in W\);
that is, \(v - av_1' \in W\).
So by definition of \(av_1' + W\), \(av_1' + (v - av_1') \in av_1' + W\);
that is, \(v \in av_1' + W\), as desired.

Now suppose \(v \in av_1' + W\).
Then \(v = av_1' + w\) where \(w \in W\).
In particular \(w = v - av_1'\).
Again by part(b) we have \(v_1 - v_1' \in W\).
And by closed scalar multiplication of \(W\), we have \(-(v_1 - v_1') = v_1' - v_1 \in W\) and \(a(v_1' - v_1) = av_1' - av_1 \in W\).
so we have \(w + (av_1' - av_1) \in W\);
that is, \((v - av_1') + (av_1' - av_1) \in W\);
that is, \(v - av_1 \in W\).
So by definition of \(av_1 + W\), \(av_1 + (v - av_1) \in av_1 + W\);
that is, \(v \in av_1 + W\), as desired.

So we have shown \(av_1 + W = av_1' + W\), that is, \(a(v_1 + W) = a(v_1' + W)\), so the scalar multiplication is well defined.

\item
Since in this case we just directly construct \(S\), we cannot use something like \THM{1.3} to show that \(S\) is a vector space(a subsapce of some other vector space);
instead, we have to show \(S\) satisfies all the conditions in \DEF{1.1}.
BTW, it's in fact trivial by using (VS 1) to (VS 8) of \(V\), although tedious.

So first we show \(+\) and \(\cdot\) is closed under \(S\).
So let \(v_1 + W\), \(v_2 + W\) and \(v_3 + W\) belong to \(S\).
By definition of \(S\), \(v_1, v_2 \in V\).
But since \(V\) is a vector space, \(v_1 + v_2 \in V\).
So again by definition of \(S\), \((v_1 + v_2) + W \in S\).
That is, by definition of \(+\) of \(S\), \((v_1 + W) + (v_2 + W) \in S\), so \(+\) is closed under \(S\).
Now let \(c \in F\).
Since \(v_1 \in V\), \(c v_1 \in V\).
So again by definition of \(W\), \(c v_1 + W \in S\).
That is, by definition of \(\cdot\) of \(S\), \(c(v_1 + W) \in S\), so \(\cdot\) is closed under \(S\).

For (VS 1),
\begin{align*}
    (v_1 + W) + (v_2 + W) & = (v_1 + v_2) + W & \text{by def of \(+\) of \(S\)} \\
                          & = (v_2 + v_1) + W & \text{by (VS 1) of \(V\)} \\
                          & = (v_2 + W) + (v_1 + W). & \text{by def of \(+\) of \(S\)}
\end{align*}

For (VS 2),
\begin{align*}
    ((v_1 + W) + (v_2 + W)) + (v_3 + W) & = ((v_1 + v_2) + W) + (v_3 + W) & \text{by def of \(+\) of \(S\)} \\
                          & = ((v_1 + v_2) + v_3) + W & \text{by def of \(+\) of \(S\)} \\
                          & = (v_1 + (v_2 + v_3)) + W & \text{by (VS 2) of \(V\)} \\
                          & = (v_1 + (v_2 + v_3)) + W & \text{by def of \(+\) of \(S\)} \\
                          & = (v_1 + W) + ((v_2 + v_3) + W) & \text{by def of \(+\) of \(S\)} \\
                          & = (v_1 + W) + ((v_2 + W) + (v_3 + W)). & \text{by def of \(+\) of \(S\)}
\end{align*}

For (VS 3), there exists the zero vector \(\OV + W\) in \(S\), where \(\OV\) is the zero vector of \(V\), s.t.
\begin{align*}
    (v_1 + W) + (\OV + W) & = (v_1 + \OV) + W & \text{by def of \(+\) of \(S\)} \\
                          & = v_1 + W. & \text{by (VS 3) of \(V\)}
\end{align*}

For (VS 4), given any \(v + W\) in \(S\), there exists the inverse \((-v) + W\), where \(-v\) is the inverse of \(v\) in the vector space \(V\), s.t.
\begin{align*}
    (v + W) + ((-v) + W & = (v + (-v)) + W & \text{by def of \(+\) of \(S\)} \\
                        & = \OV + W. & \text{by (VS 4) of \(V\)}
\end{align*}

For (VS 5), there exists scalar \(\IF \in F\), the multiplicative identity of vector space \(V\), s.t. given any \(v + W \in S\),
\begin{align*}
    \IF(v + W)& = \IF v + W & \text{by def of \(\cdot\) of \(S\)} \\
                          & = v + W. & \text{by (VS 5) of \(V\)}
\end{align*}

For (VS 6), given \(a, b \in F\), and given any \(v + W \in S\),
\begin{align*}
    (ab)(v + W) & = (ab)v + W & \text{by def of \(\cdot\) of \(S\)} \\
                & = a(bv) + W & \text{by (VS 6) of \(V\)} \\
                & = a(bv + W) & \text{by def of \(\cdot\) of \(S\)} \\
                & = a(b(v + W)). & \text{same}
\end{align*}

For (VS 7),
\begin{align*}
    a((v_1 + W) + (v_2 + W)) & = a((v_1 + v_2) + W) & \text{by def of \(+\) of \(S\)} \\
                             & = a(v_1 + v_2) + W & \text{by def of \(\cdot\) of \(S\)} \\
                             & = (av_1 + av_2) + W & \text{by (VS 7) of \(V\)} \\
                             & = (av_1 + W) + (av_2 + W) & \text{by def of \(+\) of \(S\)} \\
                             & = a(v_1 + W) + a(v_2 + W). & \text{by def of \(\cdot\) of \(S\)}
\end{align*}

For (VS 8),
\begin{align*}
    (a + b)((v + W) & = ((a + b)v) + W & \text{by def of \(\cdot\) of \(S\)} \\
                             & = (av + bv) + W & \text{by (VS 8) of \(V\)} \\
                             & = (av + W) + (bv + W) & \text{by def of \(+\) of \(S\)} \\
                             & = a(v + W) + b(v + W). & \text{by def of \(\cdot\) of \(S\)}
\end{align*}

So all the conditions in \DEF{1.1} are satisfied, so \(S\) with the corresponding addition and scalar multiplication is a vector space.
\end{enumerate}
\end{proof}

\begin{additional theorem} \label{athm 1.2}
This is the placeholder theorem for

\BLUE{(1)} \EXEC{1.3.3}(1): \((aA + bB)^\top = aA^\top + bB^\top\).

\BLUE{(2)} \EXEC{1.3.4}: \((A^\top)^\top = A\).

\BLUE{(3)} \EXEC{1.3.5}: \(A + A^\top\) is symmetric.
\end{additional theorem}

\begin{additional theorem} \label{athm 1.3}
This is the placeholder theorem for \EXEC{1.3.6}: \(\TRACE(aA + bB) = a \TRACE(A) + b \TRACE(B)\).
\end{additional theorem}

\begin{additional theorem} \label{athm 1.4}
In addition to \THM{1.3}, we can use \EXEC{1.3.17} and \EXEC{1.3.18} to show a subset of a vector space is in fact a subspace.
\end{additional theorem}

\begin{additional theorem} \label{athm 1.5}
This is the placeholder theorem for \EXEC{1.3.19}, the condition of when the union of any two subspaces is still a subspace.
\end{additional theorem}

\begin{additional theorem} \label{athm 1.6}
This is the placeholder theorem for \EXEC{1.3.20}, the \emph{linear combination} of any finite vectors of a vector space is still in the vector space.
\end{additional theorem}

\begin{additional theorem} \label{athm 1.7}
This is the placeholder theorem for \EXEC{1.3.23}:

\BLUE{(1)} the the \emph{sum} of any two subspaces is still a subspace.

\BLUE{(2)} given two subspaces \(W_1, W_2\), \(W_1 + W_2\) is the \emph{smallest subspace} that contains \(W_1 \cup W_2\).
\end{additional theorem}

\begin{additional theorem} \label{athm 1.8}
This is the placeholder theorem for \EXEC{1.3.28} and \EXEC{1.3.29}, which show some special and different pairs of subspaces which have \(M_{n \X n}(F)\) as the direct sum.
\end{additional theorem}

\begin{additional theorem} \label{athm 1.9}
This is the placeholder theorem for \EXEC{1.3.30}, which shows a good property of direct sum: uniquely representing any vector.
\end{additional theorem}

\begin{additional theorem} \label{athm 1.10}
This is the placeholder theorem for \EXEC{1.3.31}, which shows some basic facts about \emph{quotient spaces}.
\end{additional theorem}
