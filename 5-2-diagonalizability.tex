\section{Diagonalizability} \label{sec 5.2}

In \SEC{5.1}, we presented the diagonalization problem and observed that \emph{not all} linear operators or matrices are diagonalizable.
Although we are able to diagonalize operators and matrices and even obtain a \emph{necessary and sufficient condition} for diagonalizability (\THM{5.1}),
we have not yet \emph{solved} the diagonalization problem.
What is still needed is a \textbf{simple test} to determine whether an operator or a matrix can be diagonalized, as well as a \textbf{method for actually finding} a basis of eigenvectors.
In this section, we develop such a test and method.

In \EXAMPLE{5.1.6}, we \emph{accidentally} obtained a basis of eigenvectors by choosing \emph{one} eigenvector corresponding to each eigenvalue.
In general, such a procedure does \emph{not} yield a basis, but the following theorem shows that any set constructed in this manner \emph{is \LID{}}.

\begin{theorem} \label{thm 5.5}
Let \(\T\) be a linear operator on a vector space, and let \(\lambda_1, \lambda_2, ..., \lambda_k\) be distinct eigenvalues of \(\T\).
For each \(i = 1, 2, ..., k\), let \(S_i\) be a finite set of \emph{eigenvectors} of \(\T\) corresponding to \(\lambda_i\).
If each \(S_i\) (\(i = 1, 2, ..., k\)), is \LID{}, then \(S_1 \cup S_2 \cup ... \cup S_k\) is \LID{}.
\end{theorem}

\begin{proof}
The proof is by mathematical induction on \(k\).
If \(k = 1\), there is nothing to prove.
So assume that the theorem holds for \(k - 1\) distinct eigenvalues, where \(k > 1\), and that we have \(k\) distinct eigenvalues \(\lambda_1, \lambda_2, ..., \lambda_k\) of \(\T\).
For each \(i = 1, 2, ..., k\), let \(S_{\RED{i}} = \{ v_{i1}, v_{i2}, ..., v_{i\RED{n_i}} \}\) be a \LID{} set of eigenvectors of \(\T\) corresponding to \(\lambda_i\).
(So each \(S_i\) has \(n_i\) elements for \(1 \le i \le k\).)
We wish to show that \(S = S_1 \cup S_2 \cup ... \cup S_k\) is \LID{}.

That is, consider any set of scalars \(\{ a_{ij} \}\), where \(i = 1, 2, ... , k\) and \(j = 1, 2, ..., n_i\), such that
\[
    \sum_{i = 1}^k \sum_{j = 1}^{n_i} a_{ij} v_{ij} = \OV. \quad \quad \MAROON{(1)}
\]
It suffices to show \(a_{ij} = 0\) for all \(1 \le i \le k, 1 \le j \le n_i\).
Then we apply a \emph{trick} on \MAROON{(1)}: Applying both sides by the linear operator \(\T - \lambda_k I\).
By the derivation below, we will have
\[
    \sum_{i = 1}^{\RED{k - 1}} \sum_{j = 1}^{n_i} (\lambda_i - \lambda_k) a_{ij} v_{ij} = \OV. \quad \quad \MAROON{(2)}
\]
The heuristic is that by applying the trick, the term having \(\lambda_k\) in the coefficient will be \emph{removed}, and hence we can apply the inductive hypothesis.
And the equation is derived by the steps:
\begin{align*}
             & \sum_{i = 1}^k \sum_{j = 1}^{n_i} a_{ij} v_{ij} = \OV & \text{by \MAROON{(1)}} \\
    \implies & (\T - \lambda_k \ITRAN{})\left( \sum_{i = 1}^k \sum_{j = 1}^{n_i} a_{ij} v_{ij} \right) = (\T - \lambda_k \ITRAN{})(\OV) = \OV \\
    \implies & \T \left( \sum_{i = 1}^k \sum_{j = 1}^{n_i} a_{ij} v_{ij} \right) - \lambda_k \ITRAN{} \left( \sum_{i = 1}^k \sum_{j = 1}^{n_i} a_{ij} v_{ij} \right) = \OV \\
             & \quad \quad \text{(since function ``\(-\)'' is linear)} \\
    \implies & \sum_{i = 1}^k \sum_{j = 1}^{n_i} a_{ij} \T \left( v_{ij} \right) - \lambda_k \ITRAN{} \left( \sum_{i = 1}^k \sum_{j = 1}^{n_i} a_{ij} v_{ij} \right) = \OV \\
             & \quad \quad \text{(since \(\T\) is linear)} \\
    \implies & \sum_{i = 1}^k \sum_{j = 1}^{n_i} \lambda_i a_{ij} v_{ij} - \sum_{i = 1}^k \sum_{j = 1}^{n_i} \lambda_{\RED{k}} a_{ij} v_{ij} = \OV \\
             & \quad \quad \text{(of course)} \\
    \implies & (\lambda_1 - \lambda_{\RED{k}}) \sum_{j = 1}^{n_1} a_{1j} v_{1j} + (\lambda_2 - \lambda_{\RED{k}}) \sum_{j = 1}^{n_1} a_{2j} v_{2j} + ... + (\lambda_{\RED{k}} - \lambda_{\RED{k}}) \sum_{j = 1}^{n_1} a_{kj} v_{kj} \\
            & \quad \quad \text{(by rearranging finite summation)} \\
    \implies & (\lambda_1 - \lambda_k) \sum_{j = 1}^{n_1} a_{1j} v_{1j} + (\lambda_2 - \lambda_k) \sum_{j = 1}^{n_1} a_{2j} v_{2j} + ... + (\lambda_{k - 1} - \lambda_k) \sum_{j = 1}^{n_{k - 1}} a_{(k - 1)j} v_{(k - 1)j} \\
             & \quad \quad \text{(that is, the last sum has no effect)} \\
    \implies & \sum_{i = 1}^{\RED{k - 1}} \sum_{j = 1}^{n_i} (\lambda_i - \lambda_k) a_{ij} v_{ij} = \OV.
\end{align*}
But \MAROON{(2)} is a linear combination from \(S_1 \cup S_2 \cup ... \cup S_{k - 1}\), so by the inductive hypothesis, \(v_{ij}\) are \LID{} for \(1 \le i \le k - 1\) and \(1 \le j \le n_i\), so \((\lambda_i - \lambda_k) a_{ij} = 0\) for \(1 \le i \le k - 1\) and \(1 \le j \le n_i\).
But since \(\lambda_i\) are distinct, that implies \(\lambda_i - \lambda_k \ne 0\) for \(1 \le i \le k - 1\), so (by zero product property,) \(a_{ij} = 0\) for \(1 \le i \le k - 1\) and \(1 \le j \le n_i\).
So \MAROON{(1)} can be reduced to
\[
    \sum_{j = 1}^{n_k} a_{ij} v_{ij} = \OV,
\]
which is a linear combination from \(S_k\).
But \(S_k\) is also \LID{}, so \(a_{\RED{k}j} = 0\) for \(j = 1, 2, ... n_k\).
Consequently \(a_{ij} = 0\) for \(i = 1, 2, ..., k\) and \(j = 1, 2, ..., n_i\), proving that \(S\) is \LID{}.
This closes the induction.
\end{proof}

\begin{corollary} \label{corollary 5.5.1}
Let \(\T\) be a linear operator on an \(n\)-dimensional vector space \(V\).
If \(\T\) has \(n\) \emph{distinct} eigenvalues, then \(\T\) is diagonalizable.
\end{corollary}

\begin{proof}
Suppose that \(\T\) has \(n\) distinct eigenvalues \(\lambda_1, \lambda_2, ..., \lambda_n\).
For each \(i\) choose \emph{an} eigenvector \(v_i\) corresponding to \(\lambda_i\).
By \THM{5.5}, \(\{v_1, ..., v_n \}\) is \LID{} having \(n\) eigenvectors, and since \(\dim(V) = n\), this set is a basis for \(V\) and consists eigenvectors for \(\T\).
Thus, by \THM{5.1}, \(\T\) is diagonalizable.
\end{proof}

\begin{example} \label{example 5.2.1}
Let
\[
    A = \begin{pmatrix} 1 & 1 \\ 1 & 1 \end{pmatrix} \in M_{2 \X 2}(\SET{R}).
\]
The \CPOLY{} of \(A\) (and hence of \(\LMTRAN_A)\) is
\[
    \det(A - tI) = \det \begin{pmatrix} 1 - t & 1 \\ 1 & 1 - t \end{pmatrix} = t(t - 2),
\]
and thus the eigenvalues of \(\LMTRAN_A\) are \(0\) and \(2\).
Since \(\LMTRAN_A\) is a linear operator on the \emph{two}-dimensional vector space \(\SET{R}^2\) and we have \emph{two} distinct eigenvalues, we conclude from \CORO{5.5.1} that \(\LMTRAN_A\) (and hence \(A\)) is diagonalizable.
\end{example}

\begin{remark} \label{remark 5.2.1}
The \emph{converse} of the \CORO{5.5.1} is false.
That is, it is not true that if \(\T\) is diagonalizable, then it has \(n\) distinct eigenvalues.
For example, the identity operator is diagonalizable even though it has only one eigenvalue, namely, \(\lambda = 1\).
\end{remark}

We have seen that diagonalizability requires the existence of eigenvalues.
Actually, diagonalizability \textbf{imposes a stronger condition} on the \CPOLY{}.

\begin{definition} \label{def 5.5}
A polynomial \(f(t)\) in \(\mathcal{P}(F)\) \textbf{splits over} \(F\) if there are scalars \(c, a_1, ..., a_n\) (\emph{not necessarily distinct}) in \(F\) such that \(f(t) = c(t - a_1)(t - a_2) ... (t - a_n)\).
\end{definition}

\begin{remark} \label{remark 5.2.2}
For example, \(t^2 - 1 = (t + 1)(t - 1)\) splits over \(\SET{R}\), but \((t^2 + 1)(t - 2)\) does not split over \(\SET{R}\) because \(t^2 + 1\) cannot be factored into a product of linear factors.
However, \((t^2 + 1)(t - 2)\) \textbf{does split} over \(\SET{C}\) because it factors into the product \((t + i)(t - i)(t - 2)\).
If \(f(t)\) is the \CPOLY{} of a linear operator or a matrix over a field \(F\), then the statement that ``\(f(t)\) splits'' is understood to mean that ``it splits over \(F\)''.
That is, the corresponding field is very important.
\end{remark}

\begin{theorem} \label{thm 5.6}
The \CPOLY{} of any \emph{diagonalizable} linear operator on a vector space \(V\) over a field \(F\) splits over \(F\).
\end{theorem}

\begin{note}
The \emph{contrapositive} is useful: If the \CPOLY{} of a linear operator does not split(over \(F\)), then the linear operator is not diagonalizable.
\end{note}

\begin{proof}
Let \(\T\) be a diagonalizable linear operator on the \(n\)-dimensional vector space \(V\), and let \(\beta\) be an ordered basis for \(V\) such that \([\T]_{\beta} = D\) is a diagonal matrix.
Suppose that
\[
    D = \begin{pmatrix}
        \lambda_1 & 0         & ...    & 0 \\
        0         & \lambda_2 & ...    & 0 \\
        \vdots    & \vdots    & \ddots & \vdots \\
        0         & ...       & 0      & \lambda_n
    \end{pmatrix},
\]
and let \(f(t)\) be the \CPOLY{} of \(\T\).
Then
\begin{align*}
    f(t) & = \det(D - tI) = \det \begin{pmatrix}
                \lambda_{1}-t & 0 & \cdots & 0 \\
                0 & \lambda_{2}-t & \cdots & 0 \\
                \vdots & \vdots & & \vdots \\
                0 & 0 & \cdots & \lambda_{n}-t
            \end{pmatrix} \\
         & = (\lambda_1 - t)(\lambda_2 - t) ... (\lambda_n - t) & \text{by \EXEC{4.2.23}} \\
         & = (-1)^n (t - \lambda_1)(t - \lambda_2) ... (t - \lambda_n)
\end{align*}
which splits by \DEF{5.5}.
\end{proof}

\begin{remark} \label{remark 5.2.3}
From \THM{5.6}, it is clear that if \(\T\) is a diagonalizable linear operator on an \(n\)-dimensional vector space that \emph{fails to have} \(n\) distinct eigenvalues, then (since the \CPOLY{} splits,) the \CPOLY{} of \(\T\) must have \emph{repeated zeros}.

Also, the \emph{converse} of \THM{5.6} is false;
that is, the \CPOLY{} of \(\T\) may split, but \(\T\) \emph{need not} be diagonalizable.
(See \EXAMPLE{5.2.3}.)
The following concept helps us determine when an operator whose \CPOLY{} splits is diagonalizable.
\end{remark}

\begin{definition} \label{def 5.6}
Let \(\lambda\) be an eigenvalue of a linear operator or matrix with \CPOLY{} \(f(t)\).
The \textbf{multiplicity} (sometimes called the \textbf{algebraic multiplicity}) of \(\lambda\) is the largest positive integer \(k\) for which \((t - \lambda)^k\) is a \emph{factor} of \(f(t)\).
\end{definition}

\begin{example} \label{example 5.2.2}
Let
\[
    A = \begin{pmatrix} 3 & 1 & 0 \\ 0 & 3 & 4 \\ 0 & 0 & 4 \end{pmatrix},
\]
which has \CPOLY{} \(f(t) = -(t - 3)^2(t - 4)\).
Hence \(\lambda = 3\) is an eigenvalue of \(A\) with algebraic multiplicity \(2\), and \(\lambda = 4\) is an eigenvalue of \(A\) with algebraic multiplicity \(1\).
\end{example}

\begin{remark} \label{remark 5.2.4}
If \(\T\) is a diagonalizable linear operator on a \emph{finite}-dimensional vector space \(V\), then by \THM{5.1}, there is an ordered basis \(\beta\) for \(V\) consisting of eigenvectors of \(\T\).
And by \THM{5.1} again, \([\T]_{\beta}\) is a diagonal matrix in which the diagonal entries are the eigenvalues of \(\T\).

Since the \CPOLY{} of \(\T\) is \(\det([\T]_{\beta} - tI)\), \textbf{it is easily seen that each eigenvalue of \(\T\) must occur as a diagonal entry of \([\T]_{\beta}\), exactly as many times as its algebraic multiplicity.}
Hence \(\beta\) contains \textbf{as many} (\LID{}) eigenvectors corresponding to an eigenvalue \textbf{as} the \textbf{algebraic multiplicity} of that eigenvalue.
So \textbf{the number} of \LID{} eigenvectors corresponding to a given eigenvalue is \emph{of interest in determining whether an operator can be diagonalized}.

Recalling from \THM{5.4} that the eigenvectors of \(\T\) corresponding to the eigenvalue \(\lambda\) are the \emph{nonzero} vectors in \(\NULL(\T - \lambda \ITRAN{})\), we are led naturally to the study of this set.
\end{remark}

\begin{definition} \label{def 5.7}
Let \(\T\) be a linear operator on a vector space \(V\), and let \(\lambda\) be an eigenvalue of \(\T\).
Define \(E_{\lambda} = \{x \in V : \T(x) = \lambda x \} = \NULL(\T - \lambda \ITRANV{})\).
The set \(E_{\lambda}\) is called the \textbf{eigenspace of \(\T\) corresponding to the eigenvalue \(\lambda\)}.
Analogously, we define the \textbf{eigenspace} of a square matrix \(A\) corresponding to the eigenvalue \(\lambda\) to be the eigenspace of \(\LMTRAN_A\) corresponding to \(\lambda\).
Finally, we define the \textbf{geometric multiplicity} of \(\lambda\) is the \emph{dimension} of the corresponding \textbf{eigenspace} \(E_{\lambda}\).
\end{definition}

\begin{note}
The definition of geometric multiplicity is given in many literature, so I give the definition here.

And clearly, \(E_{\lambda}\) is a \emph{subspace} of \(V\) consisting of the zero vector and the eigenvectors of \(\T\) corresponding to the eigenvalue \(\lambda\).
\end{note}

\begin{theorem} \label{thm 5.7}
Let \(\T\) be a linear operator on a \emph{finite}-dimensional vector space \(V\), and let \(\lambda\) be an eigenvalue of \(\T\) having algebraic multiplicity \(m\).
Then \(1 \le \dim(E_{\lambda}) \le m\).
(That is, by \DEF{5.7}, the geometric multiplicity of \(\lambda\) is less than or equal to the algebraic multiplicity of \(\lambda\).)
\end{theorem}

\begin{proof}
First, since \(E_{\lambda} \ne \{ \OV \}\) and is a subspace of \(V\), we have \(\dim(E_{\lambda}) \ge 1\). \BLUE{(1)}.

\sloppy Now choose an ordered basis \(\{ v_1, v_2, ..., v_p \}\) for \(E_{\lambda}\), (hence \(\dim(E_{\lambda} = p\),)
and extend it to an ordered basis \(\beta = \{ v_1, v_2, ..., v_p, v_{p + 1}, ..., v_n \}\) for \(V\), and let \(A = [\T]_{\beta}\).
Observe that \(v_i (1 \le i \le p)\) is an eigenvector of \(\T\) corresponding to \(\lambda\),
that is,
\begin{align*}
    \T(v_1) & = \lambda v_1 = \lambda \cdot v_1 + 0 \cdot v_2 + ... + 0 \cdot v_p + \RED{0} \cdot v_{p + 1} + ... + \RED{0} \cdot v_n, \\
    \T(v_2) & = \lambda v_2 = 0 \cdot v_1 + \lambda \cdot v_2 + 0 \lambda \cdot v_3 + ... + 0 \cdot v_{p - 1} + 0 \cdot v_p + \RED{0} \cdot v_{p + 1} + ... + \RED{0} \cdot v_n, \\
    \vdots & \\
    \T(v_p) & = \lambda v_p = 0 \cdot v_1 + 0 \cdot v_2 + ... + 0 \cdot v_{p - 1} + \lambda \cdot v_p + \RED{0} \cdot v_{p + 1} + ... + \RED{0} \cdot v_n,
\end{align*}
and therefore
\[
    A = \begin{pmatrix} \lambda I_p & B \\ \RED{O} & C \end{pmatrix}. \quad \quad \MAROON{(1)}
\]
By \EXEC{4.3.21}, the \CPOLY{} of \(\T\) is
\begin{align*}
    f(t) & = \det(A - tI_n) = \det \begin{pmatrix} (\lambda - t) I_p & B \\ \RED{O} & C - tI_{n - p} \end{pmatrix} & \text{by \MAROON{(1)}} \\
         & = \det((\lambda - t) I_p) \cdot \det(C - tI_{n - p}) & \text{by \EXEC{4.3.21}} \\
         & = (\lambda - t)^p \cdot \det(C - tI_{n - p}).
\end{align*}
Then by definition of \(m\), the algebraic multiplicity for \(\lambda\), \(m = p + q\) where \(q\) is the ``power'' of the factor \(\det(C - tI_{n - p})\), and that implies \(p \le m\).
But \(p\) is equal to \(\dim(E_{\lambda})\), so we have \(\dim(E_{\lambda}) \le m\).

With \BLUE{(1)}, we have \(1 \le \dim(E_{\lambda}) \le m\), as desired.
\end{proof}

\begin{example} \label{example 5.2.3}
Let \(\T\) be the linear operator on \(\POLYRR\) defined by \(\T(f(x)) = f'(x)\).
The matrix representation of \(\T\) with respect to the standard ordered basis \(\beta\) for \(\POLYRR\) is
\[
    [\T]_{\beta} = \begin{pmatrix} 0 & 1 & 2 \\ 0 & 0 & 2 \\ 0 & 0 & 0 \end{pmatrix}.
\]
Consequently, the \CPOLY{} of \(\T\) is
\[
    \det([\T]_{\beta} - tI) = \det \begin{pmatrix} -t & 1 & 2 \\ 0 & -t & 2 \\ 0 & 0 & -t \end{pmatrix} = -t^3.
\]
Thus the \CPOLY{} of \(\T\) \emph{splits}, and \(\T\) has only one eigenvalue \(\lambda = 0\) with algebraic multiplicity \(3\).
Solving \(f'(x) = \T(f(x)) = \lambda f(x) = 0 \cdot f(x) = 0\) (zero function) shows that \(E_{\lambda} = \NULL(\T - \lambda \ITRAN{}) = \NULL(\T - 0 \cdot \ITRAN{}) = \NULL(\T)\) is the subspace of \(\POLYRR\) consisting of the constant polynomials.
So \(\{ 1 \}\) is a basis for \(E_{\lambda}\), and therefore \(\dim(E_{\lambda}) = 1\).
Consequently, there is \emph{no basis for \(\POLYRR\) consisting of eigenvectors of \(\T\)}, and therefore \(\T\) is not diagonalizable.
(So this is an example that the converse of \THM{5.6} is false.)
\end{example}

\begin{remark} \label{remark 5.2.5}
Even though \(\T\) is not diagonalizable, we will see in \CH{7} that \emph{its eigenvalue and eigenvectors are still useful for describing the behavior of \(\T\)}.
\end{remark}

\begin{example} \label{example 5.2.4}
Let \(\T\) be the linear operator on \(\SET{R}^3\) defined by
\[
    \T \begin{pmatrix} a_1 \\ a_2 \\ a_3 \end{pmatrix} = \begin{pmatrix} 4 a_1 + a_3 \\ 2 a_1 + 3 a_2 + 2 a_3 \\ a_1 + 4 a_3 \end{pmatrix}.
\]
We determine the eigenspace of \(\T\) corresponding to each eigenvalue.
Let \(\beta\) be the standard ordered basis for \(\SET{R}^3\).
Then
\[
    [\T]_{\beta} = \begin{pmatrix} 4 & 0 & 1 \\ 2 & 3 & 2 \\ 1 & 0 & 4 \end{pmatrix}.
\]
and hence the \CPOLY{} of \(\T\) is
\[
    \det([\T]_{\beta} - tI) = \begin{pmatrix} 4 - t & 0 & 1 \\ 2 & 3 - t & 2 \\ 1 & 0 & 4 - t \end{pmatrix} = -(t - 5)(t - 3)^2.
\]
So the eigenvalues of \(\T\) are \(\lambda_1 = 5\) and \(\lambda_2 = 3\) with algebraic multiplicities \(1\) and \(2\), respectively.

Since
\[
    E_{\lambda_1} = \NULL(\T - \lambda_{1} \ITRAN{} )
    = \left\{
        \begin{pmatrix} x_{1} \\ x_{2} \\ x_{3} \end{pmatrix}
        \in \SET{R}^3 : \begin{pmatrix} -1 & 0 & 1 \\ 2 & -2 & 2 \\ 1 & 0 & -1 \end{pmatrix}
        \begin{pmatrix} x_{1} \\ x_{2} \\ x_{3} \end{pmatrix}
        = \begin{pmatrix} 0 \\ 0 \\ 0 \end{pmatrix}
    \right\}
\]
\(E_{\lambda_1}\) is the solution space of the system of linear equations
\[
    \sysdelim..\systeme{
        -x_1 + x_3 = 0,
        2x_1 - 2x_2 + 2x_3 = 0,
        x_1 - x_3 = 0
    }.
\]
It is easily seen (using the techniques of \CH{3}) that
\[
    \left\{ \begin{pmatrix} 1 \\ 2 \\ 1 \end{pmatrix} \right\}
\]
is a basis for \(E_{\lambda_1}\).
Hence \(\dim(E_{\lambda_1}) = 1\).
Similarly, \(E_{\lambda_2} = \NULL(\T - \lambda_2 \ITRAN{})\) is the solution space of the system
\[
    \sysdelim..\systeme{
        x_1 + x_3 = 0,
        2x_1 + 2x_3 = 0,
        x_1 + x_3 = 0
    }.
\]
Since the unknown \(x_2\) does not appear in this system, we assign it a parametric value, say, \(x_2 = s\), and solve the system for \(x_1\) and \(x_3\), introducing another parameter \(t\).
The result is the general solution to the system
\[
    \begin{pmatrix} x_{1} \\ x_{2} \\ x_{3} \end{pmatrix}
    = s\begin{pmatrix} 0 \\ 1 \\ 0 \end{pmatrix}
    + t \begin{pmatrix} -1 \\ 0 \\ 1 \end{pmatrix},
    \text { for } s, t \in \SET{R}.
\]
It follows that
\[
    \left\{
        \begin{pmatrix} 0 \\ 1 \\ 0 \end{pmatrix},
        \begin{pmatrix} -1 \\ 0 \\ 1 \end{pmatrix}
    \right\}
\]
is a basis for \(E_{\lambda_2}\), and \(\dim(E_{\lambda_2}) = 2\).
In this case, the algebraic multiplicity of each eigenvalue \(\lambda_2\) is equal to the dimension of the corresponding eigenspace \(E_{\lambda_i}\).
Observe that the union of the two bases just derived, namely,
\[
    \left\{
        \begin{pmatrix} 1 \\ 2 \\ 1 \end{pmatrix},
        \begin{pmatrix} 0 \\ 1 \\ 0 \end{pmatrix},
        \begin{pmatrix} -1 \\ 0 \\ 1 \end{pmatrix}
    \right\},
\]
is \LID{} by \THM{5.5} and have \(\RED{3}\) elements hence is a basis for \(\SET{R}^{\RED{3}}\) consisting of eigenvectors of \(\T\).
Consequently, \(\T\) is diagonalizable.
\end{example}

\EXAMPLE{5.2.3} and \EXAMPLE{5.2.4} suggest that an operator on \(V\) whose \CPOLY{} splits is diagonalizable \emph{if and only if} the dimension of \emph{each} eigenspace is equal to the algebraic multiplicity of the corresponding eigenvalue.
This is indeed true, as our next theorem shows.
Moreover, when the operator is diagonalizable, we can use \THM{5.5} to construct a \emph{basis} for \(V\) consisting of eigenvectors of the operator by \emph{collecting bases for the individual eigenspaces}.

\begin{theorem} \label{thm 5.8}
Let \(\T\) be a linear operator on a \emph{finite}-dimensional vector space \(V\) such that the \CPOLY{} of \(\T\) splits.
Let \(\lambda_1, \lambda_2, ..., \lambda_k\) be the \emph{distinct} eigenvalues of \(\T\).
Then
\begin{enumerate}
\item \(\T\) is diagonalizable if and only if the algebraic multiplicity of \(\lambda_i\) is equal to \(\dim(E_{\lambda_i})\), the geometric multiplicity of \(\lambda_i\), for all \(i\).
\item If \(\T\) is diagonalizable and \(\beta_i\) is an ordered basis for \(E_{\lambda_i}\) for each \(i\), then \(\beta = \beta_1 \cup \beta_2 \cup ... \cup \beta_k\) is an ``ordered basis''\RED{*} for \(V\) consisting of eigenvectors of \(\T\).
\end{enumerate}
\end{theorem}

\begin{remark} \label{remark 5.2.6}
We only consider the diagonalizability when the corresponding \CPOLY{} splits, since if it's not, then by (contrapositive of) \THM{5.6}, the operator is not diagonalizable.
\end{remark}

\begin{remark} \label{remark 5.2.7}
\RED{*}We regard \(\beta_1 \cup \beta_2 \cup ... \cup \beta_k\) as an ordered basis in the \emph{natural way} -- the vectors in \(\beta_1\) are listed first (in the same order as in \(\beta_1\)), then the vectors in \(\beta_2\) (in the same order as in \(\beta_2\)), etc.
\end{remark}

\begin{proof}
For each \(i\), let \(m_i\) denote the algebraic multiplicity of \(\lambda_i\), \(d_i = \dim(E_{\lambda_i})\), and \(n = \dim(V)\).

First, suppose that \(\T\) is diagonalizable.
So there exists a basis \(\beta\) for \(V\) consisting of \emph{eigenvectors} of \(\T\).
Now for each \(i\), let \(\beta_i = \beta \cap E_{\lambda_i}\), the set of vectors in \(\beta\) that are eigenvectors corresponding to \(\lambda_i\),
and let \(n_i\) denote the number of vectors in \(\beta_i\).
Then
\begin{center}
    \(n_i \le d_i\) for each \(i\), \quad \quad \MAROON{(1)}
\end{center}
because \(\beta_i\) is a \LID{} subset of a subspace of \(E_{\lambda_i}\) having dimension \(d_i\), and
\begin{center}
    \(d_i \le m_i\) by \THM{5.7}. \quad \quad \MAROON{(2)}
\end{center}

Clearly, \(\beta_i\) are a \emph{partition} of \(\beta\).
Hence
\begin{center}
    \(n_i\)'s sum to \(n\) because \(\beta\) contains \(n\) vectors. \quad \quad \MAROON{(3)}
\end{center}
\begin{center}
    The \(m_i\)'s also sum to \(n\), \quad \quad \MAROON{(4)}
\end{center}
since \(m_i\)'s by definition sum to the degree of the \RED{\emph{splitting}} \CPOLY{}, and (by \THM{5.3}) the degree of the \CPOLY{} of \(\T\) is \(n\).
(Note that if the \CPOLY{} does \emph{not} split, then \(\sum_{i = 1}^k m_i \ne n\).)
Thus
\begin{align*}
    n & = \sum_{i = 1}^k n_i & \text{by \MAROON{(3)}} \\
      & \le \sum_{i = 1}^k d_i & \text{by \MAROON{(1)}} \\
      & \le \sum_{i = 1}^k m_i & \text{by \MAROON{(2)}} \\
      & = n, & \text{by \MAROON{(4)}}
\end{align*}
which implies the ``less than or equal to sign'' is fact the ``equal'' sign
So
\[
    \sum_{i = 1}^k d_i = \sum_{i = 1}^k m_i \text{ and } \sum_{i = 1}^k (m_i - d_i) = 0. \quad \quad \MAROON{(5)}
\]
But again since \(m_i \ge d_i\) by \MAROON{(3)}, \(m_i - d_i \ge 0\), which by \MAROON{(5)} implies \(m_i - d_i = 0\) for all \(i\).
(If the sum of some \emph{nonnegative} numbers is zero, then all of them are equal to \(0\).)
Hence \(m_i = d_i\) for all \(i\).

Conversely, suppose that \(m_i = d_i\) for all \(i\).
We simultaneously show that \(\T\) is diagonalizable and prove part(b).
For each \(i\), let \(\beta_i\) be an ordered basis for \(E_{\lambda_i}\), and let \(\beta = \beta_1 \cup \beta_2 \cup ... \cup \beta_k\).
By \THM{5.5}, \(\beta\) is \LID{}.
Furthermore, since \(d_i = m_i\) for all \(i\), \(\beta\) contains
\begin{align*}
    \sum_{i = 1}^k d_i & = \sum_{i = 1}^k m_i & \text{by supposition} \\
        & = n
\end{align*}
vectors, where the last equal sign is similar to \MAROON{(4)}.
Therefore \(\beta\) is an ordered basis for \(V\) consisting of eigenvectors of \(V\), and we conclude that \(\T\) is diagonalizable.
\end{proof}

This theorem completes our study of the diagonalization problem.
We summarize our results.

\subsection{Test for Diagonalizability} \label{sec 5.2.1}

Let \(\T\) be a linear operator on an \(n\)-dimensional vector space \(V\).
Then \(\T\) is diagonalizable if and only if both of the following conditions hold.
\begin{enumerate}
\item[1.] The \CPOLY{} of \(\T\) splits. 
\item[2.] For each eigenvalue \(\lambda\) of \(\T\), the algebraic multiplicity of \(\lambda\) equals \(\nullity(\T - \lambda \ITRAN{})\), 
that is, the algebraic multiplicity of \(\lambda\) equals \(n - \rank(\T - \lambda \ITRAN{})\).
\end{enumerate}
These same conditions can be used to test if a square \emph{matrix} \(A\) is diagonalizable because diagonalizability of \(A\), by \DEF{5.1}, is equivalent to diagonalizability of the operator \(\LMTRAN_A\).

If \(\T\) is a diagonalizable operator and \(\beta_1, \beta_2, ..., \beta_k\) are ordered bases for the eigenspaces of \(\T\), then the union \(\beta = \beta_1 \cup \beta_2 \cup ... \cup \beta_k\) is an ordered basis for \(V\) consisting of eigenvectors of \(\T\),
and hence \([\T]_{\beta}\) is a diagonal matrix.

When testing \(\T\) for diagonalizability, it is usually easiest to \emph{choose a convenient basis \(\alpha\)} for \(V\) and work with \(B = [\T]_{\alpha}\).
If the \CPOLY{} of \(B\) splits, then use condition 2 above to check if the algebraic multiplicity of each eigenvalue of \(B\) equals \(n - \rank(B - \lambda I)\).
(Note that by \THM{5.7}, we can skip checking for condition 2, since condition 2 is \emph{automatically satisfied} for eigenvalues of algebraic multiplicity \(1\).)
If so, then \(B\), and hence \(\T\), is diagonalizable.

If \(\T\) is diagonalizable and a basis \(\beta\) for \(V\) consisting of eigenvectors of \(\T\) \emph{is} desired, then we first \emph{find a basis for each eigenspace} of \(B\).
The union of these bases is a basis \(\gamma\) \emph{for \(F^n\)} consisting of eigenvectors of \(B\).
Each vector in \(\gamma\) is the \emph{coordinate vector} relative to a of an eigenvector of \(\T\).
The set consisting of these \(n\) eigenvectors of \(\T\) is the desired basis \(\beta\).

Furthermore, if \(A\) is an \(n \X n\) diagonalizable matrix, we can use the \CORO{2.23.1} to find an \emph{invertible} \(n \X n\) matrix \(Q\) and a diagonal \(n \X n\) matrix \(D\) such that \(Q^{-1} A Q = D\).
The matrix \(Q\) has as its \emph{columns} the vectors in a basis of eigenvectors of \(A\), and \(D\) has as its \(j\)th diagonal entry the eigenvalue of \(A\) corresponding to the \(j\)th column of \(Q\).
We now consider some examples illustrating the preceding ideas.

\begin{example} \label{example 5.2.5}
We test the matrix
\[
    A = \begin{pmatrix} 3 & 1 & 0 \\ 0 & 3 & 0 \\ 0 & 0 & 4 \end{pmatrix} \in M_{3 \X 3}(\SET{R})
\]
for diagonalizability.
The \CPOLY{} of \(A\) is \(\det(A - t I) = -(t-4)(t-3)^2\), which splits, and so condition 1 of the the \ref{sec 5.2.1} is satisfied.
Also \(A\) has eigenvalues \(\lambda_1 = 4\) and \(\lambda_2 = 3\) with algebraic multiplicities \(1\) and \(2\), respectively.
Since \(\lambda_1\) has algebraic multiplicity \(1\), condition 2 is satisfied for \(\lambda_1\).
Thus we need only test condition 2 for \(\lambda_2\).
Because
\[
    A - \lambda_2 I = \begin{pmatrix} 0 & 1 & 0 \\ 0 & 0 & 0 \\ 0 & 0 & 1 \end{pmatrix}
\]
has rank \(2\), we see that \(3 - \rank(A - \lambda_2 I) = 1\), which is \emph{not} the algebraic multiplicity of \(\lambda_2\).
Thus condition 2 fails for \(\lambda_2\), and \(A\) is therefore not diagonalizable.
\end{example}

\begin{example} \label{example 5.2.6}
Let \(\T\) be the linear operator on \(\POLYRR\) defined by
\[
    \T(f(x)) = f(1) + f'(0)x + (f'(0) + f''(0))x^2.
\]
We first test \(\T\) for diagonalizability.
Let \(\alpha\) denote the (convenient) standard ordered basis for \(\POLYRR\) and \(B = [\T]_{\alpha}\).
Then
\[
    B = \begin{pmatrix} 1 & 1 & 1 \\ 0 & 1 & 0 \\ 0 & 1 & 2 \end{pmatrix}.
\]
The \CPOLY{} of \(B\), and hence of \(\T\), is \(-(t - 1)^2(t - 2)\), which splits.
Hence condition 1 of the test for diagonalization is satisfied.
Also \(B\) has the eigenvalues \(\lambda_1 = 1\) and \(\lambda_2 = 2\) with algebraic multiplicities \(2\) and \(1\), respectively.
Condition 2 is satisfied for \(\lambda_2\) because it has algebraic multiplicity \(1\).
So we need only verify condition 2 for \(\lambda_1 = 1\).
For this case,
\[
    3 - \rank(B - \lambda_1 I) = 3 - \rank \begin{pmatrix} 0 & 1 & 1 \\ 0 & 0 & 0 \\ 0 & 1 & 1 \end{pmatrix} = 3 - 1 = 2,
\]
which is equal to the algebraic multiplicity of \(\lambda_1\).
Therefore \(\T\) is diagonalizable.
We now find an ordered basis \(\gamma\) for \(\SET{R}^3\) consisting of eigenvectors of \(B\).
We consider each eigenvalue separately.
The eigenspace corresponding to \(\lambda_1 = 1\) is
\[
    E_{\lambda_1} = \left\{ \begin{pmatrix}
        x_1 \\ x_2 \\ x_3
    \end{pmatrix} \in \SET{R}^3 : \begin{pmatrix}
        0 & 1 & 1 \\ 0 & 0 & 0 \\ 0 & 1 & 1
    \end{pmatrix} \begin{pmatrix}
        x_1 \\ x_2 \\ x_3
    \end{pmatrix} = 0
    \right\},
\]
which is the solution space for the system
\[
    x_2 + x_3 = 0,
\]
and has
\[
    \gamma_1 = \left\{ \begin{pmatrix}
        1 \\ 0 \\ 0
    \end{pmatrix}, \begin{pmatrix}
        0 \\ -1 \\ 1
    \end{pmatrix} \right\}
\]
as a basis.
The eigenspace corresponding to \(\lambda = 2\) is
\[
    E_{\lambda_2} = \left\{ \begin{pmatrix}
        x_1 \\ x_2 \\ x_3
    \end{pmatrix} \in \SET{R}^3 : \begin{pmatrix}
        -1 & 1 & 1 \\ 0 & -1 & 0 \\ 0 & 1 & 0
    \end{pmatrix} \begin{pmatrix}
        x_1 \\ x_2 \\ x_3
    \end{pmatrix} = 0
    \right\},
\]
which is the solution space for the system
\[
    \sysdelim..\systeme{
        -x_1 + x_2 + x_3 = 0,
        x_2 = 0
    },
\]
and has
\[
    \gamma_2 = \left\{ \begin{pmatrix} 1 \\ 0 \\ 1 \end{pmatrix}\right\}
\]
as a basis.
Let
\[
    \gamma = \gamma_1 \cup \gamma_2 = 
    \left\{
        \begin{pmatrix} 1 \\ 0 \\ 0 \end{pmatrix},
        \begin{pmatrix} 0 \\ -1 \\ 1 \end{pmatrix},
        \begin{pmatrix} 1 \\ 0 \\ 1 \end{pmatrix}
    \right\}
\]
Then \(\gamma\) is an ordered basis for \(\SET{R}^3\) consisting of eigenvectors of \(B\).
Finally, observe that the vectors in \(\gamma\) are the coordinate vectors relative to \(\alpha\) of the vectors in the set
\[
    \beta = \{ 1, -x + x^2, 1 + x^2 \},
\]
which is an ordered basis for \(\POLYRR\) consisting of eigenvectors of \(\T\).
Thus
\[
    [\T]_{\beta} = \begin{pmatrix}
        \lambda_1 &         0 & 0 \\
                0 & \lambda_1 & 0 \\
                0 &         0 & \lambda_2
    \end{pmatrix} = \begin{pmatrix} 1 & 0 & 0 \\ 0 & 1 & 0 \\ 0 & 0 & 2 \end{pmatrix}.
\]
\end{example}

Our next example is an application of diagonalization that is of interest in \SEC{5.3}.

\begin{example} \label{example 5.2.7}
Let
\[
    A = \begin{pmatrix} 0 & -2 \\ 1 & 3 \end{pmatrix}.
\]
We show that \(A\) is diagonalizable and find a \(2 \X 2\) matrix \(Q\) such that \(Q^{-1} A Q\) is a diagonal matrix.
We then show how to \emph{use this result to compute \(A^{n}\)
for any positive integer \(n\)}.
First observe that the \CPOLY{} of \(A\) is \((t - 1)(t- 2)\), and hence \(A\) has two distinct eigenvalues, \(\lambda_1 = 1\) and \(\lambda_2 = 2\).
By applying the \CORO{5.5.1} to the operator \(\LMTRAN_A\), we see that \(A\) is diagonalizable.
Moreover,
\[
    \gamma_1 = \left\{ \begin{pmatrix}
        -2 \\ 1
    \end{pmatrix} \right\}
    \text{ and }
    \gamma_2 = \left\{ \begin{pmatrix}
        -1 \\ 1
    \end{pmatrix} \right\}
\]
are bases for the eigenspaces \(E_{\lambda_1}\) and \(E_{\lambda_2}\), respectively. Therefore
\[
    \gamma = \gamma_1 \cup \gamma_2 = \left\{ \begin{pmatrix}
        -2 \\ 1
    \end{pmatrix}, \begin{pmatrix}
        -1 \\ 1
    \end{pmatrix} \right\}
\]
is an ordered basis for \(\SET{R}^2\) consisting of eigenvectors of \(A\).
Let
\[
    Q = \begin{pmatrix}
        -2 & -1 \\ 1 & 1
    \end{pmatrix},
\]
the matrix whose \emph{columns} are the vectors in \(\gamma\).
Then, by the \CORO{2.23.1},
\[
    D = Q^{-1} A Q = [\LMTRAN_A]_{\gamma} = \begin{pmatrix} 1 & 0 \\ 0 & 2 \end{pmatrix}.
\]
To find \(A^n\) for any positive integer \(n\), observe that \(A = QDQ^{-1}\).
Therefore
\begin{align*}
    A^n & = (QDQ^{-1})^n \\
        & = (QDQ^{-1})(QDQ^{-1}) ... (QDQ^{-1}) \\
        & = QD^nQ^{-1} = Q \begin{pmatrix} 1 & 0 \\ 0 & 2 \end{pmatrix}^n Q^{-1} \\
        & =\RED{*} Q \begin{pmatrix} 1^n & 0 \\ 0 & 2^n \end{pmatrix} Q \\
        & = \begin{pmatrix} -2 & 1 \\ 1 & 1 \end{pmatrix}\begin{pmatrix} 1 & 0 \\ 0 & 2^n \end{pmatrix}\begin{pmatrix} -1 & -1 \\ 1 & 2 \end{pmatrix} \\
        & = \begin{pmatrix} 2 - 2^n & 2 - 2^{n + 1} \\ -1 + 2^n & -1 + 2^{n + 1} \end{pmatrix}.
\end{align*}
\end{example}

\begin{note}
\RED{*}: The \(n\)th power of a diagonal matrix is equal to the diagonal matrix whose \(ii\)-entry is equal to the \(n\)th power of the \(ii\)-entry of the original matrix.
Need to prove, but trivial.
\end{note}

We now consider an application that uses diagonalization to \emph{solve a system of differential equations}.

\subsection{Systems of Differential Equations} \label{sec 5.2.2}

Consider the system of differential equations
\[
    \sysdelim..\systeme{
        x'_1 = 3 x_1 + x_2 + x_3,
        x'_2 = 2 x_1 + 4 x_2 + 2 x_3,
        x'_3 = - x_1 - x_2 + x_3
    }
\]
where, for each \(i\), \(x_i = x_i(t)\) is a \emph{differentiable real-valued function of the real variable \(t\)}.
Clearly, this system has a (trivial) solution, namely, the solution in which each \(x_i(t)\) is the zero function.
We determine all of the solutions to this system
Let \(x : \SET{R} \to \SET{R}^3\) be the function defined by
\[
    x(t) = \begin{pmatrix} x_1(t) \\ x_2(t) \\ x_3(t) \end{pmatrix}.
\]
The \textbf{derivative} of \(x\), denoted \(x'\), is \textbf{defined} by
\[
    x'(t) = \begin{pmatrix} x'_1(t) \\ x'_2(t) \\ x'_3(t) \end{pmatrix}.
\]
Let
\[
    A = \begin{pmatrix} 3 & 1 & 1 \\ 2 & 4 & 2 \\ -1 & -1 & 1 \end{pmatrix}
\]
be the \emph{coefficient matrix of} the given system, so that with these definitions, we can rewrite the system as the matrix equation \(x' = Ax\).
Now we test whether \(A\) is diagonalizable.
Then is can be verified that for
\[
    Q = \begin{pmatrix}
        -1 & 0 & -1 \\ 0 & -1 & -2 \\ 1 & 1 & 1
    \end{pmatrix}
    \text{ and }
    D = \begin{pmatrix}
        2 & 0 & 0 \\ 0 & 2 & 0 \\ 0 & 0 & 4
    \end{pmatrix},
\]
we have \(Q^{-1}AQ = D\), so that \(A = Q D Q^{-1}\), and
\begin{align*}
             & A = Q D Q^{-1} \land x' = Ax \\
    \implies & x' = QDQ^{-1} x \\
    \implies & Q^{-1}x' = Q^{-1} Q D Q^{-1} x \\
    \implies & Q^{-1}x' = D \RED{Q^{-1} x} \quad \quad \quad \MAROON{(1)}
\end{align*}
The function \(y = y(t) = \RED{Q^{-1} x}(t)\) can be shown to be differentiable, and \(y' = Q^{-1} x'\)
(see \EXEC{5.2.17}.)
Hence by replacing \(y\) and \(y'\) into \MAROON{(1)}, we have \(y' = Dy\).
Since \(D\) is a diagonal matrix, the system \(y' = Dy\) is easy to solve;
and after solving \(y\), since \(y = Q^{-1}x\), we have
\[
    x = Qy, \quad \quad \quad \MAROON{(2)}
\]
hence we can solve \(x\).

Although we let \(y = Q^{-1}x\), we just denote \(y = y(t)\) as
\[
    y(t) = \begin{pmatrix}
        y_1(t) \\ y_2(t) \\ y_3(t)
    \end{pmatrix}.
\]
And we can rewrite \(y' = Dy\) as
\[
    \begin{pmatrix}
        y_1'(t) \\ y_2'(t) \\ y_3'(t)
    \end{pmatrix}
    = \begin{pmatrix}
        2 & 0 & 0 \\ 0 & 2 & 0 \\ 0 & 0 & 4
    \end{pmatrix}
    \begin{pmatrix}
        y_1(t) \\ y_2(t) \\ y_3(t)
    \end{pmatrix}
    = \begin{pmatrix}
        2y_1(t) \\ 2y_2(t) \\ 4y_3(t)
    \end{pmatrix}.
\]
So \(y' = Dy\) corresponds to the system
\begin{align*}
    y_1' = 2y_1 \\
    y_2' = 2y_2 \\
    y_3' = 4y_3.
\end{align*}
Each equation of the system is independent of each other, and thus can be solved individually.
It is easily seen (as in \EXAMPLE{5.1.3}) that the general solution to these equations is \(y_1(t) = c_1e^{2t}, y_2(t) = c_2e^{2t}\), and \(y_3(t) = c_3e^{4t}\),
where \(c_1, c_2\), and \(c_3\) are arbitrary constants.
Finally,
\begin{align*}
    \begin{pmatrix} x_{1}(t) \\ x_{2}(t) \\ x_{3}(t)     \end{pmatrix}
    & = x(t) = Q y(t) \quad \quad \quad \text{by \MAROON{(2)}} \\
    & = \begin{pmatrix} -1 & 0 & -1 \\ 0 & -1 & -2 \\ 1 & 1 & 1 \end{pmatrix}
    \begin{pmatrix} c_{1} e^{2 t} \\ c_{2} e^{2 t} \\ c_{3} e^{4 t} \end{pmatrix} \\
    & = \begin{pmatrix}
        -c_{1} e^{2 t}-c_{3} e^{4 t} \\
        -c_{2} e^{2 t} - 2 c_3 e^{4 t} \\
        c_{1} e^{2 t}+c_{2} e^{2 t} + c_{3} e^{4 t} \\
    \end{pmatrix}
\end{align*}

yields the general solution of the original system.
Note that this solution can be written as

\begin{align*}
    x(t) = e^{2 t} \cleft[red][
        c_{1}\begin{pmatrix} -1 \\ 0 \\ 1 \end{pmatrix}
        + c_{2} \begin{pmatrix} 0 \\ -1 \\ 1 \end{pmatrix}
    \cright[red]]
    + e^{4 t} \cleft[red][
        c_{3} \begin{pmatrix} -1 \\ -2 \\ 1 \end{pmatrix}
    \cright[red]]
\end{align*}

The expressions \emph{\RED{in brackets}} are arbitrary vectors in \(E_{\lambda_1}\) and \(E_{\lambda_2}\), respectively,
where \(\lambda_1 = 2\) and \(\lambda_2 = 4\).
Thus the general solution of the original system is \(x(t) = e^{2t} z_1 + e^{4t} z_2\),
where \(z_1 \in E_{\lambda_1}\) and \(z_2 \in E_{\lambda_2}\).
This result is generalized in \EXEC{5.2.16}.

\begin{note}
因為(我)沒學過微分方程,注意最後一段,雖然從\ \(x(t) = e^{2t} z_1 + e^{4t} z_2\) 這個\ equation 可看出來,但還是提醒一下,\(x(t)\) 是一個\ ``vector of functions'',而不是一個\ function。
\end{note}

\subsection{Direct Sums} \label{sec 5.2.3}

\begin{note}
The concept of direct sum has occurred in many previous exercises.
\end{note}

Let \(\T\) be a linear operator on a \emph{finite}-dimensional vector space \(V\).
There is a way of \emph{decomposing} \(V\) into \emph{simpler subspaces} that \emph{offers insight} into the behavior of \(\T\).
This approach is especially useful in \CH{7}, where we study non-diagonalizable linear operators.
In the case of diagonalizable operators, the simpler subspaces are the eigenspaces of the operator.

\begin{definition} \label{def 5.8}
Let \(W_1, W_2, ..., W_k\) be subspaces of a vector space \(V\). we define the \textbf{sum} of these subspaces to be the set
\[
    \{ v_1 + v_2 + ... + v_k : v_i \in W_i \text{ for } 1 \le i \le k \},
\]
which we denote by \(W_1 + W_2 + ... + W_k\) or \(\sum_{i = 1}^k W_i\).
It is a simple exercise (See \EXEC{1.3.23}) to show that the sum of subspaces of a vector space is also a subspace.
\end{definition}

\begin{example} \label{example 5.2.8}
Let \(V = \SET{R}^3\), let \(W_1\) denote the \(xy\)-plane, and let \(W_2\) denote the \(yz\)-plane.
Then \(\SET{R}^3 = W_1 + W_2\) because, for any vector \((a, b, c) \in \SET{R}^3\),
we have \((a, b, c) = (a, 0, 0) + (0, b, c)\), where \((a, 0, 0) \in W_1\) and \((0, b, c) \in W_2\).
\end{example}

Notice that in \EXAMPLE{5.2.8} the representation of \((a, b, c)\) as a sum of vectors in \(W_1\) and \(W_2\) \emph{is not unique}.
For example, \((a, b, c) = (a, b, 0) + (0, 0, c)\) is
another representation.
Because we are often interested in sums for which representations are unique, we introduce a condition that assures this outcome.
The definition of \emph{direct sum} that follows is a \emph{generalization} of the definition given in the exercises of \SEC{1.3}.
(See \ADEF{1.8}.)

\begin{definition} \label{def 5.9}
Let \(W, W_1, W_2, ..., W_k\) be subspaces of a vector space \(V\) such that \(W_1 \subseteq W\) for \(i = 1, 2, ..., k\).
We call W the \textbf{direct sum} of the subspaces \(W_1, W_2, ..., W_k\), and write \(W = W_1 \oplus W_2 \oplus ... \oplus W_k\), if
\[
    W = \sum_{i = 1}^k W_i \quad \text{ and } \quad W_j \cap \sum_{i \ne j} W_i = \{ \OV \} \text{ for each } j (1 \le j \le k).
\]
\end{definition}

\begin{example} \label{example 5.2.9}
In \(\SET{R}^5\), let \(W = \{(x_1, x_2, x_3, x_4, \RED{x_5}) : \RED{x_5} = 0 \}\),
\(W_1 = \{(a, b, 0, 0, 0): a, b \in \SET{R} \}\), \(W_2 = \{ (0, 0, c, 0, 0): c \in \SET{R}\}\),
and \(W_3 = \{(0, 0, 0, d, 0): d \in \SET{R} \}\).
For any \((a, b, c, d, 0) \in W\),
\((a, b, c, d, 0) = (a, b, 0, 0, 0) + (0, 0, c, 0, 0) + (0, 0, 0, d, 0) \in W_1 + W_2 + W_3\).
Thus
\[
    W = \sum_{i = 1}^3 W_i.
\]
To show that \(W\) is the direct sum of \(W_1, W_2\), and \(W_3\), we must prove that \(W_1 \cap (W_2 + W_3) = W_2 \cap (W_1 + W_3) = W_3 \cap (W_1 + W_2) = \{ \OV \}\).
But these equalities are obvious, and so \(W = W_1 \oplus W_2 \oplus W_3\).
\end{example}

Our next result contains several conditions that are \emph{equivalent} to the definition of a direct sum.

\begin{theorem} \label{thm 5.9}
Let \(W_1, W_2, ..., W_k\) be subspaces of a finite-dimensional vector space \(V\).
The following conditions are equivalent.

\begin{enumerate}
\item \(v = W_1 \oplus W_2 \oplus ... \oplus W_k\).
\item \(V = \sum_{i = 1}^k W_i\) and, for any vectors \(v_1, v_2, ..., v_k\) such that \(v_i \in W_i
(1 \le i \le k)\), if \(v_1 + v_2 + ... + v_k = \OV\), then \(v_i = \OV\) for all \(i\).
\item Each vector \(v \in V\) can be \emph{uniquely written} as \(v = v_1 + v_2 + ... + v_k\) where \(v_i \in W_i\).
\item\RED{*} \emph{For any} set of ordered bases \(\gamma_1, \gamma_2, ..., \gamma_k\) for \(W_1, W_2, ..., W_k\), \(\gamma_1 \cup \gamma_2 \cup ... \cup \gamma_k\) is an ordered basis for \(V\).
\item\RED{**} \emph{There exists} a set of ordered bases \(\gamma_1, \gamma_2, ..., \gamma_k\) for \(W_i (1 \le i \le k)\), such that \(\gamma_1 \cup \gamma_2 \cup ... \cup \gamma_k\) is an ordered basis for \(V\).
\end{enumerate}
\end{theorem}

\begin{remark} \label{remark 5.2.8}
Note that I have modified the statements in part(d) and part(e).
It seems that the modified statements is more clear(to me).
And clearly, since each \(W_i\) of course \emph{exists} an ordered basis, this fact and part(d) imply part(e).
\end{remark}

\begin{proof}
We prove them by proving (a) \(\implies\) (b) \(\implies\) (c) \(\implies\) (d) \(\implies\) (e) \(\implies\) (a).

So assume (a).
In particular by \DEF{5.9} we have
\[
    V = \sum_{i = 1}^k W_i. \quad \quad \MAROON{(a.1)}
\]
Now suppose that \(v_1, v_2, ..., v_k\) are any vectors such that \(v_i \in W_i\) for all \(i\) and \(v_1 + v_2 + ... + v_k = \OV\).
Then for any \(j\),
\begin{align*}
    -v_j & = \sum_{i \RED{\ne} j} v_i & \text{of course} \\
         & \in \sum_{i \RED{\ne} j} W_i \quad \quad \MAROON{(a.2)} & \text{by \MAROON{(a.1)}}
\end{align*}
But \(-v_j \in W_j\), so with \MAROON{(a.2)}, \(-v_j \in W_j \cap \sum_{i \ne j} W_i\).
But by the assumption of (a), that is, by the second condition of \DEF{5.9},
\[
    -v_j \in W_j \cap \sum_{i \ne j} W_i = \{ \OV \}.
\]
Hence \(v_j = \OV\) for any \(j\), proving (b).

Now assume (b). We prove (c).
Let arbitrary \(v \in V\).
By (b), since \(V = \sum_{i = 1}^k W_i\), there exist vectors \(v_1, v_2, ..., v_k\) such that \(v_i \in W_i\) and \(v = v_1 + v_2 + ... + v_k\).
So the existence part is proved.
Now we must show that this representation \emph{is unique}. Suppose also that \(v = w_1 + w_2 + ... + w_k\), where \(w_i \in W_i\) for all \(i\).
Then
\begin{align*}
             & v_1 + v_2 + ... + v_k = v = w_1 + w_2 + ... + w_k \\
    \implies & (v_1 - w_1) + (v_2 - w_2) + ... + (v_k - w_k) = \OV \quad \quad \MAROON{(b.1)}
\end{align*}
But \(v_i - w_i \in W_i\) for all \(i\), so with \MAROON{(b.1)} and assumption of (b), \(v_i - w_i = \OV\) for all \(i\).
Thus \(v_i = w_i\) for all \(i\), proving the uniqueness of the representation.

Now assume (c). We prove (d).

For each \(i\), let \(\gamma_i\) be an ordered basis for \(W_i\).
From the statement of (c), it's clear that
\[
    V = \sum_{i = 1}^k W_i,
\]
and it follows that \(\gamma_1 \cup \gamma_2 \cup ... \cup \gamma_k\) \emph{generates} \(V\).
So it suffices to show \(\gamma_1 \cup \gamma_2 \cup ... \cup \gamma_k\) is \LID{}.
Now let \(m_i\) be the number of elements of \(\gamma_i\), and consider vectors \(v_{ij}\), the \(j\)th vector in \(\gamma_i\), for \(1 \le i \le k\) and \(1 \le j \le m_i\), such that
\[
    \sum_{i, j} a_{ij} v_{ij} = \OV. \quad \quad \MAROON{(c.1)}
\]
For each \(i\), set
\[
    w_i = \sum_{j = 1}^{\RED{m_i}} a_{ij} v_{ij}. \quad \quad \MAROON{(c.2)}
\]
Then for each \(i\), \(w_i \in \spann(\gamma_i) = W_i\) and with \MAROON{(c.1)},
\[
    w_1 + w_2 + ... + w_k = \sum_{i, j} a_{ij} v_{ij} = \OV. \quad \quad \MAROON{(c.3)}
\]
But we already have the \emph{trivial} representation of \(\OV\) using \(W_i\)'s vector, that is
\[
    0 + 0 + ... + 0 = \OV \text{ for each } 0 \in W_i.
\]
So by the assumption of (c), \(0 + 0 + ... + 0\) where each \(0 \in W_i\) is \textbf{the unique} representation of \(\OV\), which (by \MAROON{(c.3)}) implies \(w_i = \OV\) for all \(i\).
Thus by \MAROON{(c.2)},
\[
    \OV = w_i = \sum_{j = 1}^{\RED{m_i}} a_{ij} v_{ij} \text{ for each } i.
\]
But each \(\gamma_i\) is \LID{}, and \(v_{ij}\) are vectors in \(\gamma_i\), and hence for each \(i\), \(a_{ij} = 0\) for all \(j\).
Consequently, by \MAROON{(c.1)}, \(\gamma_1 \cup \gamma_2 \cup ... \cup \gamma_k\) is \LID{} and a generating set of \(V\) and therefore is a basis for \(V\).

Now by \RMK{5.2.8}, (e) follows immediately from (d).

Finally, we assume (e) and prove (a).
For each \(i\), let \(\gamma_i\) be an ordered basis for \(W_i\) such that \(\gamma_1 \cup \gamma_2 \cup ... \cup \gamma_k\) is an ordered basis for \(V\).
Then we use \EXEC{1.4.14} \emph{inductively}:
\begin{align*}
    V & = \spann(\gamma_1 \cup \gamma_2 \cup ... \cup \gamma_k) & \text{of course} \\
      & = \spann(\gamma_1) + \spann(\gamma_2 \cup ... \gamma_k) & \text{by \EXEC{1.4.14}} \\
      & \quad \vdots & \text{apply \EXEC{1.4.14} inductively} \\
      & = \spann(\gamma_1) + \spann(\gamma_2) + ... + \spann(\gamma_k) \\
      & = W_1 + W_2 + ... + W_k & \text{since \(\gamma_i\) is a basis for \(W_i\)}
\end{align*}
Now it suffices to show that for all \(j\),
\[
    W_j \cap \sum_{i \ne j} W_{i} = \{ \OV \}. \quad \quad \MAROON{(e.1)}
\]
For the sake of contradiction, suppose not.
So for some \(j\), there is a \emph{nonzero} vector \(v\) such that
\[
    v \in W_j \cap \sum_{i \ne j} W_i.
\]
Then
\[
    v \in W_j = \spann(\gamma_j)
    \quad \text{ and } \quad
    v \in \sum_{i \ne j} W_i = \spann \left( \bigcup_{i \ne j} \gamma_i \right)
\]
where similar to the derivation of \(V = W_1 + ... + W_k\) before, the second equation is derived by applying \EXEC{1.4.14} inductively.
Then we have two \emph{different} linear combination of \(v \ne \OV\) of \(\gamma_1 \cup \gamma_2 \cup ... \cup \gamma_k\): one is only using \(\gamma_j\) and the other is using \(\bigcup_{i \ne j} \gamma_i\).
But since \(\gamma_1 \cup ... \cup \gamma_k\) is a \emph{basis} for \(V\) and \(v \in V\), by \THM{1.8}, \(v\) must only have unique representation using \(\gamma_1 \cup ... \cup \gamma_k\), so we have a contradiction.
Hence \MAROON{(e.1)} is true, proving (a).
\end{proof}

With the aid of \THM{5.9}, we are able to characterize diagonalizability \emph{in terms of direct sums}.

\begin{theorem} \label{thm 5.10}
A linear operator \(\T\) on a finite-dimensional vector space \(V\) is diagonalizable if and only if \(V\) is \emph{the direct sum of the eigenspaces of \(\T\)}.
\end{theorem}

\begin{proof}
Let \(\lambda_1, \lambda_2, ..., \lambda_k\) be the distinct eigenvalues of \(\T\).

First suppose that \(\T\) is diagonalizable, and for each \(i\) choose an ordered basis \(\gamma_i\) for the eigenspace \(E_{\lambda_i}\).
By \THM{5.8}, \(\gamma_1 \cup \gamma_2 \cup ... \cup \gamma_k\) is a basis for \(V\), satisfying \THM{5.9}(d), and hence by the equivalent \THM{5.9}(a), \(V\) is a direct sum of the \(E_{\lambda_i}\)'s.

Conversely, suppose that \(V\) is a direct sum of the eigenspaces of \(\T\).
(That is, suppose \THM{5.9}(a) is satisfied.)
For each \(i\), choose an ordered basis \(\gamma_i\) of \(E_{\lambda_i}\).
Then by the equivalent \THM{5.9}(d) (of \THM{5.9}(a)), the union \(\gamma_1 \cup \gamma_2 \cup ... \cup \gamma_k\) is a basis for \(V\).
Since this basis \emph{consists of eigenvectors} of \(\T\), (by \THM{5.1}) we conclude that \(\T\) is diagonalizable.
\end{proof}

\begin{example} \label{example 5.2.10}
Let \(\T\) be the linear operator on \(\SET{R}^4\) defined by
\[
    \T(a, b, c, d) = (a, b, 2c, 3d).
\]
It is easily seen that \(\T\) is diagonalizable with eigenvalues \(\lambda_1 = 1, \lambda_2 = 2\), and \(\lambda_3 = 3\).
Furthermore, if we \RED{``remove the fifth component''} of the subspaces \(W_1, W_2\), and \(W_3\) of \EXAMPLE{5.2.9}, then \(W_1, W_2, W_3\) of that example coincide with the corresponding eigenspaces of \(\lambda_1, \lambda_2, \lambda_3\).
Thus by \THM{5.10}, since \(\T\) is diagonalizable, \(\SET{R}^4 = W_1 \oplus W_2 \oplus W_3\).
\end{example}
