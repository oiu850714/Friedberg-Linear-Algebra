\setcounter{chapter}{-1}
\chapter{Precautions}

\begin{note}
The colors used in proofs are just for notational purpose.
For example: given subspaces \(W_1, W_2\) of \(V\), we have \(\OV \in W_1 + W_2\) since \(\OV = \BLUE{\OV} + \GREEN{\OV}\) where \(\BLUE{\OV} \in W_1\) and \(\GREEN{\OV} \in W_2\).
In this case, \(\OV, \BLUE{\OV}, \GREEN{\OV}\) are exactly the same.
\end{note}

\begin{note}
I changed the indexing of the theorems/corollaries/examples/exercises/etc in the textbook.

Rules:
\begin{itemize}
\item The definitions in the book have no index, so I add the index, which has the format <chapter>-<starting number>.
\item The examples in the book have index which restarts at every section, and I change the index format as <chapter>-<section>-<starting number>.
\item Exercise: similarly as examples.
\item The theorems in the book have index which restarts at every chapter, and I change the index format as <chapter>-<starting number>.
\item The corollary in the book \emph{always} follows behind a theorem, and I change the index format as <chapter>-<theorem starting number>-<corollary number>.
    Some corollary has no number, in this case I just treat it as corollary 1.
\end{itemize}

In addition, I add \emph{additional definition} for some definitions implicitly stated(without definitions or even examples) in textbook.

Similarly, for some propositions which are proved in the exercises and are likely used in later chapters, I add \emph{additional theorem} which just refer to these exercises.
This is convenient for searching all useful propositions in the book but prevent me from looking for these propositions in the tons of exercises.
\end{note}