\exercisesection

\begin{exercise} \label{exercise 3.4.1}
Label the following statements as true or false.
\begin{enumerate}
\item If \((A'|b')\) is obtained from \((A|b)\) by a finite sequence of elementary \emph{column} operations, then the systems \(Ax = b\) and \(A'x = b'\) are equivalent.
\item If \((A'|b')\) is obtained from \((A|b)\) by a finite sequence of e.r.o.s, then the systems \(Ax = b\) and \(A' x\) are equivalent.
\item If \(A\) is an \(n \X n\) matrix with rank \(n\), then the reduced row echelon form of \(A\) is \(I_n\).
\item Any matrix can be put in reduced row echelon form by means of a finite sequence of elementary row operations.
\item If \((A|b)\) is in reduced row echelon form, then the system \(Ax = b\) is consistent.
\item Let \(Ax = b\) be a system of \(m\) linear equations in \(n\) unknowns for which the augmented matrix is in reduced row echelon form.
If this system is consistent, then the dimension of the solution set of \(Ax = 0\) is \(n - r\), where \(r\) equals the number of nonzero rows in \(A\).
\item If a matrix \(A\) is transformed by elementary row operations into a matrix \(A'\) in reduced row echelon form, then the number of nonzero rows in \(A'\) equals the rank of \(A\).
\end{enumerate}
\end{exercise}

\begin{proof} \ 

\begin{enumerate}
\item False. Counterexample is easy to find.
\item True by \CORO{3.13.1}.
\item True. Let \(B\) be the r.r.e.f. of \(A\), by \THM{3.16}(b), we have that the \(i\)th column of is \(e_i\) for \(1 \le i \le \RED{n}\), which implies \(B = I_n\).
\item True by \THM{3.14}.
\item False by \EXEC{3.4.3}.
\item True.
    \(A\) has \(r\) nonzero rows, so \(Ax = b\) is a system of \(r\) nonzero equations of \(n\) unknowns.
    And \((A|b)\) is in r.r.e.f..
    Also \(Ax = b\) is consistent.
    So the requirements of \THM{3.15} are satisfied.
    So by \THM{3.15}(b), the dimension of the corresponding homogeneous system \(Ax = 0\) is \(n - r\), as desired.
\item True by \RMK{3.4.2} and \CORO{3.4.1}.
\end{enumerate}
\end{proof}

\begin{exercise} \label{exercise 3.4.2}
Calculation problem, skip.
\end{exercise}

\begin{exercise} \label{exercise 3.4.3}
Suppose that the augmented matrix of a system \(Ax = b\) is transformed into a matrix \((A'|b')\) in reduced row echelon form by a finite sequence of elementary row operations.
\begin{enumerate}
\item Prove that \(\rank(A') \ne \rank(A'|b')\) if and only if \((A'|b')\) contains a row in which the only nonzero entry lies in the last column.

\item Deduce that \(Ax = b\) is consistent if and only if \((A'|b')\) contains no row in which the only nonzero entry lies in the last column.
\end{enumerate}
\end{exercise}

\begin{proof}
WLOG, suppose \(A\) is of size \(m \X n\).
\begin{enumerate}
\item
\(\Longrightarrow\): For the sake of contradiction, suppose \(r = \rank(A') \ne \rank(A'|b')\), \RED{but} \((A'|b')\) does not contain a row in which the only nonzero entry lies in the last column.
Then (by \RMK{3.4.2}) since \(A'\) only contains \(r\) nonzero rows, \((A|b')\) must be in the form
\[
    (A|b') =
    \left(\begin{array}{rrr|r}
             A'_{11} &       ... &      A'_{1n} & b'_1 \\
              \vdots &           &       \vdots & \vdots \\
             A'_{r1} &       ... &      A'_{rn} & b'_r \\
             \RED{0} & \RED{...} &      \RED{0} & \RED{0} \\
        \RED{\vdots} &           & \RED{\vdots} & \RED{\vdots} \\
             \RED{0} & \RED{...} &      \RED{0} & \RED{0}
    \end{array}\right),
\]
that is, the \(r + 1, ..., m\) components of \(b'\) must be zero. (Otherwise \((A'|b')\) \emph{does} contain a row in which the only nonzero entry lies in the last column.)
But then since by \THM{3.16}(b) we know \(e_1, e_2, ..., e_r\) are columns of \(A'\), we have \(b' = b_1 e_1 + b_2 e_2 + ... + b_r e_r\), so \(b'\) is in the column space of \(A'\).
But that implies \(\rank(A') = \rank(A'|b')\), which is also a contradiction!
So \((A'|b')\) \textbf{must contain} a row in which the only nonzero entry lies in the last column.

\(\Longleftarrow\):
If \((A'|b')\) contains a row in which the only nonzero entry lies in the last column, then by the structure of r.r.e.f., the last column is of course \LID{} to the remaining columns of \((A'|b')\), that is, the last column is \LID{} to the columns of \(A'\).
Hence \(\rank(A') \ne \rank(A'|b')\).

\item \(Ax = b\) is consistent, if and only if (by \THM{3.13}) \(A'x = b'\) is consistent, if and only if (by part(a)) \((A'|b')\) does \emph{not} contain a row in which the only nonzero entry lies in the last column.
\end{enumerate}
\end{proof}

\begin{exercise} \label{exercise 3.4.4}
Calculation problem, skip.
\end{exercise}

\begin{exercise} \label{exercise 3.4.5}
Let the reduced row echelon form of \(A\) be
\[
    B = \left(\begin{array}{rrrrr}
        \RED{1} & \RED{0} & 2 & \RED{0} & -2 \\
        \RED{0} & \RED{1} & -5 & \RED{0} & -3 \\
        \RED{0} & \RED{0} & 0 & \RED{1} & 6
    \end{array}\right)
\]
Determine \(A\) if the first, second, and fourth columns of \(A\) are
\[
    \begin{pmatrix} 1 \\ -1 \\ 3 \end{pmatrix},
    \begin{pmatrix} 0 \\ -1 \\ 1 \end{pmatrix},
    \text { and }
    \begin{pmatrix} 1 \\ -2 \\ 0 \end{pmatrix}
\]
respectively.
\end{exercise}

\begin{proof}
Let \(a_i\) be the \(i\)th column of \(A\), and \(b_i\) be the \(i\)th column of \(B\).
Since \(b_3 = 2 e_1 + (-5) e_2 = 2 b_1 + (-5) b_2\), by \THM{3.16}(d), \(a_3 = 2 a_1 + (-5) a_2\).
So
\[
    a_3 = 2 \begin{pmatrix} 1 \\ -1 \\ 3 \end{pmatrix}
        + (-5) \begin{pmatrix} 0 \\ -1 \\ 1 \end{pmatrix}
        = \begin{pmatrix} 2 \\ 3 \\ 1 \end{pmatrix}.
\]
Similarly, since \(b_5 = (-2) e_1 + (-3) e_2 + 6 e_3 = (-2) b_1 + (-3) b_2 + 6 b_4\), by \THM{3.16}(d), \(a_5 = (-2) a_1 + (-3) a_2 + 6 a_4\).
So
\[
    a_5 = (-2) \begin{pmatrix} 1 \\ -1 \\ 3 \end{pmatrix}
        + (-3) \begin{pmatrix} 0 \\ -1 \\ 1 \end{pmatrix}
        + (6) \begin{pmatrix} 1 \\ -2 \\ 0 \end{pmatrix}
        = \begin{pmatrix} 4 \\ -7 \\ -9 \end{pmatrix}.
\]
Hence
\[
    A = \left(\begin{array}{ccccc}
        1 & 0 & 2 & 1 & 4 \\
        -1 & -1 & 3 & -2 & -7 \\
        3 & 1 & 1 & 0 & -9
    \end{array}\right)
\]
\end{proof}

\begin{exercise} \label{exercise 3.4.6}
Let the reduced row echelon form of \(A\) be
\[
    B = \left(\begin{array}{rrrrrr}
        \RED{1} & -3 & \RED{0} & 4 & \RED{0} & \GREEN{5} \\
        \RED{0} & 0 & \RED{1} & 3  & \RED{0} & \GREEN{2} \\
        \RED{0} & 0 & \RED{0} & 0  & \RED{1} & \GREEN{-1} \\
        \RED{0} & 0 & \RED{0} & 0  & \RED{0} & \GREEN{0}
    \end{array}\right)
\]
Determine \(A\) if the \RED{first}, \RED{third}, and \GREEN{six} columns of \(A\) are
\[
    \begin{pmatrix} 1 \\ -2 \\ -1 \\ 3 \end{pmatrix},
    \begin{pmatrix} -1 \\ 1 \\ 2 \\ -4 \end{pmatrix},
    \text { and }
    \begin{pmatrix} 3 \\ -9 \\ 2 \\ 5 \end{pmatrix},
\]
respectively.
\end{exercise}

\begin{proof}
This problem is a variation of \EXEC{3.4.5}.

Let \(a_i\) be the \(i\)th column of \(A\), and \(b_i\) be the \(i\)th column of \(B\).
We should solve \(\RED{a_5}\) first, since (by \THM{3.16}(c)) it is the corresponding \LID{} vector of \(b_5\).

Since \(b_6 = 5 e_1 + 2 e_2 + (-1) e_3 = 5 b_1 + 2 b_3 + (-1) b_5\), by \THM{3.16}(d), \(a_6 = 5 a_1 + 2 a_3 + (-1) a_5\).
That is, \(a_5 = 5 a_1 + 2 a_3 + (-1) a_6\).
So
\[
    a_5 = 5 \begin{pmatrix} 1 \\ -2 \\ -1 \\ 3 \end{pmatrix}
        + 2 \begin{pmatrix} -1 \\ 1 \\ 2 \\ -4 \end{pmatrix}
        + (-1) \begin{pmatrix} 3 \\ -9 \\ 2 \\ 5 \end{pmatrix}
        = \begin{pmatrix} 0 \\ 1 \\ -3 \\ 2 \end{pmatrix}.
\]
And Since \(b_2 = (-3) e_1 = (-3) b_1\), by \THM{3.16}(d), \(a_2 = (-3) a_1\).
So
\[
    a_2 = (-3) \begin{pmatrix} 1 \\ -2 \\ -1 \\ 3 \end{pmatrix}
        = \begin{pmatrix} -3 \\ 6 \\ 3 \\ -9 \end{pmatrix}.
\]
Finally, since \(b_4 = 4 e_1 + 3 e_2 = 4 b_1 + 3 b_3\), by \THM{3.16}(d), \(a_4 = 4 a_1 + 3 a_3\).
So
\[
    a_4 = 4 \begin{pmatrix} 1 \\ -2 \\ -1 \\ 3 \end{pmatrix}
        + 3 \begin{pmatrix} -1 \\ 1 \\ 2 \\ -4 \end{pmatrix}
        = \begin{pmatrix} 1 \\ -5 \\ 2 \\ 0 \end{pmatrix}.
\]
Hence
\[
    A = \left(\begin{array}{cccccc}
        1 & -3 & -1 & 1 & 0 & 3 \\
        -2 & 6 & 1 & -5 & 1 & -9 \\
        -1 & 3 & 2 & 2 & -3 & 2 \\
        3 & -9 & -4 & 0 & 2 & 5
    \end{array}\right).
\]
\end{proof}

\begin{exercise} \label{exercise 3.4.7}
\sloppy It can be shown that the vectors \(u_1 = (2, -3, 1), u_2 = (1, 4, -2), u_3 = (-8, 12, -4), u_4 = (1, 37, -17)\), and \(u_5 = (-3, -5, -8)\) generate \(\SET{R}^3\).
Find a subset of \(\{ u_1, u_2, u_3, u_4, u_5 \}\) that is a basis for \(\SET{R}^3\).
\end{exercise}

\begin{proof}
We mimic the procedure described after \EXAMPLE{3.4.2}.
We first find the nonzero solution \((c_1, c_2, c_3, c_4, c_5)\) s.t. \(c_1 u_1 + ... + c_5 u_5 = 0\).
That is,
\[
    \sysdelim..\systeme{
         2 c_1 + 1 c_2 -  8 c_3 +  1 c_4 - 3 c_5 = 0,
        -3 c_1 + 4 c_2 + 12 c_3 + 37 c_4 - 5 c_5 = 0,
         1 c_1 - 2 c_2 -  4 c_3 - 17 c_4 - 8 c_5 = 0
    }
\]
Then the corresponding r.r.e.f. system (by calculation) is,
\[
    \sysdelim..\systeme{
         1 c_1 + 0 c_2 - 4 c_3 - 3 c_4 -           11 c_5 = 0,
         0 c_1 + 1 c_2 + 0 c_3 + 7 c_4 - \frac{2}{19} c_5 = 0,
         0 c_1 + 0 c_2 + 0 c_3 + 0 c_4 +            1 c_5 = 0
    }
\]
\sloppy Then by \THM{3.16}(c), \(\{ u_1, u_2, u_5 \}\) is \LID{}.
And since \(\dim(\SET{R}^3 = 3\), \(\{ u_1, u_2, u_5 \}\) is a basis for \(\SET{R}^3\).
\end{proof}

\begin{exercise} \label{exercise 3.4.8}
Let \(W\) denote the subspace of \(\SET{R}^5\) consisting of all vectors having coordinates that \emph{sum to zero}.
The vectors
\begin{align*}
    u_1 = (2, -3, 4, -5, 2), &\ u_2 = (-6, 9, -12, 15, -6), \\
    u_3 = (3, -2, 7, -9, 1), &\ u_4 = (2, -8, 2, -2, 6), \\
    u_5 = (-1, 1, 2, 1, -3), &\ u_6 = (0. -3, -18, 9, 12), \\
    u_7 = (1, 0, -2, 3, -2), &\ u_8 = (2, -1, 1, -9, 7),
\end{align*}
generate \(W\).
Find a subset of \(\{ u_1, u_2, ..., u_8 \}\) that is a basis for \(W\).
\end{exercise}

\begin{proof}
(BTW, it's trivial that \(W\) has dimension \(4\).)
(And BTW, the vectors \(u_1\) to \(u_8\) are the same as \EXEC{1.6.8}.)
Similar to previous exercise, we reduce the problem to the system
\[
    A = [u_1\ u_2\ ... u_8] =
    \left[\begin{array}{cccccccc}
        2 & -6 & 3 & 2 & -1 & 0 & 1 & 2 \\
        -3 & 9 & -2 & -8 & 1 & -3 & 0 & -1 \\
        4 & -12 & 7 & 2 & 2 & -18 & -2 & 1 \\
        -5 & 15 & -9 & -2 & 1 & 9 & 3 & -9 \\
        2 & -6 & 1 & 6 & -3 & 12 & 2 & 7
    \end{array}\right]
\]
And a equivalent system (by calculation) is
\[
    B = \left(\begin{array}{cccccccc}
        1 & -3 & \frac{3}{2} & 1 & -\frac{1}{2} & 0 & \frac{1}{2} & 1 \\
        0 & 0 & 1 & -2 & -\frac{1}{5} & -\frac{6}{5} & \frac{3}{5} & \frac{4}{5} \\
        0 & 0 & 0 & 0 & 1 & -4 & -\frac{23}{21} & -\frac{19}{21} \\
        0 & 0 & 0 & 0 & 0 & 0 & 1 & -1 \\
        0 & 0 & 0 & 0 & 0 & 0 & 0 & 0
    \end{array}\right)
\]
If we continue to transform \(B'\) into r.r.e.f., then it's trivial that the \(1, 3, 5, 7\)th column of the r.r.e.f. is \LID{}.
By \THM{3.16}(c), \(\{ a_1, a_3, a_5, a_7 \} =\{ u_1, u_3, u_5, u_7 \}\) is \LID{}.
And since \(\dim(W) = 4\), \(\{ u_1, u_3, u_5, u_7 \}\) is a basis for \(W\).
\end{proof}

\begin{exercise} \label{exercise 3.4.9}
Let \(W\) be the subspace of \(M_{2 \X 2}(\SET{R})\) consisting of the \emph{symmetric} \(2 \X 2\) matrices.
The set
\[
    S= \left\{
        \left(\begin{array}{rr}
            0 & -1 \\
            -1 & 1
        \end{array}\right),
        \left(\begin{array}{ll}
            1 & 2 \\
            2 & 3
        \end{array}\right),
        \left(\begin{array}{ll}
            2 & 1 \\
            1 & 9
        \end{array}\right),
        \left(\begin{array}{rr}
            1 & -2 \\
            -2 & 4
        \end{array}\right),
        \left(\begin{array}{rr}
            -1 & 2 \\
            2 & -1
        \end{array}\right)
    \right\}
\]
generates \(W\).
Find a subset of \(S\) that is a basis for \(W\).
\end{exercise}

\begin{proof}
We use the \emph{standard representation} of each vector of \(S\) to make the system
\[
    A = \left(\begin{array}{ccccc}
        0 & 1 & 2 & 1 & -1 \\
        -1 & 2 & 1 & -2 & 2 \\
        -1 & 2 & 1 & -2 & 2 \\
        1 & 3 & 9 & 4 & -1
    \end{array}\right)
\]
And the r.r.e.f. (by calculation) is
\[
    B = \left(\begin{array}{ccccc}
        1 & -2 & -1 & 2 & -2 \\
        0 & 1 & 2 & 1 & -1 \\
        0 & 0 & 0 & 1 & -2 \\
        0 & 0 & 0 & 0 & 0
    \end{array}\right)
\]
So by \THM{3.16}(c), \(a_1, a_2, a_4\) are \emph{\LID{}} vectors in \(F^4\).
That is, the corresponding matrices
\[
    S' = \left\{
        \left(\begin{array}{rr}
            0 & -1 \\
            -1 & 1
        \end{array}\right),
        \left(\begin{array}{ll}
            1 & 2 \\
            2 & 3
        \end{array}\right),
        \left(\begin{array}{rr}
            1 & -2 \\
            -2 & 4
        \end{array}\right),
    \right\}
\]
are \LID{} in \(M_{2 \X 2}(\SET{R})\).
And since \(\dim(M_{2 \X 2}(\SET{R})) = 4\), \(S'\) is a basis of the set consisting of the symmetric \(2 \X 2\) matrices.
\end{proof}

\begin{exercise} \label{exercise 3.4.10}
Let
\[
    V = \{(x_1, x_2, x_3, x_4, x_5) \in \SET{R}^5 : x_1 - 2x_2 + 3x_3 - x_4 + 2x_5 = 0
 \}.
\]
\begin{enumerate}
\item Show that \(S = \{(0, 1, 1, 1, 0)\}\) is a \LID{} subset of \(V\).
\item Extend \(S\) to a basis for \(V\).
\end{enumerate}
\end{exercise}

\begin{proof}
part(a) is a really stupid question; singleton nonzero set is \LID{}, and \(x_1 - 2x_2 + 3x_3 - x_4 + 2x_5 = 0 - 2 + 3 - 1 + 0 = 0\), hence \(S\) is a \LID{} subset of \(V\).

For part(b), the procedure is exactly the same as \EXAMPLE{3.4.4}. Skip.
\end{proof}

\begin{exercise} \label{exercise 3.4.11}
Let \(V\) be as in \EXEC{3.4.10}.
\begin{enumerate}
\item Show that \(S = \{ (1 , 2, 1, 0, 0) \}\) is a \LID{} subset of \(V\).
\item Extend \(S\) to a basis for \(V\).
\end{enumerate}
\end{exercise}

\begin{proof}
Calculation problem which is similar to \EXEC{3.4.10}. Skip.
\end{proof}

\begin{exercise} \label{exercise 3.4.12}
Let \(V\) denote the set of all solutions to the system of linear equations
\[
    \sysdelim..\systeme{
        x_1 - x_2 + 2x_4 - 3x_5 + x_6 = 0,
        2x_1 - x_2 - x_3 + 3x_4 - 4x_5 + 4x_6 = 0
    }
\]
\begin{enumerate}
\item Show that \(S = \{ (0,-1,0,1,1,0), (1,0,1,1,1,0) \}\) is a \LID{} subset of \(V\).
\item Extend \(S\) to a basis for \(V\).
\end{enumerate}
\end{exercise}

\begin{proof}
Calculation problem which is similar to \EXEC{3.4.10}. Skip.
\end{proof}

\begin{exercise} \label{exercise 3.4.13}
Let \(V\) be as in \EXEC{3.4.12}.
\begin{enumerate}
\item Show that \(S = \{ (1,0,1,1,1,0), (0,2,1,1,0,0) \}\) is a \LID{} subset of \(V\).
\item Extend \(S\) to a basis for \(V\).
\end{enumerate}
\end{exercise}

\begin{proof}
Calculation problem which is similar to \EXEC{3.4.10}. Skip.
\end{proof}

\begin{exercise} \label{exercise 3.4.14}
If \((A|b)\) is in reduced row echelon form, prove that \(A\) is also in reduced row echelon form.
\end{exercise}

\begin{proof}
Suppose \((A|b)\) is in r.r.e.f..
It's enough to check that \(A\) still satisfies the definition of reduced row echelon form(\DEF{3.7}).
Given any row number \(i\) s.t. row \(i\) of \(A\) is zero row, there are two cases:
\begin{enumerate}
\item Zero row \(i\) of \(A\) is also zero row of \((A|b)\).
    Then given any nonzero row \(j\) of \(A\), of course row \(j\) of \((A|b)\) is nonzero, hence by condition \(1\) of \((A|b)\),
    row \(j\) of \((A|b)\) is above row \(i\) of \((A|b)\), and that implies row \(j\) of \(A\) is above row \(i\) of \(A\).
    Hence condition 1 of \(A\) is satisfied.
\item Zero row \(i\) of \(A\) is a nonzero row of \((A|b)\).
    Then given any nonzero row \(j\) of \(A\), of course row \(j\) of \((A|b)\) is nonzero.
    But by the \textbf{condition 3 for \((A|b)\)}, row \(j\) of \((A|b)\) must be \emph{above} row \(i\) of \((A|b)\), and that implies row \(j\) of \(A\) is above row \(i\) of \(A\).
    Hence condition 1 of \(A\) is again satisfied.
\end{enumerate}
So in all cases, the condition 1 of \(A\) is satisfied.

The second condition for \(A\) is really, automatically, implied by the second condition for \((A|b)\).
The third condition for \(A\) is also automatically satisfied.
\end{proof}

\begin{exercise} \label{exercise 3.4.15}
Prove the \CORO{3.16.1}:
The reduced row echelon form of a matrix \textbf{is unique}.
\end{exercise}

\begin{proof}
Note that if \(A\) is a zero matrix, then every elementary row operation performed on \(A\) leaves \(A\) \emph{unchanged}.
Thus the r.r.e.f. of a zero matrix is the matrix itself, and hence is unique.

Suppose then that \(A\) is an \(m \X n\) matrix of rank \(r > 0\) whose r.r.e.f. is \(B\).
Then there is an invertible \(m \X m\) matrix \(M\) representing the sequence of e.r.o.s, such that \(MA = B\).
For \(j = 1, 2, ..., n\), let \(a_j\) denote the \(j\)th column of \(A\) and \(b_j\) denote the \(j\)th column of \(B\).
By \THM{3.16}(b), \(e_i\) is ``a'' column of \(B\) for \(i = 1, 2, ..., r\).
Now let \(j_i\) denote the column number of the \emph{leftmost column} of \(B\) s.t. \(b_{j_i} = e_i\) \MAROON{(1)}.
Then by \THM{3.16}(c), \(\{ a_{j_1}, a_{j_2}, ..., a_{j_r} \}\) is \LID{}, and (since the dimension of column space of \(A\) is \(r\)) therefore is a basis for the column space of \(A\).
So, for \(k = 1, 2, ..., n\), \(a_k = c_1 a_{j_1} + c_2 a_{j_2} + ... + c_r a_{j_r}\) for \textbf{unique} scalars \(c_1, c_2, ..., c_r\).

\RED{
But then, by \ATHM{2.29}(or \EXEC{2.3.15}), we have \(b_k = M a_k = c_1 b_{j_1} + c_2 b_{j_2} + ... + c_r b_{j_r}\).
}
(Nasty, nasty, nasty.)
That is, by \MAROON{(1)}, we have \(b_k = c_1 e_1 + c_2 e_2 + ... + c_r e_r\).
Since \(c_1, c_2, ..., c_r\) are \emph{unique} and completed determined by \(A\), \(b_k\) is completely determined by \(A\).
Hence each column of \(B\) is completely determined by \(A\), hence \(B\) is unique.
\end{proof}

\begin{note}
There is \href{https://www.youtube.com/watch?v=EcgaeUUYV1U&ab_channel=DrPeyam}{Another proof}, which is more humble.
\end{note}

\begin{note}
只用噁心的\ \ATHM{2.29} 還不夠,必須要靠\ \(c_1, ..., c_r\) 是\ unique 才能唯一決定\ \(B\)。
\end{note}

\begin{additional theorem} \label{athm 3.15}
This is the placeholder theorem for \EXEC{3.4.3}:
Suppose that the augmented matrix of a system \(Ax = b\) is transformed into a matrix \((A'|b')\) in reduced row echelon form by a finite sequence of elementary row operations.
\begin{enumerate}
\item \(\rank(A') \ne \rank(A'|b')\) if and only if \((A'|b')\) contains a row in which the only nonzero entry lies in the last column.
\item \(Ax = b\) is consistent if and only if \((A'|b')\) contains no row in which the only nonzero entry lies in the last column.
\end{enumerate}
\end{additional theorem}

\begin{additional theorem} \label{athm 3.16}
This is the placeholder theorem for \EXEC{3.4.14}:
If \((A|b)\) is in reduced row echelon form, then \(A\) is also in reduced row echelon form.
\end{additional theorem}

\begin{additional theorem} \label{athm 3.17}
This is the placeholder theorem for \EXEC{3.4.15}:
r.r.e.f. is \textbf{unique}.
\end{additional theorem}
