\section{Vectir Spaces} \label{sec 1.2}

\begin{definition} \label{def 1.1}
A \textbf{vector space} (or \textbf{linear space}) \(\V\) over a \emph{field} \(F\) consists of a set on which two operations (called \textbf{addition} and \textbf{scalar multiplication}, respectively) are defined
so that for each pair of elements \(x, y \in \V\) there is a \emph{unique} element \(x + y \in \V\),
and for each element \(a \in F\) and each element \(x \in \V\) there is a \emph{unique} element \(ax \in \V\), such that the following conditions bold.
\begin{itemize}
    \item[(VS 1)] For all \(x, y \in \V\), \(x + y = y + x\) (commutativity of addition).
    \item[(VS 2)] For all \(x, y, z \in \V\), \((x + y) + z = x + (y + z)\) (associativity of addition).
    \item[(VS 3)] There exists an element in \(\V\) \emph{denoted} by \(0\) (or \(\OV\), for disambiguation with the zero in \(F\)) such that \(x + 0 = x\) for each \(x \in \V\).
    \item[(VS 4)] For each element \(x \in \V\) there exists an element \(y \in \V\) such that \(x + y = 0\).
    \item[(VS 5)] For each element \(x \in \V\), \(1 x = x\) (where \(1 \in F\), the multiplicative identity in the field \(F\)).
    \item[(VS 6)] For each pair of elements \(a, b \in F\) and each element \(x \in \V\), \((ab)x = a(bx)\).
    \item[(VS 7)] For each element \(a \in F\) and each pair of elements \(x, y \in \V\), \(a(x + y) = ax + ay\).
    \item[(VS 8)] For each pair of elements \(a, b \in F\) and each element \(x \in \V\), \((a + b)x = ax+ bx\).
\end{itemize}
The elements \(x + y\) and \(ax\) are called the \textbf{sum} of \(x\) and \(y\) and the \textbf{product} of \(a\) (in \(F\)) and \(x\), respectively.
\end{definition}

\begin{remark}
From the definition above, \(x + y\) and \(ax\) are unique actually implies the addition and scalar multiplication are \emph{functions} (\(+\) is from \((\V, \V)\) to \(\V\) and \(\cdot\) is from \(\V\) to \(\V\)).
And that also implies if \(x = x'\) and \(y = y'\), then \(x + y = x' + y'\) and \(ax = ax'\)
\end{remark}

\begin{note}
The elements of the field \(F\) are called \textbf{scalars} and the elements of the vector space \(\V\) are called \textbf{vectors}.
\end{note}

\begin{remark}
A vector space is frequently discussed in the text \emph{without} explicitly mentioning its field of scalars.
\RED{The reader is cautioned} to remember, however, that every vector space is regarded as a vector space \textbf{over a given field}, which
is denoted by \(F\).

Occasionally we restrict our attention to the fields of real and complex numbers, which are denoted \(\SET{R}\) and \(\SET{C}\), respectively.
\end{remark}

\begin{remark}
Unless otherwise noted, we assume that fields used in the examples and exercises of this book have \emph{characteristic zero} (see textbook page 549, or \href{https://www.wikiwand.com/en/Characteristic_(algebra)}{wiki}).
\end{remark}

In the remainder of this section we introduce several important examples of vector spaces that are studied throughout this text.

\begin{remark}
Observe that in describing a vector space, \textit{it is necessary to specify} not only the vectors but also \textit{the operations of addition and scalar multiplication} (See \EXAMPLE{1.2.6} and \EXAMPLE{1.2.7}, which provide ``bad'' addition or scalar multiplication).
\end{remark}

\begin{additional definition} \label{adef 1.1} \ 

\BLUE{(1)} An object of the form \((a_1, a_2, ..., a_n)\), where the \(entries\) \(a_1, a_2, ..., a_n\) are elements of a field \(F\), is called an \textbf{\(n\)-tuple} with entries from \(F\).

\BLUE{(2)} The elements \(a_1, a_2, ..., a_n\) are called the \textbf{entries} or \textbf{components} of the \(n\)-tuple.

\BLUE{(3)} Two \(n\)-tuples \((a_1, a_2, ..., a_n)\) and \((b_1, b_2, ..., b_n)\) with entries from a field \(F\) are called \textbf{equal} if \(a_i = b_i\) for \(i = 1, 2, ..., n\).
\end{additional definition}

\begin{example} \label{example 1.2.1}
The set of all \(n\)-tuples with entries from a \emph{field} \(F\) is \emph{denoted} by \(F^n\).
This set \emph{is} a vector space over \(F\) with the operations of \emph{coordinatewise} addition and scalar multiplication;
that is, if \(u = (a_1, a_2, ..., a_n) \in F^n\), \(v = (b_1, b_2, ..., b_n) \in F^n\), and \(c \in F\), then
\[
    u + v = (a_1 + b_1, a_2 + b_2, ..., a_n + b_n) \text{ and } cu = (ca_1, ca_2, ..., ca_n).
\]
(it's easy but tedious to check \(F^n\) over \(F\) with these operations is a vector space, i.e. satisfies \DEF{1.1}.)

Thus in particular, \(\SET{R}^3\) is a vector space over (the real numbers) \(\SET{R}\), \(\SET{C}^2\) is a vector space over (the complex numbers) \(\SET{C}\).
\end{example}

\begin{additional definition} \label{adef 1.2}\ 

\BLUE{(1)} Vectors in \(F^n\) may be written as \textbf{column vectors}
\[
    \begin{pmatrix}
        a_1 \\
        a_2 \\
        ... \\
        a_n
    \end{pmatrix}
\]
rather than as \textbf{row vectors} \((a_1, a_2, ..., a_n)\).

\BLUE{(2)} Since a \(1\)-tuple whose only entry is from \(F\) can be ``regarded as'' an element of \(F\), we usually write \(F\) rather than \(F^1\) for the vector space of \(1\)-tuples with entry from \(F\).
\end{additional definition}

\begin{additional definition} \label{adef 1.3}\ 

\BLUE{(1)} An \(m \X n\) \textbf{matrix} with entries from a field \(F\) is a rectangular array of the form
\[
\begin{pmatrix}
  a_{11} & a_{12} & ... & a_{1n} \\
  a_{21} & a_{22} & ... & a_{2n} \\
  \vdots & \vdots &     & \vdots  \\
  a_{m1} & a_{m2} & ... & a_{mn}
\end{pmatrix}
\]
where each entry \(a_{ij}\) \((1 \le i \le m, 1 \le j \le n)\) is an element of \(F\).

\BLUE{(2)} We call the entries \(a_{ij}\) with \(i = j\) the \textbf{diagonal entries} of the matrix.

\BLUE{(3)} The entries \(a_{il}, a_{i2}, ..., a_{in}\) compose the \textbf{\(i\)th row} of the matrix, and the entries \(a_{1j}, a_{2j}, ... , a_{nj}\) compose the \textbf{\(j\)th column} of the matrix.

\BLUE{(4)} The rows of the preceding matrix are regarded as vectors in \(F^n\), and the columns are regarded as vectors in \(F^m\).
Furthermore, we may regard a row vector in \(F^n\) as a \(1 \X n\) matrix with entries from \(F\), and we may regard a column vector in \(F^m\) as an \(m \X 1\) matrix with entries from \(F\).

\BLUE{(5)} The \(m \X n\) matrix in which each entry equals zero is called the \textbf{zero matrix} and is denoted by \(O\) (or \(O_{m \X n}\)).
In addition, if the number of rows and columns of a matrix are equal, the matrix is called \textbf{square}.

\BLUE{(6)} Two \(m \X n\) matrices \(A\) and \(B\) are called \textbf{equal} if all their corresponding entries are equal, that is, if \(A_{ij} = B_{ij}\) for \(1 \le i \le m\) and \(1 \le j \le n\).

\BLUE{(7)} In this book, we denote matrices by capital italic letters (e.g., \(A, B\), and \(C\)) , and we denote the entry of a matrix \(A\) that lies in row \(i\) and column \(j\) by \(A_{ij}\).
\end{additional definition}

\begin{example} \label{example 1.2.2}
The set of all \(m \X n\) matrices with entries from a field \(F\) is a vector space, which we \textbf{denote} by \(M_{m \X n}(F)\), with the following operations of \textbf{matrix addition} and \textbf{scalar multiplication}:
For \(A, B \in M_{m \X n}(F)\) and \(c \in F\),
\[
    (A + B)_{ij} = A_{ii} + B_{ii} \text{ and } (cA)_{ij} = cA_{ij}
\]
for \(1 \le i \le m\) and \(1 \le j \le m\).
(Again, check it satisfies \DEF{1.1} by yourself.)
\end{example}

\begin{note}
Note that in \EXAMPLE{1.2.2} we define the addition and scalar multiplication of matrix \emph{in terms of} the field addition and multiplication.
Many example will define the two operations as an extension of the two operations of the corresponding filed \(F\).
\end{note}

\begin{example} \label{example 1.2.3}
Let \(S\) be any nonempty set and \(F\) be any field, and let 
\(\mathcal{F}(S, F)\) denote the set of \emph{all functions from \(S\) to \(F\)}.
Two functions \(f\) and \(g\) in \(\mathcal{F}(S, F)\) are called \textbf{equal} if \(f(s) = g(s)\) for each \(s \in S\).
The set \(\mathcal{F}(S, F)\) is a vector space (over the field \(F\)) with the operations of addition and scalar multiplication defined for \(f, g \in \mathcal{F}(S, F)\) and \(c \in F\) by
\[
    (f + g)(s) = f(s) + g(s) \text{ and  } (cf)(s) = c[f(s)] (\text{ or } c \X f(s))
\]
for each \(s \in S\) (again, check yourself).
Note that \emph{these are the familiar operations} of addition and scalar multiplication for functions used \emph{in algebra and calculus}.
\end{example}

\begin{additional definition} \label{adef 1.4}\ 

\BLUE{(1)} A \textbf{polynomial} with coefficients from a field \(F\) is an expression of the form
\[
    f(x) = a_n x^n + a_{n - 1} x^{n - 1} + ... + a_1 x + a_0,
\]
where \(n\) is a nonnegative integer and each \(a_k\), called the \textbf{coefficient} of \(x^k\), is in \(F\).

\BLUE{(2)} If \(f(x) = 0\) (or \(\OF\)), that is, if \(a_n = a_{n-1} = ... = a_0 = \OF\) (or \(\OF\)), then \(f(x)\) is called the \textbf{zero polynomial} and, for convenience, its \textbf{degree} is \emph{defined} to be \(-1\). Otherwise, the \textbf{degree} of a polynomial is defined to be the \emph{largest exponent} of \(x\) that appears in the representation
\[
    f(x) = a_n x^n + a_{n - 1} x^{n - 1} + ... + a_1 x + a_0,
\]
with a \emph{nonzero coefficient}.

\BLUE{(3)} Note that the polynomials of degree \emph{zero} may be written in the form \(f(x) = c\) for some \emph{nonzero} scalar \(c\).

\BLUE{(4)} Two polynomials,
\[
    f(x) = a_n x^n + a_{n - 1} x^{n - 1} + ... + a_1 x + a_0,
\]
and
\[
    g(x) = b_m x^m + b_{m - 1} x^{m - 1} + ... + b_1 x + b_0,
\]
are called \textbf{equal} if \(m = n\) and \(a_i = b_i\) for \(i = 0, 1, ..., n\).
\end{additional definition}

\begin{note}
When \(F\) is a field containing \emph{infinitely} many scalars, we usually regard a polynomial with coefficients from \(F\) as a function \emph{from \(F\) into \(F\)}.
(See textbook page 564.)
In this case, the value of the function
\[
    f(x) = a_n x^n + a_{n - 1} x^{n - 1} + ... + a_1 x + a_0,
\]
at \(c \in F\) is the \emph{scalar}
\[
    f(c) = a_n c^n + a_{n - 1} c^{n - 1} + ... + a_1 c + a_0.
\]
Here either of the notations \(f\) or \(f(x)\) is used for the polynomial function
\[
    f(x) = a_n x^n + a_{n - 1} x^{n - 1} + ... + a_1 x + a_0,
\]
\end{note}

\begin{example} \label{example 1.2.4}
Let
\[
    f(x) = a_n x^n + a_{n - 1} x^{n - 1} + ... + a_1 x + a_0,
\]
and
\[
    g(x) = b_m x^m + b_{m - 1} x^{m - 1} + ... + b_1 x + b_0,
\]
be polynomials with coefficients from a field \(F\).
(WLOG) Suppose that \(m \le n\), and define \(b_{m+l} = b_{m+2} = ... = b_n = 0\).
Then \(g(x)\) can be written as
\[
    g(x) = b_n x^n + b_{n - l} x^{n - l} + ... + b_1 x + b_0.
\]
Define
\[
    f(x) + g(x) = (a_n + b_n) x^n + (a_{n - 1} + b_{n - 1}) x^{n - 1} + ... + (a_1 + b_1) x + (a_0 + b_0)
\]
and for any \(c \in F\), define
\[
    cf(x) = c {a_n} x^n + c a_{n-1} x^{n-l} + ... + c a_1 x + c a_0.
\]
With these operations of addition and scalar multiplication, the set of all polynomials with coefficients from \(F\) is a vector space, which we denote by \(\POLYF\).
\end{example}

\begin{example} \label{example 1.2.5}
Let \(F\) be any field.
A \textbf{sequence} in \(F\) is a \emph{function} \(\sigma\) from the positive integers into \(F\).
In this book, the sequence \(\sigma\) such that \(\sigma(n) = a_n\) for \(n = 1, 2, ...\) is denoted \((a_n)\).
Let \(\V\) consist of all the sequences \((a_n)\) in \(F\).
For \((a_n)\) and \((b_n)\) in \(\V\) and \(t \in F\), define
\[
    (a_n) + (b_n) = (a_n + b_n) \text{ and } t(a_n) = (t a_n).
\]
With these operations \(\V\) is a vector space (over \(F\)).
\end{example}

Our next two examples contain sets on which addition and scalar multiplication are defined, but which are \textbf{not} vector spaces.

\begin{example} \label{example 1.2.6}
Let \(S = \{ (a_1, a_2) : a_1, a_2 \in \SET{R} \}\).
For \((a_1, a_2), (b_1, b_2) \in S\) and \(c \in \SET{R}\), define
\[
    (a_1, a_2) + (b_1, b_2) = (a_1 + b_1, a_2 - b_2) \text{ and } c(a_1, a_2) = (c a_1, c a_2).
\]

Since \DEF{1.1} (VS 1), (VS 2), and (VS 8) fail to hold, \(S\) is \emph{not} a vector space with these operations.
Why?
For example for (VS 1), (suppose \(a_2 \ne b_2\)),
\begin{align*}
    (a_1, a_2) + (b_1, b_2) & = (a_1 + b_1, a_2 - b_2) & \text{by this def of \(+\)} \\
                             & = (b_1 + a_1, -(b_2 - a_2)) & \text{by field algebra} \\
                             & \ne (b_1 + a_1, b_2 - a_2) & \\
                             & = (b_1, b_2) + (a_1, a_2) & \text{by this def of \(+\)}
\end{align*}
\end{example}

\begin{example} \label{example 1.2.7}
Let \(S\) be as in \EXAMPLE{1.2.6}.
For \((a_1, a_2), (b_1, b_2) \in S\) and \(c \in \SET{R}\), define
\[
    (a_1, a_2) + (b_1, b_2) = (a_1 + b_1, 0) \text{ and } c(a_1, a_2) = (c a_1, 0).
\]
Then \(S\) is \emph{not} a vector space with these operations because \DEF{1.1} (VS 3) (hence (VS 4)) and (VS 5) fail:

For (VS 3), since there exists an element \((a_1, a_2) \in S\), where \(a_2 \ne 0\), such that for all \((b_1, b_2) \in S\), we have
\begin{align*}
    (a_1, a_2) + (b_1, b_2) & = (a_1 + b_1, 0), & \text{by this def of \(+\)}.
\end{align*}
And \((a_1 + b_1, 0)\) is impossible equal to \((a_1, a_2)\) since \(a_2 \ne 0\), the second component of them cannot be equal to the other.
So the additive identity failed to exist.
So (VS 3) is violated.
Hence since the additive identity does not exist, (VS 4), which depends on the definition of (VS 3), does not make any sense.
Similarly for (VS 5).
\end{example}

We conclude this section with a few of the elementary consequences of the definition of a vector space.
\begin{theorem} [\emph{Cancellation Law} for Vector Addition] \label{thm 1.1}
If \(x, y\), and \(z\) are vectors in a vector space \(\V\) such that \(x + z = y + z\), then \(x = y\).
\end{theorem}

\begin{proof}
(Using \DEF{1.1}.)
By (VS 4), there exists a vector \(v \in \V\) such that \(z + v = 0\) \MAROON{(1)}.
Thus
\begin{align*}
    x & = x + 0 & \text{by (VS 3)} \\
      & = x + (z + v) & \text{by \MAROON{(1)}} \\
      & = (x + z) + v & \text{by (VS 2)} \\
      & = (y + z) + v & \text{by supposition} \\
      & = y + (z + v) & \text{by (VS 2)} \\
      & = y + 0 & \text{by \MAROON{(1)}} \\
      & = y & \text{by (VS 3)}
\end{align*}
\end{proof}

\begin{corollary} \label{corollary 1.1.1}
The vector \(0\) described in \DEF{1.1} (VS 3) is unique.
\end{corollary}

\begin{proof}
Suppose \(\exists 0' \in \V\) such that \(0'\) also satisfies \DEF{1.1} (VS 3).
Then given \(x \in \V\), we have both \(x + 0 = x\) and \(x + 0' = x\).
And
\begin{align*}
             & x + 0 = x \land x + 0' = x \\
    \implies & x + 0 = x + 0' & \text{the result of \(+\) is unique, by \DEF{1.1}} \\
    \implies & 0 + x = x + 0' & \text{by (VS 1)} \\
    \implies & 0 + x = 0' + x & \text{by (VS 1)} \\
    \implies & 0 = 0' & \text{by \THM{1.1}, cancellation}.
\end{align*}
\end{proof}

\begin{note}
加法單位元素唯一。
\end{note}

\begin{corollary} \label{corollary 1.1.2}
The vector y described in \DEF{1.1} (VS 4) is unique.
\end{corollary}

\begin{proof}
Let arbitrary \(x \in \V\).
Now let both \(y, y' \in \V\) such that \(x + y = 0\) and \(x + y' = 0\).
Then
\begin{align*}
             & x + y = 0 \land x + y' = 0 \\
    \implies & x + y = x + y' \\
    \implies & y + x = y' + x & \text{applying (VS 1) twice} \\
    \implies & y = y'. & \text{by \THM{1.1}, cancellation}
\end{align*}
\end{proof}

\begin{note}
給定任一\ \(y \in \V\),存在唯一的\ \(y\) 的加法反元素。
\end{note}

\begin{additional definition} \label{adef 1.5}
The vector \(0\) in \DEF{1.1} (VS 3) is called \emph{the} \textbf{zero vector} of \(\V\), and the vector \(y\) in (VS 4) (that is, the unique vector such that \(x + y = 0\)) is called \emph{the} \textbf{additive inverse} of \(x\) and is \textbf{denoted} by \(-x\).
\end{additional definition}

\begin{theorem} \label{thm 1.2}
Given any vector space \(\V\) (over a field \(F\)), the following statements are true:
\begin{enumerate}
    \item \(0x = 0\) (or \(\OF x = \OV\)) for each \(x \in \V\).
    \item \((-a)x = -(ax) = a(-x)\) for each \(a \in F\) and each \(x \in \V\). (Note that sometimes ``\(-\)'' is denote the inverse of a vector, sometimes the inverse of a scalar.)
    \item \(a0 = 0\) (or \(a\OV = \OV\)) for each \(a \in F\).
\end{enumerate}
\end{theorem}

\begin{proof}
\begin{enumerate}
\item
    \begin{align*}
        0x + 0x & = (0 + 0)x & \text{by \DEF{1.1}(VS 8)} \\
                & = 0x & \text{by field property \(\OF + \OF = \OF\)} \\
                & = 0x + 0 & \text{by (VS 3)} \\
                & = 0 + 0x & \text{by (VS 1)}
    \end{align*}
    Then by \THM{1.1}, cancellation, we have \(0x = 0\), as desired.
\item
    By \ADEF{1.5}, \(-(ax)\) is denoted as \emph{the unique} additive inverse of \(ax\) such that \(ax + [-(ax)] = 0\).
    Thus if we can show \(ax + (-a)x = 0\), then we have \((-a)x = -(ax)\) by the uniqueness of additive inverse(or \CORO{1.1.2}).
    But
    \begin{align*}
        ax + (-a)x & = (a + (-a))x & \text{by (VS 8)} \\
                   & = 0x & \text{by field property, \(a + (-a) = \OF\)} \\
                   & = 0 & \text{by part(a)}
    \end{align*}
    So we have \((-a)x = -(ax)\) \MAROON{(1)}.
    In particular, \((-1) x = -(1x)\), which by (VS 5) \(= -x\), so we have \((-1)x = -x\) \MAROON{(2)}.
    And
    \begin{align*}
        a(-x) & = a[(-1)x] & \text{by \MAROON{(2)}} \\
              & = [a(-1)]x & \text{by (VS 6)} & \\
              & = (-a)x & \text{by field property, \(a(-1) = -a\)}.
    \end{align*}
    So we have \(a(-x) = (-a)x\) \MAROON{(3)}.
    So by \MAROON{(1)(3)}, we have \((-a)x = -(ax) = a(-x)\), as desired.
\item
    \begin{align*}
        a0 + a0 & = a(\BLUE{0} + 0) & \text{by (VS 7)} \\
                & = a\BLUE{0} & \text{by (VS 3)} \\
                & = a0 + 0 & \text{by (VS 3)} \\
                & = 0 + a0 & \text{by (VS 1)}
    \end{align*}
    Then by \THM{1.1}, cancellation, we have \(a0 = 0\).
\end{enumerate}
\end{proof}

\exercisesection

\begin{exercise} \label{exercise 1.2.1}
Label the following statements as true or false.
\begin{enumerate}
\item Every vector space contains a zero vector.
\item A vector space may have more than one zero vector.
\item In any vector space, \(ax = bx\) implies that \(a = b\).
\item In any vector space, \(ax = ay\) implies that \(x = y\).
\item A vector in \(F^n\) may be regarded as a matrix in \(M_{n \X 1}(F \)).
\item An \(m \X n\) matrix has \(m\) columns and \(n\) rows.
\item In \(\POLYF\), only polynomials of the same degree may be added.
\item If \(f\) and \(g\) are polynomials of degree \(n\), then \(f + g\) is a polynomial of degree \(n\).
\item If \(f\) is a polynomial of degree \(n\) and \(c\) is a nonzero scalar, then \(cf\) is a polynomial of degree \(n\).
\item A nonzero scalar of \(F\) may be considered to be a polynomial in \(\POLYF\) having degree zero.
\item Two functions in \(\mathcal{F}(S, F)\) are equal if and only if they have the same value at each element of \(S\).
\end{enumerate}
\end{exercise}

\begin{proof}
\begin{enumerate}
\item True by \DEF{1.1}(VS 3).
\item False by \CORO{1.1.1}.
\item False; since for example \(5 \ne 6\) but \(5 \OV = \OV = 6 \OV\) by \THM{1.2}(c).
\item False; since for example suppose \(v_1, v_2 \in \V\) such that \(v_1 \ne v_2\).
    But by \THM{1.2}(a), \(\OF v_1 = \OV = \OF v_2\).
\item True; see \ADEF{1.2}, \ADEF{1.3}.
\item False; by \ADEF{1.3}, n \(m \X n\) matrix has \(m\) rows and \(n\) columns.
\item False; polynomials of different degrees may be added. See \ADEF{1.4} and \EXAMPLE{1.2.4}.
\item False; both \(x\) and \(-x\) has degree \(1\), but \(x + (-x) = 0\) has degree \(-1\) (defined for convenience in \ADEF{1.4}).
\item True: since \(f\) has degree \(n\), by \ADEF{1.4} it can be written as
    \[
        f(x) = {a_n} x^n + a_{n-1} x^{n-l} + ... + a_1 x + a_0.
    \]
    where \(a_n \ne 0\).
    And by \EXAMPLE{1.2.4},
    \[
        cf(x) = c {a_n} x^n + c a_{n-1} x^{n-l} + ... + c a_1 x + c a_0.
    \]
    Since \(c \ne 0 \land a_n \ne 0\), we have \(c \X a_n \ne 0\) (by \href{https://www.wikiwand.com/en/Zero-product_property}{zero product property} of any field).
    So by \ADEF{1.4}, \(cf\) has degree \(n\).
\item True by \ADEF{1.4}.
\item True by \EXAMPLE{1.2.3}.
\end{enumerate}
\end{proof}

\begin{exercise} \label{exercise 1.2.2}
Skip.
\end{exercise}

\begin{exercise} \label{exercise 1.2.3}
Skip.
\end{exercise}

\begin{exercise} \label{exercise 1.2.4}
Skip.
\end{exercise}

\begin{exercise} \label{exercise 1.2.5}
Skip.
\end{exercise}

\begin{exercise} \label{exercise 1.2.6}
Skip.
\end{exercise}

\begin{exercise} \label{exercise 1.2.7}
Let \(S = \{0, 1\}\) and \(F = \SET{R}\).
In \(\mathcal{F}(S, \SET{R})\), show that \(f = g\) and \(f + g = h\),
where \(f(t) = 2t + 1\), \(g(t) = 1 + 4t - 2t^2\) and \(h(t) = 5^t + 1\).
\end{exercise}

\begin{proof}
\(f(0) = 2 \X 0 + 1 = 1, g(0) = 1 + 4 \X 0 - 2 \X 0^2 = 1\);
\(f(1) = 2 \X 1 + 1 = 3, g(1) = 1 + 4 \X 1 - 2 \X 1^2 = 3\).
So for each element \(s \in S\), \(f(s) = g(s)\), hence by definition in \EXAMPLE{1.2.3}, \(f = g\).

And
\begin{align*}
    (f + g)(0) & = f(0) + g(0) & \text{by definition in \EXAMPLE{1.2.3}} \\
               & = 1 + 1 & \text{by previous case} \\
               & = 2 \\
    h(0) & = 5^0 + 1 = 2 \\
    (f + g)(1) & = f(1) + g(1) & \text{by definition in \EXAMPLE{1.2.3}} \\
               & = 3 + 3 & \text{by previous case} \\
               & = 6 \\
    h(1) = 5^1 + 1 & = 6.
\end{align*}
So for each element \(s \in S\), \((f + g)(s) = h(s)\), hence by definition in \EXAMPLE{1.2.3}, \(f + g = h\).
\end{proof}

\begin{exercise} \label{exercise 1.2.8}
In any vector space \(\V\) (over \(F\)), show that \((a + b)(x + y) = ax+ ay + bx + by\) for any \(x, y \in \V\) and any \(a, b \in F\).
\end{exercise}

\begin{proof}
\begin{align*}
    (a + b)(x + y) & = (a + b)x + (a + b)y & \text{by \DEF{1.1}(VS 7)} \\
                   & = (ax + bx) + (ay + by) & \text{by (VS 8)} \\
                   & = ax + ay + bx + by & \text{by applying (VS 1) (VS 2) multiple times}
\end{align*}
\end{proof}

\begin{exercise} \label{exercise 1.2.9}
Prove \CORO{1.1.1} and \CORO{1.1.2} of \THM{1.1}, and \THM{1.2}(c).
\end{exercise}

\begin{proof}
See these \CORO{1.1.1}, \CORO{1.1.2} and \THM{1.2}.
\end{proof}

\begin{exercise} \label{exercise 1.2.10}
Let \(\V\) denote the set of all \emph{differentiable} real-valued functions defined on the real line.
Prove that \(\V\) is a vector space with the operations of
addition and scalar multiplication defined in \EXAMPLE{1.2.3}.
\end{exercise}

\begin{proof}
In \EXAMPLE{1.2.3} we have shown that the set of \emph{all} functions from (in particular) \(\SET{R}\) to \(\SET{R}\) with the corresponding \(+\) and \(\cdot\) is a vector space.
But if you go to \SEC{1.3} with \THM{1.3}, you will notice that we only need to verify that \emph{zero function} is in \(\V\), and \(+\) and \(\cdot\) are \emph{closed}: \(f + g \in \V\) and \(c f \in \V\).
But of course zero function is in \(\V\) since zero function is differentiable.
And by Calculus, given two differentiable \(f, g\) and scalar \(c \in \SET{R}\), \(f + g\) is still differentiable and \(c f\) is still differentiable, so we are done.
\end{proof}

\begin{exercise} \label{exercise 1.2.11}
Let \(\V = \{ 0 \}\) consist of a \emph{single vector} \(0\), and define \(0 + 0 = 0\) \MAROON{(1)} and \(c0 = 0\) for each scalar \(c\) in \(F\) \MAROON{(2)}.
Prove that \(\V\) is a vector space over \(F\).
(\(\V\) is called the \textbf{zero vector space}.)
\end{exercise}

\begin{proof}
Since \(0\) is the only single element in \(\V\), both \(x, y, z\) described in \DEF{1.1} can only be equal to \(0 \in \V\).

First, by definition \MAROON{(1)(2)}, \(+\) and \(\cdot\) is closed under \(\V\).

For (VS 1), we only have to show that \(\BLUE{0} + \GREEN{0} = \GREEN{0} + \BLUE{0}\).
But they are defined by \MAROON{(1)} to \(0\) so are equal to each other, so we are done.

For (VS 2), we only need to show \((\BLUE{0} + \GREEN{0}) + \MAROON{0} = \BLUE{0} + (\GREEN{0} + \MAROON{0})\).
But
\begin{align*}
    & (\BLUE{0} + \GREEN{0}) + \MAROON{0} \\
    & = 0 + \MAROON{0} & \text{by \MAROON{(1)}} \\
    & = 0, & \text{by \MAROON{(1)}}
\end{align*}
and
\begin{align*}
    & \BLUE{0} + (\GREEN{0} + \MAROON{0}) \\
    & = \BLUE{0} + 0 & \text{by \MAROON{(1)}} \\
    & = 0, & \text{by \MAROON{(1)}}
\end{align*}
so we are done.

For (VS 3), we only need to show given (the single) element \(\BLUE{0} \in \V\), there exists \(\GREEN{0}\) such that \(\BLUE{0} + \GREEN{0}\).
But that again is defined by \MAROON{(1)}, so we are done.

For (VS 4), we only need to show given (the single) element \(\BLUE{0} \in \V\), there exists \(\GREEN{v}\) such that \(\BLUE{0} + v\).
Just let \(v = 0\), then again by \MAROON{(1)}, \(0 + v = 0 + 0 = 0\), so we are done.

For (VS 5), we only need to show that \(1 \GREEN{0} = 0\).
But it is defined by \MAROON{(2)} to \(0\), so we are done.

For (VS 6), we only need to show \((ab)\OV = a(b\OV)\) (we use the notation \(\OV\) to prevent ambiguity with the scalar \(0 \in F\)).
But
\begin{align*}
    (ab)\OV & = \OV; & \text{by \MAROON{(2)}} \\
    a(b\OV) & = a\OV & \text{by \MAROON{(2)}} \\
            & = \OV, & \text{by \MAROON{(2)}}
\end{align*}
So \((ab)\OV = a(b\OV)\), so we are done.

For (VS 7), we only need to show for each \(a \in F\), \(a(0 + 0) = a0 + a0\).
But
\begin{align*}
    a(0 + 0) & = a0 & \text{by \MAROON{(1)}} \\
             & = 0; & \text{by \MAROON{(2)}} \\
     a0 + a0 & = 0 + a0 & \text{by \MAROON{(2)}} \\
             & = 0 + 0 & \text{by \MAROON{(2)}} \\
             & = 0. & \text{by \MAROON{(1)}}
\end{align*}
So \(a(0 + 0) = a0 + a0\), so we are done.

For (VS 8), we only need to show for each \(a, b \in F\), \((a + b)0 = a0 + b0\).
But
\begin{align*}
    (a + b)0 & = 0; & \text{by \MAROON{(2)}} \\
     a0 + b0 & = 0 + b0 & \text{by \MAROON{(2)}} \\
             & = 0 + 0 & \text{by \MAROON{(2)}} \\
             & = 0. & \text{by \MAROON{(1)}}
\end{align*}
So \((a + b)0 = a0 + b0\), so we are done.

So all conditions in \DEF{1.1} are satisfied, so \(\V\) with the operations defined in \MAROON{(1)} and \MAROON{(2)} is a vector space over \(F\).
\end{proof}

\begin{exercise} \label{exercise 1.2.12}
A real-valued function f defined on the real line is called an \textbf{even function} if \(f(-t) = f(t)\) for each real number \(t\).
Prove that the set of even functions defined on the real line with the operations of addition and scalar multiplication defined in \EXAMPLE{1.2.3} is a vector space.
\end{exercise}

\begin{proof}
Similarly as \EXEC{1.2.10}, we only have to show that zero function is an even function(but it's trivial, of course), and \(+\), \(\cdot\) are closed.
That is, given any even function \(f, g\) and scalar \(c \in \SET{R}\), both \(f + g\) and \(c f\) are still even functions.

But given arbitrary \(t \in \SET{R}\),
\begin{align*}
    (f + g)(-t) & = f(-t) + g(-t) & \text{by definition in \EXAMPLE{1.2.3}} \\
                & = f(t) + g(-t) & \text{since \(f\) is even} \\
                & = f(t) + g(t) & \text{since \(g\) is even} \\
                & = (f + g)(t), & \text{by definition in \EXAMPLE{1.2.3}}
\end{align*}
and
\begin{align*}
    (cf)(-t) & = cf(-t) & \text{by definition in \EXAMPLE{1.2.3}} \\
                & = cf(t) & \text{since \(f\) is even} \\
                & = (cf)(t) & \text{by definition in \EXAMPLE{1.2.3}}
\end{align*}
So these operations are closed, so we are done.
\end{proof}

\begin{exercise} \label{exercise 1.2.13}
Let \(\V\) denote the set of ordered pairs of real numbers.
If \((a_1, a_2)\) and \((b_1, b_2)\) are elements of \(\V\) and \(c \in \SET{R}\), define
\[
    (a_1, a_2) + (b_1, b_2) = (a_1 + b_1, \BLUE{a_2 b_2}) \text{ and } c(a_1, a_2) = (c a_1, \BLUE{a_2}).
\]
Is \(\V\) a vector space over \(\SET{R}\) with these operations?
Justify your answer.
\end{exercise}

\begin{proof}
\DEF{1.1} (VS 8) is violated, since
\begin{align*}
    (a + b)(a_1, a_2) & = ((a + b)a_1, \RED{a_2}), & \text{by definition of \(\cdot\) in this exercise}
\end{align*}
but
\begin{align*}
    a(a_1, a_2) + b(a_1, a_2) & = (a a_1, a_2) + (b a_1, a_2) & \text{by definition of \(\cdot\) in this exercise} \\
                              & = (a a_1 + b a_1, a_2 \X a_2) & \text{by definition of \(+\) in this exercise} \\
                              & = ((a + b)a_1, \RED{(a_2)^2}) & \text{by field algebra}
\end{align*}
In particular, given any \(a_2 \ne\) \(0\) and \(1\), \(a_2 \ne a_2^2\), so LHS and RHS of (VS 8) are not equal to each other.
\end{proof}

\begin{exercise} \label{exercise 1.2.14}
Let \(\V = \{(a_1, a_2, ..., a_n): a_i \in \SET{C} \text{ for } i = 1, 2, ..., n\}\);
so \(\V\) is a vector space \emph{over} \(\SET{C}\) by \EXAMPLE{1.2.1}.
Is \(\V\) a vector space \emph{over the field of real numbers} with the operations of coordinatewise addition and multiplication?
\end{exercise}

\begin{proof}
Yes.
Since \(\SET{C}^n\) over \(\SET{C}\) is a vector space, when we change the field to \(\SET{R}\), \DEF{1.1} (VS 1) to (VS 8) can still be proved(although tedious).
So we only need to show the operation \(\cdot\) is still closed with the scalar \(a \in \SET{R}\)
(we do not need to check for \(+\) since it does not care the scalar in the field).
But given \(a \in \SET{R}\), \((a_1, a_2, ..., a_n) \in \SET{C}^n\),
\begin{align*}
    c (a_1, a_2, ..., a_n) & = (c a_1, c a_2, ..., c a_n), & \text{by definition in \EXAMPLE{1.2.1}}
\end{align*}
where \(c a_1, c a_2, ..., c a_n\) are multiplication of real numbers and complex numbers, which are still complex numbers by definition of complex number.
So \((c a_1, c a_2, ..., c a_n) \in \SET{C}^n\), so the operation \(\cdot\) is closed, so we are done.
\end{proof}

\begin{exercise} \label{exercise 1.2.15}
Let \(\V = \{ (a_1, a_2, ..., a_n): a_i \in \SET{R}^n \text{ for } i = 1, 2, ..., n \}\);
so \(\V\) is a vector space \emph{over} \(\SET{R}\) by \EXAMPLE{1.2.1}.
Is \(\V\) a vector space \emph{over the field of complex numbers} with the operations of coordinatewise addition and
multiplication?
\end{exercise}

\begin{proof}
No, since given \(i \in \SET{C}\) and \((a_1, a_2, ..., a_n) \in \SET{R}^n\),
\begin{align*}
    i (a_1, a_2, ..., a_n) & = (i a_1, i a_2, ..., i a_n), & \text{by definition in \EXAMPLE{1.2.1}}
\end{align*}
where \(i a_1, ..., i a_n\) are not real numbers, so \((i a_1, i a_2, ..., i a_n) \notin \SET{R}^n\), so the multiplication is \emph{not} closed, hence \(\SET{R}^n\) over \(\SET{C}\) is \emph{not} a vector space.
\end{proof}

\begin{exercise} \label{exercise 1.2.16}
Let \(\V\) denote the set of all \(m \X n\) matrices with real entries;
so \(\V\) is a vector space \emph{over} \(\SET{R}\) by \EXAMPLE{1.2.2}.
Let \(F\) be the field of \emph{rational numbers}.
Is \(\V\) a vector space over \(F\) with the usual definitions of matrix addition and scalar multiplication?
\end{exercise}

\begin{proof}
Yes.
The overall structure of the prove is Similar with \EXEC{1.2.14};
and the scalar multiplication is closed.
\end{proof}

\begin{exercise} \label{exercise 1.2.17}
Let \(\V = \{ (a_1, a_2) : a_1, a_2 \in F \}\), where \(F\) is a field.
Define addition of elements of \(\V\) coordinatewise, and for \(c \in F\) and \((a_1, a_2) \in \V\), define
\[
    c(a_1, a_2) = (a_1, 0).
\]
Is \(\V\) a vector space over \(F\) with these operations? Justify your answer.
\end{exercise}

\begin{proof}
No.
In particular, for any scalar \(c \in F\), we have \(c(a_1, a_2) = (a_1, 0) \ne (a_1, a_2)\) when \(a_2 \ne 0\).
So there is no multiplicative identity, hence \DEF{1.1}(VS 5) is not satisfied.
\end{proof}

\begin{exercise} \label{exercise 1.2.18}
Let \(\V = \{(a_1, a_2) : a_1, a_2 \in \SET{R}\}\).
For \((a_1, a_2), (b_1, b_2) \in \V\) and \(c \in \SET{R}\), define
\[
    (a_1, a_2) + (b_1, b_2) = (a_1 + 2b_1, a_2 + 3b_2) \text{ and } c(a_1, a_2) = (ca_1, ca_2).
\]
Is \(\V\) a vector space over \(\SET{R}\) with these operations?
Justify your answer.
\end{exercise}

\begin{proof}
No.
In particular for \DEF{1.1}(VS 1).
\begin{align*}
    (a_1, a_2) + (b_1, b_2) & = (a_1 + 2b_1, a_2 + 3b_2),
\end{align*}
But
\begin{align*}
    (b_1, b_2) + (a_1, a_2) = (b_1 + 2a_1, b_2 + 3a_2)
\end{align*}
So the commutative law is not satisfied.
\end{proof}

\begin{exercise} \label{exercise 1.2.19}
Let \(\V = \{(a_1, a_2 ) : a_1, a_2 \in \SET{R}\}\).
Define addition of elements of \(\V\) coordinatewise, and for \((a_1, a_2) \in \V\) and \(c \in \SET{R}\), define
\begin{equation*}
    c(a_1, a_2) =
    \begin{cases}
        (0, 0) & \text{ if } c = 0 \\
        \Big(c a_1, \frac{a_2}{c}\Big) & \text{ if } c \ne 0.
    \end{cases}
\end{equation*}
Is \(\V\) a vector space over \(\SET{R}\) with these operations?
Justify your answer.
\end{exercise}

\begin{proof}
In fact \DEF{1.1} (VS 7) to (VS 7) \emph{are} satisfied.
For (VS 8), given any \(a, b \in \SET{R}\) such that \(a, b \ne 0\),
\begin{align*}
    (a + b)(a_1, a_2) & = ((a + b)a_1, \frac{a_2}{a + b}) & \text{by definition of \(\cdot\) in this exercise}
\end{align*}
But
\begin{align*}
    a(a_1, a_2) + b(a_1, a_2) & = (a a_1, \frac{a_2}{a}) + (b a_1, \frac{a_2}{b}) & \text{by definition of this exercise} \\
                              & = (a a_1 + b a_1, \frac{a_2}{a} + \frac{a_2}{b}), & \text{by coordinatewise \(+\)}
\end{align*}
and trivially for some \(a, b \ne 0\) the second component of these results cannot be equal to each other,
so we have \((a + b)(a_1, a_2) \ne a(a_1, a_2) + b(a_1, a_2)\), so (VS 8) is not satisfied.
\end{proof}

\begin{exercise} \label{exercise 1.2.20}
Let \(\V\) denote the set of all real-valued functions \(f\) defined on the real line such that \(f(1) = 0\).
Prove that \(\V\) is a vector space with the operations
of addition and scalar multiplication defined in \EXAMPLE{1.2.3}.
\end{exercise}

\begin{proof}
Similarly as \EXEC{1.2.10}, we only need to show that zero function is in \(\V\) and the operations \(+\) and \(\cdot\) are closed.
But since in particular, given real number \(1\), the output of the zero function is \(0\), so the zero function is in \(\V\).
And given two function \(f, g\) such that \(f(1) = 0\) and \(g(1) = 0\), and scalar \(c \in \SET{R}\),
\begin{align*}
    (f + g)(1) & = f(1) + g(1) & \text{by definition in \EXAMPLE{1.2.3}} \\
               & = 0 + 0 & \text{by definition of \(f, g\)} \\
               & = 0 & \text{by property of \(\SET{R}\)}
\end{align*}
and
\begin{align*}
    (cf)(1) & = cf(1) & \text{by definition in \EXAMPLE{1.2.3}} \\
            & = c \X 0 & \text{by definition of \(f\)} \\
            & = 0, & \text{by property of \(\SET{R}\)}
\end{align*}
so \(f + g\) and \(cf\) are still in \(\V\), so the addition and scalar multiplication are closed under \(\V\), as desired.
\end{proof}

\begin{exercise} \label{exercise 1.2.21}
Let \(\V\) and \(\W\) be vector spaces over a field \(F\).
Let
\[
    Z = \{ (v, w) : v \in V, w \in W \}.
\]
Prove that \(Z\) is a vector space over \(F\) with the operations
\[
    (v_1, w_1) + (v_2, w_2) = (v_1 + v_2, w_1 + w_2) \MAROON{(1)} \text{ and } c(v_1, w_1) = (c v_1, c w_1) \MAROON{(2)}.
\]
\end{exercise}

\begin{proof}
We denote \(\OV\) as \(\V\)'s zero vector, \(\OW\) as \(\W\)'s zero vector.
In particular, we denote \(0_{_Z}\) as zero vector of \(Z\) if \(Z\) in fact is a vector space.

We first prove the defined operations are closed.
Given \((v_1, w_1), (v_2, w_2) \in Z\),
\begin{align*}
    (v_1, w_1) + (v_2, w_2) & = (v_1 + v_2, w_1 + w_2) & \text{by def of \MAROON{(1)}}
\end{align*}
But \(v_1 + v_2 \in \V\) since \(\V\) is a vector space; similarly for \(w_1 + w_2 \in \W\);
hence by definition of \(Z\), \((v_1 + v_2, w_1 + w_2) \in Z\);
that is, by \MAROON{(1)}, \((v_1, w_1) + (v_2, w_2) \in Z\).

Now given \((v, w) \in Z\), \(c \in F\),
\begin{align*}
    c(v, w) & = (c v, c w) & \text{by def of \MAROON{(2)}}
\end{align*}
But \(c v \in \V\) since \(\V\) is a vector space over the field \(F\); similarly for \(c w \in \W\);
hence by definition of \(Z\), \((c v, c w) \in Z\);
that is, by \MAROON{(2)}, \(c(v, w) \in Z\).

For (VS 1),
\begin{align*}
    & (v_1, w_1) + (v_2, w_2) \\
    & = (v_1 + v_2, w_1 + w_2) & \text{by def of \MAROON{(1)}} \\
    & = (v_2 + v_1, w_1 + w_2) & \text{by def of \(\V\)'s \(+\)} \\
    & = (v_2 + v_1, w_2 + w_1) & \text{by def of \(\W\)'s \(+\)} \\
    & = (v_2, w_2) + (v_1, w_1) & \text{by def of \MAROON{(1)}}.
\end{align*}

For (VS 2),
\begin{align*}
    & ((v_1, w_1) + (v_2, w_2)) + (v_3, w_3) \\
    & = (v_1 + v_2, w_1 + w_2) + (v_3, w_3) & \text{by def of \MAROON{(1)}} \\
    & = ((v_1 + v_2) + v_3, (w_1 + w_2) + w_3) & \text{by def of \MAROON{(1)}} \\
    & = (v_1 + (v_2 + v_3), w_1 + (w_2 + w_3)) & \text{by \(\V\)'s and \(\W\)'s (VS 2)} \\
    & = (v_1, w_1) + (v_2 + v_3, w_2 + w_3) & \text{by def of \MAROON{(1)}} \\
    & = (v_1, w_1) + ((v_2, w_2) + (v_3, w_3)) & \text{by def of \MAROON{(1)}}
\end{align*}

For (VS 3), we claim that \(0_{_Z} = (\OV, \OW)\) \MAROON{(3)}, because:
\begin{align*}
    & (v, w) + (\OV, \OW) \\
    & = (v + \OV, w + \OW) & \text{by def of \MAROON{(1)}} \\
    & = (v, w) & \text{by \(\V\)'s and \(\W\)'s (VS 3)}
\end{align*}

For (VS 4), given any \((v, w) \in Z\), by (VS 4) of \(\V\) and (VS 4) of \(\W\), we have \(-v\) and \(-w\) such that \(v + (-v) = \OV\) and \(w + (-w) = \OW\) \MAROON{(4)};
in particular,
\begin{align*}
    (v, w) + (-v, -w) & = (v + (-v), w + (-w)) & \text{by def of \MAROON{(1)}} \\
                      & = (\OV, \OW) & \text{by \MAROON{(4)}} \\
                      & = 0_{_Z}. & \text{by \MAROON{(3)}}
\end{align*}

For (VS 5),
\begin{align*}
    1 (v, w) & = (1v, 1w) & \text{by def of \MAROON{(2)}} \\
             & = (v, w) & \text{by \(\V\)'s and \(\W\)'s (VS 5)}
\end{align*}

For (VS 6),
\begin{align*}
    (ab) (v, w) & = ((ab) v, (ab) w) & \text{by def of \MAROON{(2)}} \\
                & = (a(bv), a(bw)) & \text{by \(\V\)'s and \(\W\)'s (VS 6)} \\
                & = a(bv, bw) & \text{by def of \MAROON{(2)}} \\
                & = a(b(v, w)) & \text{by def of \MAROON{(2)}}
\end{align*}
    
For (VS 7),
\begin{align*}
    a((v_1, w_1) + (v_2, w_2)) & = a(v_1 + v_2, w_1 + w_2) & \text{by def of \MAROON{(1)}} \\
                               & = (a(v_1 + v_2), a(w_1 + w_2)) & \text{by def of \MAROON{(2)}} \\
                               & = (a v_1 + a v_2, a w_1 + a w_2) & \text{by \(\V\)'s and \(\W\)'s (VS 7)} \\
                               & = (a v_1, a w_1) + (a v_2, a w_2) & \text{by def of \MAROON{(1)}} \\
                               & = a(v_1, w_1) + a(v_2, w_2) & \text{by def of \MAROON{(2)}}
\end{align*}
    
For (VS 8)
\begin{align*}
    (a + b)(v, w) & = ((a + b)v , (a + b)w) & \text{by def of \MAROON{(2)}} \\
                               & = (av + bv, aw + bw) & \text{by \(\V\)'s and \(\W\)'s (VS 8)} \\
                               & = (av, aw) + (bv, bw) & \text{by def of \MAROON{(1)}} \\
                               & = a(v, w) + b(v, w) & \text{by def of \MAROON{(2)}}
\end{align*}

So all conditions in \DEF{1.1} are satisfied, hence \(Z\) is a vector space (over the field \(F\)).
\end{proof}

\begin{note}
It seems that \(Z\) is called the \href{https://www.wikiwand.com/en/Direct_product}{direct product} of \(\V\) and \(\W\).
\end{note}

\begin{exercise} \label{exercise 1.2.22}
How many matrices are there in the vector space \(M_{m \X n}(Z_2)\)? (See \EXAMPLE{c.4}.)
\end{exercise}

\begin{proof}
By definition of \(Z_2\) (see \EXAMPLE{c.4}), each entry can only be \(0\) or \(1\), so the number of matrices in \(M_{m \X n}(Z_2)\) is \(2^{m \X n}\).
\end{proof}