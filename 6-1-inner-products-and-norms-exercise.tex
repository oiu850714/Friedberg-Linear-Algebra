\exercisesection

\begin{exercise} \label{exercise 6.1.1}
Label the following statements as true or false.
\begin{enumerate}
\item An inner product is a scalar-valued function on the set of ordered pairs of vectors.
\item An inner product space must be over the field of real or complex numbers.
\item An inner product is linear in both components.
\item There is exactly one inner product on the vector space \(\SET{R}^n\).
\item The triangle inequality only holds in finite-dimensional inner product spaces.
\item Only square matrices have a conjugate-transpose.
\item If \(x, y\), and \(z\) are vectors in an inner product space such that \((x, y) = (x, z)\), then \(y = z\).
\item If \((x, y) = 0\) for all \(x\) in an inner product space, then \(y = \OV\).
\end{enumerate}
\end{exercise}

\begin{proof} \ 

\begin{enumerate}
\item True by \DEF{6.1}.
\item It seems that it must be true; see some \href{https://math.stackexchange.com/questions/49348/inner-product-spaces-over-finite-fields}{reference1},
\href{https://mathoverflow.net/questions/129413/what-fields-can-be-used-for-an-inner-product-space}{reference 2}.

\item False; in general (by \RMK{6.1.8}) it is only \emph{conjugate linear} in the second component.
\item False, any function that satisfies \DEF{6.1} is an inner product.
\EXAMPLE{6.1.2} is a (counter)example.

\item False. \THM{6.2}(d) does not require that the corresponding vector space need to be finite-dimensional.

\item False. \DEF{6.2} does not require the corresponding matrix should be square.
\item False. Counterexample can be found, e.g. \(x = \OV\).
\item True. In this case we have \(\LG x, \RED{\OV} \RG = 0 = \LG x, \RED{y} \RG\) for all \(x\) in the inner product space.
Then by \THM{6.1}(e), \(y = \OV\).
\end{enumerate}
\end{proof}

\begin{exercise} \label{exercise 6.1.2}
Let \(x = (2, 1 + \iu, \iu)\) and \(y = (2 - \iu, 2, 1 + 2\iu)\) be vectors in \(\SET{C}^3\).
Compute \(\LG x, y \RG, \norm{x}, \norm{y}\), and \(\norm{x + y}\).
Then verify both the Cauchy-Schwarz inequality and the triangle inequality.
\end{exercise}
We have
\begin{align*}
    \LG x, y \RG & = 2(\conjugatet{2-\iu}) + (1+\iu)(\conjugatet{2}) + (\iu) \conjugatet{(1 + 2 \iu}) \\
    & = 2(2 + \iu) + (1 + \iu) 2 + (\iu)(1 - 2\iu) \\
    & = 4 + 2\iu + 2 + 2\iu + \iu + 2 \\
    & = 8 + 5\iu
\end{align*}
\begin{align*}
    \norm{x} &= \sqrt{\LG x, x \RG} = \sqrt{2 \cdot 2 + (1+\iu)(1-\iu) + \iu(-\iu)} = \sqrt{4 + 2 + 1}= \sqrt{7} \\
    \norm{y} &= \sqrt{\LG y, y \RG} = \sqrt{(2-\iu)(2+\iu) + 2(2) + (1 + 2\iu)(1 - 2\iu)} = \sqrt{5 + 4 + 5} = \sqrt{14} \\
    x + y &= (4 - \iu, 3 + \iu, 1 + 3\iu) \\
    \norm{x + y} & = \sqrt{(4-\iu)(4 + \iu) + (3 + \iu)(3 - \iu) + (1 + 3\iu)(1 - 3\iu)} = \sqrt{17 + 10 + 10} = \sqrt{37} \\
    \text { For Cauchy,} & \norm{\LG x, y \RG} = \abs{8 + 5\iu} = \sqrt{89}, \norm{x} \cdot \norm{y} = \sqrt{98} \\
    \text { So } & \norm{\LG x, y \RG} < \norm{x} \cdot \norm{y} \\
    \text { For Triangular,} & \norm{x + y} = \sqrt{37}, \norm{x} + \norm{y} = \sqrt{7} + \sqrt{14} \\
    \text{ So } & \norm{x + y} < \norm{x} + \norm{y}.
\end{align*}
\begin{proof}
\end{proof}

\begin{exercise} \label{exercise 6.1.3}
In \(\CONT([0, 1])\), let \(f(t) = t\) and \(g(t) = e^t\).
Compute \(\LG f, g \RG\) (as defined in \EXAMPLE{6.1.3}), \(\norm{f}, \norm{g}\), and \(\norm{f + g}\). Then verify both the Cauchy-Schwarz inequality and the triangle inequality.
\end{exercise}

\begin{proof}
We put the definition here:
\[
    \LG f, g \RG = \int_0^1 f(x)g(x) dx.
\]
Thus, for \(f(x) = x\) and \(g(x) = e^x\) we have
\begin{align*}
    \LG f, g \RG & = \LG x, e^x \RG = \int_0^1 x e^x dx & \text{by definition} \\
    & = xe^x \Big|_0^1 - \int_0^1 e^x dx & \text{By Calculus, \emph{integration by parts}} \\
    & = e - e^x \Big|_0^1 = e - e - (-1) = 1.
\end{align*}
And similarly,
\begin{align*}
    \norm{f}^2 & = \LG x, x \RG = \int_0^1 x^2 dx
    = \frac{1}{3}x^3 \Big|_0^1
    = \frac{1}{3} - 0 = \frac{1}{3} \\
    \implies & \norm{f} = \frac{\sqrt{3}}{3}. \\
    \norm{g}^2 & = \LG e^x, e^x \RG = \int_0^1 e^{2x} dx
    = \frac{1}{2} e^{2x} \Big|_0^1
    = \frac{1}{2} e^2 - \frac{1}{2} \\
    \implies & \norm{g} = \sqrt{\frac{1}{2} e^2 - \frac{1}{2}}.
\end{align*}
Moreover:
\begin{align*}
    \norm{f + g}^2 & = \LG x + e^x, x + e^x \RG = \int_0^1 (x + e^x)^2 dx \\
    & = \int_0^1 e^{2x} + x^2 + 2xe^x dx = ... = \frac{1}{2}e^2 + \frac{11}{6} \\
    \implies & \norm{f + g} = \sqrt{\frac{1}{2}e^2 + \frac{11}{6}}.
\end{align*}
So Cauchy–Schwarz inequality is satisfied since:
\begin{align*}
    \norm{f} \cdot \norm{g} = \frac{\sqrt{3}}{3} \sqrt{\frac{1}{2} e^2 - \frac{1}{2}} = \sqrt{\frac{1}{6}e^2 - \frac{1}{6}} > \sqrt{1} = 1 = \abs{1} = \abs{\LG f, g \RG}.
\end{align*}
And triangular inequality is satisfied since (using calculator):
\begin{align*}
    \norm{f + g} & \approx 2.35114 \le 2.625404 
    \approx \frac{\sqrt{3}}{3} + \sqrt{\frac{1}{2} e^2 - \frac{1}{2}} = \norm{f} + \norm{g}.
\end{align*}
\end{proof}

\begin{exercise} \label{exercise 6.1.4} \ 

\begin{enumerate}
\item Complete the proof in \EXAMPLE{6.1.5} that \(\LG \cdot , \cdot \RG\) is an inner product (the Frobenius inner product) on \(M_{n \X n}(F)\).
\item Use the Frobenius inner product to compute \(\norm{A}, \norm{B}\), and \(\LG A, B \RG\) for
\[
    A = \begin{pmatrix} 1 & 2 + \iu \\ 3 & \iu \end{pmatrix}
    \quad \text{ and } \quad
    B = \begin{pmatrix} 1 + \iu & 0 \\ \iu & -\iu \end{pmatrix}.
\]
\end{enumerate}
\end{exercise}

\begin{proof} \ 

\begin{enumerate}
\item The example has shown the (a)(d) of \DEF{6.1}, so we show (b)(c).
For (b), let \(c\) be a scalar, \(A, B\) be \(n \X n\) matrices,
\begin{align*}
    \LG cA, B \RG & = \TRACE(B^* cA) & \text{by def of \(\InnerOp\)} \\
        & = \sum_{i = 1}^n (B^* cA)_{ii} & \text{by def of trace} \\
        & = \sum_{i = 1}^n \sum_{k = 1}^n (B^*)_{ik} (cA)_{ki} & \text{by def of matrix product} \\
        & = c \sum_{i = 1}^n \sum_{k = 1}^n (B^*)_{ik} (A)_{ki} & \text{put constant out of (double) sum} \\
        & = c \sum_{i = 1}^n (B^* A)_{ii} & \text{by def of matrix product} \\
        & = c \cdot \TRACE(B^* A) & \text{by def of trace} \\
        & = c \LG A, B \RG & \text{by def of \(\InnerOp\)}
\end{align*}
For (c),
\begin{align*}
    \conjugatet{\LG A, B \RG} & = \conjugatet{\TRACE(B^* A)} = \conjugatet{\sum_{i = 1}^n (B^* A)_{ii}} & \text{by def of trace} \\
        & = \sum_{i = 1}^n \conjugatet{(B^* A)_{ii}} & \text{by \THM{d.2}(b)} \\
        & = \sum_{i = 1}^n \conjugatet{\sum_{k = 1}^n (B^*)_{ik} A_{ki}} & \text{by def of matrix product} \\
        & = \sum_{i = 1}^n \sum_{k = 1}^n \conjugatet{(B^*)_{ik} A_{ki}} & \text{by \THM{d.2}(b)} \\
        & = \sum_{i = 1}^n \sum_{k = 1}^n \conjugatet{(B^*)_{ik}} \cdot \conjugatet{A_{ki}} & \text{by \THM{d.2}(c)} \\
        & = \sum_{i = 1}^n \sum_{k = 1}^n \conjugatet{\conjugatet{B_{ki}}} (A^*)_{ik} & \text{by \DEF{6.2}} \\
        & = \sum_{i = 1}^n \sum_{k = 1}^n B_{ki} (A^*)_{ik} & \text{by \THM{d.2}(a)} \\
        & = \sum_{i = 1}^n \sum_{k = 1}^n (A^*)_{ik} B_{ki} & \text{of course} \\
        & = \sum_{i = 1}^n (A^* B)_{ii} & \text{by def of matrix product} \\
        & = \TRACE(A^* B) & \text{by def of trace} \\
        & = \LG B, A \RG & \text{by def of \(\InnerOp\)}
\end{align*}
So we proved \DEF{6.1}(b), (c) for Frobenius inner product, as desired.

\item Just using the simpler way in \RMK{6.1.5} to calculate Frobenius inner product.
\begin{align*}
    \LG A, B \RG & = \sum_{i = 1}^2 \sum_{j = 1}^2 A_{ij} \conjugatet{B_{ij}} & \text{by \RMK{6.1.5}} \\
        & = A_{11} \conjugatet{B_{11}} + A_{12} \conjugatet{B_{12}} + A_{21} \conjugatet{B_{21}} + A_{22} \conjugatet{B_{22}} \\
        & = 1(1 - \iu) + (2 + \iu)(0) + 3(-\iu) + \iu(-(-\iu) \\
        & = 1 - \iu - 3\iu - 1 = -4\iu.
\end{align*}
Similarly,
\begin{align*}
    \norm{A}^2 & = \LG A, A \RG = A_{11} \conjugatet{A_{11}} + A_{12} \conjugatet{A_{12}} + A_{21} \conjugatet{A_{21}} + A_{22} \conjugatet{A_{22}} \\
        & = 1 \cdot 1 + (2 + \iu)(2 - \iu) + 3 \cdot 3 + \iu(-\iu) = 1 + 5 + 9 + 1 = 16. \\
    \implies & \norm{A} = 4.
\end{align*}
And
\begin{align*}
    \norm{B}^2 & = \LG B, B \RG = B_{11} \conjugatet{B_{11}} + B_{12} \conjugatet{B_{12}} + B_{21} \conjugatet{B_{21}} + B_{22} \conjugatet{B_{22}} \\
        & = (1 + \iu)(1 - \iu) + 0 + \iu(-\iu) + (-\iu)(-(-\iu)) = 2 + 1 + 1 = 4. \\
    \implies & \norm{B} = 2.
\end{align*}
\end{enumerate}
\end{proof}

\begin{exercise} \label{exercise 6.1.5}
In \(\SET{C}^2\), show that \(\LG x, y \RG = x A y^*\) is an inner product, where
\[
    A = \begin{pmatrix} 1 & \iu \\ -\iu & 2 \end{pmatrix}.
\]
Compute \(\LG x, y \RG\) for \(x = (1 - \iu, 2 + 3\iu)\) and \(y = (2 + \iu, 3 - 2\iu)\).
\end{exercise}

\begin{note}
According to the context, I assume the elements \(x, y, z\) in \(\SET{C}^2\) are \emph{row} vectors (hence \(x^*, y^*, z^*\) are column vectors), and we can view them as \(1 \X 2\) or \(2 \X 1\) matrices as desired.

We have said that \((A^*)^* = A\) in \DEF{6.2}, and from \CORO{6.11.1}(c), \((AB)^* = B^* A^*\);
the proof in that corollary uses the concept of adjoint operator, but it can be directly proved using \DEF{6.2}; see \EXEC{6.3.5}(b).
\end{note}

\begin{proof}
First,
\begin{align*}
    \LG x + z, y \RG & = (x + z) A y^* & \text{by def of \(\InnerOp\)} \\
        & = x A y^* + z A y^* & \text{by \THM{2.12}(a)} \\
        & = \LG x, y \RG + \LG z, y \RG & \text{by def of \(\InnerOp\)}
\end{align*}
Second, for any scalar \(c \in \SET{C}\),
\begin{align*}
    \LG cx, y \RG & = (cx) A y^* & \text{by def of \(\InnerOp\)} \\
        & = c(x A y^*) & \text{by \THM{2.12}(b)} \\
        & = c\LG x, y \RG & \text{by def of \(\InnerOp\)}
\end{align*}
Now note that
\begin{align*}
    A^* & = \conjugatet{A^\top} = \conjugatet{\begin{pmatrix} 1 & \iu \\ -\iu & 2 \end{pmatrix}^\top} \\
        & = \conjugatet{\begin{pmatrix} 1 & -\iu \\ \iu & 2 \end{pmatrix}} \\
        & = \begin{pmatrix} 1 & \iu \\ -\iu & 2 \end{pmatrix} = A.
\end{align*}
(That is, the adjoint of \(A\) is equal to \(A\).)

Then we have
\begin{align*}
    \conjugatet{\LG x, y \RG} & = \conjugatet{x A y^*} & \text{by def of \(\InnerOp\)} \\
        & = (y^*)^* A^* x^* = y A^* x^* & \text{by what we have said} \\
        & = y A x^* & \text{\emph{since} \(A^* = A\)} \\
        & = \LG y, x \RG & \text{by def of \(\InnerOp\)}
\end{align*}

Finally, if \(x = (x_1, x_2) \ne \OV\), then
\begin{align*}
    \LG x, x \RG & = x A x^*
        = \begin{pmatrix} x_1 & x_2 \end{pmatrix}
          \begin{pmatrix} 1 & \iu \\ -\iu & 2 \end{pmatrix}
          \begin{pmatrix} \conjugatet{x_1} \\ \conjugatet{x_2} \end{pmatrix} \\
        & = \begin{pmatrix} x_1 & x_2 \end{pmatrix}
            \begin{pmatrix} \conjugatet{x_1} + \iu \conjugatet{x_2} \\ -\iu \conjugatet{x_1} + 2\conjugatet{x_2} \end{pmatrix} \\
        & = (x_1 \conjugatet{x_1} + x_1 \iu \conjugatet{x_2} - x_2 \iu \conjugatet{x_1} + 2 x_2 \conjugatet{x_2}) \\
        & = (\abs{x_1}^2 + x_1 \iu \conjugatet{x_2} - x_2 \iu \conjugatet{x_1} + 2 \abs{x_2}^2) & \text{by \RMK{d.5}}
\end{align*}
Now if we let \(x_1 = a + b \iu, x_2 = c + d \iu\), where \(a, b, c, d\) are real numbers, then by calculation, \(x_1 \iu \conjugatet{x_2} - x_2 \iu \conjugatet{x_1} = 2(ad - bc)\) and (as usual), \(\abs{x_1}^2 = a^2 + b^2\), \(2 \abs{x_2}^2 = 2c^2 + 2d^2\).
Hence we have
\begin{align*}
    \LG x, x \RG & = a^2 + b^2 + 2(ad - bc) + 2c^2 + 2d^2 \\
        & = a^2 + 2ad + d^2 + d^2 + b^2 - 2bc + c^2 + c^2 & \text{of course} \\
        & = (a + d)^2 + d^2 + (b - c)^2 + c^2. \quad \MAROON{(1)}
\end{align*}
Of course \MAROON{(1)} is nonnegative, so we only need to show that is must be nonzero.
For the sake of contradiction, suppose \(x \ne (0, 0)\) but \MAROON{(1)} is zero;
then in particular, \(d^2 = c^2 = 0\) which implies \(d = c = 0\), which implies \(x_2 = c + d \iu = 0\).
But since \(x = (x_1, x_2) \ne (0, 0)\), \(x_1\) must be nonzero, that is, \(a\) or \(b\) must be nonzero.
If \(a \ne 0\), we get \((a + d)^2 = (a + 0)^2 = a^2 > 0\), and if \(b \ne 0\), then we get \((b - c)^2 = (b - 0)^2 = b^2 > 0\), hence in all cases, \MAROON{(1)} is positive, a contradiction.

Hence \(\LG x, x \RG > 0\) for any nonzero vector \(x\).

Now, for \(x = (1 - \iu, 2 + 3\iu)\) and \(y = (2 + \iu, 3 - 2\iu)\),
\begin{align*}
    \LG x, y \RG &
    = \begin{pmatrix} 1 - \iu & 2 + 3\iu \end{pmatrix}
        \begin{pmatrix} 1 & \iu \\ -\iu & 2 \end{pmatrix}
        \begin{pmatrix} 2 - \iu \\ 3 + 2\iu \end{pmatrix} \\
    & = \begin{pmatrix} 4 - 3\iu & 5 + 7\iu \end{pmatrix}
        \begin{pmatrix} 2 - \iu \\ 3 + 2\iu \end{pmatrix}
    = 6 + 21\iu
\end{align*}
\end{proof}

\begin{exercise} \label{exercise 6.1.6}
Complete the proof of \THM{6.1}.
\end{exercise}

\begin{proof}
See \THM{6.1}.
\end{proof}

\begin{exercise} \label{exercise 6.1.7}
Complete the proof of \THM{6.2}.
\end{exercise}

\begin{proof}
See \THM{6.2}.
\end{proof}

\begin{exercise} \label{exercise 6.1.8}
Provide reasons why each of the following is \emph{not} an inner product on the given vector spaces.
\begin{enumerate}
\item \(\LG (a , b), (c, d) \RG = ac - bd\) on \(\SET{R}^2\).
\item \(\LG A, B \RG = \TRACE(A + B)\) on \(M_{2 \X 2}(\SET{R})\).
\item \(\LG f(x), g(x) \RG = \int_0^1 f'(t)g(t) dt\) on \(\POLYRINF\), where \('\) denotes differentiation.
\end{enumerate}
\end{exercise}

\begin{proof} \ 

\begin{enumerate}
\item Since in particular there is a vector \((1, 2) \in \SET{R}^2\), a nonzero vector, such that \(\LG (1, 2), (1, 2) \RG = 1 \cdot 1 - 2 \cdot 2 = -3\), not a positive real number, violating \DEF{6.1}(d).

\item Since in particular there is a matrix \(A = \begin{pmatrix} -1 & 0 \\ 0 & -1 \end{pmatrix} \in M_{2 \X 2}(\SET{R})\), a nonzero matrix, such that \(\LG A, A \RG = \TRACE \begin{pmatrix} -2 & 0 \\ 0 & -2 \end{pmatrix} = -4\), not a positive real number, violating \DEF{6.1}(d).

\item Since in particular there is a (constant) nonzero polynomial \(f(t) = 1\), such that \(\LG f, f \RG = \int_0^1 f'(t)f(t) dt = \int_0^1 0 \cdot 5 dt = \int_0^1 0 dt = 0\), not a positive real number, violating \DEF{6.1}(d).
\end{enumerate}
\end{proof}

\begin{exercise} \label{exercise 6.1.9}
Let \(\beta\) be a basis for a \emph{finite}-dimensional inner product space.
\begin{enumerate}
\item Prove that if \(\LG x, z \RG = 0\) for all \(z \in \beta\), then \(x = \OV\).
\item Prove that if \(\LG x, z \RG = \LG y, z \RG\) for all \(z \in \beta\), then \(x = y\).
\end{enumerate}
\end{exercise}

\begin{note}
So we only need to check the ``behavior'' of inner product on a given basis, when the inner product space is finite-dimensional.
\end{note}

\begin{proof}
We can use \EXEC{6.1.1}(h) to prove this exercise.
Let \(\beta = \{ z_1, z_2, ..., z_n \}\) be a basis for the inner product space.
\begin{enumerate}
\item Let \(v = a_1 z_1 + a_2 z_2 + ... + a_n z_n\) be an arbitrary vector in the inner product space.
Then Then we have
\begin{align*}
    \LG x, v \RG & = \LG x, a_1 z_1 + a_2 z_2 + ... + a_n z_n \RG \\
        & = \conjugatet{a_1} \LG x, z_1 \RG + \conjugatet{a_2} \LG x, z_2 \RG + ... + \conjugatet{a_n} \LG x, z_n \RG & \text{by \THM{6.1}(a)(b)} \\
        & = \conjugatet{a_1} \cdot 0 + \conjugatet{a_2} \cdot 0 + ... + \conjugatet{a_n} \cdot 0 & \text{by supposition} \\
        & = 0
\end{align*}
But that implies
\begin{align*}
    \LG v, x \RG & = \conjugatet{\LG x, v \RG} & \text{by \DEF{6.1}(c)} \\
        & = \conjugatet{0} = 0 & \text{by what we have shown}
\end{align*}
Hence \(\LG v, x \RG = 0\) for arbitrary \(v\) in the inner product space.
Hence by \EXEC{6.1.1}(h), \(x = \OV\).

\item For all \(z \in \beta\)
\begin{align*}
             & \LG x, z \RG = \LG y, z \RG & \text{by supposition} \\
    \implies & \LG x, z \RG - \LG y, z \RG = 0 & \text{of course} \\
    \implies & \LG x - y, z \RG = 0 & \text{by \DEF{6.1}(a)} \\
\end{align*}
So by part(a), \(x - y = \OV\), which implies \(x = y\).
\end{enumerate}
\end{proof}

\begin{exercise} \label{exercise 6.1.10}
Let \(\V\) be an inner product space, and suppose that \(x\) and \(y\) are \emph{orthogonal} vectors in \(\V\).
Prove that \(\norm{x + y}^2 \RED{=} \norm{x}^2 + \norm{y}^2\).
Deduce the \emph{Pythagorean theorem} in \(\SET{R}^2\).
\end{exercise}

\begin{proof}
We have
\begin{align*}
    \norm{x + y}^2 & = \LG x + y, x + y \RG & \text{by \DEF{6.3}} \\
        & = \LG x, x \RG + \LG x, y \RG + \LG y, x \RG + \LG y, y \RG & \text{by \DEF{6.1}(a) and \THM{6.1}(a)} \\
        & = \LG x, x \RG + 0 + 0 + \LG y, y \RG & \text{since \(x, y\) are orthogonal} \\
        & = \norm{x}^2 + \norm{y}^2 & \text{by \DEF{6.3}}
\end{align*}

For \(\SET{R}^2\) with standard inner product, let \(v, w\) be orthogonal vectors in \(\SET{R}^2\).
Then from the geometry, if we let \(v, w\) be the two sides of the right angle of the right triangle, then the length of these sides is \(\norm{v}\), \(\norm{w}\), respectively, and the length of hypotenuse is equal to the length(norm) of \(v + w\).
And from what we have shown, since \(v, w\) are orthogonal, we have \(\norm{v + w}^2 = \norm{v}^2 + \norm{w}^2\).
That is, the square of the length of hypotenuse is the sum of the square of length of the two sides, proving Pythagorean theorem.
\end{proof}

\begin{exercise} \label{exercise 6.1.11}
Prove the \emph{parallelogram law} on an inner product space \(\V\); that is , show that
\[
    \norm{x + y}^2 + \norm{x - y}^2 = 2\norm{x}^2 + 2\norm{y}^2 \text{ for all } x, y \in \V.
\]
What does this equation state about parallelograms in \(\SET{R}^2\)?
\end{exercise}

\begin{note}
平行四邊形兩對角線的平方的和等於四個邊長的平方的和。
高中通常用餘弦定理解釋。
\end{note}

\begin{proof}
\begin{align*}
    \norm{x + y}^2 + \norm{x - y}^2
    & = \LG x + y, x + y \RG + \LG x - y, x - y \RG \\
    & \quad \quad \text{(by \DEF{6.3})} \\
    & = (\LG x, x \RG + \LG y, x \RG + \LG x, y \RG + \LG y, y \RG) + (\LG x, x \RG + \LG -y, x \RG + \LG x, -y \RG + \LG -y, -y \RG) \\
    & \quad \quad \text{(by \DEF{6.1}(a) and \THM{6.1}(a))} \\
    & = (\LG x, x \RG + \LG y, x \RG + \LG x, y \RG + \LG y, y \RG) + (\LG x, x \RG - \LG y, x \RG + (-\LG x, y \RG) + \LG y, y \RG) \\
    & \quad \quad \text{(by \DEF{6.1}(b) and \THM{6.1}(b)} \\
    & = 2 \LG x, x \RG + 2 \LG y, y \RG \\
    & = 2 \norm{x}^2 + 2 \norm{y}^2 \\
    & \quad \quad \text{(by \DEF{6.3})}
\end{align*}

This law says that given arbitrary parallelogram in \(\SET{R}^2\), the sum of the squares of the lengths of the four sides of the parallelogram equals the sum of the squares of the lengths of the two diagonals.
\end{proof}

\begin{exercise} \label{exercise 6.1.12}
Let \(\{ v_1, v_2, ..., v_k \}\) be an (arbitrary) orthogonal set in \(\V\), and let \(a_1, a_2, ..., a_k\) be (arbitrary) scalars.
Prove that
\[
    \norm{\sum_{i = 1}^k a_i v_i}^2 \RED{=} \sum_{i = 1}^k \abs{a_i}^2 \norm{v_i}^2.
\]
\end{exercise}

\begin{proof}
Since \(\{ v_1, ..., v_k \}\) is an orthogonal set, it's of course that \(\{ a_1 v_1, ..., a_k v_k \}\) is also an orthogonal set.
And by \EXEC{6.1.10} and induction, we have
\[
    \norm{\sum_{i = 1}^k a_i v_i}^2 = \sum_{i = 1}^k \norm{ a_i v_i }^2
\]
And
\begin{align*}
    \sum_{i = 1}^k \norm{ a_i v_i }^2 & = \sum_{i = 1}^k (\abs{a_i} \norm{v_i})^2
    = \sum_{i = 1}^k \abs{a_i}^2 \norm{v_i}^2, & \text{by \THM{6.2}(a)}
\end{align*}
as desired.
\end{proof}

\begin{exercise} \label{exercise 6.1.13}
Suppose that \(\InnerOp_1\) and \(\InnerOp_2\) are two inner products on a vector space \(\V\).
Prove that \(\InnerOp = \InnerOp_1 + \InnerOp_2\) is another inner product on \(\V\).
\end{exercise}

\begin{proof}
We have
\begin{align*}
    \LG x + z, y \RG & = \LG x + z, y \RG_1 + \LG x + z, y \RG_2 & \text{by def of \(\InnerOp\)} \\
        & = (\LG x, y \RG_1 + \LG z, y \RG_1) + (\LG x, y \RG_2 + \LG z, y \RG_2) & \text{by their \DEF{6.1}(a), respectively} \\
        & = (\LG x, y \RG_1 + \LG x, y \RG_2) + (\LG z, y \RG_1 + \LG z, y \RG_2) & \text{of course} \\
        & = \LG x, y \RG + \LG z, y \RG & \text{by def of \(\InnerOp\)}
\end{align*}
And
\begin{align*}
    \LG cx, y \RG & = \LG cx, y \RG_1 + \LG cx, y \RG_2 & \text{by def of \(\InnerOp\)} \\
        & = c\LG x, y \RG_1 + c\LG x, y \RG_2 & \text{by their \DEF{6.1}(b), respectively} \\
        & = c(\LG x, y \RG_1 + \LG x, y \RG_2) & \text{of course} \\
        & = c\LG x, y \RG & \text{by def of \(\InnerOp\)}
\end{align*}
And
\begin{align*}
    \conjugatet{\LG x, y \RG} & = \conjugatet{\LG x, y \RG_1 + \LG x, y \RG_2} & \text{by def of \(\InnerOp\)} \\
        & = \conjugatet{\LG x, y \RG_1} + \conjugatet{\LG x, y \RG_2} & \text{by \THM{d.2}(a)} \\
        & = \LG y, x \RG_1 + \LG y, x \RG_2 & \text{by their \DEF{6.1}(c), respectively} \\
        & = \LG y, x \RG & \text{by def of \(\InnerOp\)}
\end{align*}
Finally,
\begin{align*}
    \LG x, x \RG & = \LG x, x \RG_1 + \LG x, x \RG_2 & \text{by def of \(\InnerOp\)} \\
        & > 0 + 0 & \text{by their \DEF{6.1}(d), respectively} \\
        & = 0.
\end{align*}
So \(\InnerOp\) is also an inner product on \(\V\).
\end{proof}

\begin{exercise} \label{exercise 6.1.14}
Let \(A\) and \(B\) be \(n \X n\) matrices, and let \(c\) be a scalar.
Prove that \((A + cB)^* = A^* + \conjugatet{c}B^*\).
\end{exercise}

\begin{proof}
For all \(i, j\), we have
\begin{align*}
    (A + cB)^*_{ij} & = \conjugatet{(A + cB)_{ji}} & \text{\DEF{6.2}} \\
    & = \conjugatet{A_{ji} + cB_{ji}} & \text{by \THM{2.12}(a)(b)} \\
    & = \conjugatet{A_{ji}} + \conjugatet{cB_{ji}} & \text{by \THM{d.2}(b)} \\
    & = \conjugatet{A_{ji}} + \conjugatet{c}\conjugatet{B_{ji}} & \text{by \THM{d.2}(c)} \\
    & = (A^*)_{ij} + \conjugatet{c}(B^*)_{ij} & \text{by \DEF{6.2}} \\
    & = (A^* + \conjugatet{c}B^*)_{ij} & \text{by \THM{2.12}(a)(b)}
\end{align*}
Hence \((A + cB)^* = A^* + \conjugatet{c}B^*\).
\end{proof}

\begin{exercise} \label{exercise 6.1.15} \ 

\begin{enumerate}
\item Prove that if \(\V\) is an inner product space, then \(\abs{ \LG x, y \RG } = \norm{x} \cdot \norm{y}\) if and only if \emph{one of the vectors \(x\) or \(y\) is a multiple of the other}.
Hint: If the identity holds and \(y \ne \OV\), let
\[
    a = \frac{\LG x, y \RG}{\norm{y}^2},
\]
and let \(z = x - ay\).
Prove that \(y\) and \(z\) are \emph{orthogonal} and
\[
    \abs{a} = \frac{\norm{x}}{\norm{y}}.
\]
Then apply \EXEC{6.1.10} to \(\norm{x}^2 = \norm{ay + z}^2\) to obtain \(\norm{z} = 0\).

\item Derive a similar result for the equality \(\norm{x + y} = \norm{x} + \norm{y}\), and generalize it to the case of \(n\) vectors.
\end{enumerate}
\end{exercise}

\begin{proof} \ 

\begin{enumerate}
\item
\(\Longleftarrow\):
We first exclude the case that one of them is zero vector, since both sides of the equation is equal to zero.
So without loss of generality we can assume \(x = cy\) where both \(x, y \ne \OV\) and \(c \ne 0\).
Then
\begin{align*}
    \abs{\LG x, y \RG} & = \abs{\LG cy, y \RG} & \text{by supposition} \\
        & = \abs{c \LG y, y \RG} & \text{by \DEF{6.1}(b)} \\
        & = \abs{c} \abs{\LG y, y \RG} & \text{by \THM{d.2}(c)} \\
        & = \abs{c} \LG y, y \RG & \text{since by \DEF{6.4} the operand is positive} \\
        & = \abs{c} \norm{y}^2 = \abs{c} \norm{y} \norm{y} & \text{by \DEF{6.3}} \\
        & = \norm{cy} \norm{y} = \norm{x} \norm{y} & \text{by \THM{6.2}(a)}
\end{align*}

\(\Longrightarrow\):
Again we exclude the case that \(x\) or \(y\) is zero.
Suppose \(\abs{ \LG x, y \RG } = \norm{x} \cdot \norm{y}\). \MAROON{(1)}

Then let
\[
    a = \frac{\LG x, y \RG}{\norm{y}^2}, \quad \text{ and } \quad z = x - ay. \quad \quad \MAROON{(2)}
\]
We will show that \(z = \OV\) hence \(y = ax\).

First we show that \(y\) and \(z\) are orthogonal:
\begin{align*}
    \LG z, y \RG & = \LG x - \frac{\LG x, y \RG}{\norm{y}^2}y , y \RG & \text{by def in \MAROON{(2)}} \\
        & = \LG x, y \RG - \frac{\LG x, y \RG}{\norm{y}^2} \LG y, y \RG & \text{by \DEF{6.1}(a)(b)} \\
        & = \LG x, y \RG - \frac{\LG x, y \RG}{\norm{y}^2} \norm{y}^2 & \text{by \DEF{6.3}} \\
        & = \LG x, y \RG - \LG x, y \RG = 0 & \text{of course}
\end{align*}
And
\begin{align*}
    a & = \frac{\LG x, y \RG}{\norm{y}^2} & \text{by def in \MAROON{(2)}} \\
      & = \frac{\norm{x} \cdot \norm{y}}{\norm{y}^2} & \text{by \MAROON{(1)}} \\
      & = \frac{\norm{x}}{\norm{y}}. \quad \quad \MAROON{(3)}
\end{align*}
Finally, since \(z, y\) are orthogonal, \(z, ay\) are also orthogonal. \MAROON{(4)}

And
\begin{align*}
    x & = z + ay & \text{by \MAROON{(2)}} \\
    \implies \norm{x}^2 & = \norm{z + ay}^2 & \text{of course} \\
        & = \norm{z}^2 + \norm{ay}^2& \text{by \MAROON{(4)} and \EXEC{6.1.10}} \\
        & = \norm{z}^2 + \abs{a}^2 \norm{y}^2 & \text{by \THM{6.2}(a)} \\
        & = \norm{z}^2 + \left| \frac{\norm{x}}{\norm{y}} \right|^2 \norm{y}^2 & \text{by \MAROON{(3)}} \\
        & = \norm{z}^2 + \left(\frac{\norm{x}}{\norm{y}}\right)^2 \norm{y}^2 & \text{since the operand is nonnegative} \\
        & = \norm{z}^2 + \norm{x}^2.
\end{align*}
Hence \(\norm{x}^2 = \norm{x}^2 + \norm{z}^2\), so \(\norm{z}^2 = 0\), hence by \THM{6.2}(b), \(z = \OV\), so by \MAROON{(2)}, \(x = ay\), as desired.

\item The statement is: \(\norm{x + y} = \norm{x} + \norm{y}\) if and only if \(x = ay\) for some \textbf{nonnegative} scalar \(a\).

\(\Longleftarrow\), suppose \(x = ay\) for some nonnegative scalar \(a\).
Then
\begin{align*}
    \norm{x + y} & = \norm{ay + y} = \norm{(a + 1)y} & \text{of course} \\
        & = \abs{a + 1}\norm{y} & \text{by \THM{6.2}(a)} \\
        & = (\abs{a} + \abs{1})\norm{y} & \text{since \(a, 1\) are both nonnegative, \(\abs{a + 1} = \abs{a} + \abs{1}\)} \\
        & = \abs{a}\norm{y} + \abs{1}\norm{y} & \text{of course} \\
        & = \norm{ay} + \norm{1y} = \norm{ay} + \norm{y} & \text{by \THM{6.2}(a)} \\
        & = \norm{x} + \norm{y}.
\end{align*}

\(\Longrightarrow\): Suppose \(\norm{x + y} = \norm{x} + \norm{y}\).
Then this causes the ``\(\le\)'' sign in the proof of \THM{6.2}(d) to become ``\(=\)'' sign.
That is,
\begin{align*}
    \norm{x + y}^2 & = \LG x + y, x + y \RG = \LG x, x \RG + \LG x, y \RG + \LG y, x \RG + \LG y, y \RG \\
        & = \norm{x}^2 + 2 \cdot \mathcal{R}\LG x, y \RG + \norm{y}^2 \\
        & \RED{=} \norm{x}^2 + 2 \cdot \left| \LG x, y \RG \right| + \norm{y}^2 \\
        & \RED{=} \norm{x}^2 + 2 \cdot \norm{x} \norm{y} + \norm{y}^2 & \text{by part(c)} \\
        & = \left(\norm{x} + \norm{y}\right)^2.
\end{align*}
So in particular,
\[
    2 \abs{\LG x, y \RG} = 2 \norm{x} \norm{y} \implies \abs{\LG x, y \RG} = \norm{x} \norm{y}.
\]
Then by the part(a) of the exercise, \(y = ax\) for \emph{some scalar} \(a\).
We also need to show that \(a\) is nonnegative.
Suppose for the sake of contradiction that \(a\) is negative, then similar to the \(\Longleftarrow\) direction of the proof, we have
\begin{align*}
    \norm{x + y} & = \norm{ay + y} = \norm{(a + 1)y} & \text{of course} \\
        & = \abs{a + 1}\norm{y} & \text{by \THM{6.2}(a)} \\
        &\ \RED{<}\ (\abs{a} + \abs{1})\norm{y} & \text{since \(a < 0\) but \(1 > 0\), \(\abs{a + 1} < \abs{a} + \abs{1}\)} \\
        & = \abs{a}\norm{y} + \abs{1}\norm{y} & \text{of course} \\
        & = \norm{ay} + \norm{1y} = \norm{ay} + \norm{y} & \text{by \THM{6.2}(a)} \\
        & = \norm{x} + \norm{y}.
\end{align*}
So we have \(\norm{x + y} < \norm{x} + \norm{y}\), a contradiction.
\end{enumerate}
\end{proof}

\begin{note}
所以柯西不等式只要求兩個向量是平行即可;三角不等式除了要求平行之外還要同方向。
\end{note}

\begin{exercise} \label{exercise 6.1.16} \ 

\begin{enumerate}
\item Show that the vector space \(\textsf{H}\) with \(\InnerOp\) defined on \RMK{6.1.8} is an inner product space.
\item Let \(\V = \CONT([0, \RED{1}])\), and define
\[
    \LG f, g \RG = \int_0^{\RED{1/2}} f(t)g(t) dt.
\]
\emph{Is this an inner product on \(\V\)}?
\end{enumerate}
\end{exercise}

\begin{proof} \ 

\begin{enumerate}
\item
Let \(f, g, h\) be functions in \(\textsf{H}\), \(c\) be a scalar.
First,
\begin{align*}
    \LG f + h, g \RG & = \frac{1}{2\pi} \int_{0}^{2\pi} (f(t) + h(t)) \conjugatet{g(t)} d t & \text{by def of \(\InnerOp\)} \\
    & = \frac{1}{2 \pi} \int_{0}^{2 \pi} f(t) \conjugatet{g(t)} d t + \frac{1}{2 \pi} \int_{0}^{2 \pi} h(t) \conjugatet{g(t)} d t & \text{since integration is linear} \\
    & = \LG f, g \RG + \LG h, g \RG. & \text{by def of \(\InnerOp\)}
\end{align*}
And
\begin{align*}
    \LG c f, g \RG & = \frac{1}{2 \pi} \int_{0}^{2 \pi} c f(t) \conjugatet{g(t)} d t & \text{by def of \(\InnerOp\)} \\
    & = c\left(\frac{1}{2 \pi} \int_{0}^{2 \pi} f(t) \conjugatet{g(t)} d t \right) & \text{again since integration is linear} \\
    & = c \LG f, g \RG. & \text{by def of \(\InnerOp\)}
\end{align*}
And
\begin{align*}
    \conjugatet{\LG f, g \RG} & = \conjugatet{\frac{1}{2 \pi} \int_{0}^{2 \pi} f(t) \conjugatet{g(t)} d t} & \text{by def of \(\InnerOp\)} \\
        & = \conjugatet{\frac{1}{2 \pi}} \conjugatet{\int_{0}^{2 \pi} f(t) \conjugatet{g(t)} d t} = \frac{1}{2 \pi} \conjugatet{\int_{0}^{2 \pi} f(t) \conjugatet{g(t)} d t} & \text{by \THM{d.2}(c)} \\
        & = \frac{1}{2 \pi} \int_{0}^{2 \pi} \overline{f(t) \conjugatet{g(t)}} d t & \text{see \RMK{6.1.8}} \\
        & = \frac{1}{2 \pi} \int_{0}^{2 \pi} g(t) \conjugatet{f(t)} d t & \text{by \THM{d.2}(a)(c)} \\
        & = \LG g, f \RG. & \text{by def of \(\InnerOp\)}
\end{align*}
Finally, for any \(f\) that is not zero function,
\begin{align*}
    \LG f, f \RG & = \frac{1}{2 \pi} \int_{0}^{2 \pi} f(t) \conjugatet{f(t)} d t & \text{by def of \(\InnerOp\)} \\
        & = \frac{1}{2 \pi} \int_{0}^{2 \pi} \abs{f(t)}^2 d t & \text{by \RMK{d.5}} \\
        & > 0, &
\end{align*}
since \(f(t)^2\) is continuous and is bounded away from zero on some subinterval of \([0, 2\pi]\).
Hence \(\textsf{H}\) with corresponding \(\InnerOp\) is an inner product space.

\item
We define \(f(t)\) as:
\begin{equation*}
    f(t) = \begin{cases}
        0 & \text{if} \quad t < \frac{1}{2} \\
        t - 1/2 & \text{if} \quad t \ge \frac{1}{2}
    \end{cases}
\end{equation*}
Then \(f\) nonzero function, and is continuous on \([0, 1]\) hence is in \(\CONT([0, 1])\).
But
\begin{align*}
    \LG f, f \RG & = \int_0^{1/2} f(t) f(t) dt & \text{by def of \(\InnerOp\)} \\
        & = \int_0^{1/2} 0 \cdot 0 dt & \text{by def of \(f\)} \\
        & = 0.
\end{align*}
Hence \DEF{6.1}(d) is not satisfied, hence \(\InnerOp\) is \emph{not} an inner product.
\end{enumerate}
\end{proof}

\begin{exercise} \label{exercise 6.1.17}
Let \(\T\) be a linear operator on an inner product space \(\V\), and suppose that \(\norm{\T(x)} = \norm{x}\) for all \(x\).
Prove that \(\T\) is one-to-one.
\end{exercise}

\begin{proof}
We have to show that \(\NULLT = \{ \OV \}\).
(So by \THM{2.4} \(\T\) is one-to-one.)
That is, suppose \(\T(v) = \OV\), we have to show \(v = \OV\).
Then since \(\T(v) = \OV\), by \THM{6.2}(b), \(\norm{\T(v)} = 0\), so by supposition that \(\norm{\T(x)} = \norm{v}\), we have \(\norm{v} = 0\).
So by \THM{6.2}(b) again, \(v = \OV\), as desired.
\end{proof}

\begin{exercise} \label{exercise 6.1.18}
Let \(\V\) be a vector space over \(F\), where \(F = \SET{R}\) or \(F = \SET{C}\), and let \(\W\) be an inner product space over \(F\) with inner product \(\InnerOp\).
If \(\T : \V \to \W\) is linear, prove that \(\LG x, y \RG' = \LG \T(x), \T(y) \RG\) defines an inner product on \(\V\) if and only if \(\T\) is one-to-one.
\end{exercise}

\begin{note}
Notice the exact meaning of every \(\InnerOp\) in the description.
\(\InnerOp'\) now is a function defined on \(\V \X \V \to F\), and its definition depends on \BLUE{(1)} \(\InnerOp\), defined on \(\W \X \W \to F\), and \BLUE{(2)} \(\T\), since it needs to transform vectors in \(\V\) into vectors in \(\W\).
\end{note}

\begin{proof} \ 

\(\Longleftarrow\): Suppose \(\T\) is one-to-one.
Then for \DEF{6.1}(a)(b),
\begin{align*}
    \LG cx + z, y \RG' & = \LG \T(cx + z), \T(y) \RG & \text{by def of \(\InnerOp'\)} \\
        & = \LG c\T(x) + \T(z), \T(y) \RG & \text{since \(\T\) is linear} \\
        & = c \LG \T(x), \T(y) \RG + \LG \T(z), \T(y) \RG & \text{since \(\InnerOp\) is linear} \\
        & = c \LG x, y \RG' + \LG z, y \RG' & \text{by def of \(\InnerOp'\)}
\end{align*}
For (c),
\begin{align*}
    \conjugatet{\LG y, x \RG'} & = \conjugatet{\LG \T(y), \T(x) \RG} & \text{by def of \(\InnerOp\)} \\
        & = \conjugatet{\conjugatet{\LG \T(x), \T(y) \RG}} & \text{by \DEF{6.1}(c) of \(\InnerOp\)} \\
        & = \LG \T(x), \T(y) \RG & \text{by \THM{d.2}(a)} \\
        & = \LG x, y \RG' & \text{by def of \(\InnerOp\)}
\end{align*}

For (d), if \(v \in \V\) is nonzero vector, then \(\T(x)\) is also nonzero vector since \(\T\) is one-to-one.
Hence by \DEF{6.1}(d) of \(\InnerOp\), \(\LG \T(x), \T(x) \RG\) is positive, and by definition of \(\InnerOp'\), \(\LG x, x \RG'\) is positive, as desired.

Hence \(\InnerOp\) satisfies \DEF{6.1} and hence defines an inner product on \(\V\).

\(\Longrightarrow\): We prove the equivalent contrapositive.
That is, suppose \(\T\) is not one-to-one, then \(\InnerOp\) does not define an inner product on \(\V\).
But in particular, since \(\T\) is not one-to-one, \(\T(v) = \OW\) for some nonzero vector \(v \in \V\), hence \(\LG \T(v), \T(v) \RG = \LG \OW, \OW \RG = 0\) by \THM{6.2}(c) of \(\InnerOp\).
But that implies \(\LG v, v \RG' = \LG \T(v), \T(v) \RG = 0\).
So \(\InnerOp'\) violates \DEF{6.1}(d) since there is a nonzero vector such that the result is \(0\), hence \(\InnerOp'\) does not define an inner product on \(\V\).
\end{proof}

\begin{exercise} \label{exercise 6.1.19}
Let \(\V\) be an inner product space.
Prove that
\begin{enumerate}
\item \(\norm{x \pm y}^2 = \norm{x}^2 \pm 2\mathcal{R}\LG x,y \RG + \norm{y}^2\) for all \(x, y \in \V\), where \(\mathcal{R}\LG x,y \RG\) denotes the \emph{real part} of the complex number \(\LG x, y \RG\).

\item \(\left| \norm{x} - \norm{y} \right| \le \norm{x- y}\) for all \(x, y \in \V\).
\end{enumerate}
\end{exercise}

\begin{proof}
Without loss of generality, let \(\LG x, y \RG = a + b\iu\) where \(a, b\) are real numbers.
Then \(\mathcal{R}\LG x, y \RG = a\).
\begin{enumerate}
\item We have
\begin{align*}
    \norm{x + y}^2 & = \LG x + y, x + y \RG & \text{by \DEF{6.3}} \\
        & = \LG x, x \RG + \LG x, y \RG + \LG y, x \RG + \LG y, y \RG & \text{by \DEF{6.1}(a) and \THM{6.1}(a)} \\
        & = \norm{x}^2 + \LG x, y \RG + \LG y, x \RG + \norm{y}^2 & \text{by \DEF{6.3}} \\
        & = \norm{x}^2 + \LG x, y \RG + \conjugatet{\LG x, y \RG} + \norm{y}^2 & \text{by \DEF{6.1}(c)} \\
        & = \norm{x}^2 + (a + b\iu) + \conjugatet{a + b\iu} + \norm{y}^2 & \text{by supposition} \\
        & = \norm{x}^2 + (a + b\iu) + a - b\iu + \norm{y}^2 & \text{of course} \\
        & = \norm{x}^2 + 2a + \norm{y}^2 \\
        & = \norm{x}^2 + 2\mathcal{R}\LG x, y \RG + \norm{y}^2 & \text{by supposition}
\end{align*}

\item
By \THM{6.2}(d), we have
\[
    \norm{x} = \norm{(x - y) + y} \RED{\le} \norm{x - y} + \norm{y}
\]
which implies \(\norm{x} - \norm{y} \le \norm{x - y}\). \MAROON{(1)}

Similarly,
\[
    \norm{y} = \norm{(y - x) + x} \RED{\le} \norm{y - x} + \norm{x}
\]
which implies \(\norm{y} - \norm{x} \le \norm{y - x}\), which by multiplying \(-1\) implies \(\norm{x} - \norm{y} \ge -\norm{y - x}\).
But of course \(\norm{y - x} = \norm{(-1)(x - y)} = \abs{-1}\norm{x - y} = 1\norm{x - y} = \norm{x - y}\), so we have \(\norm{x} - \norm{y} \ge -\norm{x - y}\). \MAROON{(2)}

So by \MAROON{(1)(2)} we have \(-\norm{x - y} \le \norm{x} - \norm{y} \le \norm{x - y}\), which (by the property of real number) is equivalent to \(\left| \norm{x} - \norm{y} \right| \le \norm{x- y}\), as desired.
\end{enumerate}
\end{proof}

\begin{exercise} \label{exercise 6.1.20}
Let \(\V\) be an inner product space over \(F\).
Prove the \emph{\href{https://www.wikiwand.com/en/Polarization_identity}{polar identities}}:
For all \(x, y \in \V\),
\begin{enumerate}
\item \(\LG x, y \RG = \frac{1}{4} \norm{x + y}^2 - \frac{1}{4} \norm{x - y}^2\) if \(F = \SET{R}\);
\item \(\LG x, y \RG = \frac{1}{4} \sum_{k = 1}^4 \iu^k \norm{x + \iu^k y}^2\) if \(F = \SET{C}\) where \(\iu^2 = -1\).
\end{enumerate}
\end{exercise}

\begin{proof} \ 

\begin{enumerate}
\item The heuristic is deriving the equation from right to left, but discard the fraction \(\frac{1}{4}\) first.
That is
\begin{align*}
    & \norm{x + y}^2 - \norm{x - y}^2 \\
    & = \LG x + y, x + y \RG - \LG x - y, x - y \RG & \text{by \DEF{6.3}} \\
    & = \LG x, x \RG + \LG y, x \RG + \LG x, y \RG + \LG y, y \RG - (\LG x, x \RG - \LG y, x \RG - \LG x, y \RG + \LG y, y \RG) & \text{of course} \\
    & = \LG x, x \RG + \LG x, y \RG + \LG x, y \RG + \LG y, y \RG - (\LG x, x \RG - \LG x, y \RG - \LG x, y \RG + \LG y, y \RG) & \text{since \(F = \SET{R}\)} \\
    & = 4 \LG x, y \RG & \text{of course}
\end{align*}
Then by multiplying \(\frac{1}{4}\) to both side, we get \(\LG x, y \RG = \frac{1}{4} \norm{x + y}^2 - \frac{1}{4} \norm{x - y}^2\), as desired.

\item Just use \emph{brute force}.
We have
\begin{align*}
    \iu^1 \norm{x + \iu^1 y}^2
        & = \iu \norm{x + \iu y}^2 = \iu \norm{x}^2 + \iu \norm{y}^2 + \LG x, y \RG - \LG y, x \RG \\
    \iu^2 \norm{x - \iu^2 y}^2
        & = -\norm{x - y}^2 = -\norm{x}^2 - \norm{y}^2 + \LG x, y \RG + \LG y, x \RG \\
    \iu^3 \norm{x - \iu^3 y}^2
        & = -\iu\norm{x - \iu y}^2 = -\iu\norm{x}^2 - \iu\norm{y}^2 + \LG x, y \RG - \LG y, x \RG \\
    \iu^4 \norm{x + \iu^4 y}^2
        & = \norm{x + y}^2 = \norm{x}^2 + \norm{y}^2 + \LG x, y \RG + \LG y, x \RG
\end{align*}
Adding both sides of this four equalities, we have
\[
    \sum_{k = 1}^4 \iu^k \norm{x + \iu^k y}^2 = 4 \LG x, y \RG.
\]
Then by multiplying \(\frac{1}{4}\) to both side, we get \(\LG x, y \RG = \frac{1}{4} \sum_{k = 1}^4 \iu^k \norm{x + \iu^k y}^2\), as desired.
\end{enumerate}
\end{proof}

\begin{exercise} \label{exercise 6.1.21}
Let \(A\) be an \(n \X n\) matrix.
Define
\[
    A_1 = \frac{1}{2} (A + A^*) \quad \text{ and } \quad A_2 = \frac{1}{2\iu} (A - A^*).
\]
\begin{enumerate}
\item Prove that \(A_1^* = A_1, A_2^* = A_2\), and \(A = A_1 + \iu A_2\).
Would it be reasonable to define \(A_1\) and \(A_2\) to be the real and imaginary parts, respectively, of the matrix \(A\)?
\item Let \(A\) be an \(n \X n\) matrix.
Prove that the representation in (a) is \emph{unique}.
That is, prove that if \(A = B_1 + \iu B_2\), where \(B_1^* = B_1\) and \(B_2^* = B_2\), then \(B_1 = A_1\) and \(B_2 = A_2\).
\end{enumerate}
\end{exercise}

\begin{note}
There is some \href{https://www.wikiwand.com/en/Hermitian_matrix#/Decomposition_into_Hermitian_and_skew-Hermitian_matrices}{related wiki} for this exercise.
\end{note}

\begin{proof} \ 

\begin{enumerate}
\item
We have
\begin{align*}
    A_1^* & = \left( \frac{1}{2} A + \frac{1}{2} A^* \right)^* \\
        & = (\frac{1}{2} A)^* + \conjugatet{\left(\frac{1}{2}\right)} (A^*)^* = (\frac{1}{2} A)^* + \frac{1}{2} (A^*)^* & \text{by \EXEC{6.1.14}} \\
        & = \conjugatet{\left(\frac{1}{2}\right)} A^* + \frac{1}{2} (A^*)^* = \frac{1}{2} A^* + \frac{1}{2} (A^*)^* & \text{by \EXEC{6.1.14} again} \\
        & = \frac{1}{2} A^* + \frac{1}{2} A & \text{of course} \\
        & = A_1 & \text{by def of \(A_1\)}
\end{align*}
Similarly,
\begin{align*}
    A_2^* & = \left( \frac{1}{2\iu} A - \frac{1}{2\iu} A^* \right)^* \\
        & = (\frac{1}{2\iu} A)^* - \conjugatet{\left(\frac{1}{2\iu}\right)} (A^*)^* = (\frac{1}{2\iu} A)^* + \frac{1}{2\iu} (A^*)^* & \text{by \EXEC{6.1.14}} \\
        & = \conjugatet{\left(\frac{1}{2\iu}\right)} A^* + \frac{1}{2\iu} (A^*)^* = -\frac{1}{2\iu} A^* + \frac{1}{2\iu} (A^*)^* & \text{by \EXEC{6.1.14} again} \\
        & = -\frac{1}{2\iu} A^* + \frac{1}{2\iu} A = \frac{1}{2\iu} A - \frac{1}{2\iu} A^* & \text{of course} \\
        & = A_2 & \text{by def of \(A_2\)}
\end{align*}
Finally,
\begin{align*}
    A_1 + \iu A_2 & = \frac{1}{2} (A + A^*) + \iu \cdot \frac{1}{2\iu}(A - A^*) \\
        & = \frac{1}{2} (A + A^*) + \frac{1\iu}{2\iu}(A - A^*) \\
        & = \frac{1}{2} (A + A^*) + \frac{1}{2}(A - A^*) \\
        & = A^*
\end{align*}
The representation has some useful properties that split a matrix into two self-adjoint matrices; but that does not mean \(A_1\) only has real entries and \(A_2\) only has imaginary entries.

\item Suppose \(A = B_1 + \iu B_2\) where \(B_1^* = B_1\) and \(B_2^* = B_2\).
In particular,
\begin{align*}
    A^* & = (B_1 + \iu B_2)^* \\
        & = B_1^* + \conjugatet{\iu} B_2^* = B_1^* - \iu B_2^* & \text{using \EXEC{6.1.14}} \\
        & = B_1 - \iu B_2 & \text{since \(B_1^* = B_1\) and \(B_2^* = B_2\)} \\
    \implies & A + A^* = (B_1 + \iu B_2) + (B_1 - \iu B_2) = 2 B_1 \\
    \implies & B_1 = \frac{1}{2} (A + A^*) = A_1. & \text{by def of \(A_1\)}
\end{align*}
Similarly,
\begin{align*}
    A - A^* & = (B_1 + \iu B_2) - (B_1 - \iu B_2) = 2\iu B_2 \\
    \implies & B_2 = \frac{1}{2\iu} (A - A^*) = A_2. & \text{by def of \(A_2\)}
\end{align*}
Hence the representation that \(A = A_1 + \iu A_2\) is unique.
\end{enumerate}
\end{proof}

\begin{exercise} \label{exercise 6.1.22}
Let \(\V\) be a real or complex vector space (possibly infinite-dimensional), and let \(\beta\) be an \emph{arbitrary} basis for \(\V\).
For \(x, y \in \V\) there exist \(v_1, v_2, ..., v_n \in \beta\) such that
\[
    x = \sum_{i = 1}^n a_i v_i \quad \text{ and } \quad y = \sum_{i = 1}^n b_i v_i.
\]
Define
\[
    \LG x, y \RG = \sum_{i = 1}^n a_i \conjugatet{b_i}.
\]
\begin{enumerate}
\item Prove that \(\InnerOp\) is an inner product on \(\V\) and that \(\beta\) is an \textbf{ortho\RED{normal}} basis for \(\V\).
Thus every real or complex vector space (i.e. every vector space over real or complex field) may be regarded as an inner product space, using this inner product.

\item Prove that if \(\V = \SET{R}^n\) or \(\V = \SET{C}^n\) and \(\beta\) \emph{is} the \textbf{standard} ordered basis, then the inner product defined above is the standard inner product.
\end{enumerate}
\end{exercise}

\begin{note}
The \(\InnerOp\) defined by the exercise is \emph{depend on} the given basis \(\beta\).
The meaning of this exercise is that, given a basis, if we define the ``inner product'' of the two vectors as the sum of component-wise inner product of the coordinates(of \(\beta\)) of these vectors,
then it actually is an inner product, and this basis will automatically be an orthonormal basis on this inner product on \(\V\).

This is the generalization of the standard inner product, on \emph{standard} ordered basis, on \(F^n\).
\end{note}

\begin{proof} \ 

\begin{enumerate}
\item Given any basis \(\beta\) for \(\V\), let \(c\) be a scalar, and for any \(x, y, z \in \V\), there exists \(v_1, v_2, ..., v_n \in \beta\) such that
\[
    x = \sum_{i = 1}^n a_i v_i \quad \text{ and } \quad y = \sum_{i = 1}^n b_i v_i \quad \text{ and } z = \sum_{i = 1}^n c_i v_i
\]
And in particular,
\[
    x + z = \sum_{i = 1}^n (a_i + c_i) v_i \quad \text{ and } \quad cx = \sum_{i = 1}^n c a_i v_i.
\]
So
\begin{align*}
    \LG x + z, y \RG & = \sum_{i = 1}^n (a_i + c_i) \conjugatet{b_i} & \text{by def of \(\InnerOp\)} \\
    & = \sum_{i = 1}^n a_i \conjugatet{b_i} + \sum_{i = 1}^n c_i \conjugatet{b_i} & \text{split summation} \\
    & = \LG x, y \RG + \LG z, y \RG & \text{by def of \(\InnerOp\)}
\end{align*}
And
\begin{align*}
    \LG cx, y \RG & = \sum_{i = 1}^n (ca_i) \conjugatet{b_i} & \text{by def of \(\InnerOp\)} \\
    & = c\sum_{i = 1}^n a_i \conjugatet{b_i} & \text{move constant scalar out} \\
    & = c\LG x, y \RG & \text{by def of \(\InnerOp\)}
\end{align*}
And
\begin{align*}
    \conjugatet{\LG y, x \RG} & = \conjugatet{\sum_{i = 1}^n b_i \conjugatet{a_i}} & \text{by def of \(\InnerOp\)} \\
    & = \sum_{i = 1}^n \conjugatet{b_i \conjugatet{a_i}} & \text{by \THM{d.2}(b)} \\
    & = \sum_{i = 1}^n \conjugatet{b_i} \cdot \conjugatet{\conjugatet{a_i}} & \text{by \THM{d.2}(c)} \\
    & = \sum_{i = 1}^n \conjugatet{b_i} \cdot a_i = \sum_{i = 1}^n a_i \conjugatet{b_i} & \text{by \THM{d.2}(a)} \\
    & = \LG x, y \RG & \text{by def of \(\InnerOp\)}
\end{align*}
Finally, Suppose \(x \ne \OV\), then one of the coefficient \(a_k\) must be nonzero for some \(1 \le k \le n\), hence \(\abs{a_k}^2 > 0\).
And
\begin{align*}
    0 < \abs{a_k}^2 & = a_k \conjugatet{a_k} & \text{by \RMK{d.5}} \\
    & \le \sum_{i = 1}^n a_i \conjugatet{a_i} & \text{of course} \\
    & = \LG x, x \RG & \text{by def of \(\InnerOp\)}
\end{align*}
Therefore, \(\InnerOp\) is an inner product on \(\V\).

Now we have to show \(\beta\) is an orthonormal basis for this inner product on \(\V\).
But for any vector \(v \in \beta\), \(v = \sum_{i = 1}^1 1 \cdot v\), hence \(\LG v, v \RG = 1 \cdot \conjugatet{1} = 1\).
And for any \(v, w \in \beta\) where \(v \ne w\), we have \(v = 1 \cdot v + 0 \cdot w\) and \(w = 0 \cdot v + 1 \cdot w\), so \(\LG v, w \RG = 1 \cdot 0 + 0 \cdot 1 = 0\).
Hence \(\beta\) is an orthonormal basis for this inner product on \(\V\).

\item If \(\beta = \{ e_1, e_2, ..., e_n \}\) is the standard ordered basis, then for any \(v, w \in \V\), we can write \(v = a_1 e_1 + ... + a_n e_n\) and \(w = b_1 e_1 + ... + b_n e_n\).
By the definition of the standard inner product, the output of the product of \(v, w\) is \(\sum_{i = 1}^n a_i \conjugatet{b_i}\).
But by the definition of the inner product defined in this exercise, the output is also equal to \(\sum_{i = 1}^n a_i \conjugatet{b_i}\).
So the output of the inner product defined in this exercise is equal to the output of the standard inner product, for all \(v, w \in \V\), hence they are equal to each other.
\end{enumerate}
\end{proof}

\begin{exercise} \label{exercise 6.1.23}
Let \(\V = F^n\) (so in particular \(\V\) is an inner product space and we use standard inner product), and let \(A\) be an arbitrary matrix in \(M_{n \X n}(F)\).
\begin{enumerate}
\item Prove that \(\LG x, Ay \RG = \LG A^*x, y \RG\) for all \(x, y \in \V\).
\item Suppose that \emph{for some} \(B \in M_{n \X n}(F)\), we have \(\LG x, Ay \RG = \LG B x, y \RG\) for all \(x, y \in \V\).
Prove that \(B = A^*\).
\item Let \(\alpha\) be the \emph{standard} ordered basis for \(\V\).
(In particular, \(\alpha\) is an orthonormal basis.)
For any orthonormal basis \(\beta\) for \(\V\), let \(Q\) be the \(n \X n\) matrix whose columns are the vectors in \(\beta\).
Prove that \(Q^* = Q^{-1}\).
\item Define linear operators \(\T\) and \(\U\) on \(\V\) by \(\T(x) = Ax\) and \(\U(x) = A^* x\).
Show that \([\U]_{\beta} = [\T]_{\beta}^*\) for any \emph{orthonormal} basis \(\beta\) for \(\V\).
\end{enumerate}
\end{exercise}

\begin{note}
(a) feels like the generalization of \(\LG x, cy \RG = \conjugatet{c} \LG x, y \RG = \LG \conjugatet{c}x, y \RG\).

For (c), that \(Q\) is an \emph{unitary matrix}; see \SEC{6.5}.

BTW, this exercise is highly related to \SEC{6.3}.

\RED{Warning}: I use some fact in \SEC{6.2} to prove this exercise, since the definition and related properties of \emph{orthonormal basis} are given in \SEC{6.2}.
\end{note}

\begin{proof} \ 

\begin{enumerate}
\item Without loss of generality, let \(x = (x_1, x_2, ..., x_n)\) and \(y = (y_1, y_2, ..., y_n)\).
We have
\begin{align*}
    & \LG x, Ay \RG \\
    & = \LG (x_1, x_2, ..., x_n), \left( \sum_{j = 1}^n A_{1j} y_j, \sum_{j = 1}^n A_{2j} y_j, ..., \sum_{j = 1}^n A_{nj} y_j \right) \RG & \text{of course} \\
    & = x_1 \conjugatet{\left( \sum_{j = 1}^n A_{1j} y_j \right)}
      + x_2 \conjugatet{\left( \sum_{j = 1}^n A_{2j} y_j \right)}
      + ...
      + x_n \conjugatet{\left( \sum_{j = 1}^n A_{nj} y_j \right)} & \text{by def of std inner prod.} \\
    & = \sum_{i = 1}^n x_i \conjugatet{\sum_{j = 1}^n A_{ij} y_j}
      = \sum_{i = 1}^n x_i \sum_{j = 1}^n \conjugatet{A_{ij} y_j}
      = \sum_{i = 1}^n x_i \sum_{j = 1}^n \conjugatet{A_{ij}} \conjugatet{y_j} & \text{combine the summation} \\
    & = \sum_{i = 1}^n \sum_{j = 1}^n x_i \conjugatet{A_{ij}} \conjugatet{y_j} & \text{move in ``constant''} \\
    & = \sum_{i = 1}^n \sum_{j = 1}^n x_i A^*_{ji} \conjugatet{y_j} & \text{by \DEF{6.2}} \\
    & = \sum_{j = 1}^n \sum_{i = 1}^n x_i A^*_{ji} \conjugatet{y_j} & \text{change summation order} \\
    & = \sum_{j = 1}^n \conjugatet{y_j} \sum_{i = 1}^n x_i A^*_{ji} & \text{move out ``constant''} \\
    & = \conjugatet{y_1} \left( \sum_{i = 1}^n A^*_{1i} x_i \right)
      + \conjugatet{y_2} \left( \sum_{i = 1}^n A^*_{2i} x_i \right)
      + ...
      + \conjugatet{y_n} \left( \sum_{i = 1}^n A^*_{ni} x_i \right) & \text{split the summation} \\
    & = \left( \sum_{j = 1}^n A^*_{1j} x_j \right) \conjugatet{y_1}
      + \left( \sum_{j = 1}^n A^*_{2j} x_j \right) \conjugatet{y_2}
      + ...
      + \left( \sum_{j = 1}^n A^*_{nj} x_j \right) \conjugatet{y_n} & \text{of course} \\
    & = \LG \left( \sum_{j = 1}^n A^*_{1j} x_j, \sum_{j = 1}^n A^*_{2j} x_j, ..., \sum_{j = 1}^n A^*_{nj} x_j \right), (y_1, y_2, ..., y_n) \RG & \text{by def of std inner prod.} \\
    & = \LG A^* x, y \RG & \text{of course}
\end{align*}

\item By part (a) we have \(\LG x, Ay \RG = \LG A^* x, y \RG\) for all \(x, y \in \V\), hence with the supposition we have \(\LG A^* x, y \RG = \LG B x, y \RG\) for all \(x, y \in \V\), and by \THM{6.1}(e) we have \(A^* x = B x\) for all \(x \in \V\), which implies \(A^* = B\) if we just view the matrices as linear operators on \(\V\).

\item In fact, in \EXEC{6.2.11}, we will show the ``row version'' of this problem and the ``column version'' is similar;
furthermore, that exercise does not depends on the previous exercises, so it's OK to forward reference that exercise.

\item (Note that \THM{6.5} and \CORO{6.5.1} is in \SEC{6.2})
Let \(\beta = \{ v_1, v_2, ..., v_n \}\) be an orthonormal basis for \(\V\).
Then in particular for any \(x \in \V\), by \THM{6.5},
\[
    x = \sum_{i = 1}^n \LG x, v_i \RG v_i.
\]
And by \CORO{6.5.1}, \(([\T]_{\beta})_{ij} = \LG \T(v_j), v_i \RG\). \MAROON{(d.1)}
Similarly, \(([\U]_{\beta})_{ij} = \LG \U(v_j), v_i \RG\). \MAROON{(d.2)}
Finally,
\begin{align*}
    \LG \T(v_j), v_i \RG & = \LG A v_j, v_i \RG & \text{by supposition that \(\T(v) = Av\)} \\
        & = \LG (A^*)^* v_j, v_i \RG & \text{of course} \\
        & = \LG v_j, A^* v_i \RG & \text{by part(a)} \\
        & = \conjugatet{\LG A^* v_i, v_j \RG} & \text{by \DEF{6.1}(c)} \\
        & = \conjugatet{\LG \U(v_i), v_j \RG} & \text{by supposition that \(\U(v) = A^*v\)} \\
        & = \conjugatet{([\U]_{\beta})_{ji}} & \text{by \MAROON{(d.2)}} \\
        & = ([\U]_{\beta}^*)_{ij} & \text{by \DEF{6.2}}
\end{align*}
All in all, we have \(([\T]_{\beta})_{ij} = ([\U]_{\beta}^*)_{ij}\) for all \(i, j\).
Hence \([\T]_{\beta} = [\U]_{\beta}^*\).
In particular, \([\T]_{\beta}^* = ([\U]_{\beta}^*)^* = [\U]_{\beta}\).
\end{enumerate}
\end{proof}

\begin{exercise} \label{exercise 6.1.24}
Let \(\V\) be a complex inner product space with an inner product \(\InnerOp\).
Let \([ \cdot, \cdot ]\) be the real-valued function such that \([x, y]\) is the \emph{real part} of the complex number \(\LG x, y \RG\) for all \(x, y \in \V\).
Prove that \([ \cdot, \cdot ]\) is an inner product for \(\V\), where \(\V\) is \textbf{regarded as a vector space \RED{over \(\SET{R}\)}}.
Prove, furthermore, that \([x, \iu x] = 0\) for all \(x \in \V\).
\end{exercise}

\begin{proof}
\end{proof}

\begin{exercise} \label{exercise 6.1.25}
Let \(\V\) be a vector space over \(\SET{C}\), and suppose that \([ \cdot, \cdot ]\) is a \emph{real} inner product on \(\V\), where \(\V\) is regarded as a vector space over \(\SET{R}\), such that \([x, \iu x] = 0\) for all \(x \in \V\).
Let \(\InnerOp\) be the \emph{complex}-valued function defined by
\[
    \LG x, y \RG = [x, y] + \iu[x, \iu y] \text{ for } x, y \in \V.
\]
Prove that \(\InnerOp\) is a complex inner product on \(\V\).
\end{exercise}

\begin{proof}
\end{proof}

The following definition is used in Exercises 26 -- 30.

\begin{additional definition} \label{adef 6.2}
Let \(\V\) be a vector space over \(F\), where \(F\) is either \(\SET{R}\) or \(\SET{C}\).
Regardless of whether \(\V\) is or is not an inner product space, \emph{we may still define a norm} \(\norm{ \cdot }_{\V}\) as a \emph{real}-valued function on \(\V\) satisfying the following three conditions for all \(x, y \in \V\) and \(a \in F\):
\begin{enumerate}
\item[(1)] \(\norm{x}_{\V} \ge 0\), and \(\norm{x}_{\V} = 0\) if and only if \(x = \OV\).
\item[(2)] \(\norm{ax}_{\V} = \abs{a} \cdot \norm{x}_{\OV}\).
\item[(3)] \(\norm{x + y}_{\V} \le \norm{x}_{\V} + \norm{y}_{\V}\).
\end{enumerate}
\end{additional definition}

\begin{exercise} \label{exercise 6.1.26}
Prove that the following are norms on the given vector spaces \(\V\).
\begin{enumerate}
\item \(\V = \SET{R}^2\); \quad \(\norm{(a, b)}_{\V} = \abs{a} + \abs{b}\) for all \(a, b \in V\)
\item \(\V = \CONT([0, 1])\); \quad \(\norm{f}_{\V} = \max_{t \in [0, 1]} \abs{f(t)}\) for all \(f \in \V\)
\item \(\V = \CONT([0, 1])\); \(\norm{f}_{\V} = \int_0^1 \abs{f(t)} dt\) for all \(f \in \V\)
\item \(\V = M_{m \X n}(F)\); \(\norm{A}_{\V} = \max_{i, j} \abs{A_{ij}}\) for all \(A \in \V\)
\end{enumerate}
\end{exercise}

\begin{proof}
\end{proof}

\begin{exercise} \label{exercise 6.1.27}
Use \EXEC{6.1.11} to show that there is \emph{no} inner product \(\InnerOp\) on \(\SET{R}^2\) such that \(\norm{x}_{\V}^2 = \LG x, x \RG\) for all \(x \in \SET{R}^2\) if the norm \(\norm{\cdot}_{\V}\) is defined as in \EXEC{6.1.26}(a).
\end{exercise}

\begin{proof}
\end{proof}

\begin{exercise} \label{exercise 6.1.28}
Let \(\norm{\cdot}_{\V}\) be a norm on a vector space \(\V\), and define, for each ordered pair of vectors, the scalar \(d(x, y) = \norm{x - y}_{\V}\), called the \textbf{distance} between \(x\) and \(y\).
Prove the following results for all \(x, y, z \in \V\).
\begin{enumerate}
\item \(d(x, y) \ge 0\).
\item \(d(x, y) = d(y, x)\).
\item \(d(x, y) \le d(x, z) + d(z, y)\).
\item \RED* \(d(x, x) = 0\) if and only if \(x = 0\).
\item \(d(x, y) \ne 0\) if \(x \ne y\).
\end{enumerate}
\end{exercise}

\begin{note}
(a), (b), (c), and (e) are the requirements for a \emph{metric} \(d\) defined on a set.
But what about (d)?
\end{note}

\begin{proof}
\end{proof}

\begin{exercise} \label{exercise 6.1.29}
Let \(\norm{ \cdot }_{\V}\) be a norm on a real vector space \(\V\) satisfying the parallelogram law given in \EXEC{6.1.11}.
Define
\[
    \LG x, y \RG = \frac{1}{4} \left[ \norm{x + y}_{\V}^2 - \norm{x - y}_{\V}^2 \right].
\]
Prove that \(\InnerOp\) defines an inner product on \(\V\) such that \(\norm{x}_{\V} = \LG x, x \RG\) for all \(x \in \V\).
Hints:
\begin{enumerate}
\item Prove \(\LG x, 2y \RG = 2 \LG x, y \RG\) for all \(x, y \in \V\).
\item Prove \(\LG x + u, y \RG = \LG x, y \RG + \LG u, y \RG\) for all \(x, u, y \in \V\).
\item Prove \(\LG nx, y \RG = n \LG x, y \RG\) for every \emph{positive integer} \(n\) and every \(x, y \in \V\).
\item Prove \(m \LG \frac{1}{m} x, y \RG = \LG x, y \RG\) for every positive integer \(m\) and every \(x, y \in \V\).
\item Prove \(\LG rx, y \RG = r \LG x, y \RG\) for every \emph{rational} number rand every \(x, y \in \V\).
\item Prove \(\abs{ \LG x, y \RG } \le \norm{x}_{\V} \norm{y}_{\V}\) for every \(x, y \in \V\).
Hint: Condition (3) in the definition of norm can be helpful.
\item Prove that for every \(c \in \SET{R}\), every rational number \(r\), and every \(x, y \in \SET{V}\),
\[
    \abs{ c \LG x, y \RG - \LG cx, y \RG } = \abs{ (c - r) \LG x, y \RG - \LG(c - r)x, y \RG} \le 2\abs{c - r} \norm{x}_{\V} \norm{y}_{\V}.
\]
\item Use the fact that for any \(c \in \SET{R}\), \(\abs{c - r}\) can be made \textbf{arbitrarily small}, where \(r\) \emph{varies over the set of \textbf{rational} numbers}, to establish item (b) of the definition of inner product.
\end{enumerate}
\end{exercise}

\begin{proof}
\end{proof}

\begin{exercise} \label{exercise 6.1.30}
Let \(\norm{\cdot}_{\V}\) be a norm (as defined with \ADEF{6.2}) on a complex vector space \(\V\) satisfying the parallelogram law given in \EXEC{6.1.11}.
Prove that \emph{there is} an inner product \(\InnerOp\) on \(\V\) such that \(\norm{x}_{\V}^2 = \LG x, x \RG\) for all \(x \in \V\).
Hint: Apply \EXEC{6.1.29} to \(\V\) regarded as a vector space over \(\SET{R}\).
Then apply \EXEC{6.1.25}.
\end{exercise}
\begin{proof}
\end{proof}
