\exercisesection

\begin{exercise} \label{exercise 6.4.1}
Label the following statements as true or false.
Assume that the underlying inner product spaces are \emph{finite}-dimensional.
\begin{enumerate}
\item Every self-adjoint operator is normal.
\item Operators and their adjoints have the same eigenvectors.
\item If \(\T\) is an operator on an inner product space \(\V\), then \(\T\) is normal if and only if \([\T]_{\beta}\) is normal, where \(\beta\) is \emph{any} ordered basis for \(\V\).
\item A real or complex matrix \(A\) is normal if and only if \(\LMTRAN_A\) is normal.
\item The eigenvalues of a self-adjoint operator must all be real.
\item The identity and zero operators are self-adjoint.
\item Every normal operator is diagonalizable.
\item Every self-adjoint operator is diagonalizable.
\end{enumerate}
\end{exercise}

\begin{proof} \ 

\begin{enumerate}
\item True, since
\begin{align*}
    \T \RED{\T^*} & = \BLUE{\T} \RED{\T} & \text{since \(\RED{\T^*} = \RED{\T}\)} \\
        & = \BLUE{\T^*} \T & \text{since \(\BLUE{\T} = \BLUE{\T^*}\)}
\end{align*}

\item False.
Counterexample, let \(A = \begin{pmatrix} 1 & 1 \\ 0 & 1 \end{pmatrix}\) hence \(A^* = \begin{pmatrix} 1 & 0 \\ 1 & 1 \end{pmatrix}\), then we have \(\begin{pmatrix} 1 \\ 0 \end{pmatrix}\) as \(\LMTRAN_A\)'s eigenvector but not as \((\LMTRAN_A)^* = \LMTRAN_{A^*}\)'s eigenvector.

\item False, \(\beta\) should also be orthonormal.

\item True.
\(\LMTRAN_A\) is normal, iff (by \RMK{6.4.2}) \([\LMTRAN_A]_{\beta}\) is normal where \(\beta\) is the standard ordered (orthonormal) basis for \(F^n\), if and only if (by \THM{2.15}(a)) \(A\) is normal.

\item True by \LEM{6.5}(a).
\item True, since (by \THM{6.11}(d)) \(\ITRAN{}^* = \ITRAN{}\) and (similarly) \(\TZERO^* = \TZERO\).

\item False, a counterexample is \EXAMPLE{6.4.1}, the rotation transformation (over \(\SET{R}\)).

\item True.
For \emph{complex} inner product space, since (by part(a)) self-adjoint implies normal, by \THM{6.16}, we can find a orthonormal basis \(\beta\) such that the matrix representation of the operator using \(\beta\) is diagonal.
For \emph{real} inner product space, the same result is guaranteed by \THM{6.17}.
\end{enumerate}
\end{proof}

\begin{exercise} \label{exercise 6.4.2}
For each linear operator \(\T\) on an inner product space \(\V\), determine whether \(\T\) is normal, self-adjoint, or neither.
If possible, produce an orthonormal basis of eigenvectors of \(\T\) for \(\V\) and list the corresponding eigenvalues.
\begin{enumerate}
\item \(\V = \SET{R}^2\) and \(\T\) is defined by \(\T(a, b) = (2a - 2b, -2a + 5b)\).
\item \(\V = \SET{R}^3\) and \(\T\) is defined by \(\T(a, b, c) = (-a + b, 5b, 4a - 2b + 5c)\).
\item \(\V = \SET{C}^2\) and \(\T\) is defined by \(\T(a, b) = (2a + \iu b, a + 2b)\).
\item \(\V = \POLYRR\) and \(\T\) is defined by \(\T(f) = f'\), where
\[
    \LG f(x), g(x) \RG = \int_0^1 f(t)g(t) dt.
\]
\item \(\V = M_{2 \X 2}(\SET{R})\) and \(\T\) is defined by \(\T(A) = A^\top\).
\item \(\V = M_{2 \X 2}(\SET{R})\) and \(\T\) is defined by \(\T \begin{pmatrix} a & b \\ c & d \end{pmatrix} = \begin{pmatrix} c & d \\ a & b \end{pmatrix}\).
\end{enumerate}
\end{exercise}

\begin{note}
If an operator is self-adjoint, then by \EXEC{6.4.1}(a) it is normal.

To find an orthonormal basis of eigenvectors of \(\T\) for \(\V\), just find an \emph{orthonormal basis for each eigenspace} and take the union of them as the desired basis.
\end{note}

\begin{proof} \ 

\begin{enumerate}
\item Let \(\beta = \{ e_1, e_2 \}\), and let  \(A = [\T]_{\beta} = \begin{pmatrix}2 & -2 \\ -2 & 5\end{pmatrix}\),
then \([\T^*]_{\beta} = \begin{pmatrix}2 & -2 \\ -2 & 5\end{pmatrix}^* = \begin{pmatrix}2 & -2 \\ -2 & 5\end{pmatrix} = A\).
So by \RMK{6.4.6}, \(\T\) is self-adjoint, hence is normal.

\item Let  \(\beta = \{ e_1, e_2, e_3 \}\), and let \(A = [\T]_{\beta} = \begin{pmatrix}-1 & 1 & 0 \\ 0 & 5 & 0 \\ 4 & -2 & 5\end{pmatrix}\), then \([\T^*]_{\beta} = A^{*} = \begin{pmatrix}-1 & 0 & 4 \\ 1 & 5 & -2 \\ 0 & 0 & 5\end{pmatrix} \neq A\), So by \RMK{6.4.6}, \(\T\) is \emph{not} self-adjoint.
And in particular, since \((A A^*)_{11} = 2\) and \((A^* A)_{11} = 17\), \(A A^* \ne A^*A\), and by \RMK{6.4.2}, \(\T\) is not normal.
So in particular, since \(\V\) is over \(\SET{R}\), by \THM{6.17}, we cannot find a orthonormal basis consisting of eigenvectors of \(\T\).

\item Let \(\beta = \{ e_1, e_2 \}\), and let \(A = [\T]_{\beta} = \begin{pmatrix} 2 & \iu \\ 1 & 2 \end{pmatrix}\).
By calculation, \(A\) is normal but not self-adjoint.
Hence (by \RMK{6.4.6} and \RMK{6.4.2}) \(\T\) is normal but not self-adjoint.
Since the \(\V\) is over \(\SET{C}\), and \(\T\) is normal, by \THM{6.16} there exists an orthonormal basis for \(\V\) consisting eigenvectors of \(\T\).
We can compute this basis as the aforementioned note, but skip, tedious.
But one such basis is
\[
    \left\{ \left( \frac{1}{\sqrt{2}}, -\frac{1}{2} + \frac{1}{2}\iu \right), \left( \frac{1}{\sqrt{2}}, \frac{1}{2} - \frac{1}{2}\iu \right) \right\}
\]

\item We have calculated a orthonormal basis with respect to this inner product on \EXEC{6.3.2}:
\[
    \beta = \{ f_1, f_2, f_3 \} = \left\{ 1, 2\sqrt{3} \left( t - \frac{1}{2} \right), 6\sqrt{5} \left( t^2 - t + \frac{1}{6} \right) \right\}.
\]
And
\begin{align*}
    \T(f_1) & = 0 = 0 \cdot f_1 + 0 \cdot f_2 + 0 \cdot f_3 \\
    \T(f_2) & = 2\sqrt{3} = 2\sqrt{3} \cdot f_1 + 0 \cdot f_2 + 0 \cdot f_3 \\
    \T(f_3) & = 12\sqrt{5}t - 6\sqrt{5} = 0 \cdot f_1 + 2\sqrt{15} \cdot f_2 + 0 \cdot f_3 \\
    \implies &
    [\T]_{\beta} = \begin{pmatrix}
        0 & 2\sqrt{3} & 0 \\
        0 & 0 & 2\sqrt{15} \\
        0 & 0 & 0
    \end{pmatrix}
\end{align*}
So by calculation, \([\T]_{\beta}\) is neither self-adjoint nor normal, hence (by \RMK{6.3.6} and \RMK{6.3.2}) \(\T\) is neither self-adjoint nor normal.
So in particular, since \(\V\) is over \(\SET{R}\), by \THM{6.17}, we cannot find an orthonormal basis consisting of eigenvectors of \(\T\).

\item Let \(\beta = \{ E_{11}, E_{12}, E_{21}, E_{22} \}\) the ``standard'' ordered basis.
Note that the ``standard'' inner product on matrix is Frobenius inner product(\ADEF{6.1}), and \(\beta\) is orthonormal with respect to this inner product.
Then, of course,
\[
    [\T]_{\beta} = \begin{pmatrix}
        1 & 0 & 0 & 0 \\
        0 & 0 & 1 & 0 \\
        0 & 1 & 0 & 0 \\
        0 & 0 & 0 & 1
    \end{pmatrix},
\]
which by calculation is self-adjoint and normal, hence (by \RMK{6.3.6} and \RMK{6.3.2}) \(\T\) is both self-adjoint and normal.
Since the \(\V\) is over \(\SET{R}\), and \(\T\) is self-adjoint, by \THM{6.17} there exists an orthonormal basis for \(\V\) consisting eigenvectors of \(\T\).
We can compute this basis as the aforementioned note, but skip, tedious.
But one such basis is
\[
    \left\{ (1,0,0,0), \frac{1}{\sqrt{2}}(0,1,1,0),(0,0,0,1), \frac{1}{\sqrt{2}}(0,1,-1,0)\right \}
\]

\item Pick \(\beta\) as part(e), then, of course,
\[
    [\T]_{\beta} = \begin{pmatrix}
        0 & 0 & 1 & 0 \\
        0 & 0 & 0 & 1 \\
        1 & 0 & 0 & 0 \\
        0 & 1 & 0 & 0
    \end{pmatrix},
\]
which by calculation is self-adjoint and normal, hence (by \RMK{6.3.6} and \RMK{6.3.2}) \(\T\) is both self-adjoint and normal.
Since the \(\V\) is over \(\SET{R}\), and \(\T\) is self-adjoint, by \THM{6.17} there exists an orthonormal basis for \(\V\) consisting eigenvectors of \(\T\).
We can compute this basis as the aforementioned note, but skip, tedious.
But one such basis is
\[
    \left\{\frac{1}{\sqrt{2}}(1,0,-1,0), \frac{1}{\sqrt{2}}(0,1,0,-1), \frac{1}{\sqrt{2}}(1,0,1,0), \frac{1}{\sqrt{2}}(0,1,0,1)\right\}
\]
\end{enumerate}
\end{proof}

\begin{exercise} \label{exercise 6.4.3}
Give an example of a linear operator \(\T\) on \(\SET{R}^2\) and an \emph{ordered basis}(but not orthonormal) for \(\SET{R}^2\) that provides a \emph{counterexample} to the statement in \EXEC{6.4.1}(c).
\end{exercise}

\begin{proof}
Let \(\T(a, b) = (2a, b)\), and \(\beta = \{ (1, 1), (1, 0) \}\).
In particular for \(\alpha\) be the standard ordered (orthonormal) basis, \([\T]_{\alpha} = \begin{pmatrix} 2 & 0 \\ 0 & 1 \end{pmatrix}\), which is symmetric hence of course normal, hence (by \RMK{6.3.2}) \(\T\) is normal.
But \([\T]_{\beta} = \begin{pmatrix} 1 & 0 \\ 1 & 2 \end{pmatrix}\), which is \emph{not} normal.

Furthermore, the converse is also not true;
that is, we can find an operator \(\T\), which is not normal, but we can find an ordered basis \(\beta\) such that \([\T]_{\beta}\) is normal.
The process is similar.
\end{proof}

\begin{exercise} \label{exercise 6.4.4}
Let \(\T\) and \(\U\) be self-adjoint operators on an inner product space \(\V\).
Prove that \(\T\U\) is self-adjoint if and only if \(\T\U = \U\T\).
\end{exercise}

\begin{note}
根據這個陳述可知,兩個\ self-adjoint 的算子的合成\textbf{不一定}會是\ self-adjoint。
\end{note}

\begin{proof}
Suppose \(\T\U\) is self-adjoint.
Then
\begin{align*}
    \T\U & = (\T\U)^* & \text{since \(\T\U\) is self-adjoint} \\
        & = \U^*\T^* & \text{by \THM{6.11}(c)} \\
        & = \U\T^* = \U\T. & \text{since \(\T\) and \(\U\) are self-adjoint}
\end{align*}

Now suppose \(\T\U = \U\T\).
Then
\begin{align*}
    (\T\U)^* & = \U^*\T^* & \text{by \THM{6.11}(c)} \\
        & = \U\T^* = \U\T & \text{since \(\T\) and \(\U\) are self-adjoint} \\
        & = \T\U & \text{since \(\T\U = \U\T\)}
\end{align*}
Hence \(\T\U\) is self-adjoint.
\end{proof}

\begin{exercise} \label{exercise 6.4.5}
Prove \THM{6.15}(b).
\end{exercise}

\begin{proof}
See \THM{6.15}.
\end{proof}

\begin{exercise} \label{exercise 6.4.6}
Let \(\V\) be a \emph{complex} inner product space, and let \(\T\) be a linear operator on \(\V\).
Define
\[
    \T_1 = \frac{1}{2}(\T + \T^*) \quad \text{ and } \quad \T_2 = \frac{1}{2\MAROON{\iu}} (\T - \T^*).
\]
\begin{enumerate}
\item Prove that \(\T_1\) and \(\T_2\) are self-adjoint and that \(\T = \T_1 + \iu \T_2\).
\item Suppose also that \(\T = \U_1 + \iu\U_2\), where \(\U_1\) and \(\U_2\) are self-adjoint.
Prove that \(\U_1 = \T_1\) and \(\U_2 = \T_2\).
(Hence the representation is unique.)
\item Prove that \(\T\) is normal if and only if \(\T_1 \T_2 = \T_2 \T_1\).
\end{enumerate}
\end{exercise}

\begin{proof}
There is a related \EXEC{6.1.21}.
\begin{enumerate}
\item
We have
\begin{align*}
    (\T_1)^* & = \left[ \frac{1}{2}(\T + \T^*) \right]^* \\
        & = \conjugatet{\left(\frac{1}{2}\right)} (\T + \T^*)^* = \frac{1}{2} (\T + \T^*)^*& \text{by \THM{6.11}(b)} \\
        & = \frac{1}{2} (\T^* + (\T^*)^*) & \text{by \THM{6.11}(a)} \\
        & = \frac{1}{2} (\T^* + \T) = \frac{1}{2} (\T + \T^*) & \text{by \THM{6.11}(d)} \\
        & = \T_1
\end{align*}
and
\begin{align*}
    (\T_2)^* & = \left[ \frac{1}{2\iu}(\T - \T^*) \right]^* \\
        & = \conjugatet{\left(\frac{1}{2\iu}\right)} (\T - \T^*)^* = -\frac{1}{2\iu} (\T - \T^*)^*& \text{by \THM{6.11}(b)} \\
        & = -\frac{1}{2\iu} (\T^* - (\T^*)^*) & \text{by \THM{6.11}(a)} \\
        & = -\frac{1}{2\iu} (\T^* - \T) = \frac{1}{2\iu} (\T - \T^*) & \text{by \THM{6.11}(d)} \\
        & = \T_2
\end{align*}
Finally,
\begin{align*}
    \T_1 + \iu \T_2 & = \frac{1}{2}(\T + \T^*) + \iu \cdot \frac{1}{2\MAROON{\iu}} (\T - \T^*) \\
        & = \frac{1}{2}(\T + \T^*) + \frac{1\iu}{2\iu} (\T - \T^*) \\
        & =\frac{1}{2}(\T + \T^*) + \frac{1}{2} (\T - \T^*)
        = \T.
\end{align*}

\item Suppose \(\T = \U_1 + \iu \U_2\) where \(\U_1\) and \(\U_2\) are self-adjoint.
In particular,
\begin{align*}
    \T^* & = (\U_1 + \iu \U_2)^* \\
        & = \U_1^* + \conjugatet{\iu} \U_2^* = \U_1^* - \iu \U_2^* & \text{by \THM{6.11}(b)} \\
        & = \U_1 - \iu \U_2 & \text{since \(\U_1, \U_2\) are self-adjoint} \\
    \implies & \T + \T^* = (\U_1 + \iu \U_2) + (\U_1 - \iu \U_2) = 2 \U_1 \\
    \implies & \U_1 = \frac{1}{2} (\T + \T^*) = \T_1. & \text{by def of \(\T_1\)}
\end{align*}
Similarly,
\begin{align*}
    \T - \T^* & = (\U_1 + \iu \U_2) - (\U_1 - \iu \U_2) = 2\iu \U_2 \\
    \implies & \U_2 = \frac{1}{2\iu} (\T - \T^*) = \T_2. & \text{by def of \(\T_2\)}
\end{align*}
Hence the representation that \(\T = \T_1 + \iu \T_2\) is unique.

\item By tedious but skipped calculation, we have the equation
\[
    \T_1\T_2 - \T_2\T_1 = ... = \frac{1}{2\iu} (\T^*\T - \T\T^*).
\]
Then from this equation it's clear that \(\T_1\T_2 = \T_2\T_1\) if and only if \(\T^*\T = \T\T^*\), that is, if and only if \(\T\) is normal.
\end{enumerate}
\end{proof}

\begin{exercise} \label{exercise 6.4.7}
Let \(\T\) be a linear operator on an inner product space \(\V\), and let \(\W\) be a \(\T\)-invariant subspace of \(\V\).
Prove the following results.
\begin{enumerate}
\item If \(\T\) is self-adjoint, then \(\T_\W\) is self-adjoint.
\item \(\W^{\perp}\) is \(\T^{\RED{*}}\)-invariant.
\item If \(\W\) is both \(\T\)- and \(\T^*\)-invariant, then \((\T_\W)^* = (\T^*)_\W\).
\item If \(\W\) is both \(\T\)- and \(\T^*\)-invariant and \(\T\) is normal, then \(\T_\W\) is normal.
\end{enumerate}
\end{exercise}

\begin{proof}
Although \(\V\) (and \(\W\)) may be infinite-dimensional, by \RMK{6.3.4}, we assume the existence of the adjoint of any linear operators.

Note that \(\W\) itself can be seen as an inner product space, using the inner product defined on \(\V\).
\begin{enumerate}
\item Suppose \(\T\) is self-adjoint, we have to show \(\T_\W = (\T_\W)^*\).
Then for any \(x, y \in \W\),
\begin{align*}
    \LG x, (\T_\W)^*(y) \RG & = \LG \T_\W(x), y \RG & \text{by \THM{6.9}} \\
        & = \LG \T(x), y \RG & \text{by def of \(\T_\W\)} \\
        & = \LG x, \T^*(y) \RG & \text{by \THM{6.9} again} \\
        & = \LG x, \T(y) \RG & \text{since \(\T\) is self-adjoint} \\
        & = \LG x, \T_\W(y) \RG & \text{by def of \(\T_\W\)}
\end{align*}
which (by \THM{6.1}(e)) implies \((\T_\W)^*(y) = \T_\W(y)\) for all \(y \in \W\), hence \((\T_\W)^* = \T_\W\).

\item Suppose \(x \in \W^{\perp}\), we have to show \(\T^*(x) \in \W^{\perp}\).
To do this, (by \DEF{6.7}) we can show that \(\LG \T^*(x), y \RG = 0\) for all \(y \in \W\).
But for all \(y \in \W\),
\begin{align*}
    \LG \T^*(x), y \RG & = \LG \T(x), y \RG & \text{since \(\T\) is self-adjoint} \\
        & = \LG x, \T^*(y) \RG & \text{by \THM{6.9}} \\
        & = \LG x, \T(y) \RG & \text{since \(\T\) is self-adjoint} \\
        & = 0, &
\end{align*}
where the last step is because that \(\T(y) \in \W\) (since \(y \in \W\) and \(\W\) is \(\T\)-invariant) and \(x \in \W^{\perp}\).
Hence \(\T^*(x) \in \W^{\perp}\), so \(\W^{\perp}\) is \(\T^*\)-invariant.

\item (By the assumption that \(\W\) is \(\T^*\)-invariant, the notation \((\T^*)_\W\) now does make sense.)
For all \(x, y \in \W\),
\begin{align*}
    \LG x, (\T_\W)^*(y) \RG & = \LG \T_\W(x), y \RG & \text{by \THM{6.9}} \\
        & = \LG \T(x), y \RG & \text{by def of \(\T_\W\)} \\
        & = \LG x, \T^*(y) \RG & \text{by \THM{6.9} again} \\
        & = \LG x, (\T^*)_\W(y) \RG & \text{by def of \((\T^*)_\W\)}
\end{align*}
Hence (by \THM{6.1}(e)) \((\T_\W)^*(y) = (\T^*)_\W(y)\) for all \(y \in \W\), hence \((\T_\W)^* = (\T^*)_\W\).

\item We have to show \((\T_\W)^* \T_\W = \T_\W (\T_\W)^*\).
And for all \(x \in \W\),
\begin{align*}
    (\T_\W)^* \T_\W(x) & = \RED{(\T^*)_\W} \T_\W(x) & \text{by part(c)} \\
        & = \T^* \T (x) & \text{by def of \((\T^*)_\W\) and \(\T_\W\)} \\
        & = \T \T^* (x) & \text{since \(\T\) is normal} \\
        & = \T_\W (\T^*)_\W (x) & \text{by def of \(\T_\W\) and \((\T^*)_\W\) again} \\
        & = \T_\W \RED{(\T_\W)^*} (x) & \text{by part(c)}
\end{align*}
Hence \((\T_\W)^* \T_\W = \T_\W (\T_\W)^*\), as desired.
\end{enumerate}
\end{proof}

\begin{exercise} \label{exercise 6.4.8}
Let \(\T\) be a \emph{normal} operator on a \emph{finite}-dimensional \textbf{complex} inner product space \(\V\), and let \(\W\) be a subspace of \(\V\).
Prove that if \(\W\) is \(\T\)-invariant, then \(\W\) is also \(\T^*\)-invariant.
\emph{Hint}: Use \EXEC{5.4.24}.
\end{exercise}

\begin{proof}
Since \(\V\) is over \(\SET{C}\) and \(\T\) is normal, by \THM{6.16} there exists an orthonormal basis consisting of eigenvectors of \(\V\).
In particular, \(\T\) is diagonalizable, and by \EXEC{5.4.24}, \(\T_\W\) is diagonalizable.
So let \(\beta_\W = \{ v_1, v_2, ..., v_k \}\) be the basis for \(\W\) and consist of eigenvectors of \(\T_\W\).
Then in particular, \(v_1, ..., v_k\) are eigenvectors of \(\T\); and we can let the corresponding eigenvalues as \(\lambda_1, ..., \lambda_k\).
By \THM{6.15}(c), \(v_1, ..., v_k\) are eigenvectors of \(\T^*\) corresponding to \(\conjugatet{\lambda_1}, ..., \conjugatet{\lambda_k}\).

Now we show that \(\W\) is \(\T^*\)-invariant, that is, if \(v \in \W\), then \(\T^*(v) \in \W\).
So let \(v \in \W\).
Since \(\beta_\W\) is a basis for \(\W\), we can let \(v = a_1 v_1 + ... + a_k v_k\) for some coefficients \(a_1, ..., a_k\), and
\begin{align*}
    \T^*(v) & = \T^*(a_1 v_1 + ... + a_k v_k) \\
        & = a_1 \T^* (v_1) + ... + a_k \T^* (v_k) & \text{since \(\T^*\) is linear} \\
        & = a_1 \conjugatet{\lambda_1} v_1 + ... + a_k \conjugatet{\lambda_k} v_k
\end{align*}
which is still a linear combination of \(v_1, ..., v_k\) hence is in \(\W\).
So \(\T^*\) is \(\W\)-invariant.
\end{proof}

\begin{exercise} \label{exercise 6.4.9}
Let \(\T\) be a \emph{normal} operator on a \emph{finite}-dimensional inner product space \(\V\).
Prove that \(\NULLT = \NULL(\T^*)\) and \(\RANGET = \RANGE(\T^*)\).
\emph{Hint}: Use \THM{6.15} and \EXEC{6.3.12}.
\end{exercise}

\begin{note}
所以一個\ operator 如果\ normal,則它跟它的\ adjoint 的零空間還有值域都會一樣。
\end{note}

\begin{proof}
We have
\begin{align*}
    x \in \NULLT & \iff \T(x) = \OV \iff \norm{\T(x)} = 0 & \text{by \THM{6.2}(b)} \\
        & \iff \norm{\T^*(x)} = 0 & \text{since \(\T\) is normal and by \THM{6.15}(a)} \\
        & \iff \T^*(x) = \OV \iff x \in \NULL(\T^*) & \text{by \THM{6.2}(b)}
\end{align*}

Now by \EXEC{6.3.12}(b), (since \(\V\) is finite-dimensional) we have \(\RANGE(\T^*) = \NULLT^{\perp}\) for any linear operator \(\T\).
In particular, we have \(\RANGE(\RED{(\T^*)}^*) = \NULL(\RED{\T^*})^{\perp}\). \MAROON{(1)}

For the second equation \(\RANGET = \RANGE(\T^*)\), we have
\begin{align*}
    \RANGET & = \RANGE(\RED{(\T^*)}^*) & \text{by \THM{6.11}(d)} \\
        & = \NULL(\RED{\T^*})^{\perp} & \text{by \MAROON{(1)}} \\
        & = \NULL(\BLUE{\T})^{\perp} & \text{by the first equation that we have shown} \\
        & = \RANGE(\BLUE{\T}^*) & \text{by \EXEC{6.3.12}(c) again, where the operator is \(\BLUE{\T}\)}
\end{align*}
\end{proof}

\begin{exercise} \label{exercise 6.4.10}
Let \(\T\) be a \emph{self-adjoint} operator on a finite-dimensional inner product space \(\V\).
Prove that for all \(x \in \V\)
\[
    \norm{\T(x) \pm \iu x}^2 = \norm{\T(x)}^2 + \norm{x}^2.
\]
Deduce that \(\T - \iu \ITRAN{}\) is invertible and that \emph{the adjoint of} \((\T - \iu \ITRAN{})^{-1}\) is \((\T + \iu \ITRAN{})^{-1}\).
\end{exercise}

\begin{proof}
We have
\begin{align*}
    \norm{\T(x) + \iu x}^2
        & = \LG \T(x) + \iu x, \T(x) + \iu x \RG & \text{by \DEF{6.3}} \\
        & = \LG \T(x), \T(x) \RG
          + \LG \T(x), \iu x \RG
          + \LG \iu x, \T(x) \RG
          + \LG \iu x, \iu x \RG \\
          & \quad \quad \quad \text{(by \THM{6.1}, \DEF{6.1}(a))} \\
        & = \norm{\T(x)}^2
          + \LG \T(x), \iu x \RG
          + \LG \iu x, \T(x) \RG
          + \norm{\iu x}^2 & \text{by \DEF{6.3}} \\
        & = \norm{\T(x)}^2 - \iu \LG \T(x), x \RG + \iu \LG x, \T(x) \RG + \norm{\iu x}^2 \\
          & \quad \quad \quad \text{(by \THM{6.1}, \DEF{6.1}(b))} \\
        & = \norm{\T(x)}^2 - \iu \LG x, \RED{\T^*}(x) \RG + \iu \LG x, \T(x) \RG + \norm{\iu x}^2 & \text{by \THM{6.9}} \\
        & = \norm{\T(x)}^2 - \iu \LG x, \RED{\T}(x) \RG + \iu \LG x, \T(x) \RG + \norm{\iu x}^2 & \text{since \(\T\) is self-adjoint} \\
        & = \norm{\T(x)}^2 + \norm{\iu x}^2 & \text{of course} \\
        & = \norm{\T(x)}^2 + \abs{\iu}^2\norm{x}^2 & \text{by \THM{6.2}(a)} \\
        & = \norm{\T(x)}^2 + \norm{x}^2 & \text{of course}
\end{align*}
Similarly,
\begin{align*}
    \norm{\T(x) - \iu x}^2
        & = \LG \T(x) - \iu x, \T(x) - \iu x \RG & \text{by \DEF{6.3}} \\
        & = \LG \T(x), \T(x) \RG
          + \LG \T(x), -\iu x \RG
          + \LG -\iu x, \T(x) \RG
          + \LG -\iu x, -\iu x \RG \\
          & \quad \quad \quad \text{(by \THM{6.1}, \DEF{6.1}(a))} \\
        & = \norm{\T(x)}^2 + \iu \LG \T(x), x \RG - \iu \LG x, \T(x) \RG + \norm{\iu x}^2 \\
          & \quad \quad \quad \text{(by \DEF{6.3}, \THM{6.1}, \DEF{6.1}(b))} \\
        & = \norm{\T(x)}^2 + \iu \LG x, \RED{\T^*}(x) \RG - \iu \LG x, \T(x) \RG + \norm{\iu x}^2 & \text{by \THM{6.9}} \\
        & = \norm{\T(x)}^2 + \iu \LG x, \RED{\T}(x) \RG - \iu \LG x, \T(x) \RG + \norm{\iu x}^2 & \text{since \(\T\) is self-adjoint} \\
        & = \norm{\T(x)}^2 + \norm{\iu x}^2 & \text{of course} \\
        & = \norm{\T(x)}^2 + \abs{\iu}^2\norm{x}^2 & \text{by \THM{6.2}(a)} \\
        & = \norm{\T(x)}^2 + \norm{x}^2 & \text{of course}
\end{align*}

In particular, \(\T - \iu \ITRAN{}\) is one-to-one, since:
\begin{align*}
         & (\T - \iu \ITRAN{}) (x) = \OV \\
    \iff & \norm{(\T - \iu \ITRAN{})(x)} = \norm{\T(x) - \iu x} = 0 & \text{by \THM{6.2}(b)} \\
    \iff & \norm{\T(x) - \iu x}^2 = 0 & \text{of course} \\
    \iff & \norm{\T(x)}^2 + \norm{x}^2 = 0 & \text{by what we have shown} \\
    \iff & \norm{\T(x)} = \norm{x} = 0 & \text{of course (by field algebra)} \\
    \iff & \T(x) = x = \OV & \text{by \THM{6.2}(b) again}
\end{align*}
Hence \((\T - \iu \ITRAN{}) (x) = \OV \iff x = \OV\), so \(\T - \iu \ITRAN{}\) is one-to-one, and  (by \THM{2.5}) invertible, so \((\T - \iu \ITRAN{})^{-1}\) exists, hence \(\left[ (\T - \iu \ITRAN{})^{-1} \right]^*\) exists.

Finally, we need to show that \((\T + \iu \ITRAN{})^{-1}\) also exists and \((\T + \iu \ITRAN{})^{-1} = \left[ (\T - \iu \ITRAN{})^{-1} \right]^*\).
And it suffices to show that \(\left[ (\T - \iu \ITRAN{})^{-1} \right]^* (\T + \iu \ITRAN{}) = (\T + \iu \ITRAN{}) \left[ (\T - \iu \ITRAN{})^{-1} \right]^* = \ITRAN{}\).

We use the ``inner product trick'' again: for any \(x, y \in \V\),
\begin{align*}
    & \quad \LG x, \left[ (\T - \iu \ITRAN{})^{-1} \right]^* (\T + \iu \ITRAN{}) (y) \RG \\
    & = \LG \left[ (\T - \iu \ITRAN{})^{-1} \right](x), (\T + \iu \ITRAN{}) (y) \RG & \text{by \THM{6.9}} \\
    & = \LG \left[ (\T - \iu \ITRAN{})^{-1} \right](x), (\RED{\T^*} + \iu \ITRAN{}) (y) \RG & \text{since \(\T\) is self-adjoint} \\
    & = \LG \left[ (\T - \iu \ITRAN{})^{-1} \right](x), (\T - \iu \ITRAN{})^* (y) \RG & \text{of course by \THM{6.11}} \\
    & = \LG (\T - \iu \ITRAN{}) \left[ (\T - \iu \ITRAN{})^{-1} \right](x), y \RG & \text{by \THM{6.9} again} \\
    & = \LG x, y \RG = \LG x, \ITRAN{}(y) \RG. & \text{of course}
\end{align*}
hence \(\left[ (\T - \iu \ITRAN{})^{-1} \right]^* (\T + \iu \ITRAN{}) (y) = \ITRAN{}(y)\) for all \(y \in \V\), so \(\left[ (\T - \iu \ITRAN{})^{-1} \right]^* (\T + \iu \ITRAN{}) = \ITRAN{}\).
The same process can be used to show that \((\T + \iu \ITRAN{}) \left[ (\T - \iu \ITRAN{})^{-1} \right]^* = \ITRAN{}\).
Hence we have \((\T + \iu \ITRAN{})^{-1} = \left[ (\T - \iu \ITRAN{})^{-1} \right]^*\).
\end{proof}
\begin{note}
\EXEC{6.4.10} 不知道有什麼用意? 給定一個\ self-adjoint \(\T\),\(\T \pm \iu \ITRAN{}\) 有什麼特別的用途嗎?
\end{note}

\begin{exercise} \label{exercise 6.4.11}
Assume that \(\T\) is a linear operator on a \emph{complex} (\emph{not necessarily finite} dimensional) inner product space \(\V\) with an adjoint \(\T^*\).
Prove the following results.
\begin{enumerate}
\item If \(\T\) is self-adjoint, then \(\LG \T(x), x \RG\) is \textbf{real} for all \(x \in \V\).
\item If \(\T\) satisfies \(\LG \T(x), x \RG = 0\) for all \(x \in \V\), then \(\T = \TZERO\).
\emph{Hint}: Replace \(x\) by \(x + y\) and then by \(x + \iu y\), and expand the resulting inner products.
\item If \(\LG \T(x), x \RG\) is real for all \(x \in \V\), then \(\T\) is self-adjoint.
\end{enumerate}
\end{exercise}

\begin{note}
Part(a) and part(c) form a if and only if statement.
And it seems that part(b) is used in \SEC{6.5} (by my professor).
\TODOREF{(which part?)}
\end{note}

\begin{proof} \ 

\begin{enumerate}
\item For all \(x \in \V\), we have
\begin{align*}
    \LG \T(x), x \RG & = \LG x, \T^*(x) \RG & \text{by \THM{6.9}} \\
        & = \LG x, \T(x) \RG & \text{since \(\T\) is self-adjoint} \\
        & = \conjugatet{\LG \T(x), x \RG} & \text{by \DEF{6.1}(c)}
\end{align*}
which by \THM{d.2}(e) implies \(\LG \T(x), x \RG\) is a real number.

\item In particular, for \(x, y \in \V\), \(x + y \in \V\) and \(x + \iu y \in \V\), so we have \(\LG \T(x + y), x + y \RG = 0\) and \(\LG \T(x + \iu y), x + \iu y \RG = 0\).
And that implies
\begin{align*}
    0 & = \LG \T(x + y), x + y \RG \\
      & = \LG \T(x) + \T(y), x + y \RG & \text{since \(\T\) is linear} \\
      & = \LG \T(x), x \RG + \LG \T(x), y \RG + \LG \T(y), x \RG + \LG \T(y), y \RG & \text{just expand} \\
      & = 0 + \LG \T(x), y \RG + \LG \T(y), x \RG + 0 & \text{since \(x \in \V\) and \(y \in \V\)} \\
    \implies & \LG \T(x), y \RG = -\LG \T(y), x \RG.
\end{align*}
and
\begin{align*}
    0 & = \LG \T(x + \iu y), x + \iu y \RG \\
      & = \LG \T(x) + \iu \T(y), x + \iu y \RG & \text{since \(\T\) is linear} \\
      & = \LG \T(x), x \RG + \LG \T(x), \iu y \RG + \LG \iu \T(y), x \RG + \LG \iu \T(y), \iu y \RG & \text{just expand} \\
      & = \LG \T(x), x \RG - \iu \LG \T(x), y \RG + \iu \LG \T(y), x \RG + \abs{\iu}^2 \LG \T(y), y \RG \\
      & \quad \quad \quad \text{(by \THM{6.1},\DEF{6.1}(b))} \\
      & = 0 - \iu \LG \T(x), y \RG + \iu \LG \T(y), x \RG + 1 \cdot 0 & \text{since \(x \in \V\) and \(y \in \V\)} \\
      & = -\iu \LG \T(x), y \RG + \iu \LG \T(y), x \RG \\
    \implies & \iu \LG \T(x), y \RG = \iu \LG \T(y), x \RG \\
    \implies & \LG \T(x), y \RG = \LG \T(y), x \RG
\end{align*}
Hence we have \(\LG \T(x), y \RG = -\LG \T(y), x \RG\) and \(\LG \T(x), y \RG = \LG \T(y), x \RG\), which implies \(-\LG \T(y), x \RG = \LG \T(y), x \RG\) and implies \(\LG \T(y), x \RG = 0\). 
In particular, \(\LG \T(x), y \RG = \LG \T(y), x \RG = 0\), and \(\LG y, \T(x) \RG = \conjugatet{\LG \T(x), y \RG} = \conjugatet{0} = 0\).

So we have for all \(x, y \in \V\), \(\LG y, \T(x) \RG = 0\), which (by \THM{6.1}(e)) implies \(\T(x) = \OV\) for all \(x\), hence \(\T = \TZERO\).

\item We have, for all \(x \in \V\),
\begin{align*}
    \LG \T(x), x \RG & = \conjugatet{\LG \T(x), x \RG} & \text{by \THM{d.2}(e)} \\
        & = \conjugatet{\LG x, \T^*(x) \RG} & \text{by \THM{6.9}} \\
        & = \LG \T^*(x), x \RG & \text{by \DEF{6.1}(c)} \\
    \implies & \LG \T(x) - \T^*(x), x \RG = 0 & \text{by \DEF{6.1}} \\
    \implies & \LG (\T - \T^*)(x), x \RG = 0 & \text{of course}
\end{align*}
By part(b), we have \(\T - \T^* = \TZERO\), hence \(\T = \T^*\).
\end{enumerate}
\end{proof}

\begin{exercise} \label{exercise 6.4.12}
Let \(\T\) be a \emph{normal} operator on a \emph{finite}-dimensional \textbf{real} inner product space \(\V\) whose \CPOLY{} \emph{splits}.
Prove that \(\V\) has an orthonormal basis of eigenvectors of \(\T\).
Hence (since \(\V\) is over \(\SET{R}\) and consists of orthonormal basis of eigenvectors of \(\T\), by \THM{6.17}, with the statement from right to left,) \(\T\) is self-adjoint.
\end{exercise}

\begin{note}
So on \emph{real} inner product space, normal and splitted-\CPOLY{} imply self-adjoint.
\end{note}

\begin{proof}
(The proof is similar to \THM{6.16}.)
Suppose \(\T\) is a normal operator on real inner product space \(\V\) such that its \CPOLY{} splits.
In particular, all eigenvalues of \(\T\) are real. \MAROON{(1)}
And by Schur's Theorem (\THM{6.14}), there exists an ordered orthonormal basis \(\beta = \{ v_1, v_2, ..., v_n \}\) such that \([\T]_{\beta}\) is upper triangular.
In particular, (sine \([\T]_{\beta}\) is upper triangular,) \(v_1\) is an eigenvector of \(\T\).
(Similar to the proof of \THM{6.16},) Assume that \(v_1, v_2, ..., v_{k - 1}\) are eigenvectors of \(\T\) for some \(k > 1\).
We claim that \(v_k\) is also an eigenvector of \(\T\).
It then follows by mathematical induction on \(k\) that all of the \(v_i\)'s are eigenvectors of \(\T\).

Consider any \(j < k\), and let \(\lambda_j\) denote the eigenvalue of \(\T\) corresponding to \(v_j\).
Since \(\T\) is normal, by \THM{6.15}(c), \(\T^*(v_j) = \conjugatet{\lambda_j} v_j\).
But (since the inner product space is over \(\SET{R}\)), by \MAROON{(1)} we have \(\conjugatet{\lambda_j} = \lambda_j\), that is, \(\T^*(v_j) = \lambda_j v_j\). \MAROON{(2)}

And since \(A\) is upper triangular,
\begin{align*}
    \T(v_k) & = A_{1k} v_1 + A_{2k} v_2 + ... + A_{jk} v_j + ... + A_{kk}v_k + 0 \cdot v_{k + 1} + ... + 0 \cdot v_{n} \\
    & = A_{1k} v_1 + A_{2k} v_2 + ... + A_{jk} v_j + ... + A_{kk}v_k. \quad \quad \MAROON{(3)}
\end{align*}
Furthermore, by the \CORO{6.5.1}, since \(\beta\) is orthonormal,
\begin{align*}
    A_{jk} & = \LG \T(v_k), v_j \RG & \text{by \CORO{6.5.1}} \\
        & = \LG v_k, \T^*(v_j) \RG & \text{by \THM{6.9}} \\
        & = \LG v_k, \lambda_j v_j \RG & \text{by \MAROON{(2)}} \\
        & = \conjugatet{\lambda_j} \LG v_j, v_k \RG = \lambda \LG v_j, v_k \RG & \text{by \THM{6.1}(b) and \MAROON{(1)}} \\
        & = \lambda_j \cdot 0 = 0. & \text{since \(v_j\)'s are orthonormal}
\end{align*}
So \(A_{jk} = 0\) for all \(j < k\), hence it follows from \MAROON{(3)} that \(\T(v_k) = A_{kk} v_k\), and hence \(v_k\) is an eigenvector of \(\T\).
So by induction, all the vectors in \(\beta\) are eigenvectors of \(\T\);
so \(\V\) has an orthonormal basis, \(\beta\), of eigenvectors of \(\T\).
\end{proof}

\begin{exercise} \label{exercise 6.4.13}
An \(n \X n\) \emph{real} matrix \(A\) is said to be a \textbf{Gramian} matrix if there exists a \emph{real} (square) matrix \(B\) such that \(A = B^\top B\).
Prove that \(A\) is a Gramian matrix if and only if \(A\) is symmetric and all of its eigenvalues are \emph{nonnegative}.
\emph{Hint}: Apply \THM{6.17} to \(\T = \LMTRAN_A\) to obtain an orthonormal basis \(\{ v_1, v_2, ..., v_n \}\)
of eigenvectors with the associated eigenvalues \(\lambda_1, \lambda_2, ..., \lambda_n\).
Define the linear operator \(\U\) by \(\U(v_i) = \sqrt{\lambda_i} v_i\).
\end{exercise}

\begin{proof}
Note that since the matrix is real, \(A^*\) is reduced to \(A^\top\).

\(\Longrightarrow\): Suppose \(A\) is Gramian such that \(A = B^\top B\) for some real square matrix \(B\).
Then we have (by the properties of matrix transpose)
\[
    A^\top = (B^\top B)^\top = B^\top (B^\top)^\top = B^\top B = A
\]
Hence \(A\) is symmetric.
Now suppose \(\lambda\) is an eigenvalue of \(A\), hence we have \(Ax = \lambda x\) for some \textbf{unit vector} \(x\).
In particular, (if we view \(Ax, Bx\) as \(\LMTRAN_A(x), \LMTRAN_B(x)\), with \THM{6.9},)
\begin{align*}
    0 & \le \LG Bx, Bx \RG & \text{of course} \\
      & = \LG x, B^* B x \RG & \text{by \THM{6.9}} \\
      & = \LG x, B^\top B x \RG & \text{since \(B\) is real matrix} \\
      & = \LG x, Ax \RG & \text{by def of \(A\)} \\
      & = \LG x, A^\top x \RG = \LG x, A^* \top \RG & \text{since \(A\) is symmetric and real} \\
      & = \LG Ax, x \RG & \text{by \THM{6.9} again} \\
      & = \LG \lambda x, x \RG & \text{by supposition} \\
      & = \lambda \LG x, x \RG & \text{by \DEF{6.1}(b)} \\
      & = \lambda \cdot 1 = \lambda & \text{since \(x\) is unit vector}.
\end{align*}
So all eigenvalues of \(A\) are nonnegative.

\(\Longleftarrow\): Now suppose \(A\) is symmetric and all eigenvalues of \(A\) are nonnegative.
In particular, since \(A\) is real and symmetric, by definition it's self-adjoint, and \(A = [\LMTRAN_A]_{\alpha}\) where \(\alpha\) is the standard ordered (orthonormal) basis, hence (by \RMK{6.4.6}) \(\LMTRAN_A\) is also self-adjoint.

So by \THM{6.17}, there exists an orthonormal basis \(\beta = \{ v_1, v_2, ..., v_n \}\) consisting eigenvectors of \(\LMTRAN_A\), in particular, \([\LMTRAN_A]_{\beta}\) is diagonal.

Now the tricky part: define \(D\) be the diagonal matrix such that its \(ii\)-entry is \(\sqrt{\lambda_i}\) where \(\lambda_i\) is the \(ii\)-entry of \([\LMTRAN_A]_{\beta}\).
Then since each \(\lambda_i\) is nonnegative, \(D\) is also a \textbf{real} matrix.
And of course \(D\) is symmetric \MAROON{(1)} and \(D^2 = [\LMTRAN_A]_{\beta}\). \MAROON{(2)}

Note that we have let \(\alpha\) be the standard ordered basis, and
\begin{align*}
    A & = [\LMTRAN_A]_{\alpha} & \text{by \THM{2.15}(a)} \\
      & = [\ITRAN{}]_{\beta}^{\alpha} [\LMTRAN_A]_{\beta} [\ITRAN{}]_{\alpha}^{\beta} & \text{by \THM{2.23}} \\
      & = [\ITRAN{}]_{\beta}^{\alpha} D^2 [\ITRAN{}]_{\alpha}^{\beta} & \text{by \MAROON{(2)}} \\
      & = ([\ITRAN{}]_{\beta}^{\alpha} D) (D [\ITRAN{}]_{\alpha}^{\beta}) & \text{of course} \\
      & = ([\ITRAN{}]_{\beta}^{\alpha} D) (D^\top) [\ITRAN{}]_{\alpha}^{\beta}) & \text{by \MAROON{(1)}}
\end{align*}
Now from \THM{2.23}, let \(Q^{-1} = [\ITRAN{}]_{\beta}^{\alpha}\), then \(Q = [\ITRAN{}]_{\alpha}^{\beta}\), so from the last equation we have \(A = (Q^{-1} D) (D^\top Q)\). \MAROON{(3)}.

Since \(Q^{-1} = [\ITRAN{}]_{\beta}^{\alpha}\), the \(j\)th column of \(Q^{-1}\) is \(v_j\), that is, the columns of \(Q^{-1}\) are the vectors of the orthonormal basis \(\beta\).
So by \EXEC{6.1.23}(c), we have \((Q^{-1})^* = (Q^{-1})^{-1} = Q\).
But since \((Q^{-1})^*\) is real matrix, the equation reduces to \((Q^{-1})^\top = Q\). \MAROON{(4)}

Hence
\begin{align*}
    A & = (Q^{-1} D) (D^\top \RED{Q}) & \text{by \MAROON{(3)}} \\
      & = (Q^{-1} D) (D^\top \RED{(Q^{-1})^\top}) & \text{by \MAROON{(4)}} \\
      & = (Q^{-1} D) (Q^{-1} D)^\top & \text{of course by the property of transpose} \\
      & = ((Q^{-1} D)^\top)^\top (Q^{-1} D)^\top. & \text{of course by the property of transpose}
\end{align*}
Hence we have found a matrix \(B = (Q^{-1} D)^\top\) such that \(A = B^\top B\);
in particular, since \(Q^{-1}\) and \(D\) are \textbf{real} matrices, \(B\) is also real matrix, hence \(A\) is a Gramian matrix, as desired.
\end{proof}

\begin{exercise} \label{exercise 6.4.14}
\emph{Simultaneous Diagonalization}.
Let \(\V\) be a finite-dimensional \textbf{real} inner product space, and let \(\U\) and \(\T\) be \textbf{self-adjoint} linear operators on \(\V\) such that \(\U\T = \T\U\).
Prove that there exists an \textbf{orthonormal} basis for \(\V\) consisting of vectors that are eigenvectors of both \(\U\) and \(\T\).
(The complex version of this result appears as \EXEC{6.6.10}.)
\emph{Hint}: For any eigenspace \(\W = E_{\lambda}\) of \(\T\), we have that \(\W\) is both \(\T\)- and \(\U\)-invariant.
By \EXEC{6.4.7}, we have that \(\W^{\perp}\) is both \(\T\)- and \(\U\)-invariant.
Apply \THM{6.17} and \THM{6.6}.
\end{exercise}

\begin{proof}
Let \(\V\) be an arbitrary \(n\)-dimensional \textbf{real} inner product space, and let \(\U\) and \(\T\) be \textbf{self-adjoint} linear operators on \(\V\) such that \(\U\T = \T\U\).

We use induction on the dimension of the underlying (real) inner product space.
So for \(n = 1\), by \EXEC{5.4.25} \(\T\) and \(\U\) can already be diagonalized simultaneously by an ordered basis.
But since the basis is a singleton set, by normalizing the vector in it, we get an orthonormal basis, hence \(\T\) and \(\U\) can be diagonalized simultaneously by an orthonormal basis.

Now suppose the statement is true for \(n \le k - 1\), where \(k > 1\);
that is, for arbitrary finite-dimensional real inner product space \(\V\) such that \(\dim(\V) \le k - 1\),
given \(\U\) and \(\T\) be \textbf{self-adjoint} linear operators on \(\V\) such that \(\U\T = \T\U\), there exists an orthonormal basis for \(\V\) consisting of vectors that are eigenvectors of both \(\U\) and \(\T\).

We have to show the statement is true for \(n = k\); that is, for arbitrary \(k\)-dimensional real inner product space \(\V\),
given \(\U\) and \(\T\) be \textbf{self-adjoint} linear operators on \(\V\) such that \(\U\T = \T\U\), there exists an orthonormal basis for \(\V\) consisting of vectors that are eigenvectors of both \(\U\) and \(\T\).

So let \(\V\) be a \(k\)-dimensional real inner product space, and let \(\U\) and \(\T\) be \textbf{self-adjoint} linear operators on \(\V\) such that \(\U\T = \T\U\).
Let assume \(\W = E_{\lambda}\) is an eigenspace corresponding to \(\lambda\) of \(\T\).
Then (by \EXAMPLE{5.4.1}) \(\W\) is \(\T\)-invariant.
Now we show that \(\W\) is also \(\U\)-invariant:
Suppose \(x \in \W\); in particular, \(\T(x) = \lambda x\). \MAROON{(1)}
Then
\begin{align*}
    \U\T = \T\U \implies & \U\T(x) = \T\U(x) \\
        \implies & \U(\T(x)) = \T(\U(x)) & \text{of course} \\
        \implies & \U(\lambda x) = \T(\U(x)) & \text{by \MAROON{(1)}} \\
        \implies & \lambda \U(x) = \T(\U(x)) & \text{since \(\U\) is linear} \\
        \implies & \U(x) \text{ is an eigenvector of \(\T\) corresponding to } \lambda
\end{align*}
So by definition, \(\U(x) \in E_{\lambda}\); that is, \(\U(x) \in \W\), hence \(\W\) is \(\U\)-invariant. \MAROON{(2)}
(BTW the same strategy was used in the more general exercise \EXEC{5.4.25}(a).)

Now by \EXEC{6.4.7}(b), since \(\W\) is \(\T\)- and \(\U\)-invariant, \(\W^{\perp}\) is \(\T^*\)- and \(\U^*\)-invariant.
But since \(\T\) and \(\U\) are self-adjoint, that implies \(\W^{\perp}\) is also \(\T\)- and \(\U\)-invariant.
And by \EXEC{6.4.7}(a), \(\T_\W\), \(\T_{\W^{\perp}}\), \(\U_\W\), and \(\U_{\W^{\perp}}\) are all self-adjoint. \MAROON{(3)}

Since both \(\W\) and \(\W^{\perp}\) are \emph{proper subsets} of \(\V\), that is, \(\dim(\W) < \dim(\V)\) and \(\dim(\W^{\perp}) < \dim(\V)\),
with \MAROON{(3)}, we can apply the induction hypothesis on \(\W\) with \(\T_\W, \U_\W\), and on \(\W^{\perp}\) with \(\T_{\W^{\perp}}, \U_{\W^{\perp}}\), respectively.

That is, there exist orthonormal bases \(\beta_1\) for \(\W\) consisting of eigenvectors of both \(\T_\W\) and \(\U_\W\),
and there exist orthonormal bases \(\beta_2\) for \(\W^{\perp}\) consisting of eigenvectors of both \(\T_{\W^{\perp}}\) and \(\U_{\W^{\perp}}\).
In particular, \(\beta_1\) consists of eigenvectors of both \(\T\) and \(\U\), and \(\beta_2\) consists of eigenvectors of both \(\T\) and \(\U\).

Finally, (by \EXEC{6.2.10}) since \(\V = \W \oplus \W^{\perp}\), (by the equivalent (e) of \THM{5.9},) \(\beta = \beta_1 \cup \beta_2\) is an ordered basis for \(\V\), and consists of eigenvectors of both \(\T\) and \(\U\).
So the statement is true for \(k\).
This closes the induction.
\end{proof}

\begin{exercise} \label{exercise 6.4.15}
Let \(A\) and \(B\) be symmetric \(n \X n\) matrices such that \(AB = BA\).
Use \EXEC{6.4.14} to prove that there exists an \emph{orthogonal} matrix \(P\) such that \(P^\top A P\) and \(P^\top B P\) are both diagonal matrices.
\end{exercise}

\begin{note}
The definition of orthogonal matrix is in \DEF{6.12}, in \SEC{6.5}.
\end{note}

\begin{proof}
\end{proof}

\begin{exercise} \label{exercise 6.4.16}
Prove the \emph{Cayley-Hamilton} theorem for a \textbf{complex} \(n \X n\) matrix \(A\).
That is, if \(f(t)\) is the \CPOLY{} of \(A\), prove that \(f(A) = O\).
Hint: Use Schur's theorem (\THM{6.14}) to show that \(A\) may be assumed to be upper triangular, in which case
\[
    f(t) = \sum_{i = 1}^n (A_{ii} - t).
\]
Now if \(\T = \LMTRAN_A\), we have \((A_{jj} \ITRAN{} - \T)(e_j) \in \spann(\{ e_1, e_2, ..., e_{j - 1} \})\) for
\(j \ge 2\), where \(\{ e_1, e_2, ..., e_n \}\) is the standard ordered basis for \(\SET{C}^n\).
(The general case is proved in \THM{5.21}.)
\end{exercise}

\begin{proof}
\end{proof}

The following definitions are used in Exercises 17 through 23.
\begin{additional definition} \label{adef 6.6}
A linear operator \(\T\) on a \emph{finite}-dimensional inner product space is called \textbf{positive definite} [\textbf{positive semidefinite}] if \(\T\) is self-adjoint and \(\LG \T(x), x \RG > 0\) [\(\LG \T(x), x \RG \ge 0\)] for all \(x \ne \OV\).
An \(n \X n\) matrix \(A\) with entries from \(\SET{R}\) or \(\SET{C}\) is called \textbf{positive definite}
[\textbf{positive semidefinite}] if \(\LMTRAN_A\) is positive definite [positive semidefinite].
\end{additional definition}

\begin{exercise} \label{exercise 6.4.17}
Let \(\T\) and \(\U\) be self-adjoint linear operators on an \(n\)-dimensional inner product space \(\V\), and let \(A = [\T]_{\beta}\), where \(\beta\) is an \emph{orthonormal} basis for \(\V\).
Prove the following results.
\begin{enumerate}
\item \(\T\) is positive definite [semidefinite] if and only if all of its eigenvalues are positive [nonnegative].
\item \(\T\) is positive definite if and only if
\[
    \sum_{i, j} A_{ij} a_j \conjugatet{a_i} > 0 \text{ for all nonzero \(n\)-tuples } (a_1, a_2, ..., a_n).
\]
\item \(\T\) is positive semidefinite if and only if \(A = B^* B\) for some square matrix \(B\).
\item If \(\T\) and \(\U\) are positive semidefinite operators such that \(\T^2 = \U^2\), then \(\T = \U\).
\item If \(\T\) and \(\U\) are positive definite operators such that \(\T\U = \U\T\), then \(\T\U\) is positive definite.
\item \(\T\) is positive definite [semidefinite] if and only if \(A\) is positive definite [semidefinite].
\end{enumerate}
Because of (f), results analogous to items (a) through (d) hold for matrices as well as operators.
\end{exercise}

\begin{proof}
\end{proof}

\begin{exercise} \label{exercise 6.4.18}
Let \(\T: \V \to \W\) be a \LTRAN{}, where \(\V\) and \(\W\) are finite-dimensional inner product spaces.
Prove the following results.
\begin{enumerate}
\item \(\T^*\T\) and \(\T\T^*\) are positive semidefinite. (See \EXEC{6.3.15}.)
\item \(\rank(\T^*\T) = \rank(\T\T^*) = \rankT\).
\end{enumerate}
\end{exercise}

\begin{proof}
\end{proof}

\begin{exercise} \label{exercise 6.4.19}
Let \(\T\) and \(\U\) be positive definite operators on an inner product space \(\V\).
Prove the following results.
\begin{enumerate}
\item \(\T + \U\) is positive definite.
\item If \(c > 0\), then \(c \T\) is positive definite.
\item \(\T^{-1}\) is positive definite.
\end{enumerate}
\end{exercise}

\begin{proof}
Visit goo. gl/ cQch7i for a solution.
\end{proof}

\begin{exercise} \label{exercise 6.4.20}
Let \(\V\) be an inner product space with inner product \(\InnerOp\), and let \(\T\) be a positive definite linear operator on \(\V\).
Prove that \(\LG x, y \RG' = \LG \T(x), y \RG\) defines another inner product on \(\V\).
\end{exercise}

\begin{proof}
\end{proof}

\begin{exercise} \label{exercise 6.4.21}
Let \(\V\) be a finite-dimensional inner product space, and let \(\T\) and \(\U\) be self-adjoint operators on \(\V\) such that \(\T\) is positive definite.
Prove that both \(\T\U\) and \(\U\T\) are diagonalizable linear operators \emph{that have only real eigenvalues}.
\emph{Hint}: Show that \(\U\T\) is self-adjoint \emph{with respect to} the inner product \(\LG x, y \RG' = \LG \T(x), y \RG\).
To show that \(\T\U\) is self-adjoint, repeat the argument with \(\T^{-1}\) in place of \(\T\).
\end{exercise}

\begin{proof}
\end{proof}

\begin{exercise} \label{exercise 6.4.22}
This exercise provides a converse to \EXEC{6.4.20}.
Let \(\V\) be a finite-dimensional inner product space with inner product \(\InnerOp\), and let \(\InnerOp'\) be any other inner product on \(\V\).
\begin{enumerate}
\item Prove that there exists a \textbf{unique} linear operator \(\T\) on \(\V\) such that \(\LG x, y \RG' = \LG \T(x), y \RG\) for all \(x\) and \(y\) in \(\V\).
\emph{Hint}: Let \(\beta = \{ v_1, v_2, ..., v_n \}\) be an \emph{orthonormal} basis for \(\V\) \emph{with respect to} \(\InnerOp\), and define a matrix \(A\) by \(A_{ij} = \LG v_j, v_i \RG'\) for all \(i\) and \(j\).
Let \(\T\) be the unique linear operator on \(\V\) such that \([\T]_{\beta} = A\).
\item Prove that the operator \(\T\) of (a) is positive definite with respect to \textbf{both} inner products.
\end{enumerate}
\end{exercise}

\begin{proof}
\end{proof}

\begin{exercise} \label{exercise 6.4.23}
Let \(\U\) be a diagonalizable linear operator on a finite-dimensional inner product space \(\V\) such that all of the eigenvalues of \(\U\) are \emph{real}.
Prove that there exist positive definite linear operators \(\T_1\) and \(\T_1'\) and self-adjoint linear operators \(\T_2\) and \(\T_2'\) such that \(\U = \T_2 \T_1 = \T_1' \T_2'\).
\emph{Hint}: Let \(\InnerOp\) be the inner product associated with \(\V\), \(\beta\) a basis of eigenvectors for \(\U\), \(\InnerOp'\) the inner product on \(\V\) with respect to which \(\beta\) is orthonormal (see \EXEC{6.1.22}(a)), and \(\T_1\) the positive definite operator according to \EXEC{6.4.22}.
Show that \(\U\) is self-adjoint with respect to \(\InnerOp'\) and \(\U = \T_1^{-1} \U^* \T_1\) (the adjoint is \emph{with respect to} \(\InnerOp\)).
Let \(\T_2 = \T_1^{-1} \U^*\).
\end{exercise}

\begin{proof}
\end{proof}
