\exercisesection

\begin{exercise} \label{exercise 3.2.1}
Label the following statements as true or false.
\begin{enumerate}
\item The rank of a matrix is equal to the number of its nonzero columns.
\item The product of two matrices always has rank equal to the lesser of the ranks of the two matrices.
\item The \(m \X n\) zero matrix is the only \(m \X n\) matrix having rank \(0\).
\item e.r.o.s preserve rank.
\item e.c.o.s do not necessarily preserve rank.
\item The rank of a matrix is equal to the maximum number of linearly independent rows in the matrix.
\item The inverse of a matrix can be computed exclusively by means of elementary row operations.
\item The rank of an \(n \X n\) matrix is at most \(n\).
\item An \(n \X n\) matrix having rank \(n\) is invertible.
\end{enumerate}
\end{exercise}

\begin{proof} \ 

\begin{enumerate}
\item False. Counterexample: \(\begin{pmatrix} 1 & 1 \\ 1 & 1 \end{pmatrix}\) has rank \(1\) but has two nonzero columns.

\item False. Counterexample: \(\begin{pmatrix} 1 & 0 \\ 0 & 0 \end{pmatrix} \begin{pmatrix} 0 & 0 \\ 1 & 0 \end{pmatrix} = O_{2 \X 2}\), the product has rank \(0\) but both of the factors have rank \(1\).

\item True. Any nonzero matrix has rank \(\ge 1\).
\item True by \CORO{3.4.1}.
\item False by \CORO{3.4.1}.
\item True by \CORO{3.6.2}(b).
\item True by the whole description in \RMK{3.2.6}.
\item True by \THM{3.6}.
\item True by \RMK{3.2.1}.
\end{enumerate}
\end{proof}

\begin{exercise} \label{exercise 3.2.2}
Calculation problem, skip.
\end{exercise}


\begin{exercise} \label{exercise 3.2.3}
Prove that for any \(m \X n\) matrix \(A\), \(\rank(A) = 0\) if and only if \(A\) is the zero matrix.
\end{exercise}

\begin{proof}
If \(A = O\) then by \THM{3.5} \(\rank(A) = 0\).
Now if \(\rank(A) = 0\), again by \THM{3.5}, \(\dim(\spann(\{a_1, ..., a_n\})) = 0\) where \(a_i\) is the \(i\)th column of \(A\).
But that implies \(\spann(\{a_1, ..., a_n\}) = \{ \OV \}\), which must implies \(a_1 = a_2 = ... = a_n = \OV\).
Hence \(A = O_{m \X n}\).
\end{proof}

\begin{exercise} \label{exercise 3.2.4}
Calculation problem, skip.
\end{exercise}

\begin{exercise} \label{exercise 3.2.5}
Calculation problem, skip.
\end{exercise}

\begin{exercise} \label{exercise 3.2.6}
For each of the following \LTRAN{}s \(\T\), determine whether \(\T\) is invertible, and compute \(\T^{-1}\) if it exists.
\begin{enumerate}
\item \(\T: \mathcal{P}_{2}(\SET{R}) \to \mathcal{P}_{2}(\SET{R})\) defined by \(\T(f(x)) = f''(x) + 2f'(x) - f(x)\).

\item \(\T:\mathcal{P}_{2}(\SET{R}) \to \mathcal{P}_{2}(\SET{R})\) defined by \(\T(f(x)) = (x + 1) f'(x)\).

\item \(\T: \SET{R}^{3} \to \SET{R}^{3}\) defined by
\[
    \T(a_1, a_2, a_3) = (a_1 + 2 a_2 + a_3, - a_1 + a_2 + 2 a_3, a_1 + a_3)
\]

\item \(\T: \SET{R}^{3} \to \mathcal{P}_{2}(\SET{R})\) defined by
\[
    \T(a_1, a_2, a_3) = (a_1 + a_2 + a_3) + (a_1 - a_2 + a_3)x + a_1 x^2.
\]

\item \(\T: \mathcal{P}_{2}(\SET{R}) \to \SET{R}^{3}\) defined by  \(\T(f(x)) = (f(-1), f(0), f(1))\).

\item \(\T: M_{2 \X 2}(\SET{R}) \to \SET{R}^4\) defined by
\[
    \T(A)= (\TRACE(A), \TRACE(A^\top), \TRACE(EA), \TRACE(AE)),
\]
where
\[
    E = \begin{pmatrix} 0 & 1 \\ 1 & 0 \end{pmatrix}.
\]
\end{enumerate}
\end{exercise}

\begin{proof}
We use \CORO{2.18.1} to test \(\T\) for invertibility and compute the inverse if \(\T\) is invertible.

In each item below, we let \(\alpha\) and \(\beta\) be the standard ordered basis of the domain and codomain of \(\T\), respectively.
(\(\alpha\) may be equal to \(\beta\).)

\begin{enumerate}
\item
We have
\begin{align*}
    \T(1) & = -1 = -1 \cdot 1 + 0 \cdot x + 0 \cdot x^2 \\
    \T(x) & = -2 - x = -2 \cdot 1 + (-1) \cdot x + 0 \cdot x^2 \\
    \T(x^2) & = -2 + 4x - x^2 = -2 \cdot 1 + 4 \cdot x + (-1) \cdot x^2
\end{align*}
Thus
\[
    [\T]_{\alpha}^{\beta} = \begin{pmatrix}
        -1 & -2 & -2 \\ 0 & -1 & 4 \\ 0 & 0 & -1
    \end{pmatrix}.
\]
Using the method of \EXAMPLE{3.2.5} and \EXAMPLE{3.2.6}, we can show that \([\T]_{\alpha}^{\beta}\) is invertible with inverse
\[
    ([\T]_{\alpha}^{\beta})^{-1} = \begin{pmatrix}
        -1 & -2 & -10 \\ 0 & -1 & -4 \\ 0 & 0 & -1
    \end{pmatrix}.
\]
Thus (by \CORO{2.18.1}) \(\T\) is invertible, and \(([\T]_{\alpha}^{\beta})^{-1} = [\T^{-1}]_{\beta}^{\alpha}\).
Hence by \THM{2.14}, we have
\begin{align*}
    [\T^{-1}(a_0 + a_1 x + a_2 x^2)]_{\alpha}
        & = [\T^{-1}]_{\beta}^{\alpha} [(a_0 + a_1 x + a_2 x^2)]_{\beta} & \text{by \THM{2.14}} \\
        & = \left(\begin{array}{rrr}
                -1 & -2 & -10 \\
                0 & -1 & -4 \\
                0 & 0 & -1
            \end{array}\right)
            \left(\begin{array}{l}
                a_0 \\ a_1 \\ a_2
            \end{array}\right) \\
        & = \left(\begin{array}{c}
                \BLUE{-a_0 - 2a_1 - 10a_2} \\ \RED{-a_1 - 4a_2} \\ \GREEN{-a_2}
            \end{array}\right)
\end{align*}
Therefore
\[
    \T^{-1}(a_0 + + a_1 x + a_2 x^2) = \BLUE{(-a_0 - 2a_1 - 10a_2)} + \RED{(-a_1 - 4a_2)} x + \GREEN{-a_2} x^2.
\]

\item We have
\begin{align*}
    \T(1) & = (x + 1) \cdot 0 = 0 \cdot 1 + 0 \cdot x + 0 \cdot x^2 \\
    \T(x) & = (x + 1) \cdot 1 = 1 \cdot 1 + 1 \cdot x + 0 \cdot x^2 \\
    \T(x^2) & = (x + 1) \cdot 2x = 0 \cdot 1 + 2 \cdot x + 2 \cdot x^2
\end{align*}
Thus
\[
    [\T]_{\alpha}^{\beta} = \begin{pmatrix} 0 & 1 & 0 \\ 0 & 1 & 2 \\ 0 & 0 & 2 \end{pmatrix}
\]
Which is not invertible.

\item Skip.

\item We have
\begin{align*}
    \T(1,0,0) & = 1 + x + x^{2} \\
    \T(0,1,0) & = 1 - x \\
    \T(0,0,1) & = 1 + x \\
\end{align*}
So
\[
    [\T]_{\alpha}^{\beta} = \begin{pmatrix}
        1 & 1 & 1 \\ 1 & -1 & 1 \\ 1 & 0 & 0
    \end{pmatrix}.
\]
Using the method of \EXAMPLE{3.2.5} and \EXAMPLE{3.2.6}, we can show that \([\T]_{\alpha}^{\beta}\) is invertible with inverse
\[
    ([\T]_{\alpha}^{\beta})^{-1} = \begin{pmatrix}
        0 & 0 & 1 \\
        \frac1{2} & -\frac1{2} & 0 \\
        \frac1{2} & \frac1{2} & -1
    \end{pmatrix}.
\]
Thus (by \CORO{2.18.1}) \(\T\) is invertible, and \(([\T]_{\alpha}^{\beta})^{-1} = [\T^{-1}]_{\beta}^{\alpha}\).
Hence by \THM{2.14}, we have
\begin{align*}
    [\T^{-1}(a_0 + a_1 x + a_2 x^2)]_{\alpha}
        & = [\T^{-1}]_{\beta}^{\alpha} [(a_0 + a_1 x + a_2 x^2)]_{\beta} & \text{by \THM{2.14}} \\
        & = \left(\begin{array}{rrr}
                0 & 0 & 1 \\
                \frac1{2} & -\frac1{2} & 0 \\
                \frac1{2} & \frac1{2} & -1
            \end{array}\right)
            \left(\begin{array}{l}
                a_0 \\ a_1 \\ a_2
            \end{array}\right) \\
        & = \left(\begin{array}{c}
                \BLUE{a_2} \\ \RED{\frac1{2}a_0 - \frac1{2}a_1} \\ \GREEN{\frac1{2}a_0 + \frac1{2}a_1 - a_2}
            \end{array}\right)
\end{align*}
Therefore
\[
    \T^{-1}(a_0 + + a_1 x + a_2 x^2) = (\BLUE{a_2}, \RED{\frac1{2}a_0 - \frac1{2}a_1}, \GREEN{\frac1{2}a_0 + \frac1{2}a_1 - a_2}).
\]

\item Skip.
\item Skip.
\end{enumerate}
\end{proof}

\begin{exercise} \label{exercise 3.2.7}
Express the invertible matrix
\[
    \begin{pmatrix} 1 & 2 & 1 \\ 1 & 0 & 1 \\ 1 & 1 & 2 \end{pmatrix}
\]
as a product of elementary matrices.
\end{exercise}

\begin{proof}
We can do the Gaussian elimination and record what operation we've done.
\begin{align*}
    \left(\begin{array}{lll}
        1 & 2 & 1 \\
        1 & 0 & 1 \\
        1 & 1 & 2
    \end{array}\right)
    & \leadsto
    \left(\begin{array}{ccc}
        1 & 2 & 1 \\
        0 & -2 & 0 \\
        1 & 1 & 2
    \end{array}\right)
    \text{, with } & E_1 = \left(\begin{array}{ccc}
        1 & 0 & 0 \\
        -1 & 1 & 0 \\
        0 & 0 & 1
    \end{array}\right) \\
    & \leadsto
        \left(\begin{array}{ccc}
        1 & 2 & 1 \\
        0 & -2 & 0 \\
        0 & -1 & 1
    \end{array}\right)
    \text{, with } & E_2 = \left(\begin{array}{ccc}
        1 & 0 & 0 \\
        0 & 1 & 0 \\
        -1 & 0 & 1
    \end{array}\right) \\
    & \leadsto
        \left(\begin{array}{ccc}
        1 & 2 & 1 \\
        0 & 1 & 0 \\
        0 & -1 & 1
    \end{array}\right)
    \text{, with } & E_3 = \left(\begin{array}{ccc}
        1 & 0 & 0 \\
        0 & -\frac1{2} & 0 \\
        0 & 0 & 1
    \end{array}\right) \\
    & \leadsto
        \left(\begin{array}{ccc}
        1 & 0 & 1 \\
        0 & 1 & 0 \\
        0 & -1 & 1
    \end{array}\right)
    \text{, with } & E_4 = \left(\begin{array}{ccc}
        1 & -2 & 0 \\
        0 & 1 & 0 \\
        0 & 0 & 1
    \end{array}\right) \\
    & \leadsto
    \left(\begin{array}{ccc}
        1 & 0 & 1 \\
        0 & 1 & 0 \\
        0 & 0 & 1
    \end{array}\right)
    \text{, with } & E_5 = \left(\begin{array}{ccc}
        1 & 0 & 0 \\
        0 & 1 & 0 \\
        0 & 1 & 1
    \end{array}\right) \\
        & \leadsto
    \left(\begin{array}{ccc}
        1 & 0 & 0 \\
        0 & 1 & 0 \\
        0 & 0 & 1
    \end{array}\right)
    \text{, with } & E_6 = \left(\begin{array}{ccc}
        1 & 0 & -1 \\
        0 & 1 & 0 \\
        0 & 1 & 1
    \end{array}\right)
\end{align*}
So we have \(\I_3 = E_6 E_5 E_4 E_3 E_2 E_1 A\), hence
\(A = E_{1}^{-1} E_{2}^{-1} E_{3}^{-1} E_{4}^{-1} E_{5}^{-1} E_{6}^{-1}\).
\end{proof}

\begin{exercise} \label{exercise 3.2.8}
Let \(A\) be an \(m \X n\) matrix.
Prove that if \(c\) is any nonzero scalar, then \(\rank(cA) = \rank(A)\).
\end{exercise}

\begin{proof}
Clearly,
\[
    cA = \begin{pmatrix} ca_1 \\ ca_2 \\ \vdots \\ ca_m
    \end{pmatrix}
\]
where \(a_i\) is the \(i\)th row of \(A\).
If we perform \(m\) type 2 e.r.o.s on each row of \(cA\) with scalar \(\frac1{c}\) then we get \(A\).
And since e.r.o.s are rank-preserving, we have \(\rank(cA) = \rank(A)\).
\end{proof}

\begin{exercise} \label{exercise 3.2.9}
Complete the proof of the \CORO{3.4.1} by showing that e.\RED{c}.o.s preserve rank.
\end{exercise}

\begin{proof}
See the second part of the proof of \CORO{3.4.1}.
\end{proof}

\begin{exercise} \label{exercise 3.2.10}
Prove \THM{3.6} for the case that \(A\) is an \(m \X 1\) matrix.
\end{exercise}

\begin{proof}
Given \(m \X 1\) nonzero matrix \(A\), there exists a position \((i, 1)\) such that \(A_{i1} \ne 0\).
By means of at most one type 1 row operation and at most one type 2 row operation, \(A\) can be transformed into a matrix with a \(1\) in the \((1, 1)\) position.
By means of at most \(m - 1\) type 3 row operations, this matrix can in turn be transformed into the matrix
\[
\begin{pmatrix} 1 \\ 0 \\ \vdots \\ 0 \end{pmatrix},
\]
which has the desired form, and from \THM{3.5}, it has rank \(1\).
Since e.r.o.s are rank-preserving, the original \(A\) has rank \(1\). which is both \(\le 1\), the number of columns of \(A\), and \(\le m\), the number of rows of \(A\).
\end{proof}

\begin{exercise} \label{exercise 3.2.11}
Let
\[
    B = \left(\begin{array}{c|ccc}
        1 & 0 & \cdots & 0 \\
        \hline 0 & & & \\
        \vdots & & B^{\prime} & \\
        0 & & &
    \end{array}\right),
\]
where \(B'\) is an \(m \X n\) submatrix of \(B\).
Prove that if \(\rank(B) = r\), then \(\rank(B') = r - 1\).
\end{exercise}

\begin{proof}
By \THM{3.5}, the maximum number of \(B\)'s \LID{} rows is equal to \(r\).
Furthermore, by the structure of the form of \(B\), the first column
\[
    \begin{pmatrix} 1 \\ 0 \\ \vdots \\ 0 \end{pmatrix}
\]
\emph{is \LID{} to the remaining columns of \(B\)} since these columns have \(0\) in the first entry.
And that means \(r = \rank(B) = \dim(\spann(\{ \RED{b_1}, b_2, ..., b_n \})) \RED{=} \dim(\spann(\{ b_2, b_3, ..., b_n \})) \RED{+ 1}\), or \(\rank(\spann(\{ b_2, b_3, ..., b_n \})) = r - 1\).

Now it suffices to show \(\rank(B') = \dim(\spann(\{ b_2, b_3, ..., b_n \}))\).
(This is in fact very intuitive since the first \(0\) of each \(b_i\) ``does not contribute any linear independency'', and by removing the it the columns become the corresponding columns of \(B'\), hence \(B'\) and \(\spann(\{ b_2, b_3, ..., b_n \})\) have same dimensions.)

To do this, we define \(\T : F^{m - 1} \to W\), where \(W\) is a subspace of \(F^m\) with elements having \(0\) as the first entry,
and
\[
    \T(a_1, a_2, ..., a_m) = (0, a_1, a_2, ..., a_m).
\]
It's clear that \(\T\) is an isomorphism.
Now let \(v_1, v_2, ..., v_{r - 1}\) be the selected \LID{} vectors of \(\{ b_2, b_3, ..., b_n \}\).
And let \(v_1', v_2', ..., v_{r - 1}'\) be the columns of \(B'\) such that \(v_i'\) lies in the same column position as \(v_i\).
Then it's clear that \(\T(v_i') = v_i\) for \(i = 1, ..., r - 1\).
And by \ATHM{2.2}(2.b), \(\{ v_1', v_2', ..., v_{r - 1}'\}\) is also \LID{}, which implies the dimension of subspace generated by the columns of \(B'\) is at least \(r - 1\).
But since \(\rank(\spann(\{ b_2, b_3, ..., b_n \})) = r - 1\) any \(r\) vectors  of \(b_2, ..., b_n\) must be \LDP{}.
And by \ATHM{2.2}(2.b), the corresponding \(r\) vectors mapped by \(\T\) in columns of \(B'\) must also be \LDP{}, and that implies \(\rank(B') < r\).
Hence \(\rank(B') = r - 1\), as desired.
\end{proof}

\begin{exercise} \label{exercise 3.2.12}
Let \(B'\) and \(D'\) be \(m \X n\) matrices, and let \(B\) and \(D\) be \((m + 1) \X (n + 1)\) matrices respectively defined by
\[
    B = \left(\begin{array}{c|ccc}
        1 & 0 & \cdots & 0 \\
        \hline 0 & & & \\
        \vdots & & B^{\prime} & \\
        0 & & &
    \end{array}\right)
    \text { and }
    D = \left(\begin{array}{c|ccc}
        1 & 0 & \cdots & 0 \\
        \hline 0 & & \\
        \vdots & & D^{\prime} & \\
        0 & &
    \end{array}\right).
\]
Prove that if \(B'\) can be transformed into \(D'\) by an elementary row [column] operation, then \(B\) can be transformed into \(D\) by an elementary row [column] operation.
\end{exercise}

\begin{proof}
Suppose \(B'\) can be transformed into \(D'\) by an elementary row operation.
Then by \THM{3.1} we can write \(D' = E'B'\) for some \(m \X m\) elementary matrix \(E'\).
Now let
\[
    E = \left(\begin{array}{c|ccc}
        1 & 0 & \ldots & 0 \\
        \hline 0 & & & \\
        \vdots & & E' & \\
        0 & & &
    \end{array}\right)
\]
Then it's clear that \(E\) is also an elementary matrix with dimension \((m + 1) \X (m + 1)\).
And
\[
    EB = \left(\begin{array}{c|ccc}
        1 & 0 & \ldots & 0 \\
        \hline 0 & & & \\
        \vdots & & E' & \\
        0 & & &
    \end{array}\right) \left(\begin{array}{c|ccc}
        1 & 0 & \cdots & 0 \\
        \hline 0 & & & \\
        \vdots & & B' & \\
        0 & & &
    \end{array}\right) \\
    \RED{=} \left(\begin{array}{c|ccc}
        1 & 0 & \ldots & 0 \\
        \hline 0 & & & \\
        \vdots & & E' B' & \\
        0 & & &
    \end{array}\right) =
    \left(\begin{array}{c|ccc}
        1 & 0 & \ldots & 0 \\
        \hline 0 & & & \\
        \vdots & & D' & \\
        0 & & &
    \end{array}\right)
    = D.
\]
(Note that the red \RED{\(=\)} needs proof, but it's similar to the proof of \EXEC{3.2.15}.)
Hence by \THM{3.1} again \(B\) can be transformed into \(D\) by an elementary row operation.

The case of column operation is similar to prove.
\end{proof}

\begin{exercise} \label{exercise 3.2.13}
Prove (b) and (c) of \CORO{3.6.2}.
\end{exercise}

\begin{proof}
See \CORO{3.6.2}.
\end{proof}

\begin{exercise} \label{exercise 3.2.14}
Let \(\T, \U: V \to W\) be linear transformations.
\begin{enumerate}
\item Prove that \(\RANGE(\T + \U) \subseteq \RANGET + \RANGE(\U)\).
(See the definition \ADEF{1.8} of the sum of subsets of a vector space.)
\item Prove that if \(W\) is finite-dimensional, then \(\rank(\T + \U) \le \rankT + \rank(\U)\).
\item Deduce from (b) that \(\rank(A + B) \le \rank(A) + \rank(B)\) for any \(m \X n\) matrices \(A\) and \(B\).
\end{enumerate}
\end{exercise}

\begin{proof} \ 

\begin{enumerate}
\item Suppose arbitrary \(w \in \RANGE(\T + \U)\).
Then \(\exists v \in V such that (\T + \U)(v) = w\), or \(\T(v) + \U(v) = w\).
Then by definition of sum, \(w \in \RANGET + \RANGE(\U)\) since \(w = \T(v) + \U(v)\) where \(\T(v) \in \RANGET\) and \(\U(v) \in \RANGE(\U)\).
Hence \(\RANGE(\T + \U) \subseteq \RANGET + \RANGE(\U)\).

\sloppy
\item From \ATHM{1.27}(1.1), \(\dim(\RANGET + \RANGE(\U)) = \dim(\RANGET) + \dim(\RANGE(\U)) - \dim(\RANGET \cap \RANGE(\U))\).
In particular, \(\dim(\RANGET + \RANGE(\U)) \le \dim(\RANGET) + \dim(\RANGE(\U))\).
And from part(a) we have \(\dim(\RANGE(\T + \U)) \le \dim(\RANGET + \RANGE(\U))\).
Put them together, we have \(\dim(\RANGE(\T + \U)) \le \dim(\RANGET) + \dim(\RANGE(\U))\).
That is, by definition of rank, \(\rank(\T + \U)) \le \rankT + \rank(\U)\).

\item
\begin{align*}
    \rank(A + B) & = \rank(\LMTRAN_{A + B}) & \text{by \DEF{3.3}} \\
                 & = \rank(\LMTRAN_A + \LMTRAN_B) & \text{by \THM{2.15}(c)} \\
                 & \le \rank(\LMTRAN_A) + \rank(\LMTRAN_B) & \text{by part(b)} \\
                 & = \rank(A) + \rank(B) & \text{by \DEF{3.3}}
\end{align*}
\end{enumerate}
\end{proof}

\begin{exercise} \label{exercise 3.2.15}
Suppose that \(A\) and \(B\) are matrices having \(n \X p\) and \(n \X q\) rows, respectively.
Prove that \(M(A|B) = (MA|MB)\) for any \(m \X n\) matrix \(M\).
\end{exercise}

\begin{proof}
We have
\begin{equation*}
    [M(A | B)]_{ij} =
    \begin{cases}
        \sum_{k = 1}^n M_{ik} A_{kj} = (MA)_{ij} & \text{ if } 1 \le j \le p \\
        \sum_{k = 1}^n M_{ik} B_{k(\RED{j - p})} = (MB)_{i(j - p)} & \text{ if } p < j \le p + q \\
    \end{cases}
\end{equation*}
So
\begin{align*}
    M(A | B) & = \left(\begin{array}{lll|lll}
        (MA)_{11} & ... & (MA)_{1p} & (MB)_{1(p+1-p)} & ... & (MB)_{1(p+q-p)} \\
        \vdots    &     & \vdots    & \vdots          &     & \vdots \\
        (MA)_{n1} & ... & (MA)_{np} & (MB)_{n(p+1-p)} & ... & (MB)_{n(p+q-p)} \\
    \end{array}\right) \\
             & = \left(\begin{array}{lll|lll}
        (MA)_{11} & ... & (MA)_{1p} & (MB)_{11} & ... & (MB)_{1q} \\
        \vdots    &     & \vdots    & \vdots          &     & \vdots \\
        (MA)_{n1} & ... & (MA)_{np} & (MB)_{n1} & ... & (MB)_{nq} \\
    \end{array}\right),
\end{align*}
which is equal to \((MA|MB)\).
\end{proof}

\begin{exercise} \label{exercise 3.2.16}
Supply the details to the proof of (b) of \THM{3.4}.
\end{exercise}

\begin{proof}
See the proof of \THM{3.4}.
\end{proof}

\begin{exercise} \label{exercise 3.2.17}
Prove that if \(B\) is a \(3 \X 1\) matrix and \(C\) is a \(1 \X 3\) matrix, then the \(3 \X 3\) matrix \(BC\) \textbf{has rank at most \(1\)}.
Conversely, show that if \(A\) is any \(3 \X 3\) matrix having rank \(1\), then there exist a \(3 \X 1\) matrix \(B\) and a \(1 \X 3\) matrix \(C\) such that \(A = BC\).
\end{exercise}

\begin{proof} \ 

\(\Longrightarrow\): By \THM{3.7}(c)(d), \(\rank(BC) \le \min(\rank(B), \rank(C)) = 1\).

\(\Longleftarrow\): Suppose \(A\) is any \(3 \X 3\) matrix having rank \(1\).
Then \(A\) must have the form
\[
    A = \begin{pmatrix} a & c_1 a & c_2 a \end{pmatrix} or \begin{pmatrix} c_1 a' & a' & c_2 a' \end{pmatrix} or \begin{pmatrix} c_1 a'' & c_2 a'' & a'' \end{pmatrix}
\]
where \(c_1, c_2 \in F\) and \(a, a', a''\) is the first/second/third column of \(A\).
(If \(A\) is not in one of these forms then it will trivially to get contradiction that \(\rank(A) \ne 1\).)
We just prove the case that \(A\) has the first form, the other cases are similar to prove.
Now let \(a = \begin{pmatrix} a_1 \\ a_2 \\ a_3 \end{pmatrix}\).
Then \(A\) can be represented as the product of \(3 \X 1\) ``matrix'' \(a\) and \(1 \X 3\) matrix \(c = \begin{pmatrix} 1 & c_1 & c_2 \end{pmatrix}\) since
\[
    a c = \begin{pmatrix} a_1 \\ a_2 \\ a_3 \end{pmatrix} \begin{pmatrix} 1 & c_1 & c_2 \end{pmatrix}
    = \begin{pmatrix}
        a_1 & c_1 a_1 & c_2 a_1 \\
        a_2 & c_1 a_2 & c_2 a_2 \\
        a_3 & c_1 a_3 & c_2 a_3 \\
    \end{pmatrix}
    = \begin{pmatrix} a & c_1 a & c_2 a \end{pmatrix} = A.
\]
\end{proof}

\begin{exercise} \label{exercise 3.2.18}
Let \(A\) be an \(m \X n\) matrix and \(B\) be an \(n \X p\) matrix.
Prove that \(AB\) can be written as a sum of \(n\) matrices of rank at most one.
\end{exercise}

\begin{proof}
In fact, we can split \(AB\) as:
\begin{align*}
    AB & = \begin{pmatrix}
                (AB)_{11} & ... & (AB)_{1p} \\
                \vdots    &     & \vdots \\
                (AB)_{m1} & ... & (AB)_{mp} \\
            \end{pmatrix} & \\
       & = \begin{pmatrix}
                \sum_{k = 1}^n A_{1k} B_{k1} & ... & \sum_{k = 1}^n A_{1k} B_{kp} \\
                \vdots    &     & \vdots \\
                \sum_{k = 1}^n A_{mk} B_{k1} & ... & \sum_{k = 1}^n A_{mk} B_{kp} \\
            \end{pmatrix} & \\
        & \RED{=} \begin{pmatrix}
                A_{1\RED{1}} B_{\RED{1}1} & ... & A_{1\RED{1}} B_{1p} \\
                \vdots    &     & \vdots \\
                A_{m\RED{1}} B_{\RED{1}1} & ... & A_{m\RED{1}} B_{\RED{1}p} \\
            \end{pmatrix}
            + \begin{pmatrix}
                A_{1\RED{2}} B_{\RED{2}1} & ... & A_{1\RED{2}} B_{\RED{2}p} \\
                \vdots    &     & \vdots \\
                A_{m\RED{2}} B_{\RED{2}1} & ... & A_{m\RED{2}} B_{\RED{2}p} \\
            \end{pmatrix}
            + ... + \begin{pmatrix}
                A_{1\RED{n}} B_{\RED{n}1} & ... & A_{1\RED{n}} B_{\RED{n}p} \\
                \vdots    &     & \vdots \\
                A_{m\RED{n}} B_{\RED{n}1} & ... & A_{m\RED{n}} B_{\RED{n}p} \\
            \end{pmatrix}, \\
            & \text{\ \ \ \ \ \ \ (by splitting into \(n\) matrices)}
\end{align*}
and each splitted matrix can be considered as the product of \(m \X 1\) and \(1 \X p\) matrices;
that is,
\[
    \begin{pmatrix} A_{11} \\ A_{21} \\ \vdots \\ A_{m1} \end{pmatrix} \begin{pmatrix} B_{11} & B_{12} & ... & B_{1p} \end{pmatrix}
    + \begin{pmatrix} A_{12} \\ A_{22} \\ \vdots \\ A_{m2} \end{pmatrix} \begin{pmatrix} B_{21} & B_{22} & ... & B_{2p} \end{pmatrix}
    + ... + \begin{pmatrix} A_{1n} \\ A_{2n} \\ \vdots \\ A_{mn} \end{pmatrix} \begin{pmatrix} B_{n1} & B_{n2} & ... & B_{np} \end{pmatrix}.
\]
By (the generalization of) \EXEC{3.2.17}, each product has at most rank one.
Hence we have written \(AB\) as the sum of \(n\) matrices of rank at most one, as desired.
\end{proof}

\begin{exercise} \label{exercise 3.2.19}
Let \(A\) be an \(m \X n\) matrix with rank \(m\) and \(B\) be an \(n \X p\) matrix with rank \(n\).
Determine the rank of \(AB\).
Justify your answer.
\end{exercise}

\begin{proof}
We have
\begin{align*}
             & \rank(A) = m \\
    \implies & \rank(\LMTRAN_A) = m & \text{by \DEF{3.3}} \\
    \implies & \LMTRAN_A \text{ is onto } & \text{since the codomain of \(\LMTRAN_A : F^n \to F^m\) has \(\dim = m\)}
\end{align*}
Similarly, \(\LMTRAN_B: F^p \to F^n\) is also onto.

Then
\begin{align*}
    \rank(AB) & = \rank(\LMTRAN_{AB}) & \text{by \DEF{3.3}} \\
              & = \rank(\LMTRAN_A \LMTRAN_B) & \text{by \THM{2.15}(e)} \\
              & = \dim(\LMTRAN_A(\LMTRAN_B(F^p))) & \text{just by def of range and rank and composition} \\
              & = \dim(\LMTRAN_A(F^n)) & \text{since \(\LMTRAN_B\) is onto} \\
              & = \dim(F^m) & \text{since \(\LMTRAN_A\) is onto} \\
              & = m.
\end{align*}
\end{proof}

\begin{exercise} \label{exercise 3.2.20}
Let
\[
    A = \begin{pmatrix}
        1 & 0 & -1 & 2 & 1 \\
        -1 & 1 & 3 & -1 & 0 \\
        -2 & 1 & 4 & -1 & 3 \\
        3 & -1 & -5 & 1 & -6
    \end{pmatrix}.
\]
\begin{enumerate}
\item Find a \(5 \X 5\) matrix \(M\) with rank \(\RED{2}\) such that \(AM = O\), where \(O\) is the \(4 \X 5\) zero matrix.

\item Suppose that \(B\) is a \(5 \X 5\) matrix such that \(AB = O\).
Prove that \(\rank(B) \le \RED{2}\).
\end{enumerate}
\end{exercise}

\begin{note}
I feel that the exercise is really strange, because the criteria of rank \(\RED{2}\) is just like a magic number.
It seems more reasonable to given some other fact, e.g. \(\rank(A) = 3\).
\end{note}

\begin{proof}
(Note that I use \THM{3.8}, which is in \SEC{3.3}, so this proof is a forward reference.)
\begin{enumerate}
\item
First, if \(AM = O\), that means \textbf{each} column of \(M\) is in the null space of \(\LMTRAN_A\), because \(A m_i = 0 \in F^m\) for each column \(m_i\) of \(M\).
Hence given any that kind of \(M\) the columns space of \(M\) is a subspace of \(\NULL(\LMTRAN_A)\).
Hence the \(\rank(M) \le \dim(\NULL(\LMTRAN_A))\).
Then why not we just first find the null space of \(\LMTRAN_A\)?
By the technique in \SEC{3.3}, this is equivalent to find the solution set of \(Ax = 0\),
which is
\[
    \left\{
        t_1 \begin{pmatrix} 1 \\ -2 \\ 1 \\ 0 \\ 0 \end{pmatrix}
        + t_2 \begin{pmatrix} 3 \\ 1 \\ 0 \\ -2 \\ 1 \end{pmatrix}
        : t_1, t_2 \in \SET{R}
    \right\}.
\]
Hence the dimension of \(\NULL(\LMTRAN_A)\) is \(2\), and the basis is
\[
    \left\{
        \begin{pmatrix} 1 \\ -2 \\ 1 \\ 0 \\ 0 \end{pmatrix}
        \begin{pmatrix} 3 \\ 1 \\ 0 \\ -2 \\ 1 \end{pmatrix}
    \right\}.
\]
So we just let
\[
    M = \left[\begin{array}{ccccc}
        1 & 3 & 0 & 0 & 0 \\
        -2 & 1 & 0 & 0 & 0 \\
        1 & 0 & 0 & 0 & 0 \\
        0 & -2 & 0 & 0 & 0 \\
        0 & 1 & 0 & 0 & 0
    \end{array}\right]
\]
then \(\rank(M) = 2\) and \(AM = O\).

\item We have shown in the previous case that any kind of \(B\) such that \(AB = O\) has the column space as a subspace of \(\NULL(\LMTRAN_A)\), and we have shown that \(\dim(\NULL(\LMTRAN_A)) = 2\), hence \(\rank(B) \le 2\).
\end{enumerate}
\end{proof}

\begin{exercise} \label{exercise 3.2.21}
Let \(A\) be an \(m \X n\) matrix with rank \(m\).
Prove that there exists an \(n \X m\) matrix \(B\) such that \(AB = I_m\).
\end{exercise}

\begin{proof}
With same argument in \EXEC{3.2.19}, \(\LMTRAN_A : F^n \to F^m\) is onto.
So for each vector \(e_1, e_2, ..., e_m\) in the standard basis of \(F^m\), we can find \(v_1, v_2, ..., v_m \in F^n\) such that \(\LMTRAN_A(v_i) = e_i\);
that is, \(A v_i = e_i\).
Now let \(B = \begin{pmatrix} v_1 & v_2 & ... & v_m \end{pmatrix}\).
Then
\begin{align*}
    AB & = \begin{pmatrix} A v_1 & A v_2 & ... & A v_m \end{pmatrix} & \text{by \THM{2.13}(a)} \\
       & = \begin{pmatrix} e_1 & e_2 & ... & e_m \end{pmatrix} = I_m,
\end{align*}
as desired.
\end{proof}

\begin{exercise} \label{exercise 3.2.22}
Let \(B\) be an \(n \X m\) matrix with rank \(m\).
Prove that there exists an \(m \X n\) matrix \(A\) such that \(AB = I_m\).
\end{exercise}

\begin{proof}
By \CORO{3.6.2}(a), \(B^\top\) is an \(m \X n\) matrix that also has rank \(m\).
And by \EXEC{3.2.21}, we can find an \(n \X m\) matrix \(A^\top\) such that \(B^\top A^\top = I_m\).
(I intentionally denote that matrix with transpose.)
Then we have \((B^\top A^\top)^\top = I_m^\top = I_m\);
that is \(A B = I_m\).
So we have found a \(m \X n\) matrix \(A\) such that \(A B = I_m\).
\end{proof}

\begin{additional theorem} \label{athm 3.4}
This is a placeholder theorem for \EXEC{3.2.8}, given any nonzero scalar \(c\), \(\rank(cA) = \rank(A)\) any any \(m \X n\) matrix \(A\).
\end{additional theorem}

\begin{additional theorem} \label{athm 3.5}
This is a placeholder theorem for \EXEC{3.2.11} and \EXEC{3.2.12}, which are intermediate steps of \THM{3.6}.
\end{additional theorem}

\begin{additional theorem} \label{athm 3.6}
This is a placeholder theorem for \EXEC{3.2.14}:
Let \(\T, \U: V \to W\) be linear transformations.
\begin{enumerate}
\item \(\RANGE(\T + \U) \subseteq \RANGET + \RANGE(\U)\).
\item If \(W\) is finite-dimensional, then \(\rank(\T + \U) \le \rankT + \rank(\U)\).
\item \(\rank(A + B) \le \rank(A) + \rank(B)\) for any \(m \X n\) matrices \(A\) and \(B\).
\end{enumerate}
\end{additional theorem}

\begin{additional theorem} \label{athm 3.7}
\sloppy This is a placeholder theorem for \EXEC{3.2.15}:
\(M(A|B) = (MA|MB)\) (where these matrices have compatible dimensions).
\end{additional theorem}

\begin{additional theorem} \label{athm 3.8}
This is a placeholder theorem for \EXEC{3.2.17}:
If \(B\) is a \(3 \X 1\) matrix and \(C\) is a \(1 \X 3\) matrix, then the \(3 \X 3\) matrix \(BC\) \textbf{has rank at most \(1\)}.
Conversely, if \(A\) is any \(3 \X 3\) matrix having rank \(1\), then there exist a \(3 \X 1\) matrix \(B\) and a \(1 \X 3\) matrix \(C\) such that \(A = BC\).

This theorem can be generalized with \(n \X 1\) and \(1 \X n\) matrices using induction.
\end{additional theorem}

\begin{additional theorem} \label{athm 3.9}
This is a placeholder theorem for \EXEC{3.2.18}:
Let \(A\) be an \(m \X n\) matrix and \(B\) be an \(n \X p\) matrix.
Then \(AB\) can be written as a sum of \(n\) matrices of rank at most one.
\end{additional theorem}

\begin{additional theorem} \label{athm 3.10}
This is a placeholder theorem for \EXEC{3.2.19}, which has concepts related to ``onto''.
\end{additional theorem}

\begin{additional theorem} \label{athm 3.11}
This is a placeholder theorem for \EXEC{3.2.21} and \EXEC{3.2.22}, which let us find ``some kind of inverse'' for \emph{non-square} matrices.
\end{additional theorem}