\exercisesection

\begin{exercise} \label{exercise 6.2.1}
Label the following statements as true or false.
\begin{enumerate}
\item The Gram-Schmidt orthogonalization process produces an ortho\textbf{normal} set from an arbitrary \emph{\LID{}} set.
\item Every nonzero finite-dimensional inner product space has an orthonormal basis.
\item The orthogonal complement of any \emph{set} is a subspace.
\item If \(\{ v_1, v_2, ..., v_n \}\) is a basis for an inner product space \(\V\), then for any \(x \in \V\) the scalars \(\LG x, v_i \RG\) are the Fourier coefficients of \(x\).
\item An orthonormal basis must be an ordered basis.
\item Every orthogonal set is linearly independent.
\item Every orthonormal set is linearly independent.
\end{enumerate}
\end{exercise}

\begin{proof} \ 

\begin{enumerate}
\item Strictly speaking, false, since the process only produces an ortho\RED{gonal} set from an arbitrary \LID{} set.
But it also seems reasonable to add a normalization step as last step in the process.

\item True by \THM{6.5}.

\item True.
Let \(S\) be any set of an inner product space, we have to show \(S^{\perp}\) is a subspace.
We use \THM{1.3} to show \(S^{\perp}\) is a subspace.
So first, \(\OV \in S^{\perp}\) since \(\LG \OV, s \RG = 0\) for all \(s \in S\).
Now suppose \(c\) is a scalar and \(v_1, v_2 \in S^{\perp}\), we have to show \(c v_1 + v_2 \in S^{\perp}\).
But for all \(s \in S\),
\begin{align*}
    \LG c v_1 + v_2, s \RG & = c \LG v_1, s \RG + \LG v_2, s \RG & \text{by \DEF{6.1}(a)(b)} \\
        & = c \cdot 0 + 0 & \text{since \(v_1, v_2 \in S^{\perp}\)} \\
        & = 0
\end{align*}
Hence \(c v_1 + v_2 \in S^{\perp}\).
So by \THM{1.3}, \(S^{\perp}\) is a subspace.

\item False, by \DEF{6.6}, the basis also need to be orthonormal.

\item Weird question, but true by \DEF{6.5}.

\item False, the set cannot have zero vector.

\item True, since by definition any vector in the set has norm equal to \(1\) (so the vector is nonzero), and the set is orthogonal, by \CORO{6.3.2}, the set is \LID{}.
\end{enumerate}
\end{proof}

\begin{exercise} \label{exercise 6.2.2}
In each part, apply the Gram-Schmidt process to the given subset \(S\) of the inner product space \(\V\) to obtain an orthogonal basis for \(\spann(S)\).
Then normalize the vectors in this basis to obtain an orthonormal basis \(\beta\) for \(\spann(S)\), and compute the Fourier coefficients of the given vector relative to \(\beta\).
Finally, use \THM{6.5} to verify your result.

\begin{enumerate}
\item \(\V = \SET{R}^3, S = \{ (1, 0, 1), (0, 1, 1), (1, 3, 3) \}\), and \(x = (1, 1, 2)\)
\item \(\V = \SET{R}^3, S = \{ (1, 1, 1), (0, 1, 1), (0, 0, 1) \}\), and \(x = (1, 0, 1)\)
\item \(\V = \POLYRR\) with the inner product \(\LG f(x), g(x) \RG = \int_0^1 f(t) g(t) dt, S = \{1, x, x^2 \}\), and \(h(x) = 1 + x\)
\item \(\V = \spann(S)\), where \(S = \{(1, \iu, 0), (1 - \iu, 2, 4\iu) \}\), and \(x = (3 + \iu, 4\iu, -4)\)
\item \(\V = \SET{R}^4, S = \{ (2, -1, -2, 4), (-2, 1, -5, 5), (-1, 3, 7, 11) \}\), and \(x = (-11, 8, -4, 18)\)
\item \(\V = \SET{R}^4, S = \{ (1, -2, -1, 3), (3, 6, 3, -1), (1, 4, 2, 8) \}\), and \(x = (-1, 2, 1, 1)\)
\item \(\V = M_{2 \X 2}(\SET{R}),
    S = \left\{
        \begin{pmatrix}3 & 5 \\ -1 & 1\end{pmatrix},
        \begin{pmatrix}-1 & 9 \\ 5 & -1\end{pmatrix},
        \begin{pmatrix}7 & -17 \\ 2 & -6\end{pmatrix}\right
    \}\),
    and \(A = \begin{pmatrix}-1 & 27 \\ -4 & 8\end{pmatrix}\)
\item \(\V = M_{2 \X 2}(\SET{R}),
    S = \left\{
        \begin{pmatrix}2 & 2 \\ 2 & 1\end{pmatrix},
        \begin{pmatrix}11 & 4 \\ 2 & 5\end{pmatrix},
        \begin{pmatrix}4 & -12 \\ 3 & -16\end{pmatrix}
    \right\}\),
    and \(A = \begin{pmatrix}8 & 6 \\ 25 & -13\end{pmatrix}\)
\item \(\V = \spann(S)\) with the inner product \(\LG f, g \RG = \int_0^{\pi} f(t)g(t) dt, S = \{\sin t, \cos t, 1, t\}\), and \(h(t) = 2t + 1\)
\item \(\V = \SET{C}^4,
    S = \{ (1, i, 2 - \iu, -1), (2 + 3\iu, 3\iu, 1 - \iu, 2\iu), (-1 + 7\iu, 6 + 10\iu, 11 - 4\iu, 3 + 4\iu)\}\),
    and \(x = (-2 + 7\iu, 6 + 9\iu, 9 - 3\iu, 4 + 4\iu)\)
\item \(\V = \SET{C}^4, S = \{(-4, 3 - 2\iu, \iu, 1 - 4\iu), (-1 - 5\iu, 5 - 4\iu, -3 + 5\iu, 7 - 2\iu), (-27 - \iu, -7 - 6\iu, -15 + 25\iu, -7 - 6\iu)\}\), and \(x = (-13 - 7\iu, -12 + 3\iu, -39 - 11\iu, -26 + 5\iu)\)
\item \(\V = M_{2 \X 2}(\SET{C})\),
\[
    S = \left\{
        \begin{pmatrix} 1 - \iu & -2 - 3\iu \\ 2 + 2\iu & 4 + \iu \end{pmatrix},
        \begin{pmatrix} 8\iu & 4 \\ -3 - 3\iu & -4 + 4\iu \end{pmatrix},
        \begin{pmatrix} -25 - 38\iu & -2 - 13\iu \\ 12 - 78\iu & -7 + 24\iu \end{pmatrix}
    \right\},
\]
and \(A=\begin{pmatrix} -2 + 8\iu & -13 + \iu \\ 10 - 10\iu & 9 - 9\iu \end{pmatrix}\)
\item \(\V = M_{2 \X 2}(\SET{C})\),
\[
    S = \left\{
        \begin{pmatrix} -1 + \iu & -\iu \\ 2 - \iu & 1 + 3\iu\end{pmatrix},
        \begin{pmatrix} -1 - 7\iu & -9 - 8\iu \\ 1 + 10\iu & -6 - 2\iu\end{pmatrix},
        \begin{pmatrix} -11 - 132\iu & -34 -31\iu \\ 7 - 126\iu & -71 - 5\iu \end{pmatrix}
    \right\},
\]
and \(A=\begin{pmatrix} -7 + 5\iu & 3 + 18\iu \\ 9 - 6\iu & -3 + 7\iu \end{pmatrix}\)
\end{enumerate}
\end{exercise}

\begin{proof}
Skip. Too painful using \LaTeX.
\end{proof}

\begin{exercise} \label{exercise 6.2.3}
In \(\SET{R}^2\), let
\[
    \beta = \left\{
        \left( \frac{1}{\sqrt{2}}, \frac{1}{\sqrt{2}} \right), \left( \frac{1}{\sqrt{2}}, \frac{-1}{\sqrt{2}} \right)
    \right\}
\]
Find the Fourier coefficients of \((3, 4)\) relative to \(\beta\).
\end{exercise}

\begin{proof}
We have
\[
    \LG (3, 4) , \left( \frac{1}{\sqrt{2}}, \frac{1}{\sqrt{2}} \right) \RG = \frac{7}{\sqrt{2}}
\]
and
\[
    \LG (3, 4) , \left( \frac{-1}{\sqrt{2}}, \frac{1}{\sqrt{2}} \right) \RG = \frac{1}{\sqrt{2}}.
\]
\end{proof}

\begin{exercise} \label{exercise 6.2.4}
Let \(S = \{ (1, 0, \iu), (1, 2, 1) \}\) in \(\SET{C}^3\). Compute \(S^{\perp}\).
\end{exercise}

\begin{proof}
We have to find all vectors \(v = (v_1, v_2, v_3) \in \SET{C}^3\) such that \(\LG v, (1, 0, \iu) \RG = \LG v, (1, 2, 1) \RG = 0\).
Technically, this is equivalent to solving the system of equations
\[
    \sysdelim..\systeme{
        1 v_1 + 0 v_2 + \iu v_3 = 0,
        1 v_1 + 2 v_2 + 1 v_3 = 0
    }.
\]
Then we have \(v_1 = -\iu v_3\) and \(v_2 = \frac{-1 v_1 - v_3}{2} = \frac{\iu v_3 - v_3}{2} = \frac{-1 + \iu}{2} v_3\).
So let \(v_3\) be a parameter \(c\), then the solution set, which is equal to \(S^{\perp}\), is
\[
    \left\{ c (-\iu, \frac{-1 + \iu}{2}, 1) : c \in \SET{C} \right\}
\]
\end{proof}

\begin{exercise} \label{exercise 6.2.5}
Let \(S_0 = \{ x_0 \}\), where \(x_0\) is a nonzero vector in \(\SET{R}^3\).
Describe \(S_0^{\perp}\) geometrically.
Now suppose that \(S = \{ x_1, x_2 \}\) is a \LID{} subset of \(\SET{R}^3\).
Describe \(S^{\perp}\) geometrically.
\end{exercise}

\begin{proof}
\(S_0^{\perp}\) is the plane which has \(x_0\) as its normal.
\(S^{\perp}\) is the line which is a normal of \(S\).
\end{proof}

\begin{exercise} \label{exercise 6.2.6}
Let \(\V\) be an inner product space, and let \(\W\) be a \emph{finite}-dimensional subspace of \(\V\).
If \(x \notin \W\), prove that there \emph{exists} \(y \in \V\) such that \(y \in \W^{\perp}\), but \(\LG x, y \RG \ne 0\).
Hint: Use \THM{6.6}.
\end{exercise}

\begin{proof}
From \THM{6.6}, \(x = u + z\) where \(u \in \W\) and \(z \in \W^{\perp}\).
Notice that \(z \ne \OV\), otherwise \(x = u + z = u + \OV = u \in W\).
Then we claim such \(z\), which is a ``normal'' of the subspace \(\W\), is what we want, since
\begin{align*}
    \LG x, z \RG & = \LG u + z, z \RG \\
        & = \LG u, z \RG + \LG z, z \RG & \text{by \DEF{6.1}(a)} \\
        & = 0 + \LG z, z \RG & \text{since \(z \in \W^{\perp}\) and \(u \in \W\)} \\
        & > 0, & \text{since \(z \ne \OV\) and by \DEF{6.1}(d)}
\end{align*}
as desired.
\end{proof}

\begin{exercise} \label{exercise 6.2.7}
Let \(\beta\) be a basis for a subspace \(\W\) of an inner product space \(\V\), and let \(z \in \V\).
Prove that \(z \in \W^{\perp}\) if and only if \(\LG z, v \RG = 0\) for every \(v \in \beta\).
\end{exercise}

\begin{note}
We do \emph{not} assume \(\W\) is finite-dimensional(hence \(\beta\) may not be finite).

The purpose of this exercise is that we don't need to examine the orthogonality between \(z\) and all vectors in \(\W\);
instead we can simply examine the relation of \(z\) and an arbitrary basis of \(\W\) to determine whether \(z \in \W^{\perp}\).
\end{note}

\begin{proof} \ 

\(\Longrightarrow\): Suppose \(z \in \W^{\perp}\).
Then \(\LG z, v \RG = 0\) for all \(v \in \W\), and in particular, \(\LG z, v \RG = 0\) for all \(v \in \beta\) where \(\beta\) is a basis(subset) of \(\W\).

\(\Longleftarrow\): Suppose \(\LG z, v \RG = 0\) for every \(v \in \beta\).
Then for every \(v \in \W\),
\begin{align*}
    \LG z, v \RG & = \LG z, \sum_{i = 1}^k a_i v_i \RG & \text{where \(v_i \in \beta\)} \\
        & = \sum_{i = 1}^k \conjugatet{a_i} \LG z, v_i \RG & \text{by \THM{6.1}(a)(b)} \\
        & = \sum_{i = 1}^k \conjugatet{a_i} 0 & \text{by supposition} \\
        & = 0.
\end{align*}
So by definition \(z \in \W^{\perp}\).
\end{proof}

\begin{exercise} \label{exercise 6.2.8}
Prove that if \(\{ w_1, w_2, ..., w_n \}\) is an \emph{orthogonal} set of nonzero vectors, then the vectors \(v_1, v_2, ..., v_n\) derived from the Gram-Schmidt process \emph{satisfy} \(v_i = w_i\) for \(i = 1, 2, ..., n\).
Hint: Use mathematical induction.
\end{exercise}

\begin{proof}
The base case \(n = 1\) is trivial since the process let \(v_1 = w_1\).

Now suppose the case for \(k - 1\) is true where \(k > 1\), that is, suppose given any nonzero orthogonal set \(\{ w_1, w_2, ..., w_{k - 1} \}\) of \(k - 1\) vectors, the vectors \(\{ v_1, v_2, ..., v_{k - 1} \}\) derived from the process satisfy \(v_i = w_i\) for \(i = 1, 2, ..., k - 1\).
We have to show the case for \(k\) is true.
So let \(S = \{ w_1, w_2, ..., w_k \}\) be an arbitrary set of nonzero orthogonal vectors.
And let \(S' = \{ v_1, v_2, ..., v_k \}\) be the vectors produced by the process.

Note that the proper subset \(\{ w_1, w_2, ..., w_{k - 1} \}\) of \(S\) is a set of \(k - 1\) nonzero orthogonal vectors, hence we can apply the induction hypothesis on the subset.
That is, if we apply the process on \(\{ w_1, w_2, ..., w_{k_1} \}\) to get \(S'' = \{ v'_1, v'_2, ... v'_{k - 1} \}\), then \(v'_i = w_i\) for \(i = 1, 2, ..., k - 1\). \MAROON{(1)}
\textbf{But by the structure(or algorithm) of the process}, if we apply the process \emph{directly} on \(\{ w_1, w_2, ..., w_k \}\), the ``subprocess'' for producing the first \(k - 1\) vectors is exactly the same as the process we directly apply on \(\{ w_1, w_2, ..., w_{k - 1} \}\), so the derived vectors are also the same;
that is, \(v_i = v'_i\) for \(i = 1, 2, ..., k - 1\). \MAROON{(2)}
So by \MAROON{(1)(2)}, \(v_i = w_i\) for \(i = 1, 2, ..., k - 1\).

And for \(v_k\),
\begin{align*}
    v_k & = w_k - \sum_{j = 1}^{\RED{k - 1}} \frac{\LG w_k, v_j \RG}{\norm{v_j}^2} v_j & \text{by the process in \THM{6.4}} \\
        & = w_k - \sum_{j = 1}^{k - 1} \frac{\LG w_k, \RED{w_j} \RG}{\norm{\RED{w_j}}^2} \RED{w_j} & \text{we have shown \(v_i = w_k\) for \(i = 1, ..., k - 1\)} \\
        & = w_k - \sum_{j = 1}^{k - 1} \frac{0}{\norm{w_j}^2} w_j & \text{since \(S = \{ w_1, ..., w_k \}\) are orthogonal} \\
        & = w_k - 0 = w_k.
\end{align*}
So we have \(v_i = w_i\) for \(i = 1, 2, ..., k\), so the case for \(k\) is true.
This closes the induction.
\end{proof}

\begin{exercise} \label{exercise 6.2.9}
Let \(\W = \spann(\{ (\iu, 0, 1) \})\) in \(\SET{C}^3\).
Find orthonormal bases for \(\W\) and \(\W^{\perp}\).
\end{exercise}

\begin{proof}
Since \(\{(\iu, 0, 1)\}\) is a singleton set, we only have to normalize it to get an orthonormal basis for \(\W\), so \(\{ \frac{\sqrt{2}}{2} (\iu, 0, 1) \}\) is an orthonormal basis for \(\W\).

Now for finding orthonormal basis for \(\W^{\perp}\), we first get what \(\W^{\perp}\) actually is; but the process is similar to \EXEC{6.2.4}, and
\[
    \W^{\perp} = \{ a(1, 0, \iu) + b(0, 1, 0) : a, b \in \SET{C} \}
\]
hence \(\{ (1, 0, \iu), (0, 1, 0) \}\) is a basis for \(\W^{\perp}\);
in particular, it happens to be orthogonal, so we only need to normalize it.
So
\[
    \{ \frac{\sqrt{2}}{2} (1, 0, \iu), (0, 1, 0) \}
\]
is an orthonormal basis of \(\W^{\perp}\).
\end{proof}

\begin{exercise} \label{exercise 6.2.10}
Let \(\W\) be a \emph{finite}-dimensional \textbf{subspace} of an inner product space \(\V\).
Prove that \(\V = \W \oplus \W^{\perp}\).
Using the \ADEF{2.2}, prove that there exists a \textbf{projection} \(\T\) on \(\W\) \textbf{along} \(\W^{\perp}\) that satisfies \(\NULLT = \W^{\perp}\).
In addition, prove that \(\norm{\T(x)} \le \norm{x}\) for all \(x \in \V\).
Hint: Use \THM{6.6} and \EXEC{6.1.10}.
\end{exercise}

\begin{proof}
Since from \THM{6.6}, every vector \(v \in \V\) can be written as the \emph{unique} combination \(u + z\) where \(u \in \W\) and \(z \in \W^{\perp}\), we have \(\V = \W + \W^{\perp}\).
Furthermore, from the discussion in \DEF{6.7}, we have \(\W \cap \W^{\perp} = \{ \OV \}\).
So by definition of direct sum, \(\V = \W \oplus \W^{\perp}\).

Then using the fact that \(\V = \W \oplus \W^{\perp}\) by \ADEF{2.2} we can define the \(\T\) such that \(\T\) is the projection on \(\W\) along \(\W^{\perp}\).
And by \EXEC{2.1.27}(b), \(\W^{\perp} = \NULLT\).

Finally, for all \(x \in \V\),
\begin{align*}
    \norm{x}^2 & = \norm{u + z}^2 & \text{where \(u \in \W\) and \(z \in \W^{\perp}\)} \\
        & = \norm{u}^2 + \norm{z}^2 & \text{since \(u, z\) are orthogonal, and by \EXEC{6.1.10}} \\
        & \ge \norm{u}^2 & \text{by \THM{6.2}(b), \(\norm{z}^2 \ge 0\)} \\
        & = \norm{\T(x)}^2, & \text{by def of \(\T\)}
\end{align*}
which implies \(\norm{x} \ge \norm{\T(x)}\), as desired.
\end{proof}

\begin{exercise} \label{exercise 6.2.11}
Let \(A\) be an \(n \X n\) matrix with \emph{complex} entries.
Prove that \(A A^* = I\) if and only if the \emph{rows} of \(A\) form an \emph{orthonormal} basis for \(\SET{C}^n\).
\end{exercise}

\begin{proof}
We define the notation \(A_i\) be the \(i\)th row of \(A\).
First we have \(A A^* = I\) if and only if \((A A^*)_{ij} = \delta_{ij}\).
But
\begin{align*}
    (A A^*)_{ij} & = \sum_{k = 1}^n A_{ik} (A^{*})_{kj} & \text{by def of matrix product} \\
        & = \sum_{k = 1}^n A_{ik} \conjugatet{A_{jk}} & \text{by \DEF{6.2}} \\
        & = \LG A_i, A_j \RG. & \text{by def of standard inner product on \(\SET{C}^n\)}
\end{align*}
So we have \(A A^* = I\) if and only if \(\LG A_i, A_j \RG = \delta_{ij}\).
\emph{But this last equation is equivalent to the rows of \(A\) being an orthonormal set} of \(n\) vectors in \(\SET{C}^n\), and hence the rows of A form an orthonormal basis for \(\SET{C}^n\).
\end{proof}

\begin{exercise} \label{exercise 6.2.12}
Prove that for any matrix \(A \in M_{m \X n}(F)\), \((\RANGE(\LMTRAN_{A^*}))^{\perp} = \NULL(\LMTRAN_A)\).
\end{exercise}

\begin{note}
Try to speak the tedious equation precisely.
The null space of \(\LMTRAN_A\) (hence \(A\)) is equal to the orthogonal complement of the range of \(\LMTRAN_{A^*}\) (hence \(A^*\)).
\end{note}

\begin{note}
中文版,\(A\) 的零空間等於\ \(A\) 的伴隨矩陣的值域的正交補集。
\end{note}

\begin{proof}
Note that by \EXEC{6.1.23}(a), \(\LG Ax, y \RG = \LG x, A^* y \RG\) for all \(x, y \in F^n\). \MAROON{(1)}
We show the statement by showing \((\RANGE(\LMTRAN_{A^*}))^{\perp} \subseteq \NULL(\LMTRAN_A)\) and \(\NULL(\LMTRAN_A) \subseteq (\RANGE(\LMTRAN_{A^*}))^{\perp}\).

So suppose \(x \in (\RANGE(\LMTRAN_{A^*}))^{\perp}\).
Then by definition of orthogonal complement, \(\LG x, y \RG = 0\) for all \(y \in \RANGE(\LMTRAN_{A^*})\).
Then by definition of range and \(\LMTRAN_{A^*}\), that implies \(\LG x, A^* y' \RG = 0\) for all \(y' \in F^n\).
But by \MAROON{(1)}, that means \(\LG Ax, y' \RG = 0\) for all \(y' \in F^n\).
Then by \THM{6.1}(e), \(Ax = \OV\), which means \(x \in \NULL(\LMTRAN_A)\).
Hence \((\RANGE(\LMTRAN_{A^*}))^{\perp} \subseteq \NULL(\LMTRAN_A)\).

Now suppose \(x \in \NULL(\LMTRAN_A)\).
Then \(Ax = 0\), and \(\LG Ax, y' \RG = 0\) for all \(y' \in F^n\).
And again by \MAROON{(1)}, that implies \(\LG x, A^* y' \RG = 0\) for all \(y' \in F^n\), which, again, by definition of range and \(\LMTRAN_{A^*}\), implies \(\LG x, y \RG = 0\) for all \(y \in \RANGE(\LMTRAN_{A^*})\).
Then by definition of orthogonal complement, \(x \in (\RANGE(\LMTRAN_{A^*}))^{\perp}\).
Hence \(\NULL(\LMTRAN_A) \subseteq (\RANGE(\LMTRAN_{A^*}))^{\perp}\).
\end{proof}

\begin{exercise} \label{exercise 6.2.13}
Let \(\V\) be an inner product space, \(S\) and \(S_0\) be \emph{subsets} of \(\V\), and \(\W\) be a finite-dimensional \emph{subspace} of \(\V\).
Prove the following results.
\begin{enumerate}
\item \(S_0 \subseteq S\) implies that \(S^{\perp} \subseteq S_0^{\perp}\).
\item \(S \subseteq (S^{\perp})^{\perp}\); so \(\spann(S) \subseteq (S^{\perp})^{\perp}\).
\item \(\W = (\W^{\perp})^{\perp}\). Hint: Use \EXEC{6.2.6}.
\item \(\V = \W \oplus \W^{\perp}\). (See the exercises of \SEC{1.3}.)
\end{enumerate}
\end{exercise}

\begin{note}
For (a)(b), there is related \EXEC{2.2.16}.
\end{note}

\begin{note}
直覺想法是,一個子集合越大,它的正交補集只會越小。
\end{note}

\begin{proof} \ 

\begin{enumerate}
\item Suppose \(S_0 \subseteq S\), we have to show \(S^{\perp} \subseteq S_0^{\perp}\).
So suppose arbitrary \(v \in S^{\perp}\), we have to show \(v \in S_0^{\perp}\).
Then by \DEF{6.7} of \(S^{\perp}\), \(\LG v, s \RG = 0\) for all \(s \in S\).
Since \(S_0 \subseteq S\), we have \(\LG v, s \RG = 0\) for all \(s \in S_0\).
Then again by \DEF{6.7}, \(s \in S_0^{\perp}\).

\item Suppose \(x \in S\), we have to show \(x \in \MAROON{(S^{\perp})}^{\perp}\).
For the sake of contradiction, suppose not.
Then by \DEF{6.7}, there exists \(y \in \MAROON{S^{\perp}}\) such that \(\LG x, y \RG \ne 0\).
Then in particular, \(\LG y, x \RG \ne 0\).
Then that implies there exists \(x \in S\) such that \(\LG y, x \RG \ne 0\), which again by \DEF{6.7} implies \(y \notin S^{\perp}\).
But we've already said \(y \in S^{\perp}\)! A contradiction.
So \(x\) must in \((S^{\perp})^{\perp}\).

\item By part(b), we have \(\W \subseteq (\W^{\perp})^{\perp}\), so we only need to show \((\W^{\perp})^{\perp} \subseteq \W\).
From \EXEC{6.2.6}, if \(x \notin \W\), then there exists \(y \in \W^{\perp}\) but \(\LG x, y \RG \ne 0\).
In particular, there exists \(y \in \W^{\perp}\) such that \(\LG x, y \RG \ne 0\), which again by \DEF{6.7} implies \(x \notin (\W^{\perp})^{\perp}\).

The point is we have derived the statement: if \(x \notin \W\), then \(x \notin (\W^{\perp})^{\perp}\).
And the equivalent \emph{contrapositive} is: if \(x \in (\W^{\perp})^{\perp}\), then \(x \in \W\), which is what we want.

\item This has already been proved in \EXEC{6.2.10}.
\end{enumerate}
\end{proof}

\begin{exercise} \label{exercise 6.2.14}
Let \(\W_1\) and \(\W_2\) be subspaces of a \emph{finite}-dimensional inner product space.
Prove that \((\W_1 + \W_2)^{\perp} = \W_1^{\perp} \cap \W_2^{\perp}\) and \((\W_1 \cap \W_2)^{\perp} = \W_1^{\perp} + \W_2^{\perp}\).
\emph{Hint for the second equation}: Apply \EXEC{6.2.13}(c) to the first equation.
\end{exercise}

\begin{proof} \ 

\((\W_1 + \W_2)^{\perp} \subseteq \W_1^{\perp} \cap \W_2^{\perp}\):
We have
\begin{align*}
    & x \in (\W_1 + \W_2)^{\perp} \\
    \implies & \LG x, y \RG = 0 \quad \forall\ y \in \W_1 + \W_2 & \text{by \DEF{6.7}} \\
    \implies & \LG x, w_1 + w_2 \RG = 0 \quad \forall\ w_1 \in \W_1 \land \forall w_2 \in \W_2 & \text{by def of sum} \\
    \implies & \LG x, w_1 \RG + \LG x, w_2 \RG = 0 \quad \forall\ w_1 \in \W_1 \land \forall w_2 \in \W_2 & \text{by \THM{6.1}(a)}
\end{align*}
In particular, \(\LG x, w_1 \RG + \LG x, \RED{\OV} \RG = 0 \quad \forall w_1 \in \W_1\), that is, \(\LG x, w_1 \RG = 0 \quad \forall w_1 \in \W_1\).
And by \DEF{6.7} again, \(x \in \W_1^{\perp}\).
Similarly, \(\LG x, w_2 \RG = 0 \quad \forall w_2 \in \W_2\), hence \(x \in \W_2^{\perp}\).
So \(x \in \W_1^{\perp} \cap \W_2^{\perp}\), as desired.

\(\W_1^{\perp} \cap \W_2^{\perp} \subseteq (\W_1 + \W_2)^{\perp}\):
Suppose \(x \in \W_1^{\perp} \cap \W_2^{\perp}\).
In particular \(x \in \W_1^{\perp}\) and \(x \in \W_2^{\perp}\), hence by \DEF{6.7}, \(\LG x, w \RG = 0\) for all \(w \in \W_1\) \MAROON{(b.1)} and \(\LG x, w' \RG = 0\) for all \(w' \in \W_2\). \MAROON{(b.2)}
Then for any \(w_1 + w_2 \in \W_1 + \W_2\) where \(w_1 \in \W_1\) and \(w_2 \in \W_2\),
\begin{align*}
    \LG x, v \RG & = \LG x, w_1 + w_2 \RG \\
        & = \LG x, w_1 \RG + \LG x, w_2 \RG & \text{by \THM{6.1}(a)} \\
        & = 0 + \LG x, w_2 \RG & \text{by \MAROON{(b.1)}} \\
        & = 0 + 0 = 0 & \text{by \MAROON{(b.2)}}
\end{align*}
Hence by \DEF{6.7}, \(x \in (\W_1 + \W_2)^{\perp}\), as desired.

For the second equality, we have
\begin{align*}
    (\W_1 \cap \W_2)^{\perp} & = (\MAROON{(\W_1^{\perp})^{\perp}} \cap \RED{(\W_2^{\perp})^{\perp}})^{\perp} & \text{by \EXEC{6.2.13}(c), although nasty} \\
    & = (\MAROON{(\W_1^{\perp})^{\perp}})^{\perp} + (\RED{(\W_2^{\perp})^{\perp}})^{\perp} & \text{by applying the first equality, although nasty} \\
    & = \W_1^{\perp} + \W_2^{\perp} & \text{by \EXEC{6.2.13}(c) again}
\end{align*}
\end{proof}

\begin{exercise} \label{exercise 6.2.15}
Let \(\V\) be a \emph{finite}-dimensional inner product space over \(F\).
\begin{enumerate}
\item \emph{Parseval's Identity}.
Let \(\{ v_1, v_2, ..., v_n \}\) be an \emph{orthonormal} basis for \(\V\).
For any \(x, y \in \V\) prove that
\[
    \LG x, y \RG = \sum_{i = 1}^n \LG x, v_i \RG \conjugatet{\LG y, v_i \RG}.
\]
\item Use (a) to prove that if \(\beta\) is an orthonormal basis for \(\V\) with inner product \(\InnerOp\), then for any \(x, y \in \V\)
\[
    \LG \phi_{\beta}(x), \phi_{\beta}(y) \RG' = \LG [x]_{\beta}, [y]_{\beta} \RG' = \LG x, y \RG,
\]
where \(\InnerOp'\) is the \emph{standard} inner product on \(F^n\).
\end{enumerate}
\end{exercise}

\begin{proof} \ 

\begin{enumerate}
\item We have
\begin{align*}
    \LG x, y \RG & = \LG \sum_{i = 1}^n \LG x, v_i \RG v_i, \sum_{j = 1}^n \LG y, v_j \RG v_j \RG & \text{since \(v_i\)'s is orthonormal} \\
    & = \sum_{i = 1}^n \LG x, v_i \RG \LG v_i, \sum_{j = 1}^n \LG y, v_j \RG v_j \RG & \text{by \DEF{6.1}(a)(b)} \\
    & = \sum_{i = 1}^n \LG x, v_i \RG \sum_{j = 1}^n \conjugatet{\LG y, v_j \RG} \LG v_i, v_j \RG & \text{by \THM{6.1}(a)(b)} \\
    & = \sum_{i = 1}^n \sum_{j = 1}^n \LG x, v_i \RG  \conjugatet{\LG y, v_j \RG} \LG v_i, v_j \RG & \text{move in ``constant''} \\
    & = \sum_{i = 1}^n \LG x, v_i \RG  \conjugatet{\LG y, v_{\RED{i}} \RG} \LG v_i, v_{\RED{i}} \RG & \text{since \(v_i\)'s is orthonormal} \\
    & = \sum_{i = 1}^n \LG x, v_i \RG \conjugatet{\LG y, v_i \RG} \cdot 1 & \text{again since \(v_i\)'s is orthonormal} \\
    & = \sum_{i = 1}^n \LG x, v_i \RG \conjugatet{\LG y, v_i \RG} & \text{of course}
\end{align*}

\item We have
\begin{align*}
    \LG x, y \RG & = \sum_{i = 1}^n \LG x, v_i \RG \conjugatet{\LG y, v_i \RG} \\
    & \quad \quad \text{(by part(a))} \\
    & = \LG \left( \LG x, v_1 \RG, \LG x, v_2 \RG, ..., \LG x, v_n \RG \right), \left(\LG y, v_1 \RG, \LG y, v_2 \RG, ..., \LG y, v_n \RG\right) \RG\RED{'} \\
    & \quad \quad \text{(by definition of standard inner product)} \\
    & = \LG [x]_{\beta}, [y]_{\beta} \RG\RED{'} \\
    & \quad \quad \text{(since \(\beta\) is orthonormal, and by \THM{6.5})} \\
    & = \LG \phi_{\beta}(x), \phi_{\beta}(y) \RG\RED{'}. \\
    & \quad \quad \text{(by def of \(\phi_{\beta}\))}
\end{align*}
\end{enumerate}
\end{proof}

\begin{exercise} \label{exercise 6.2.16} \ 

\begin{enumerate}
\item \emph{Bessel's Inequality}.
Let \(\V\) be an inner product space, and let \(S = \{ v_1, v_2, ..., v_n \}\) be an orthonormal subset of \(\V\).
Prove that for any \(x \in \V\) we have
\[
    \norm{x}^2 \ge \sum_{i = 1}^n \abs{\LG x, v_i \RG}^2.
\]
Hint: Apply \THM{6.6} to \(x \in \V\) and \(\W = \spann(S)\).
Then use \EXEC{6.1.10}.
\item In the context of (a), prove that Bessel's inequality is an equality if and only if \(x \in \spann(S)\).
\end{enumerate}
\end{exercise}

\begin{note}
For (a), 不等式左邊就是向量的長度的平方,不等式右邊就是向量在\ \(\W = \spann(S)\) 的投影的長度的平方。
\end{note}

\begin{proof} \ 

\begin{enumerate}
\item Let arbitrary \(x \in \V\), and let \(\W = \spann(S)\).
By \THM{6.6}, we have the unique combination \(x = u + z\) where \(u \in \W\) and \(z \in \W^{\perp}\).
Furthermore, by \THM{6.6}, we have
\[
    u = \sum_{i = 1}^n \LG x, v_i \RG v_i. \quad \quad \MAROON{(1)}
\]
And in particular, \(u\) and \(z\) are perpendicular, so by \EXEC{6.1.10},
\[
    \norm{x}^2 = \norm{u + z}^2 = \norm{u}^2 + \norm{z}^2. \quad \quad \MAROON{(2)}
\]
Hence we have
\begin{align*}
    \norm{x}^2 & = \norm{u}^2 + \norm{z}^2 & \text{by \MAROON{(2)}} \\
    & \ \RED{\ge} \norm{u}^2 & \text{by \THM{6.2}(b)} \\
    & = \LG u, u \RG & \text{by \DEF{6.3}} \\
    & = \LG \sum_{i = 1}^n \LG x, v_i \RG v_i, \sum_{j = 1}^n \LG x, v_j \RG v_j \RG & \text{by \MAROON{(1)}} \\
    & = \sum_{i = 1}^n \LG x, v_i \RG \LG v_i, \sum_{j = 1}^n \LG x, v_j \RG v_j \RG & \text{by \DEF{6.1}(a)(b)} \\
    & = \sum_{i = 1}^n \sum_{j = 1}^n \LG x, v_i \RG \conjugatet{\LG x, v_j \RG} \LG v_i, v_j \RG & \text{by \THM{6.1}(a)(b)} \\
    & = \sum_{i = 1}^n \LG x, v_i \RG \conjugatet{\LG x, v_{\RED{i}} \RG} \LG v_i, v_{\RED{i}} \RG & \text{since \(S\) is orthonormal} \\
    & = \sum_{i = 1}^n \LG x, v_i \RG \conjugatet{\LG x, v_i \RG} \cdot 1 = \sum_{i = 1}^n \LG x, v_i \RG \conjugatet{\LG x, v_i \RG} & \text{again since \(S\) is orthonormal} \\
    & = \sum_{i = 1}^n \abs{\LG x, v_i \RG}^2 & \text{by \RMK{d.5}}
\end{align*}

\item
From part(a), we have \(\norm{x}^2 = \norm{u}^2 + \norm{z}^2 \ge \norm{u}^2 = \sum_{i = 1}^n \abs{\LG x, v_i \RG}^2\), where \(x = u + z\), \(u \in \W\), and \(z \in \W^{\perp}\).
And that inequality becomes equality, if and only if \(\norm{z}^2 = 0\), if and only if (by \THM{6.2}(b)) \(z = \OV\), if and only if \(x = u + z = u + \OV = u \in \W\), if and only if \(x \in \W = \spann(S)\).
\end{enumerate}
\end{proof}

\begin{exercise} \label{exercise 6.2.17}
Let \(\T\) be a linear operator on an inner product space \(\V\).
If \(\LG \T(x), y \RG = 0\) for all \(x, y \in \V\), prove that \(\T = \TZERO\).
In fact, prove this result if the equality holds for all \(x\) and \(y\) in some \emph{basis} for \(\V\).
\end{exercise}

\begin{proof}
Suppose \(\LG \T(x), y \RG = 0\) for all \(x, y \in \V\).
Then in particular let \(y = \T(x)\), we have \(\LG \T(x), \T(x) \RG = 0\) for all \(x \in \V\).
Then by \THM{6.1}(d), \(\T(x) = \OV\) for all \(x \in \V\), which implies \(\T = \TZERO\).

Now let \(\beta\) be a basis for \(\V\).
And suppose that \(\LG \T(v), w \RG = 0\) for all \(v, w \in \beta\).
Using the similar proof in the previous case, we will get \(\LG \T(x), \T(x) \RG = 0\) for all \(x \in \beta\), hence \(\T(x) = \OV\) for all \(x \in \beta\).
Then \(\T\) sends all vectors of a basis to zero vector; by (infinite-dimensional version of) \THM{2.6} (or \EXEC{2.1.41}), \(\T = \TZERO\).
\end{proof}

\begin{exercise} \label{exercise 6.2.18}
Let \(\V = \CONT([-1, 1])\).
Suppose that \(\W_e\) and \(\W_o\) denote the subspaces of \(\V\) consisting of the \emph{even} and \emph{odd} functions, respectively.
Prove that \(\W_e^{\perp} = \W_o\), where the inner product on \(\V\) is defined by
\[
    \LG f, g \RG = \int_{-1}^1 f(t) g(t) dt.
\]
\end{exercise}

\begin{proof}
We show the statement by showing \(\W_o \subseteq \W_e^{\perp}\) and \(\W_e^{\perp} \subseteq \W_o\).

Note that the integral which has symmetric interval \([-1, 1]\) has the value zero if the integrand is an odd function.

So first, suppose \(f \in \W_o\).
But given any \(g \in \W_e\), by definition \(g\) is even function.
And if we define \(f \cdot g\) as \(f \cdot g (t) = f(t)g(t)\), then
\begin{align*}
    f \cdot g(-t) & = f(-t)g(-t) \\
        & = -f(t)g(-t) & \text{since \(f\) is odd function} \\
        & = -f(t)g(t) & \text{since \(g\) is even function} \\
        & = - f \cdot g(t).
\end{align*}
Hence \(f \cdot g\) is an odd function, which implies
\[
    \int_{-1}^1 f(t)g(t) dt = 0.
\]
Hence by \DEF{6.7}, \(f \in \W_e^{\perp}\), hence \(\W_o \subseteq \W_e^{\perp}\).

Before we show the other set containment, notice that every function \(g\) can be written as \(g = g_1 + g_2\) where
\[
    g_1(t) = \frac{1}{2} (g(t) + g(-t)) \quad \text{ and } \quad g_2(t) = \frac{1}{2} (g(t) - g(-t))
\]
and it can be showed that \(g_1\) is an even function and \(g_2\) is an odd function:
\begin{align*}
    g_1(-t) & = \frac{1}{2} (g(-t) + g(-(-t))) \\
        & = \frac{1}{2} (g(-t) + g(t))) = \frac{1}{2} (g(t) + g(-t)) = g_1(t),
\end{align*}
and
\begin{align*}
    g_2(-t) & = \frac{1}{2} (g(-t) - g(-(-t))) \\
        & = \frac{1}{2} (g(-t) - g(t))) = \frac{1}{2} (-g(t) + g(-t)) = -\frac{1}{2} (g(t) - g(-t)) = -g_2(t),
\end{align*}

Now suppose \(g \in \W_e^{\perp}\).
Then by definition, \(\LG g, f \RG = 0\) for all \(f \in \W_e\), that is, for all even function \(f\).
In particular, we let \(g = g_1 + g_2\) where \(g_1, g_2\) are the even and odd functions discussed above.
Also, since \(g_1\) is an even function, in particular \(\LG g, g_1 \RG = 0\).
And we have
\begin{align*}
    0 = \LG g, g_1 \RG & = \LG g_1 + g_2, g_1 \RG \\
        & = \LG g_1, g_1 \RG + \LG g_2, g_1 \RG & \text{by \DEF{6.1}(a)} \\
        & = \LG g_1, g_1 \RG + 0 & \text{since \(g_1\) is even and \(g_2\) is odd,} \\
        & & \text{we have shown that \(g_2(t)g_1(t)\) is an odd function} \\
        & = \norm{g_1}^2, & \text{by \DEF{6.3}}
\end{align*}
which by \THM{6.2}(b) implies \(g_1\) is zero function, hence \(g = g_1 + g_2 = 0 + g_2 = g_2\), an odd function, and by definition \(g_1 \in \W_o\).
Hence \(\W_e^{\perp} \in \W_o\).
\end{proof}

\begin{exercise} \label{exercise 6.2.19}
In each of the following parts, find the orthogonal projection of the given vector on the given subspace \(\W\) of the inner product space \(\V\).
\begin{enumerate}
\item \(\V = \SET{R}^2, u = (2, 6)\), and \(\W = \{ (x, y): y = 4x \}\).
\item \(\V = \SET{R}^3, u = (2, 1, 3)\), and \(\W = \{ (x, y, z): x + 3y - 2z = 0 \}\).
\item \(\V = \POLYRINF\) with the inner product \(\LG f(x), g(x) \RG = \int_{\RED{0}}^1 f(t) g(t) dt, h(x) = 4 + 3x - 2x^2\), and \(\W = \POLYR\).
\end{enumerate}
\end{exercise}

\begin{proof}
We skip the process of calculating the orthonormal basis for \(\W\) in each item.
\begin{enumerate}
\item By calculation, \(\beta = \{ \sqrt{\frac{1}{17}} (1, 4) \}\) is an orthonormal basis for \(\W\).
And by \THM{6.6}, the orthogonal projection of \(u\) on \(\W\) is
\[
    \LG u, \frac{1}{\sqrt{17}} (1, 4) \RG \cdot \sqrt{\frac{1}{17}} (1, 4) = \frac{26}{\sqrt{17}} \sqrt{\frac{1}{17}} (1, 4). = \frac{26}{17} (1, 4).
\]

\item By calculation, \(\beta =  \left\{ \frac{1}{\sqrt{10}}(-3,1,0), \frac{1}{\sqrt{35}}(1,3,5) \right\}\) is a basis for \(\W\).
Thus by \THM{6.6}, the orthogonal projection of \(u\) is
\begin{align*}
    & \LG u, \frac{1}{\sqrt{10}}(-3,1,0)\RG \frac{1}{\sqrt{10}}(-3,1,0)
    + \LG u, \frac{1}{\sqrt{35}}(1,3,5)\RG \frac{1}{\sqrt{35}}(1,3,5) \\
    & = -\frac{1}{2}(-3,1,0) + \frac{4}{7}(1,3,5) = \frac{1}{14}(29,17,40)
\end{align*}

\item Skip, tedious, \EXAMPLE{6.2.5} is enough.
\end{enumerate}
\end{proof}

\begin{exercise} \label{exercise 6.2.20}
In each part of \EXEC{6.2.19}, find the distance from the given vector to the subspace \(\W\).
\end{exercise}

\begin{proof}
By \THM{6.6}, the distance is \(\norm{u - u'}\) where \(u'\) is the orthogonal projection of \(u\) on \(\W\).
\begin{enumerate}
\item We have \(u - u' = (2, 6) - \frac{26}{17}(1, 4) = \left( \frac{8}{17}, -\frac{2}{17} \right)\).
\begin{align*}
    \norm{u - u'}^2 & = \LG u - u', u - u' \RG \\
        & = \LG \left( \frac{8}{17}, -\frac{2}{17} \right), \left( \frac{8}{17}, -\frac{2}{17} \right) \RG \\
        & = \frac{68}{289}
\end{align*}
hence \(\norm{u - u'} = 2 \frac{\sqrt{17}}{17}\).

\item We have \(u - u' = (2, 1, 3) - \frac{1}{14}(29, 17, 40) = \left( \frac{8}{17}, -\frac{2}{17} \right)\).
\begin{align*}
    \norm{u - u'}^2 & = \LG u - u', u - u' \RG \\
        & = ... = \frac{1}{14}
\end{align*}
hence \(\norm{u - u'} = \frac{1}{\sqrt{14}}\).

\item Skip.
\end{enumerate}
\end{proof}

\begin{exercise} \label{exercise 6.2.21}
Let \(\V = \CONT([-1, 1])\) with the inner product \(\LG f,g \RG = \int_{-1}^1 f(t)g(t) dt\), and let \(\W\) be the subspace \(\POLYRR\), viewed as a space \emph{of functions}.
(See the difference between polynomial and polynomial \emph{functions} in \RMK{e.7}.)
Use the orthonormal basis obtained in \EXAMPLE{6.2.5} to compute the ``best'' (closest) second-degree polynomial approximation of the function \(h(t) = e^t\) on the interval \([-1, 1]\).
\end{exercise}

\begin{proof} \ 
By \EXAMPLE{6.2.5}, we have
\[
    \beta = \{ u_1, u_2, u_3 \} =
    \left\{
        \frac{1}{\sqrt{2}},
        \sqrt{\frac{3}{2}} x,
        \sqrt{\frac{5}{8}}(3x^2 - 1)
    \right\}
\]
as an orthonormal basis for the subspace \(\POLYRR\).
So by \THM{6.6}, the closest approximation is the orthogonal projection of \(h(t) = e^t\) on \(\POLYRR\), that is,
\begin{align*}
    & \LG u_1, h \RG h + \LG u_2, h \RG h + \LG u_3, h \RG h \\
    & = ... \quad \quad \text{by tedious integral calculation, skip} \\
    & = \frac{1}{4} \left[\left(15 e - 105 e^{-1}\right) t^2 + 12t - 3e^{2} + 33 \right]
\end{align*}
\end{proof}

\begin{exercise} \label{exercise 6.2.22}
Let \(\V = \CONT([0, 1])\) with the inner product \(\LG f, g \RG = \int_0^1 f(t)g(t) dt\).
Let \(\W\) be the subspace spanned by the \LID{} set \(\{ t, \sqrt{t} \}\).
\begin{enumerate}
\item Find an orthonormal basis for \(\W\).
\item Let \(h(t) = t^2\).
Use the orthonormal basis obtained in (a) to obtain the ``best'' (closest) approximation of \(h\) in \(\W\).
\end{enumerate}
\end{exercise}

\begin{proof}
Skip. Similar to previous exercise.
\end{proof}

\begin{exercise} \label{exercise 6.2.23}
Let \(\V\) be the vector space defined in \EXAMPLE{1.2.5}, the space of all \emph{sequences} \(\sigma\) in \(F\) (where \(F = \SET{R}\) or \(F = \SET{C})\) such that \(\sigma(n) \ne 0\) for \emph{only finitely many} positive integers \(n\).
For \(\sigma, \mu \in \V\), we define \(\LG \sigma, \mu \RG = \sum_{n = 1}^{\infty} \sigma(n) \conjugatet{\mu(n)}\).
Since all but a finite number of terms of the series are zero, the series converges.
\begin{enumerate}
\item Prove that \(\InnerOp\) is an inner product on \(\V\), and hence \(\V\) is an inner product space.
\item For each positive integer \(n\), let \(e_n\) be the sequence defined by \(e_n(k) = \delta_{nk}\), where \(\delta_{nk}\) is the Kronecker delta.
Prove that \(\{ e_1, e_2, ... \}\) is an orthonormal basis for \(\V\).
\item Let \(\sigma_n = e_1 + e_n\) and \(\W = \spann(\{ \sigma_n : n \ge 2 \})\).
    \begin{enumerate}
    \item[(i)] Prove that \(e_1 \notin \W\), so \(\W \ne \V\).
    \item[(ii)] Prove that \(\W^{\perp} = \{ \OV \}\), and conclude that \(\W \ne (\W^{\perp})^{\perp}\).
    Thus the assumption in \EXEC{6.2.13}(c) that \(\W\) is \emph{finite}-dimensional is essential.
    \end{enumerate}
\end{enumerate}
\end{exercise}

\begin{proof}
\end{proof}
