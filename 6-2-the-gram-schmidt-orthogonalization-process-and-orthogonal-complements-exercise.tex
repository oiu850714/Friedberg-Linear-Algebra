\exercisesection

\begin{exercise} \label{exercise 6.2.1}
Label the following statements as true or false.
\begin{enumerate}
\item The Gram-Schmidt orthogonalization process produces an orthonormal set from an arbitrary \emph{\LID{}} set.
\item Every nonzero finite-dimensional inner product space has an orthonormal basis.
\item The orthogonal complement of any set is a subspace.
\item If \(\{ v_1, v_2, ..., v_n \}\) is a basis for an inner product space \(\V\), then for any \(x \in \V\) the scalars \(\LG x, v_i \RG\) are the Fourier coefficients of \(x\).
\item An orthonormal basis must be an ordered basis.
\item Every orthogonal set is linearly independent.
\item Every orthonormal set is linearly independent.
\end{enumerate}
process 那提很尷尬;gen 出來的嚴格來說是 orthogonal,但是把 process 多一步做 normalize 也可以吧
Fourier coefficients 那個,basis 要 orthonormal ==
\end{exercise}

\begin{proof}
\end{proof}

\begin{exercise} \label{exercise 6.2.2}
In each part, apply the Gram-Schmidt process to the given subset \(S\) of the inner product space \(\V\) to obtain an orthogonal basis for \(\spann(S)\).
Then normalize the vectors in this basis to obtain an orthonormal basis \(\beta\) for \(\spann(S)\), and compute the Fourier coefficients of the given vector relative to \(\beta\).
Finally, use \THM{6.5} to verify your result.

\begin{enumerate}
\item \(\V = \SET{R}^3, S = \{ (1, 0, 1), (0, 1, 1), (1, 3, 3) \}\), and \(x = (1, 1, 2)\)
\item \(\V = \SET{R}^3, S = \{ (1, 1, 1), (0, 1, 1), (0, 0, 1) \}\), and \(x = (1, 0, 1)\)
\item \(\V = \POLYRR\) with the inner product \(\LG f(x), g(x) \RG = \int_0^1 f(t) g(t) dt, S = \{1, x, x^2 \}\), and \(h(x) = 1 + x\)
\item \(\V = \spann(S)\), where \(S = \{(1, \iu, 0), (1 - \iu, 2, 4\iu) \}\), and \(x = (3 + \iu, 4\iu, -4)\)
\item \(\V = \SET{R}^4, S = \{ (2, -1, -2, 4), (-2, 1, -5, 5), (-1, 3, 7, 11) \}\), and \(x = (-11, 8, -4, 18)\)
\item \(\V = \SET{R}^4, S = \{ (1, -2, -1, 3), (3, 6, 3, -1), (1, 4, 2, 8) \}\), and \(x = (-1, 2, 1, 1)\)
\item \(\V = M_{2 \X 2}(\SET{R}),
    S = \left\{
        \begin{pmatrix}3 & 5 \\ -1 & 1\end{pmatrix},
        \begin{pmatrix}-1 & 9 \\ 5 & -1\end{pmatrix},
        \begin{pmatrix}7 & -17 \\ 2 & -6\end{pmatrix}\right
    \}\),
    and \(A = \begin{pmatrix}-1 & 27 \\ -4 & 8\end{pmatrix}\)
\item \(\V = M_{2 \X 2}(\SET{R}),
    S = \left\{
        \begin{pmatrix}2 & 2 \\ 2 & 1\end{pmatrix},
        \begin{pmatrix}11 & 4 \\ 2 & 5\end{pmatrix},
        \begin{pmatrix}4 & -12 \\ 3 & -16\end{pmatrix}
    \right\}\),
    and \(A = \begin{pmatrix}8 & 6 \\ 25 & -13\end{pmatrix}\)
\item \(\V = \spann(S)\) with the inner product \(\LG f, g \RG = \int_0^{\pi} f(t)g(t) dt, S = \{\sin t, \cos t, 1, t\}\), and \(h(t) = 2t + 1\)
\item \(\V = \SET{C}^4,
    S = \{ (1, i, 2 - \iu, -1), (2 + 3\iu, 3\iu, 1 - \iu, 2\iu), (-1 + 7\iu, 6 + 10\iu, 11 - 4\iu, 3 + 4\iu)\}\),
    and \(x = (-2 + 7\iu, 6 + 9\iu, 9 - 3\iu, 4 + 4\iu)\)
\item \(\V = \SET{C}^4, S = \{(-4, 3 - 2\iu, \iu, 1 - 4\iu), (-1 - 5\iu, 5 - 4\iu, -3 + 5\iu, 7 - 2\iu), (-27 - \iu, -7 - 6\iu, -15 + 25\iu, -7 - 6\iu)\}\), and \(x = (-13 - 7\iu, -12 + 3\iu, -39 - 11\iu, -26 + 5\iu)\)
\item \(\V = M_{2 \X 2}(\SET{C})\),
\[
    S = \left\{
        \begin{pmatrix} 1 - \iu & -2 - 3\iu \\ 2 + 2\iu & 4 + \iu \end{pmatrix},
        \begin{pmatrix} 8\iu & 4 \\ -3 - 3\iu & -4 + 4\iu \end{pmatrix},
        \begin{pmatrix} -25 - 38\iu & -2 - 13\iu \\ 12 - 78\iu & -7 + 24\iu \end{pmatrix}
    \right\},
\]
and \(A=\begin{pmatrix} -2 + 8\iu & -13 + \iu \\ 10 - 10\iu & 9 - 9\iu \end{pmatrix}\)
\item \(\V = M_{2 \X 2}(\SET{C})\),
\[
    S = \left\{
        \begin{pmatrix} -1 + \iu & -\iu \\ 2 - \iu & 1 + 3\iu\end{pmatrix},
        \begin{pmatrix} -1 - 7\iu & -9 - 8\iu \\ 1 + 10\iu & -6 - 2\iu\end{pmatrix},
        \begin{pmatrix} -11 - 132\iu & -34 -31\iu \\ 7 - 126\iu & -71 - 5\iu \end{pmatrix}
    \right\},
\]
and \(A=\begin{pmatrix} -7 + 5\iu & 3 + 18\iu \\ 9 - 6\iu & -3 + 7\iu \end{pmatrix}\)
\end{enumerate}
\end{exercise}

\begin{proof}
手寫吧... 跳過
\end{proof}

\begin{exercise} \label{exercise 6.2.3}
In \(\SET{R}^2\), let
\[
    \beta = \left\{
        \left( \frac{1}{\sqrt{2}}, \frac{1}{\sqrt{2}} \right), \left( \frac{1}{\sqrt{2}}, \frac{-1}{\sqrt{2}} \right)
    \right\}
\]
Find the Fourier coefficients of \((3, 4)\) relative to \(\beta\).
\end{exercise}

\begin{proof}
\end{proof}

\begin{exercise} \label{exercise 6.2.4}
Let \(S = \{ (1, 0, \iu), (1, 2, 1) \}\) in \(\SET{C}^3\). Compute \(S^{\perp}\).
\end{exercise}

\begin{proof}
\end{proof}

\begin{exercise} \label{exercise 6.2.5}
Let \(S_0 = \{ x_0 \}\), where \(x_0\) is a nonzero vector in \(\SET{R}^3\).
Describe \(S_0^{\perp}\) geometrically.
Now suppose that \(S = \{ x_1, x_2 \}\) is a \LID{} subset of \(\SET{R}^3\).
Describe \(S^{\perp}\) geometrically.
\end{exercise}

\begin{proof}
\end{proof}

\begin{exercise} \label{exercise 6.2.6}
Let \(\V\) be an inner product space, and let \(\W\) be a \emph{finite}-dimensional subspace of \(\V\).
If \(x \notin \W\), prove that there \emph{exists} \(y \in \V\) such that \(y \in W^{\perp}\), but \(\LG x, y \RG \ne 0\).
Hint: Use \THM{6.6}.
\end{exercise}

\begin{proof}
\end{proof}

\begin{exercise} \label{exercise 6.2.7}
Let \(\beta\) be a basis for a subspace \(\W\) of an inner product space \(\V\), and let \(z \in \V\).
Prove that \(z \in \W^{\perp}\) if and only if \(\LG z, v \RG = 0\) for every \(v \in \beta\).
\end{exercise}

\begin{note}
又一題,只要檢查 basis 行為就可決定整體行為的。
\end{note}

\begin{proof}
\end{proof}

\begin{exercise} \label{exercise 6.2.8}
Prove that if \(\{ w_1, w_2, ..., w_n \}\) is an \emph{orthogonal} set of nonzero vectors, then the vectors \(v_1, v_2, ..., v_n\) derived from the Gram-Schmidt process \emph{satisfy} \(v_i = w_i\) for \(i = 1, 2, ..., n\).
Hint: Use mathematical induction.
\end{exercise}

\begin{proof}
\end{proof}

\begin{exercise} \label{exercise 6.2.9}
Let \(\W = \spann(\{ ( \iu, 0, 1) \})\) in \(\SET{C}^3\).
Find orthonormal bases for \(\W\) and \(\W^{\perp}\).
\end{exercise}

\begin{proof}
\end{proof}

\begin{exercise} \label{exercise 6.2.10}
Let \(\W\) be a \emph{finite}-dimensional subspace of an inner product space \(\V\).
Prove that \(\V = \W \oplus \W^{\perp}\).
Using the \ADEF{2.2}, prove that there exists a projection \(\T\) on \(\W\) along \(\W^{\perp}\) that satisfies \(\NULLT = \W^{\perp}\).
In addition, prove that \(\norm{\T(x)} \le \norm{x}\) for all \(x \in \V\).
Hint: Use \THM{6.6} and \EXEC{6.1.10}.
\end{exercise}

\begin{proof}
這感覺用 1.6 3x 題 dim v = dim w + dim w perp 可以吧?
\end{proof}

\begin{exercise} \label{exercise 6.2.11}
Let \(A\) be an \(n \X n\) matrix with \emph{complex} entries.
Prove that \(A A^* = I\) if and only if the \emph{rows} of \(A\) form an \emph{orthonormal} basis for \(\SET{C}^n\).
\end{exercise}

\begin{proof}
\end{proof}

\begin{exercise} \label{exercise 6.2.12}
Prove that for any matrix \(A \in M_{m \X n}(F)\), \((\RANGE(\LMTRAN_{A^*}))^{\perp} = \NULL(\LMTRAN_A)\).
\end{exercise}

\begin{note}
幹,怕
A 的零空間會跟 A 的共軛轉置矩陣的 range 的正交補集一樣
\end{note}

\begin{proof}
\end{proof}

\begin{exercise} \label{exercise 6.2.13}
Let \(\V\) be an inner product space, \(S\) and \(S_0\) be \emph{subsets} of \(\V\), and \(\W\) be a finite-dimensional \emph{subspace} of \(\V\).
Prove the following results.
\begin{enumerate}
\item \(S_0 \subseteq S\) implies that \(S^{\perp} \subseteq S_0^{\perp}\).
\item \(S \subseteq (S^{\perp})^{\perp}\); so \(\spann(S) \subseteq (S^{\perp})^{\perp}\).
\item \(\W = (\W^{\perp})^{\perp}\). Hint: Use \EXEC{6.2.6}.
\item \(\V = \W \oplus \W^{\perp}\). (See the exercises of \SEC{1.3}.)
\end{enumerate}
\end{exercise}

\begin{proof}
(d) 在銃殺小? 第十題不就有了?
\end{proof}

\begin{exercise} \label{exercise 6.2.14}
Let \(\W_1\) and \(\W_2\) be subspaces of a \emph{finite}-dimensional inner product space.
Prove that \((\W_1 + \W_2)^{\perp} = \W_1^{\perp} \cap \W_2^{\perp}\) and \((\W_1 \cap \W_2)^{\perp} = \W_1^{\perp} + \W_2^{\perp}\).
Hint for the second equation: Apply \EXEC{6.2.13}(c) to the first equation.
\end{exercise}

\begin{proof}
\end{proof}

\begin{exercise} \label{exercise 6.2.15}
Let \(\V\) be a \emph{finite}-dimensional inner product space over \(F\).
\begin{enumerate}
\item \emph{Parseval's Identity}.
Let \(\{ v_1, v_2, ..., v_n \}\) be an orthonormal basis for \(\V\).
For any \(x, y \in \V\) prove that
\[
    \LG x, y \RG = \sum_{i = 1}^n \LG x, v_i \RG \conjugatet{\LG y, v_i \RG}.
\]
\item Use (a) to prove that if \(\beta\) is an orthonormal basis for \(\V\) with inner product \(\InnerOp\), then for any \(x, y \in \V\)
\[
    \LG \phi_{\beta}(x), \phi_{\beta}(y) \RG' = \LG [x]_{\beta}, [y]_{\beta} \RG' = \LG x, y \RG,
\]
where \(\InnerOp'\) is the \emph{standard} inner product on \(F^n\).
\end{enumerate}
\end{exercise}

\begin{proof}
\end{proof}

\begin{exercise} \label{exercise 6.2.16} \ 

\begin{enumerate}
\item \emph{Bessel's Inequality}.
Let \(\V\) be an inner product space, and let \(S = \{ v_1, v_2, ..., v_n \}\) be an orthonormal subset of \(\V\).
Prove that for any \(x \in \V\) we have
\[
    \norm{x}^2 \ge \sum_{i = 1}^n \abs{\LG x, v_i \RG}^2.
\]
Hint: Apply \THM{6.6} to \(x \in \V\) and \(\W = \spann(S)\).
Then use \EXEC{6.1.10}.
\item In the context of (a), prove that Bessel's inequality is an equality if and only if \(x \in \spann(S)\).
\end{enumerate}
\end{exercise}

\begin{proof}
\end{proof}

\begin{exercise} \label{exercise 6.2.17}
Let \(\T\) be a linear operator on an inner product space \(\V\).
If \(\LG \T(x), y \RG = 0\) for all \(x, y \in \V\), prove that \(\T = \TZERO\).
In fact, prove this result if the equality holds for all \(x\) and \(y\) in some \emph{basis} for \(\V\).
\end{exercise}

\begin{proof}
\end{proof}

\begin{exercise} \label{exercise 6.2.18}
Let \(\V = \CONT([-1, 1])\).
Suppose that \(\W_e\) and \(\W_o\) denote the subspaces of \(\V\) consisting of the \emph{even} and \emph{odd} functions, respectively.
Prove that \(\W_e^{\perp} = W_o\), where the inner product on \(\V\) is defined by
\[
    \LG f, g \RG = \int_{-1}^1 f(t) g(t) dt.
\]
\end{exercise}

\begin{proof}
\end{proof}

\begin{exercise} \label{exercise 6.2.19}
In each of the following parts, find the orthogonal projection of the given vector on the given subspace \(\W\) of the inner product space \(\V\).
\begin{enumerate}
\item \(\V = \SET{R}^2, u = (2, 6)\), and \(\W = \{ (x, y): y = 4x \}\).
\item \(\V = \SET{R}^3, u = (2, 1, 3)\), and \(\W = \{ (x, y, z): x + 3y - 2z = 0 \}\).
\item \(\V = \POLYRINF\) with the inner product \(\LG f(x), g(x) \RG = \int_0^1 f(t) g(t) dt, h(x) = 4 + 3x - 2x^2\), and \(\W = \POLYR\).
\end{enumerate}
\end{exercise}

\begin{proof}
\end{proof}

\begin{exercise} \label{exercise 6.2.20}
In each part of \EXEC{6.2.19}, find the distance from the given vector to the subspace \(\W\).
\end{exercise}

\begin{proof}
\end{proof}

\begin{exercise} \label{exercise 6.2.21}
Let \(\V = \CONT([-1, 1])\) with the inner product \(\LG f,g \RG = \int_{-1}^1 f(t)g(t) dt\), and let \(\W\) be the subspace \(\POLYRR\), viewed as a space \emph{of functions}.
(See the difference between polynomial and polynomial \emph{functions} in \RMK{e.7}.)
Use the orthonormal basis obtained in \EXAMPLE{6.2.5} to compute the ``best'' (closest) second-degree polynomial approximation of the function \(h(t) = e^t\) on the interval \([-1, 1)\).
\end{exercise}

\begin{proof}
\end{proof}

\begin{exercise} \label{exercise 6.2.22}
Let \(\V = \CONT([0, 1])\) with the inner product \(\LG f, g \RG = \int_0^1 f(t)g(t) dt\).
Let \(\W\) be the subspace spanned by the \LID{} set \(\{ t, \sqrt{t} \}\).
\begin{enumerate}
\item Find an orthonormal basis for \(\W\).
\item Let \(h(t) = t^2\).
Use the orthonormal basis obtained in (a) to obtain the ``best'' (closest) approximation of \(h\) in \(\W\).
\end{enumerate}
\end{exercise}

\begin{proof}
\end{proof}

\begin{exercise} \label{exercise 6.2.23}
Let \(\V\) be the vector space defined in \EXAMPLE{1.2.5}, the space of all \emph{sequences} \(\sigma\) in \(F\) (where \(F = \SET{R}\) or \(F = \SET{C})\) such that \(\sigma(n) \ne 0\) for \emph{only finitely many} positive integers \(n\).
For \(\sigma, \mu \in \V\), we define \(\LG \sigma, \mu \RG = \sum_{n = 1}^{\infty} \sigma(n) \conjugatet{\mu(n)}\).
Since all but a finite number of terms of the series are zero, the series converges.
\begin{enumerate}
\item Prove that \(\InnerOp\) is an inner product on \(\V\), and hence \(\V\) is an inner product space.
\item For each positive integer \(n\), let \(e_n\) be the sequence defined by \(e_n(k) = \delta_{nk}\), where \(\delta_{nk}\) is the Kronecker delta.
Prove that \(\{ e_1, e_2, ... \}\) is an orthonormal basis for \(\V\).
\item Let \(\sigma_n = e_1 + e_n\) and \(\W = \spann(\{ \sigma_n : n \ge 2 \})\).
    \begin{enumerate}
    \item[(i)] Prove that \(e_1 \notin \W\), so \(\W \ne \V\).
    \item[(ii)] Prove that \(\W^{\perp} = \{ \OV \}\), and conclude that \(\W \ne (\W^{\perp})^{\perp}\).
    Thus the assumption in \EXEC{6.2.13}(c) that \(\W\) is \emph{finite}-dimensional is essential.
    \end{enumerate}
\end{enumerate}
\end{exercise}

\begin{proof}
\end{proof}
