\exercisesection

\begin{exercise} \label{exercise 5.2.1}
Label the following statements as true or false.
\begin{enumerate}
\item Any linear operator on an \(n\)-dimensional vector space that has fewer than \(n\) distinct eigenvalues is not diagonalizable.
\item Two distinct eigenvectors corresponding to the same eigenvalue are always linearly dependent.
\item If \(\lambda\) is an eigenvalue of a linear operator \(\T\), then each vector in \(E_{\lambda}\) is an eigenvector of \(\T\).
\item If \(\lambda_1\) and \(\lambda_2\) are distinct eigenvalues of a linear operator \(\T\), then \(E_{\lambda_1} \cap E_{\lambda_2} = \{ \OV \}\).
\item Let \(A \in M_{n \X n}(F)\) and \(\beta = \{ v_1, v_2, ..., v_n \}\) be an ordered basis for \(F^n\) consisting of eigenvectors of \(A\).
If \(Q\) is the \(n \X n\) matrix whose \(j\)th column is \(v_j (1 \le j \le n)\), then \(Q^{-1} A Q\) is a diagonal matrix.
\item A linear operator \(\T\) on a finite-dimensional vector space is diagonalizable if and only if the algebraic multiplicity of each eigenvalue \(\lambda\) equals the dimension of \(E_{\lambda}\).
\item Every diagonalizable linear operator on a nonzero vector space has at least one eigenvalue.
\item If a vector space is the direct sum of subspaces \(\W_1, \W_2, ..., \W_k\), then \(\W_i \cap \W_j = \{ \OV \}\) for \(i \ne j\).

\item If
\[
    \V = \sum_{i = 1}^k W_i \quad \text{ and } \quad W_i \cap W_j = \{ \OV \} \text{ for } i \ne j,
\]
then \(\V = \W_1 \oplus \W_2 \oplus ... \oplus \W_k\).
\end{enumerate}
\end{exercise}

\begin{proof} \ 

\begin{enumerate}
\item False by \RMK{5.2.1}.
\item False. Again the identity map is a counterexample such that \(e_1, e_2\) correspond to the same eigenvalue \(\lambda = 1\) but are \LID{}.
\item False. Zero vector by definition is in \(E_{\lambda}\), but is not an eigenvector by definition.

\item True.
Suppose \(x \in E_{\lambda_1} \cap E_{\lambda_2}\).
Then by definition \(\T(x) = \lambda_1 x = \lambda_2 x\), which implies \((\lambda_1 - \lambda_2) x = \OV\).
But that implies \(x = \OV\), since \(\lambda_1 \ne \lambda_2\).
Hence \(E_{\lambda_1} \cap E_{\lambda_2} = \{ \OV \}\).

\item True by \CORO{5.1.1}(b).
\item True by \THM{5.8}(a).
\item True.
By \THM{5.6}, if \(\T\) is diagonalizable, then the \CPOLY{} splits, which implies the zeroes of the \CPOLY{} must exists.

\item True.
Given \(i, j\) such that \(i \ne j\), suppose \(x \in \W_i \cap \W_j\), we have to show \(x = \OV\).
Then in particular, \(x \in \W_i\) and \(x \in \W_j\).
And of course \(\RED{\frac{1}{2}x} \in \W_i\) and \(\BLUE{\frac{1}{2}x} \in \W_j\) since \(\W_i, \W_j\) are subspaces.
Similarly, \(\MAROON{\frac{1}{3}x} \in \W_i\) and \(\GREEN{\frac{2}{3}x} \in \W_j\).
Then we have
\begin{align*}
    x & = \RED{\frac{1}{2}x} + \BLUE{\frac{1}{2}x} \text{ where } \RED{\frac{1}{2}x} \in W_i \text{ and } \BLUE{\frac{1}{2}x} \in W_j \\
    x & = \MAROON{\frac{1}{3}x} + \GREEN{\frac{2}{3}x} \text{ where } \MAROON{\frac{1}{3}x} \in W_i \text{ and } \GREEN{\frac{1}{3}x} \in W_j
\end{align*}
But by the direct sum's equivalent condition, \THM{5.9}(c), the representation of \(x\) using \(\W_i\)'s vectors \emph{is unique}.
That implies \(\RED{\frac{1}{2}x} = \MAROON{\frac{1}{3}x}\) and \(\BLUE{\frac{1}{2}x} = \GREEN{\frac{2}{3}x}\), which trivially implies \(x = \OV\).

\item False.
Consider a counterexample such that \(\V = \SET{R}^3\) and
\begin{align*}
    W_1 & = \{ (x, y, 0) : x, y \in \SET{R} \} \\
    W_2 & = \{ (0, 0, z) : z \in \SET{R} \} \\
    W_3 & = \{ (0, y, y) : y \in \SET{R} \}
\end{align*}
Clearly, \(\W_1 \cap \W_2 = \W_1 \cap \W_3 = \W_2 \cap \W_3 = \{ \OV \}\).
But
\begin{align*}
    (0, 0, 0) & = (0, y, 0) + (0, 0, y) + (0, -y, -y) \\
    & \quad \quad \text{ where } (0, y, 0) \in W_1, (0, 0, y) \in W_2, (0, -y, -y) \in W_3, \text{ for \emph{all} } y \in \SET{R}.
\end{align*}
Hence \((0, 0, 0) \in \SET{R}^3\) has cannot be uniquely written as \(v_1 + v_2 + v_3\) where \(v_i \in \W_i\).
So the condition of \THM{5.9}(c) is false, hence equivalently by \THM{5.9}(a), \(\V \ne \W_1 \oplus \W_2 \oplus \W_3\).
\end{enumerate}
\end{proof}

\begin{note}
從\ \EXEC{5.2.1}(h) 可知,任一集合與剩下集合的\ sum 的交集為零空間,可推得任兩兩交集為零空間,不能推得。

但從\ \EXEC{5.2.1}(i) 可知\ the converse is false:任兩兩交集為零空間,不能推得任一集合與剩下集合的\ sum 的交集為零空間。
\end{note}

\begin{exercise} \label{exercise 5.2.2}
For each of the following matrices \(A \in M_{n \X n}(\SET{R})\), test \(A\) for diagonalizability, and if \(A\) is diagonalizable, find an invertible matrix \(Q\) and a diagonal matrix \(D\) such that \(Q^{-1}AQ = D\).

(a) \(\left(\begin{array}{ll}1 & 2 \\ 0 & 1\end{array}\right)\)
\quad \quad \quad
(b) \(\left(\begin{array}{ll}1 & 3 \\ 3 & 1\end{array}\right)\)
\quad \quad \quad
(c) \(\left(\begin{array}{ll}1 & 4 \\ 3 & 2\end{array}\right)\)

(d) \(\left(\begin{array}{lll}7 & -4 & 0 \\ 8 & -5 & 0 \\ 6 & -6 & 3\end{array}\right)\)
\quad
(e) \(\left(\begin{array}{rrr}0 & 0 & 1 \\ 1 & 0 & -1 \\ 0 & 1 & 1\end{array}\right)\)
\quad
(f) \(\left(\begin{array}{lll}1 & 1 & 0 \\ 0 & 1 & 2 \\ 0 & 0 & 3\end{array}\right)\)

(g) \(\left(\begin{array}{rrr}3 & 1 & 1 \\ 2 & 4 & 2 \\ -1 & -1 & 1\end{array}\right)\)
\end{exercise}

\begin{proof}
Use the 2 conditions in the subsection \ref{sec 5.2.1}.
\begin{enumerate}
\item The \CPOLY{} is
\[
    \det \begin{pmatrix} 1 - t & 2 \\ 0 & 1 - t \end{pmatrix} = (1 - t)^2,
\]
so the \CPOLY{} splits, and the only eigenvalue is \(\lambda = 1\), and it has algebraic multiplicity \(2\).
Now
\[
    E_{\lambda} = \NULL \begin{pmatrix} 1 - 1 & 2 \\ 0 & 1 - 1 \end{pmatrix} = \NULL \begin{pmatrix} 0 & 2 \\ 0 & 0 \end{pmatrix},
\]
and the matrix trivially has rank \(\RED{1}\), so \(\dim(E_{\lambda}) = \dim(\SET{R}^2) - \RED{1} = 1\), which is \emph{not} equal to the algebraic multiplicity, so by \THM{5.8}(a), \(\T\) is not diagonalizable.

\item The \CPOLY{} is
\[
    \det \begin{pmatrix} 1 - t & 3 \\ 3 & 1 - t \end{pmatrix} = (1 - t)^2 - 9 = t^2 - 2t + 8 = (t + 2)(t - 4),
\]
so the \CPOLY{} splits, and the eigenvalues are \(\lambda_1 = -2\), and \(\lambda_2 = 4\).
But now we have \emph{two} distinct eigenvalues, and \(2 = \dim(\SET{R}^2)\), so by \CORO{5.5.1}, \(A\) is diagonalizable.
Now we find the basis for each eigenspace.
For \(\lambda_1 = -2\),
\[
    E_{\lambda} = \NULL \begin{pmatrix} 1 - (-2) & 3 \\ 3 & 1 - (-2) \end{pmatrix} = \NULL \begin{pmatrix} 3 & 3 \\ 3 & 3 \end{pmatrix},
\]
which has equivalent matrix \(\begin{pmatrix} 1 & 1 \\ 0 & 0 \end{pmatrix}\), which trivially has the null space, that is, \(E_{\lambda_1}\),
\[
    E_{\lambda_1} = \left\{ t \begin{pmatrix} -1 \\ 1 \end{pmatrix} : t \in \SET{R} \right\},
\]
so \(\left\{ \begin{pmatrix} -1 \\ 1 \end{pmatrix} \right\}\) is a basis for \(E_{\lambda_1}\).

Similarly, for \(\lambda_2 = 4\),
\[
    E_{\lambda_2} = \NULL \begin{pmatrix} 1 - 4 & 3 \\ 3 & 1 - 4 \end{pmatrix} = \NULL \begin{pmatrix} -3 & 3 \\ 3 & -3 \end{pmatrix} = \left\{ t \begin{pmatrix} 1 \\ 1 \end{pmatrix} : t \in \SET{R} \right\}
\]
so \(\left\{ \begin{pmatrix} 1 \\ 1 \end{pmatrix} \right\}\) is a basis for \(E_{\lambda_2}\).
Then
\[
    Q = \begin{pmatrix} -1 & 1 \\ 1 & 1 \end{pmatrix}.
\]
By \CORO{5.1.1}(b), \(D = Q^{-1} A Q\) is a diagonal matrix such that
\[
    D = \begin{pmatrix} \lambda_1 & 0 \\ 0 & \lambda_2 \end{pmatrix} = \begin{pmatrix} -2 & 0 \\ 0 & 4 \end{pmatrix}.
\]

\item \(A\) is diagonalizable by calculation, and the process is similar to part(b), skip.

\item \(A\) is also diagonalizable by calculation, similart to part(b), skip.

\item The \CPOLY{} is
\[
    \det \begin{pmatrix} -t & 0 & 1 \\ 1 & -t & -1 \\ 0 & 1 & 1 - t \end{pmatrix} = (-t)\left[ (-t)(1 - t) + 1 \right] - 1(-1) = -t^3 + t^2 - t + 1 = (t^2 + 1)(-t + 1),
\]
which does \emph{not} split over \(\SET{R}\), so by (contrapositive of) \THM{5.6}, \(A\) is not diagonalizable.

\item The \CPOLY{} is
\[
    \det \begin{pmatrix} 1-t & 1 & 0 \\ 0 & 1-t & 2 \\ 0 & 0 & 3-t \end{pmatrix} = (1-t)^2(3-t)
\]
so the eigenvalues are \(\lambda_1 = 1\) with algebraic multiplicity \(2\), and \(\lambda_2 = 3\) with algebraic multiplicity \(1\).

For \(\lambda_1 = 1\),
\[
    E_{\lambda_1} = \NULL \begin{pmatrix} 1-1 & 1 & 0 \\ 0 & 1-1 & 2 \\ 0 & 0 & 3-1 \end{pmatrix} = \NULL \begin{pmatrix} 0 & 1 & 0 \\ 0 & 0 & 2 \\ 0 & 0 & 2 \end{pmatrix}.
\]
Since the rank of the matrix is \(\RED{2}\), the nullity is \(\dim(\SET{R}^3) - \RED{2} = 1\), so \(\dim(E_{\lambda_1}) = 1\) which is not equal to the algebraic multiplicity of \(\lambda_1\).
So by \THM{5.8}(a), \(A\) is not diagonalizable.

\item By calculation, \(A\) is diagonalizable, similar to part(b), skip.
\end{enumerate}
\end{proof}

\begin{exercise} \label{exercise 5.2.3}
For each of the following linear operators \(\T\) on a vector space \(\V\), test \(\T\) for diagonalizability, and if \(\T\) is diagonalizable, find a basis \(\beta\) for \(\V\) such that \([\T]_{\beta}\) is a diagonal matrix.

\begin{enumerate}
\item \(\V = \POLYRRR\) and \(\T\) is defined by \(\T(f(x)) = f'(x) + f''(x)\).
\item \(\V = \POLYRR\) and \(\T\) is defined by \(\T(ax^2 + bx + c) = cx^2 + bx + a\).
\item \(\V = \SET{R}^3\) and \(\T\) is defined by
\[
    \T \begin{pmatrix} a_1 \\ a_2 \\ a_3 \end{pmatrix}
    = \begin{pmatrix} -a_2 \\ -a_1 \\ 2a_3 \end{pmatrix}
\]
\item \(\V = \POLYRR\) and \(\T\) is defined by \(\T(f(x)) = f(0) + f(1)(x + x^2)\).
\item \(\V = \SET{C}^2\) and \(\T\) is defined by \(\T(z, w) = (z + \iu w, \iu z + w)\).
\item \(\V = M_{2 \X 2}(\SET{R})\) and \(\T\) is defined by \(\T(A) = A^\top\).
\end{enumerate}
\end{exercise}

\begin{proof}
WLOG, let \(\alpha\) be the corresponding standard ordered basis for \(\V\) on each item.
\begin{enumerate}
\item We have
\begin{align*}
    \T(1) = 0, \T(x) = 1 + 0 = 1, \T(x^2) = 2 + 2x, \T(x^3) = 6x + 3x^2
    \implies [\T]_{\alpha}
    = \begin{pmatrix}
        0 & 1 & 2 & 0 \\
        0 & 0 & 2 & 6 \\
        0 & 0 & 0 & 3 \\
        0 & 0 & 0 & 0
    \end{pmatrix}
\end{align*}
so the \CPOLY{} of \(\T\) is
\[
    \det([\T]_{\alpha} - tI) = \det \begin{pmatrix}
        -t & 1 & 2 & 0 \\
        0 & -t & 2 & 6 \\
        0 & 0 & -t & 3 \\
        0 & 0 & 0 & -t
    \end{pmatrix} = t^4,
\]
so the only eigenvalue is \(\lambda = 0\), with algebraic multiplicity \(4\).
And
\[
    \NULL(E_{\lambda}) = \NULL \begin{pmatrix}
        0-0 & 1 & 2 & 0 \\
        0 & 0-0 & 2 & 6 \\
        0 & 0 & 0-0 & 3 \\
        0 & 0 & 0 & 0-0
    \end{pmatrix}
    = \NULL \begin{pmatrix}
        0 & 1 & 2 & 0 \\
        0 & 0 & 2 & 6 \\
        0 & 0 & 0 & 3 \\
        0 & 0 & 0 & 0
    \end{pmatrix}
    = \left\{ t \begin{pmatrix} 1 \\ 0 \\ 0 \\ 0 \end{pmatrix} : t \in \SET{R} \right\},
\]
which has dimension \(1\), which is not equal to the algebraic multiplicity.
So by \THM{5.8}(a), \([\T]_{\alpha}\), and therefore \(\T\), is not diagonalizable.

\item We have
\begin{align*}
    \T(1) = x^2, \T(x) = x, \T(x^2) = 1
    \implies [\T]_{\alpha} = \begin{pmatrix} 0 & 0 & 1 \\ 0 & 1 & 0 \\ 1 & 0 & 0 \end{pmatrix}
\end{align*}
So the \CPOLY{} of \(\T\) is
\[
    \det \begin{pmatrix} -t & 0 & 1 \\ 0 & 1-t & 0 \\ 1 & 0 & -t \end{pmatrix} = -(t - 1)^2(t + 1). 
\]
So the eigenvalues are \(\lambda_1 = 1\) with algebraic multiplicity \(2\), and \(\lambda_2 = -1\), with algebraic multiplicity \(1\).
We only need to check \(\dim(E_{\lambda_1})\) to check diagonalizability.
For \(\lambda_1 = 1\),
\[
    E_{\lambda_1} = \NULL \begin{pmatrix} -1 & 0 & 1 \\ 0 & 0 & 0 \\ 1 & 0 & -1 \end{pmatrix}
    = \left\{ t_1 \begin{pmatrix} 0 \\ 1 \\ 0 \end{pmatrix} + t_2 \begin{pmatrix} 1 \\ 0 \\ 1 \end{pmatrix} \right\}
\]
so \(E_{\lambda_1}\) has dimension \(2\), which is equal to the algebraic multiplicity of \(\lambda_1\).
Hence \([\T]_{\alpha}\), and therefore \(\T\), is diagonalizable.

For \(\lambda_2 = -1\),
\[
    E_{\lambda_2} = \NULL \begin{pmatrix} 1 & 0 & 1 \\ 0 & 2 & 0 \\ 1 & 0 & 1 \end{pmatrix}
    = \left\{ t \begin{pmatrix} 1 \\ 0 \\ -1 \end{pmatrix} : t \in \SET{R} \right\}
\]
So by \THM{5.8}(b),
\[
    \beta = \left\{
        \begin{pmatrix} 0 \\ 1 \\ 0 \end{pmatrix},
        \begin{pmatrix} 1 \\ 0 \\ 1 \end{pmatrix},
        \begin{pmatrix} 1 \\ 0 \\ -1 \end{pmatrix}
    \right\}
\]
is a basis of eigenvectors such that \([\T]_{\beta}\) is a diagonal matrix.

\item
By calculation,
\[
    [\T]_{\alpha} = \begin{pmatrix}
        0 & 1 & 0 \\
        -1 & 0 & 0 \\
        0 & 0 & 2
    \end{pmatrix}
\]
and the \CPOLY{} of \(\T\) is
\[
    \det \begin{pmatrix}
        -t & 1 & 0 \\
        -1 & -t & 0 \\
        0 & 0 & 2-t
    \end{pmatrix} = (2 - t)(t^2 + 1),
\]
which does \emph{not} split over \(\SET{R}\).
So by (contrapositive of) \THM{5.6}, \([\T]_{\alpha}\), and therefore \(\T\), is not diagonalizable.

\item Skip, but it's diagonalizable.

\item Note that the field is \(\SET{C}\), and the standard ordered basis is still \(\alpha = \{(1, 0), (0, 1)\}\).
By calculation
\begin{align*}
    \T(1, 0) & = (1 + \iu \cdot 0, \iu \cdot 1 + 0) = (1, \iu) = 1 \cdot (1, 0) + \iu \cdot (0, 1) \\
    \T(0, 1) & = (0 + \iu \cdot 1, \iu \cdot 0 + 1) = (\iu, 1) = \iu \cdot (1, 0) + 1 \cdot (0, 1) \\
    \implies & [\T]_{\beta} = \begin{pmatrix} 1 & \iu \\ \iu & 1 \end{pmatrix}
\end{align*}
And the \CPOLY{} of \(\T\) is
\[
    \det \begin{pmatrix} 1 - t & \iu \\ \iu & 1 - t \end{pmatrix}
    = (1 - t)^2 - \iu^2 = 1 - 2t + t^2 + 1 = t^2 - 2t + 2
\]
So we have two distinct eigenvalues \(\cfrac{2 \pm 2 \iu}{2}\), that is,
\[
    \lambda_1 = 1 + \iu \quad \lambda_2 = 1 - \iu
\]
since \(\dim(\SET{C}^2)\) (over \(\SET{C}\)) is \(2\), and we have \(2\) distinct eigenvalues, by \CORO{5.5.1} \(\T\) is diagonalizable.
Now we find a basis such \(\beta\) that \([\T]_{\beta}\) is a diagonal matrix.

For \(\lambda_1 = 1 + \iu\),
\[
    E_{\lambda_1} = \NULL \begin{pmatrix}
        -\iu & \iu \\ \iu & -\iu
    \end{pmatrix} = \NULL \begin{pmatrix}
        \iu & -\iu \\ 0 & 0
    \end{pmatrix} = \left\{
        t \begin{pmatrix}
            1 \\ 1
        \end{pmatrix} : t \in \SET{C}
    \right\}
\]
For \(\lambda_2 = 1 - \iu\),
\[
    E_{\lambda_1} = \NULL \begin{pmatrix}
        \iu & \iu \\ \iu & \iu
    \end{pmatrix} = \NULL \begin{pmatrix}
        \iu & \iu \\ 0 & 0
    \end{pmatrix} = \left\{
        t \begin{pmatrix}
            1 \\ -1
        \end{pmatrix} : t \in \SET{C}
    \right\}
\]
So by \THM{5.8}(b),
\[
    \beta = \left\{
        \begin{pmatrix} 1 \\ 1 \end{pmatrix},
        \begin{pmatrix} 1 \\ -1 \end{pmatrix}
    \right\}
\]
is a basis for \(\V\) such that \([\T]_{\beta}\) is a diagonal matrix.
\end{enumerate}

(f) See \EXEC{5.1.18}(c).
\end{proof}

\begin{remark} \label{remark 5.2.9}
Note that in part(e), if the scalar field is still \(\SET{R}\), then \(\V\) has dimension \(4\);
in particular, by calculation, \(\T\) is \emph{not} diagonalizable.
\end{remark}

\begin{exercise} \label{exercise 5.2.4}
Prove the matrix version of the \CORO{5.5.1}:
If \(A \in M_{n \X n}(F)\) has \(n\) distinct eigenvalues, then \(A\) is diagonalizable.
\end{exercise}

\begin{proof}
If \(A\) has \(n\) distinct eigenvalues, then (by \RMK{5.1.5}) \(\LMTRAN_A\) has \(n\) distinct eigenvalues;
so by \CORO{5.5.1}, \(\LMTRAN_A\) is diagonalizable, and finally, by \DEF{5.1}, \(A\) is diagonalizable.
\end{proof}

\begin{exercise} \label{exercise 5.2.5}
State and prove the matrix version of \THM{5.6}.
\end{exercise}

\begin{proof}
The statement is:
The \CPOLY{} of any diagonalizable \(A \in M_{n \X n}(F)\) splits over \(F\).

So if \(A\) is diagonalizable, then by \DEF{5.1}, \(\LMTRAN_A\) is diagonalizable, so by \THM{5.6}, the \CPOLY{} of \(\LMTRAN_A\) splits over \(F\).
But by \DEF{5.4}, the \CPOLY{} of \(\LMTRAN_A\) \emph{is} the \CPOLY{} of \([\LMTRAN_A]_{\beta}\) where \(\beta\) is the standard ordered basis.
But \([\LMTRAN_A]_{\beta} = A\) by \THM{2.15}(a).
So the \CPOLY{} of \(A\) splits over \(F\).
\end{proof}

\begin{exercise} \label{exercise 5.2.6} \ 

\begin{enumerate}
\item \emph{Justify} the test for diagonalizability and the method for diagonalization stated in the subsection \ref{sec 5.2.1}.
\item Formulate the results in (a) for matrices.
\end{enumerate}
\end{exercise}

\begin{proof}
The condition 1 for the test in \SEC{5.2.1} is because of \THM{5.6};
if the \CPOLY{} of \(\T\) does \emph{not} split, then by (the contrapositive of) \THM{5.6}, \(\T\) is not diagonalizable.
The condition 2 is because of \THM{5.8}(a);
if there is an eigenvalue \(\lambda\) such that the algebraic multiplicity is not equal to \(\dim(E_{\lambda})\), then by \THM{5.8}(a), \(\T\) is not diagonalizable.

For the matrix version, just replace \(\T\) in the argument above with matrix \(A\).
\end{proof}

\begin{exercise} \label{exercise 5.2.7}
For
\[
    A = \begin{pmatrix} 1 & 4 \\ 2 & 3 \end{pmatrix} \in M_{2 \X 2}(\SET{R}),
\]
find an expression for \(A^n\), where \(n\) is an arbitrary positive integer.
\end{exercise}

\begin{proof}
By calculation, \(A\) is diagonalizable and we have
\[
    Q = \begin{pmatrix}
        1 & -2 \\ 1 & 1
    \end{pmatrix}
    \quad \text{ and } \quad
    Q^{-1} = \frac{1}{3} \begin{pmatrix}
        1 & 2 \\ -1 & 1
    \end{pmatrix}
    \text{ such that } D = Q^{-1} A Q = \begin{pmatrix} 5 & 0 \\ 0 & -1 \end{pmatrix} \text{ is diagonal}.
\]
Using the same technique in \EXAMPLE{5.2.7},
\begin{align*}
    A^n & = Q D^n Q^{-1} = Q \begin{pmatrix}
        5^n & 0 \\ 0 & (-1)^n
    \end{pmatrix} Q^{-1} \\
        & = \frac{1}{3}
            \begin{pmatrix} 1 & -2 \\ 1 & 1 \end{pmatrix}
            \begin{pmatrix} 5^n & 0 \\ 0 & (-1)^n \end{pmatrix}
            \begin{pmatrix} 1 & 2 \\ -1 & 1 \end{pmatrix} \\
        & = \frac{1}{3} \begin{pmatrix}
                5^n + 2(-1)^n & 2 \cdot 5^n - 2(-1)^n \\
                5^n - (-1)^n & 2 \cdot 5^n + (-1)^n
        \end{pmatrix}.
\end{align*}
\end{proof}

\begin{exercise} \label{exercise 5.2.8}
Suppose that \(A \in M_{n \X n}(F)\) has two distinct eigenvalues, \(\lambda_1\) and \(\lambda_2\), and that \(\dim(E_{\lambda_1})
= n - 1\).
Prove that \(A\) is diagonalizable.
\end{exercise}

\begin{proof}
Let \(m_1, m_2\) be the algebraic multiplicity of \(\lambda_1, \lambda_2\), respectively.
First, since \(\dim(E_{\lambda_1}) = n - 1\), by \THM{5.7}, the \(m_1 \ge n - 1\).
But since there exists another distinct eigenvalue \(\lambda_2\), \(m_1\) cannot be greater than \(\RED{n} - 1\), where the ``role'' of \(\RED{n}\) is the \emph{degree} of the \CPOLY{}.
Hence the \(m_1 = n - 1\), so \(m_1 = \dim(E_{\lambda_1})\).

For \(\lambda_2\), again by \THM{5.7}, \(m_2 \ge 1\).
But \(m_2\) must also be \(\le 1\) since if the \(m_2 > 1\), then the total multiplicity \(m_1 + m_2\) of \(\lambda_1\) and \(\lambda_2\) now is greater than \(1 + (n - 1) = n\), more than the \emph{degree} of \CPOLY{}, which is impossible.
So the \(m_2 = 1\).
And by \THM{5.7}, \(1 \le \dim(E_{\lambda_2}) \le m_2 = 1\), hence  \(m_2 = \dim(E_{\lambda_2})\).

Finally, since \(m_1 + m_2 = (n - 1) + 1 = n\), the \CPOLY{} of \(A\) splits.
So all conditions of \THM{5.8}(a) are satisfied, that is, the \CPOLY{} splits, \(m_1 = \dim(E_{\lambda_1})\), \(m_2 = \dim(E_{\lambda_2})\), hence \(A\) is diagonalizable.
\end{proof}

\begin{exercise} \label{exercise 5.2.9}
Let \(\T\) be a linear operator on a \(n\)-dimensional vector space \(\V\), and suppose there exists an ordered basis \(\beta\) for \(\V\) such that \([\T]_{\beta}\) is an \emph{upper triangular} matrix.
\begin{enumerate}
\item Prove that the \CPOLY{} for \(\T\) splits.
\item State and prove an analogous result for matrices.
The \emph{converse} of (a) is treated in \EXEC{5.2.12}(b).
\end{enumerate}
\end{exercise}

\begin{proof} \ 

\begin{enumerate}
\item Let \(A = [\T]_{\beta}\).
Since \(A = [\T]_{\beta}\) is an upper triangular, and \(tI\) for any \(t \in F\) is also an upper triangular, we have \(A - tI\) as an upper triangular matrix.

Then
\begin{align*}
    \det([\T]_{\beta} - tI)
        & = \det(A - tI) \\
        & = (A - tI)_{11} \cdot (A - tI)_{22} \cdot ... \cdot (A - tI)_{nn} & \text{by \EXEC{4.2.23}} \\
        & = (A_{11} - t)(A_{22} - t) ... (A_{nn} - t) & \text{of course}
\end{align*}
With this form, by \DEF{5.5}, the \CPOLY{} of \([\T]_{\beta}\), and therefore \(\T\), splits over \(F\).

\item \(A\) is upper triangular, \(\implies\) (by \THM{2.15}) \([\LMTRAN_A]_{\beta}\) is upper triangular, where \(\beta\) is the standard ordered basis, \(\implies\) (by part(a)) the \CPOLY{} of \(\LMTRAN_A\) splits, \(\implies\) (by \DEF{5.4}) the \CPOLY{} of \([\LMTRAN_A]_{\beta}\) splits, that is, the \CPOLY{} of \(A\) splits.
\end{enumerate}
\end{proof}

\begin{exercise} \label{exercise 5.2.10}
Let \(\T\) be a linear operator on a finite-dimensional vector space \(\V\) with the distinct eigenvalues \(\lambda_1, \lambda_2, ..., \lambda_k\) and corresponding algebraic multiplicities \(m_1, m_2, ..., m_k\).
Suppose that \(\beta\) is a basis for \(\V\) such that \([\T]_{\beta}\) is an \emph{upper triangular} matrix.
Prove that the \emph{diagonal entries} of \([\T]_{\beta}\) are \(\lambda_1, \lambda_2, ..., \lambda_k\) and that \textbf{each \(\lambda_i\) occurs \(m_i\) times} \((1 \le i \le k)\).
\end{exercise}

\begin{proof}
Since \(A = [\T]_{\beta}\) is an upper triangular, by \EXEC{5.2.9}(a), then the \CPOLY{} of \(\T\) splits.

But, by similar to the proof in that exercise, since \(A - tI\) is also an upper triangular, by \EXEC{4.2.23}, the \CPOLY{} of \(\T\) is
\[
    \det(A - tI) = (A_{11} - t)(A_{22} - t) ... (A_{nn} - t),
\]
which means each \(\lambda_i\) occurs \(m_i\) times in the diagonal entries of \(A\).
\end{proof}

\begin{exercise} \label{exercise 5.2.11}
Let \(A\) be an \(n \X n\) matrix that is \emph{similar} to an upper triangular matrix and has the distinct eigenvalues \(\lambda_1, \lambda_2, ..., \lambda_k\) with corresponding algebraic multiplicities \(m_1, m_2, ..., m_k\).
Prove the following statements.

\begin{enumerate}
\item \(\TRACE(A) = \sum_{i = 1}^k m_i \lambda_i\).
\item \(\det(A) = (\lambda_1)^{m_1}(\lambda_2)^{m_2} ... (\lambda_k)^{m_k}\).
\end{enumerate}
\end{exercise}

\begin{proof} \ 

\begin{enumerate}
\item Suppose \(A\) is similar to an upper triangular matrix \(B\).
By \EXEC{5.1.13}, \(A\) and \(B\) have the same \CPOLY{} hence the same eigenvalues.
And by \EXEC{5.2.10}, the diagonal entries of \(B\) are \(\lambda_1, \lambda_2, ..., \lambda_k\) where each \(\lambda_i\) occurs \(m_i\) times.
So clearly \(\TRACE(B) = \sum_{i = 1}^k m_i \lambda_i\).
And by \EXEC{2.5.10}, \(\TRACE(B) = \TRACE(A)\), hence \(\TRACE(A) = \sum_{i = 1}^k m_i \lambda_i\).

\item 
Clearly, from part(a), \(\det(B) = (\lambda_1)^{m_1} (\lambda_2)^{m_2} ... (\lambda_k)^{m_k}\).
But by \EXEC{4.3.15}, \(\det(A) = \det(B)\), hence \(\det(A) = (\lambda_1)^{m_1} (\lambda_2)^{m_2} ... (\lambda_k)^{m_k}\).
\end{enumerate}
\end{proof}

\begin{exercise} \label{exercise 5.2.12} \ 

\begin{enumerate}
\item Prove that if \(A \in M_{n \X n}(F)\) and the \CPOLY{} of \(A\) splits, then \(A\) is similar to an upper triangular matrix.
(This proves the \emph{converse} of \EXEC{5.2.9}(b).)
Hint: Use mathematical induction on \(n\).
For the general case, let \(v_1\) be an eigenvector of \(A\), and \emph{extend} \(\{ v_1 \}\) to a basis \(\{ v_1, v_2, ..., v_n \}\) for \(F^n\).
Let \(P\) be the \(n \X n\) matrix whose \(j\)th column is \(v_i\), and consider \(P^{-1} A P\).
\EXEC{5.1.13}(a) and \EXEC{4.3.21} can be helpful.

\item Prove the \emph{converse} of \EXEC{5.2.9}(a).
\end{enumerate}
\end{exercise}

\begin{proof} \ 

\begin{enumerate}
\item We prove by induction with base case \(n = 1\).
But a \(1 \X 1\) matrix is by definition an upper triangular anyway.

So for the inductive hypothesis, suppose any \((n - 1) \X (n - 1)\) matrix whose \CPOLY{} splits is similar to an upper triangular matrix for \(n > 1\);
we have to show the statement is true for any \(n \X n\) matrix.

So let \(A\) be an \(n \X n\) matrix whose \CPOLY{} splits.
Then there exists an eigenvalue of \(\lambda\) of \(A\), and let \(v_1\) be a corresponding eigenvector \MAROON{(1)}.
Now we extend \(v_1\) to an ordered basis \(\{ v_1, v_2, ..., v_n \}\) for \(F^n\).
And let \(P\) be the \emph{invertible}(of course!) matrix
\[
    P = \begin{pmatrix}
            v_1 & v_2 & ... & v_n
    \end{pmatrix}. \quad \quad \MAROON{(p.1)}.
\]
We will show the structure of \(P^{-1} A P\), but first consider the matrix product \(AP\);
by \THM{2.13}(a), the first column of \(AP\) is equal to \(A\) times the first column of \(P\), that is, by, \MAROON{(p.1)}, \(A v_1\).
But by \MAROON{(1)}, \(A v_1 = \lambda v_1\), hence
\begin{center}
    the first column of \(AP\) is equal to \(\lambda v\). \quad \quad \MAROON{(2)}
\end{center}
And by \THM{2.13}(b), the first column of \(P\) is equal to \(P e_1\), hence (again by \MAROON{(p.1)},) \(P e_1 = v_1\), hence (since \(P\) is invertible,) \(e_1 = P^{-1} v_1\). \MAROON{(3)}

Putting these together, we calculate the first column of \(P^{-1} A P\):
\begin{align*}
    & \text{The first column of } P^{-1} A P \\
    & = P^{-1} \X \text{The first column of } (AP) & \text{by \THM{2.13}(a)} \\
    & = P^{-1} (\lambda v_1) & \text{by \MAROON{(2)}} \\
    & = \lambda P^{-1} v_1 & \text{by \THM{2.12}(b)} \\
    & = \lambda e_1 & \text{by \MAROON{(3)}} \\
    & = \begin{pmatrix}
            \lambda \\
            0 \\
            \vdots \\
            0
    \end{pmatrix} \quad \quad \MAROON{(4)}
\end{align*}
So by \MAROON{(4)} we can conclude the structure of \(P^{-1} A P\) is:
\[
    \begin{pmatrix}
        \lambda & u \\
        O_{(n - 1) \X 1} & B
    \end{pmatrix}, \quad \quad \MAROON{(5)}
\]
where \(u\) is a \(1 \X (n - 1)\) matrix, and \(B\) is an \((n - 1) \X (n - 1)\) matrix.
Now since by definition \(P^{-1} A P\) and \(A\) are similar, by \EXEC{5.1.13}(a),
they have the same \CPOLY{}, hence the \CPOLY{} of \(P^{-1} A P\) splits.
And from \EXEC{4.3.21} and \MAROON{(5)}, we have conclude the \CPOLY{} of \(P^{-1} A P\) is equal to the product of the \CPOLY{} of the ``matrix'' \(\lambda\) and the \CPOLY{} of \(B\),
which implies the \CPOLY{} of \(B\) also splits.

Since \(B\) is a \((n - 1) \X (n - 1)\) matrix and the \CPOLY{} splits, by inductive hypothesis, \(B\) is similar to an upper triangular matrix.
That is, there exists an invertible matrix \(Q\) such that \(U = Q^{-1} B Q\) is an upper triangular.
Now, let
\[
    R = \begin{pmatrix}
        1 & O_{1 \X (n - 1)} \\
        O_{(n - 1) \X 1} & Q
    \end{pmatrix}
\]
its of course that \(R\) is invertible, and
\[
    R^{-1} = \begin{pmatrix}
        1 & O_{1 \X (n - 1)} \\
        O_{(n - 1) \X 1} & Q^{-1}
    \end{pmatrix}
\]
Now let \(M = PR\), then \(M\) is the product of invertible matrices, hence is invertible.

Finally, we have
\begin{align*}
    & (M)^{-1} A M \\
    & = (PR)^{-1} A (PR) = (R^{-1} P^{-1}) A (P R) & \text{by \EXEC{2.4.4}} \\
    & = R^{-1} (P^{-1} A P) R & \text{of course} \\
    & = \begin{pmatrix}
        1 & O_{1 \X (n - 1)} \\
        O_{(n - 1) \X 1} & Q^{-1}
    \end{pmatrix}
    \begin{pmatrix}
        \lambda & u \\
        O_{(n - 1) \X 1} & B
    \end{pmatrix} \begin{pmatrix}
        1 & O_{1 \X (n - 1)} \\
        O_{(n - 1) \X 1} & Q
    \end{pmatrix} \\
    & = \begin{pmatrix}
        \lambda & uQ \\
        O_{(n - 1) \X 1} & U
    \end{pmatrix},
\end{align*}
which is an upper triangular.
Hence by definition, \(A\) is similar to an upper triangular matrix, proving the statement for the case \(= n\).
This closes the induction.

\item The converse of \EXEC{5.2.9}(a) is:
Given linear operator \(\T\), if the \CPOLY{} of \(\T\) splits, then there exists an ordered bases \(\beta\) such that \([\T]_{\beta}\) is an upper triangular matrix.
So given \(\T\) such that \CPOLY{} splits, let \(\alpha\) be the \emph{standard} ordered basis, and \(A = [\T]_{\alpha}\).
Then by \DEF{5.4}, the \CPOLY{} of \(A\) also splits, so by part(a), \(A\) is similar to an upper triangular matrix \(U\).
So we can write \(U = Q^{-1} A Q\) for some invertible \(Q\).
Now let \(\beta\) be another ordered basis where the \(j\)th vector in \(\beta\) is the \(j\)th column of \(Q\).
Then by \CORO{2.23.1}, \(Q\) is the change of coordinate matrix that changes \(\beta\)-coordinates to (the \emph{standard}) \(\alpha\)-coordinates, and \(U = [\T]_{\beta}\).
Hence we have found an ordered basis \(\beta\) such that \([\T]_{\beta}\) is an upper triangular matrix, as desired.
\end{enumerate}
\end{proof}

\begin{exercise} \label{exercise 5.2.13}
Let \(\T\) be an \emph{invertible} linear operator on a \emph{finite}-dimensional vector space \(\V\).
\begin{enumerate}
\item Recall that for any eigenvalue \(\lambda\) of \(\T\), \(\lambda^{-1}\) is an eigenvalue of \(\T^{-1}\) (\EXEC{5.1.9}).
Prove that the eigenspace of \(\T\) corresponding to \(\lambda\) is \textbf{the same as} the eigenspace of \(\T^{-1}\) corresponding to \(\lambda^{-1}\).
\item Prove that if \(\T\) is diagonalizable, then \(\T^{-1}\) is diagonalizable.
\end{enumerate}
\end{exercise}

\begin{proof}
We define the notation: \(E_{\lambda}\) is the eigenspace of \(\T\) corresponding to \(\lambda\), and \(E'_{\lambda}\) is the eigenspace of \(\T^{-1}\) corresponding to \(\lambda\).

Also note that, since \(\T\) is invertible, by \EXEC{5.1.9}(a) any eigenvalue of \(\T\) is not equal to zero.

\begin{enumerate}
\item \RED{For \(v \in \V\), \(v \in E_{\lambda}\)}, if and only if \(\T(v) = \lambda v\), if and only if (by def of inverse) \(\T^{-1} (\lambda v) = v\), if and only if (since \(\T\) is linear) \(\lambda \T^{-1}(v) = v\), if and only if (since \(\lambda \ne 0\)) \(\T^{-1}(v) = \lambda^{-1} v\), \RED{if and only if \(v \in E'_{\lambda^{-1}}\)}.
So \(E_{\lambda} = E'_{\lambda^{-1}}\), as desired.

\item Suppose \(\T\) is invertible and diagonalizable.
Then by \THM{5.1}, there exists an ordered basis \(\beta\) consisting of eigenvectors of \(\T\).
But by part(a), the vectors in \(\beta\) are also eigenvectors of \(\T^{-1}\).
Hence \(\beta\) is an ordered basis consisting of eigenvectors of \(\T^{-1}\), hence by \THM{5.1} again, \(\T^{-1}\) is diagonalizable.
\end{enumerate}
\end{proof}

\begin{exercise} \label{exercise 5.2.14}
Let \(A \in M_{n \X n}(F)\).
Recall from \EXEC{5.1.15} that \(A\) and \(A^\top\) have the same \CPOLY{} and hence share the same eigenvalues with the same algebraic multiplicities.
For any eigenvalue \(\lambda\) of \(A\) and \(A^\top\) let \(E_{\lambda}\) and \(E'_{\lambda}\) denote the corresponding eigenspaces for \(A\) and \(A^{\top}\), respectively.
\begin{enumerate}
\item Show by way of example that for a given common eigenvalue, these two eigenspaces \emph{need not be the same}.

\item Prove that for any eigenvalue \(\lambda\), \(\dim(E_{\lambda}) = \dim(E'_{\lambda})\).

\item Prove that if \(A\) is diagonalizable, then \(A^\top\) is also diagonalizable.
\end{enumerate}
\end{exercise}

\begin{proof} \ 

\begin{enumerate}
\item Let
\[
    A = \begin{pmatrix} 1 & 1 \\ 0 & 1 \end{pmatrix}
    \quad \text{ hence } \quad
    A^\top = \begin{pmatrix} 1 & 0 \\ 1 & 1 \end{pmatrix}.
\]
By calculation, \(\lambda = 1\) is an eigenvalue with algebraic multiplicity \(2\), and
\[
    \begin{pmatrix} 1 & 1 \\ 0 & 1 \end{pmatrix}
    \begin{pmatrix} 1 \\ 0 \end{pmatrix} = \begin{pmatrix} 1 \\ 0 \end{pmatrix}
    = \lambda \begin{pmatrix} 1 \\ 0 \end{pmatrix}
    \quad \text{ but } \quad
    \begin{pmatrix} 1 & 0 \\ 1 & 1 \end{pmatrix}
    \begin{pmatrix} 1 \\ 0 \end{pmatrix} = \begin{pmatrix} 1 \\ 1 \end{pmatrix}
    \ne \lambda \begin{pmatrix} 1 \\ 0 \end{pmatrix}
\]
which implies \(\begin{pmatrix} 1 \\ 0 \end{pmatrix}\) is an eigenvector of \(A\) with corresponding \(\lambda = 1\), but \emph{not} an eigenvector of \(A^\top\) with corresponding \(\lambda = 1\).
Hence \(E_{\lambda} \ne E'_{\lambda}\).

\item
Note that (from \CH{3}), \(\rank(A) = \rank(A^\top)\) for any matrix \(A\) \MAROON{(1)}.
And we have
\begin{align*}
    \dim(E_{\lambda}) & = n - \rank(A - \lambda I) & \text{by definition}
    \\
        & = n - \rank((A - \lambda I)^\top) & \text{by \MAROON{(1)}} \\
        & = n - \rank(A^\top - \lambda I^\top) & \text{by \ATHM{1.2}(1)} \\
        & = n - \rank(A^\top - \lambda I) & \text{of course} \\
        & = \dim(E'_{\lambda}) & \text{by definition}
\end{align*}

\item
If \(A\) is diagonalizable, then by \CORO{5.1.1} \(A\) is similar to a diagonal matrix, that is, \(D = Q^{-1} A Q\) for some diagonal matrix \(D\) and invertible \(Q\).
By \EXEC{2.4.5}, \(Q^\top\) is invertible, and \((Q^{\top})^{-1} = (Q^{-1})^\top\).
By \ATHM{2.24}, we have \(D^\top = (Q^{-1} A Q)^\top = Q^\top A^\top (Q^{-1})^\top = Q^\top A^\top (Q^\top)^{-1}\).

But since \(D\) is diagonal, and in particular symmetric, \(D = D^\top\), so we have \(D = Q^\top A^\top (Q^\top)^{-1}\).
So \(A^\top\) is similar to a diagonal matrix, so again by \CORO{5.1.1}, \(A^\top\) is diagonalizable.
\end{enumerate}
\end{proof}

\TODOREF{Skip Exercise 15 through 17, differential equations.}

\begin{exercise} \label{exercise 5.2.15}
Find the general solution to each system of differential equations.
\begin{enumerate}
\item 
\begin{align*}
    x_1' = x_1 + x_2 \\
    x_2' = 3x_1 - x_2
\end{align*}
\item
\begin{align*}
    x_1' = 8x_1 + 10x_2 \\
    x_2' = -5x_1 - 7x_2
\end{align*}

\item
\begin{align*}
    x_1' = x_1 \quad + x_3 \\
    x_2' = x_1 + x_3 \\
    x_3' = \quad 2 x_3
\end{align*}
\end{enumerate}
\end{exercise}

\begin{proof}
\end{proof}

\begin{exercise} \label{exercise 5.2.16}
Let
\[
    A = \begin{pmatrix}
        a_{11} & a_{12} & ... & a_{1n} \\
        a_{21} & a_{22} & ... & a_{2n} \\
        \vdots & \vdots &     & \vdots \\
        a_{n1} & a_{n2} & ... & a_{nn} \\
    \end{pmatrix}
\]
be the coefficient matrix of the system of differential equations
\begin{align*}
    x_1' & = a_{11} x_1 + a_{12} x_2 + ... + a_{1n} x_n \\
    x_2' & = a_{21} x_1 + a_{22} x_2 + ... + a_{2n} x_n \\
    & \vdots \\
    x_n' & = a_{n1} x_1 + a_{n2} x_2 + ... + a_{nn} x_n
\end{align*}
Suppose that \(A\) is diagonalizable and that the distinct eigenvalues of \(A\) are \(\lambda_1, \lambda_2, ..., \lambda_k\).
Prove that a differentiable function \(x: \SET{R} \to \SET{R}^n\) is a
solution to the system if and only if \(x\) is of the form
\[
    x(t) = e^{\lambda_1 t} z_1 + e^{\lambda_2 t} z_2 + ... + e^{\lambda_k t} z_k,
\]
where \(z_i \in E_{\lambda_i}\) for \(i = 1, 2, ..., k\).
Use this result to prove that the set of solutions to the system is an \(n\)-dimensional \emph{real} vector space.
\end{exercise}

\begin{proof}
\end{proof}

\begin{exercise} \label{exercise 5.2.17}
Let \(C \in M_{m \X n}(\SET{R})\), and let \(Y\) be an \(n \X p\) matrix of differentiable functions.
Prove \((CY)' = CY'\), where \((Y')_{ij} = Y_{ij}'\) for all \(i, j\).
\end{exercise}

\begin{proof}
\end{proof}

Exercises 18 through 20 are concerned with \emph{simultaneous diagonalization}.

\begin{additional definition} \label{adef 5.1} \ 
\begin{enumerate}
\item Two linear operators \(\T\) and \(\U\) on a finite-dimensional vector space \(\V\) are called \textbf{simultaneously diagonalizable} if there exists an ordered basis \(\beta\) for \(\V\) such that both \([\T]_{\beta}\) and \([\U]_{\beta}\) are diagonal matrices.
\item Similarly, \(A, B \in M_{n \X n}(F)\) are called \textbf{simultaneously diagonalizable} if there exists an invertible matrix \(Q \in M_{n \X n}(F)\) such that both \(Q^{-1} A Q\) and \(Q^{-1} B Q\) are diagonal matrices.
\end{enumerate}
\end{additional definition}

\begin{exercise} \label{exercise 5.2.18} \ 

\begin{enumerate}
\sloppy \item Prove that if \(\T\) and \(\U\) are simultaneously diagonalizable linear operators on a finite-dimensional vector space \(\V\), then the matrices \([\T]_{\beta}\) and \([\U]_{\beta}\) are simultaneously diagonalizable \textbf{for any} ordered basis \(\beta\).
\item Prove that if \(A\) and \(B\) are simultaneously diagonalizable matrices, then \(\LMTRAN_A\) and \(\LMTRAN_B\) are simultaneously diagonalizable linear operators.
\end{enumerate}
\end{exercise}

\begin{proof} \ 

\begin{enumerate}
\item If \(\T\) and \(\U\) are simultaneously diagonalizable, then by \ADEF{5.1}(a), there exists a basis \(\alpha\) such that \([\T]_{\alpha}\) and \([\U]_{\alpha}\) are both diagonal matrices.
Now let \(\beta\) be \textbf{any} basis, and let \(Q\) be the change of coordinate matrix that changes \(\alpha\)-coordinates into \(\beta\)-coordinates.
By \THM{2.23}, we have
\[
    [\T]_{\alpha} = Q^{-1} [\T]_{\beta} Q
    \quad \text{ and } \quad
    [\U]_{\alpha} = Q^{-1} [\U]_{\beta} Q
\]
Hence we have found a invertible matrix \(Q\) such that both \(Q^{-1} [\T]_{\beta} Q\) and \(Q^{-1} [\U]_{\beta} Q\) are diagonal matrices.
Hence by \ADEF{5.1}(b), \([\T]_{\beta}\) and \([\U]_{\beta}\) are simultaneously diagonalizable.

\item If \(A\) and \(B\) are simultaneously diagonalizable, then by \ADEF{5.1}(b), there exists a invertible matrix \(Q\) such that \(Q^{-1} A Q\) and \(Q^{-1} B Q\) are both diagonal matrices.
Now let \(\beta\) be the basis such that the \(j\)th vector is the \(j\)th column of \(Q\),
Then both \([\LMTRAN_A]_{\beta}\) and \([\LMTRAN_B]_{\beta}\) are diagonal matrices, since by \CORO{2.23.1}, we have
\[
    [\LMTRAN_A]_{\beta} = Q^{-1} A Q
    \quad \text{ and } \quad
    [\LMTRAN_B]_{\beta} = Q^{-1} B Q.
\]
Hence we have found a basis \(\beta\) such that both \([\LMTRAN_A]_{\beta}\) and \([\LMTRAN_B]_{\beta}\) are diagonal matrices,
so by \ADEF{5.1}(a), \(\LMTRAN_A\) and \(\LMTRAN_B\) are simultaneously diagonalizable.
\end{enumerate}
\end{proof}

\begin{exercise} \label{exercise 5.2.19} \ 

\begin{enumerate}
\item Prove that if \(\T\) and \(\U\) are simultaneously diagonalizable operators, then \(\T\) and \(\U\) \textbf{commute} (i.e., \(\T\U = \U\T\)).
\item Show that if \(A\) and \(B\) are simultaneously diagonalizable matrices, then \(A\) and \(B\) commute.
The \emph{converses} of (a) and (b) are established in \EXEC{5.4.25}
\end{enumerate}
\end{exercise}

\begin{note}
See some cool \href{https://math.stackexchange.com/questions/236212/prove-that-simultaneously-diagonalizable-matrices-commute}{answer}.
\end{note}

\begin{proof}
We first show the fact that diagonal matrices ``commute''.
Let \(D_1, D_2\) be arbitrary diagonal matrices.
It's of course that \(D_1 D_2\) and \(D_2 D_2\) are also diagonal matrices, so we only need to check they have the same diagonal entries.
Then
\begin{align*}
    (D_1 D_2)_{ii} & = \sum_{k = 1}^n (D_1)_{ik} (D_2)_{ki} & \text{by definition} \\
        & = (D_1)_{ii} (D_2)_{ii} & \text{only diagonal entries of \(D_1, D_2\) are nonzero} \\
        & = (D_2)_{ii} (D_1)_{ii} & \text{of course} \\
        & = \sum_{k = 1}^n (D_2)_{ik} (D_1)_{ki} & \text{only diagonal entries of \(D_1, D_2\) are nonzero} \\
        & = (D_2 D_1)_{ii}, & \text{by definition}
\end{align*}
as desired.

\begin{enumerate}
\item Suppose \(\T\) and \(\U\) are simultaneously diagonalizable, so by \ADEF{2.1}(a) there exists an ordered basis \(\beta\) such that both \([\T]_{\beta}\) and \([\U]_{\beta}\) are diagonal matrices.
Then we have
\begin{align*}
    [\T\U]_{\beta} & = [\T]_{\beta} [\U]_{\beta} & \text{by \THM{2.11}} \\
        & = [\U]_{\beta} [\T]_{\beta} & \text{since diagonal matrices commute} \\
        & = [\U\T]_{\beta}, & \text{by \THM{2.11}}
\end{align*}
which implies (by the second part of \RMK{2.2.2}) \(\T\U = \U\T\), as desired.

\item Suppose \(A\) and \(B\) are simultaneously diagonalizable, so by \ADEF{2.1}(b) there exists a invertible matrix \(Q\) such that both \(Q^{-1} A Q\) and \(Q^{-1} B Q\) are diagonal matrices.
Then we have
\begin{align*}
    AB & = (QQ^{-1}) A (QQ^{-1}) B (QQ^{-1}) & \text{tricky, but true} \\
       & = Q \left[ (Q^{-1} A Q) (Q^{-1} B Q) \right] Q^{-1} & \text{of course by associativity} \\
       & = Q \left[ (Q^{-1} B Q) (Q^{-1} A Q) \right] Q^{-1} & \text{since diagonal matrices commute} \\
       & = (QQ^{-1}) B (QQ^{-1}) A (QQ^{-1}) & \text{of course by associativity} \\
       & = BA, & \text{of course}
\end{align*}
as desired.
\end{enumerate}
\end{proof}

\begin{exercise} \label{exercise 5.2.20}
Let \(\T\) be a diagonalizable linear operator on a finite-dimensional vector space, and let \(m\) be any positive integer.
Prove that \(\T\) and \(\T^m\) are simultaneously diagonalizable.
\end{exercise}

\begin{proof}
Suppose \(\T\) is diagonalizable, so (by \DEF{5.1}) there exists an ordered basis \(\beta\) such that \([\T]_{\beta}\) is a diagonal matrix.
Then (inductively) by \THM{2.11},
\begin{align*}
    [\T^m]_{\beta} & = [\T]_{\beta} [\T^{m - 1}]_{\beta} \\
        & \vdots \\
        & = ([\T]_{\beta})^m,
\end{align*}
which is the power of \(m\) diagonal matrices, hence is also diagonal.
So both \([\T]_{\beta}\) and \([\T^m]_{\beta}\) are diagonal matrices, so by \ADEF{5.1}(a), \(\T\) and \(\T^m\) are simultaneously diagonalizable.
\end{proof}

Exercises 21 through 24 are concerned with direct sums.

\begin{exercise} \label{exercise 5.2.21}
Let \(\W_1, \W_2, ..., \W_k\) be subspaces of a \emph{finite}-dimensional vector space \(\V\) such that
\[
    \sum_{i = 1}^k \W_i = \V.
\]
Prove that \(\V\) is the direct sum of \(\W_1, \W_2, ..., \W_k\) if and only if
\[
    \dim(\V) = \sum_{i = 1}^k \dim(\W_i).
\]
\end{exercise}

\begin{proof}
\(\V\) is the direct sum of \(\W_1, \W_2, ..., \W_k\),
if and only if (by \THM{5.9}(e)) there exists ordered basis \(\gamma_1, \gamma_2, ..., \gamma_k\) for each \(\W_i\) such that \(\gamma = \gamma_1 \cup \gamma_2 \cup ... \cup \gamma_k\) is an ordered basis for \(\V\) \MAROON{(1)}.

And we claim that the vectors in \(\gamma_i\) do not have duplicates, otherwise if \(\gamma_i\) and \(\gamma_j\) have duplicates, it immediately follows that \(\W_i \cap \W_j \ne \{ \OV \}\), which contradicts \EXEC{5.2.1}(h), hence \(\#\gamma = \#\gamma_1 + \#\gamma_2 + ... + \#\gamma_k\).

So continuing, \MAROON{(1)} is true, if and only if \(\#\gamma = \#\gamma_1 + \#\gamma_2 + ... + \#\gamma_k\),
if and only if (by definition of dimension) \(\dim(\V) = \dim(\W_1) + \dim(\W_2) + ... + \dim(\W_k)\), as desired.
\end{proof}

\begin{exercise} \label{exercise 5.2.22}
Let \(\V\) be a \emph{finite}-dimensional vector space with a basis \(\beta\), and let \(\beta_1, \beta_2, ..., \beta_k\) be a \href{https://www.wikiwand.com/en/Partition_of_a_set}{\emph{partition}} of \(\beta\)
(i.e., \(\beta_1, \beta_2, ..., \beta_k\) are subsets of \(\beta\) such that \(\beta = \beta_1 \cup \beta_2 \cup ... \cup \beta_k\) and \(\beta_i \cap \beta_j = \emptyset\) if \(i \ne j\)).
Prove that \(\V = \spann(\beta_1) \oplus \spann(\beta_2) \oplus ... \oplus \spann(\beta_k)\).
\end{exercise}

\begin{proof}
First, we of course have \(\V = \spann(\beta_1) + \spann(\beta_2) + ... + \spann(\beta_k)\).
So the precondition of \EXEC{5.2.21} is satisfied.
Now, since \(\beta_i\) are a partition of \(\beta\), we of course have \(\#\beta = \#\beta_1 + \#\beta_2 + ... + \#\beta_k\), and by definition of dimension, \(\V = \dim(\spann(\beta_1)) + \dim(\spann(\beta_2)) + ... + \dim(\spann(\beta_k))\).
Hence by \EXEC{5.2.21}, \(\V\) is the direct sum of \(\dim(\spann(\beta_1)), \dim(\spann(\beta_2)), ..., \dim(\spann(\beta_k))\), as desired.
\end{proof}

\begin{exercise} \label{exercise 5.2.23}
Let \(\T\) be a linear operator on a \emph{finite}-dimensional vector space \(\V\), and suppose that the distinct eigenvalues of \(\T\) are \(\lambda_1, \lambda_2, ..., \lambda_k\).
Prove that
\[
    \spann(\{ x \in \V: x \text{ is an eigenvector of } \T \}) = E_{\lambda_1} \oplus E_{\lambda_2} \oplus ... \oplus E_{\lambda_k}.
\]
\end{exercise}

\begin{note}
We do \emph{not} say that \(\T\) is diagonalizable.
\end{note}

\begin{proof}
It's of course that (by the definition of LHS)
\[
    \spann(\{ x \in \V: x \text{ is an eigenvector of } \T \}) = E_{\lambda_1} + E_{\lambda_2} + ... + E_{\lambda_k},
\]
so it suffices to show \(E_{\lambda_j} \cap \sum_{i \ne j} E_{\lambda_i} = \{ \OV \}\) for any \(j\). \BLUE{(1)}.

For the sake of contradiction, suppose not;
that is, there exists \emph{nonzero} \(v\) such that \(v \in E_{\lambda_j} \cap \sum_{i \ne j} E_{\lambda_i}\);
and WLOG, we just relabel the index such that
\(v \in E_{\lambda_1} \cap \sum_{i = 2}^k E_{\lambda_i}\).
So by definition, we have \(v = c_1 v_1\) for some eigenvector \(v_1 \in E_{\lambda_1}\) and \emph{nonzero} scalar \(c_1\), and \(v = c_2 v_2 + ... + c_k v_k\) for each \(v_i \in E_{\lambda_i}\) and \emph{nonzero} scalar \(c_i\), for \(2 \le i \le k\).
So
\[
    v = c_1 v_1 = c_2 v_2 + ... + c_k v_k, \quad \quad \MAROON{(1)}
\]
And
\begin{align*}
             & \MAROON{(1)} \\
    \implies & \T(c_1 v_1) = \T(c_2 v_2 + ... + c_k v_k) & \text{by applying \(\T\)} \\
    \implies & c_1\T(v_1) = c_2\T(v_2) + ... + c_k\T(v_k) & \text{since \(\T\) is linear} \\
    \implies & c_1 (\lambda_1 v_1) = \lambda_1 (c_1 v_1) = c_2 \lambda_2 v_2 + ... + c_k \lambda_k v_k & \text{by assumption}
\end{align*}
By replacing \(c_1 v_1\) of the LHS in the last equation above with the expression in \MAROON{(1)}, we have
\[
    \lambda_1 (\RED{c_2 v_2 + ... + c_k v_k}) = c_2 \lambda_2 v_2 + ... + c_k \lambda_k v_k,
\]
which implies
\[
    \OV = c_2 (\lambda_2 - \lambda_1) v_2 + c_3 (\lambda_3 - \lambda_1) v_3 + ... + c_k (\lambda_k - \lambda_1) v_k.
\]
But since (by \THM{5.5}) \(v_2, ..., v_k\) are \LID{}, \(c_2(\lambda_2 - \lambda_1), c_3(\lambda_3 - \lambda_1), ..., c_k(\lambda_k - \lambda_1)\) must all be zero.
But it's impossible, since \(c_i\) are nonzero, and all \(\lambda_i\) are distinct.
So we get a contradiction.
Hence \BLUE{(1)} is true, as desired.
\end{proof}

\begin{exercise} \label{exercise 5.2.24}
Let \(\W_1, \W_2, K_1, K_2, ..., K_p, M_1, M_2, ..., M_q\) be subspaces of a vector space \(\V\) such that

\BLUE{(1)} \(\W_1 = K_1 \oplus K_2 \oplus ... \oplus K_p\) and

\BLUE{(2)} \(\W_2 = M_1 \oplus M_2 \oplus ... \oplus M_q\).

Prove that if \BLUE{(3)} \(\W_1 \cap \W_2 = \{ \OV \}\), then
\[
    \W_1 + \W_2 = \W_1 \oplus \W_2 = K_1 \oplus K_2 \oplus ... \oplus K_p \oplus M_1 \oplus M_2 \oplus ... \oplus M_q.
\]
\end{exercise}

\begin{note}
\end{note}

\begin{proof}
We prove the first equality \(\W_1 + \W_2 = \W_1 \oplus \W_2\).
Then of course \(\BLUE{\W_1 + \W_2} = \RED{\W_1} + \RED{\W_2}\);
and by the assumption \BLUE{(3)}, \(\RED{\W_1} \cap \RED{\W_2} = \{ \OV \}\)
Hence by \DEF{5.9} we have \(\BLUE{\W_1 + \W_2} = \RED{\W_1} \oplus \RED{\W_2}\).

Now we prove the second equality \(\W_1 + \W_2 = K_1 \oplus K_2 \oplus ... \oplus K_p \oplus M_1 \oplus M_2 \oplus ... \oplus M_q\).
It's of course that \(\W_1 + \W_2 = K_1 + K_2 + ... + K_p + M_1 + M_2 + ... + M_q\):

\(v \in \W_1 + \W_2\),
iff, by \DEF{5.8}, \(v = k + m\) where \(k \in \W_1\) and \(m \in \W_2\),
iff, by the assumption \BLUE{(1)(2)} and \DEF{5.8} again, \(v = k + m\) where \(k = v_1 + v_2 + ... + v_p\) and \(m_2 = w_1 + w_2 + ... + w_q\) where \(v_i \in K_i\) and \(w_j \in M_j\),
iff (of course) \(v = v_1 + ... + v_p + w_1 + ... + w_q\) where \(v_i \in K_i\) and \(w_j \in M_j\),
iff, by \DEF{5.8} again, \(v \in K_1 + ... + K_p + M_1 + ... + M_q\).

So by (the second part of) \THM{5.9}(b), it suffices to show that:
\begin{center}
    if \(v_1 + v_2 + ... + v_p + w_1 + w_2 + ... + w_q = \OV\) where \(v_i \in K_i\) and \(w_j \in M_j\), then \(v_1 = v_2 = ... = v_p = w_1 = w_2 = ... = w_q = \OV\). \quad \quad \MAROON{(1)}
\end{center}
(Hence by the equivalent \THM{5.9}(a), we have \(\W_1 + \W_2 = K_1 \oplus K_2 \oplus ... \oplus K_p \oplus M_1 \oplus M_2 \oplus ... \oplus M_q\).)

So suppose the hypothesis of \MAROON{(1)} is true, and let
\[
    a = v_1 + v_2 + ... + v_p, b = w_1 + w_2 + ... + w_q. \quad \quad \MAROON{(2)}
\]
Then of course \(a + b = \OV\).
And in particular, since \(v_i \in K_i\) and \(w_j \in M_j\), (by \DEF{5.8}) \(a \in K_1 + K_2 + ... + K_p\) and \(b \in M_1 + M_2 + ... + M_q\),
which implies by the assumption \BLUE{(1)(2)}, \(a \in \W_1\) and \(b \in \W_2\).
So \(a + b = \OV\) where \(a \in \W_1\) and \(b \in \W_2\).

Then since \(\W_1 + \W_2 = \W_1 \oplus \W_2\), by \THM{5.9}(b) (in the case of \(\W_1 \oplus \W_2\)), \(a = b = \OV\).
Then by \MAROON{(2)}, \(v_1 + v_2 + ... + v_p = \OV\) and \(w_1 + w_2 + ... + w_q = \OV\).
Again, similarly, by \THM{5.9}(b) (in the case of \(K_1 \oplus K_2 \oplus ... \oplus K_p\) and \(M_1 \oplus M_2 \oplus ... \oplus M_q\), respectively), \(v_1 = v_2 = ... = v_p = \OV\), and \(w_1 = w_2 = ... = w_q = \OV\).
Hence we have shown that \(v_1 = v_2 = ... = v_p = w_1 = w_2 = ... = w_q = \OV\), as desired.
\end{proof}
